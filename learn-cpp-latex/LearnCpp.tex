\PassOptionsToPackage{quiet}{fontspec}
\documentclass[UTF8]{ctexart}

% 页面大小、宽距等
\usepackage[a4paper, total={6in, 8in}]{geometry}
% 字体
\usepackage{lmodern}
% 字体
\usepackage{amsfonts}
% 字体颜色
\usepackage{xcolor}
% 图形插入
\usepackage{graphicx}
% 引入图片文件夹
\graphicspath{\subfix{./images/}}
% 代码引用
\usepackage{listings}
% 外部链接
\usepackage{hyperref}
% 表格
\usepackage{tabularx}
% 表格合并行
\usepackage{multirow}
% 多文件处理(放置在所有依赖的末尾)
\usepackage{subfiles}


\title{Learn C++}
\author{Jacob Bishop}
\date{2022-09-01}

% adding style
\usepackage{./style}
\lstset{style=lcpp}

% a new command which severs arbitrary section number
\newcommand{\asection}[2]{
  \setcounter{section}{#1}
  \addtocounter{section}{-1}
  \section{#2}
}

% a new command which severs arbitrary subsection number
\newcommand{\asubsection}[2]{
  \setcounter{subsection}{#1}
  \addtocounter{subsection}{-1}
  \subsection{#2}
}

% a new command which simplify \lstinline
\newcommand{\acode}[1]{
  \lstinline[basicstyle=\ttfamily]{#1}
}

\begin{document}

\maketitle

\section{C++ Basics}

\subfile{Ch01/1.1 Statements and the structure of a program}
\subfile{Ch01/1.3 Introduction to objects and variables}
\subfile{Ch01/1.4 Variable assignment and initialization}
\subfile{Ch01/1.7 Keywords and naming identifiers}

\newpage

\section{Functions and Files}

\subfile{Ch02/2.1 Introduction to functions}
\subfile{Ch02/2.7 Forward declarations and definitions}
\subfile{Ch02/2.9 Naming collisions and an introduction to namespaces}
\subfile{Ch02/2.10 Introduction to the preprocessor}
\newpage % 为了使 2.11 图片正常显示
\subfile{Ch02/2.11 Header files}
\subfile{Ch02/2.12 Header guards}
\subfile{Ch02/2.13 How to design your first programs}

\newpage

\section{Debugging C++ Programs}

\subfile{Ch03-05/3.2 The debugging process}

\newpage

\section{Fundamental Data Types}

\subfile{Ch03-05/4.1 Introduction to fundamental data types}
\subfile{Ch03-05/4.3 Object sizes and the sizeof operator}
\subfile{Ch03-05/4.14 Compile-time constants, constant expressions, and constexpr}

\newpage

\section{Operators}

\subfile{Ch03-05/5.1 Operator precedence and associativity}

\newpage

\section{Scope, Duration, and Linkage}

\subfile{Ch06/6.1 Compound statements(blocks)}
\subfile{Ch06/6.2 User-defined namespaces and the scope resolution operator}
\subfile{Ch06/6.6 Internal linkage}
\subfile{Ch06/6.7 External linkage and variable forward declarations}
\subfile{Ch06/6.9 Sharing global constants across multiple files (using inline variables)}
\subfile{Ch06/6.14 Constexpr and consteval functions}
\subfile{Ch06/6.15 Unnamed and inline namespaces}

\newpage

\section{Control Flow and Error Handling}

\subfile{Ch07-08/7.1 Control flow introduction}

\newpage

\section{Type Conversion and Function Overloading}

\subfile{Ch07-08/8.5 Explicit type conversion (casting) and static_cast}
\subfile{Ch07-08/8.6 Typedefs and type aliases}
\subfile{Ch07-08/8.9 Introduction to function overloading}
\subfile{Ch07-08/8.10 Function overload differentiation}
\subfile{Ch07-08/8.11 Function overload resolution and ambiguous matches}
\subfile{Ch07-08/8.13 Function templates}
\subfile{Ch07-08/8.14 Function template instantiation}
\subfile{Ch07-08/8.15 Function templates with multiple template types}

\newpage

\section{Compound Types: References and Pointers}

\subfile{Ch09/9.1 Introduction to compound data types}
\subfile{Ch09/9.2 Value categories (lvalues and rvalues)}
\subfile{Ch09/9.3 Lvalue references}
\subfile{Ch09/9.4 Lvalue references to const}
\subfile{Ch09/9.5 Pass by lvalue reference}
\subfile{Ch09/9.6 Introduction to pointers}
\subfile{Ch09/9.7 Null pointers}
\subfile{Ch09/9.8 Pointers and const}
\subfile{Ch09/9.9 Pass by address}
\subfile{Ch09/9.11 Return by reference and return by address}
\subfile{Ch09/9.12 Type deduction with pointers, references, and const}

\newpage

\section{Compound Types: Enums and Structs}

\subfile{Ch10/10.2 Unscoped enumerations}
\subfile{Ch10/10.3 Unscoped enumeration input and output}
\subfile{Ch10/10.4 Scoped enumerations (enum classes)}
\subfile{Ch10/10.5 Introduction to structs, members, and member selection}
\subfile{Ch10/10.6 Struct aggregate initialization}
\subfile{Ch10/10.7 Default member initialization}
\subfile{Ch10/10.8 Struct passing and miscellany}
\subfile{Ch10/10.9 Member selection with pointers and references}
\subfile{Ch10/10.10 Class templates}
\subfile{Ch10/10.11 Class template argument deduction (CTAD) and deduction guides}

\newpage

\section{Arrays, Strings, and Dynamic Allocation}

\subfile{Ch11/11.1 Arrays}
\subfile{Ch11/11.3 Arrays and loops}
\subfile{Ch11/11.4 Sorting an array using selection sort}
\subfile{Ch11/11.8 Pointers and arrays}
\subfile{Ch11/11.9 Pointer arithmetic and array indexing}
\subfile{Ch11/11.11 Dynamic memory allocation with new and delete}
\subfile{Ch11/11.12 Dynamically allocating arrays}
\subfile{Ch11/11.13 For-each loops}
\subfile{Ch11/11.14 Void pointers}
\subfile{Ch11/11.16 An introduction to std::array}
\subfile{Ch11/11.17 An introduction to std::vector}
\subfile{Ch11/11.18 Introduction to iterators}
\subfile{Ch11/11.19 Introduction to standard library algorithms}

\newpage

\section{Functions}

\subfile{Ch12/12.1 Function Pointers}
\subfile{Ch12/12.2 The stack and the heap}
\subfile{Ch12/12.3 std::vector capacity and stack behavior}
\subfile{Ch12/12.4 Recursion}
\subfile{Ch12/12.5 Command line arguments}
\subfile{Ch12/12.7 Introduction to lambdas (anonymous functions)}
\subfile{Ch12/12.8 Lambda captures}

\newpage

\section{Basic Object-oriented Programming}

\subfile{Ch13/13.2 Classes and class members}
\subfile{Ch13/13.3 Public vs private access specifiers}
\subfile{Ch13/13.4 Access functions and encapsulation}
\subfile{Ch13/13.5 Constructors}
\subfile{Ch13/13.6 Constructor member initializer lists}
\subfile{Ch13/13.7 Non-static member initialization}
\subfile{Ch13/13.8 Overlapping and delegating constructors}
\subfile{Ch13/13.9 Destructors}
\subfile{Ch13/13.10 The hidden this pointer}
\subfile{Ch13/13.11 Class code and header files}
\subfile{Ch13/13.12 Const class objects and member functions}
\subfile{Ch13/13.13 Static member variables}
\subfile{Ch13/13.14 Static member functions}
\subfile{Ch13/13.15 Friend functions and classes}
\subfile{Ch13/13.16 Anonymous objects}
\subfile{Ch13/13.17 Nested types in classes}

\newpage

\section{Operator overloading}

\subfile{Ch14/14.1 Introduction to operator overloading}
\subfile{Ch14/14.2 Overloading the arithmetic operators using friend functions}
\subfile{Ch14/14.3 Overloading operators using normal functions}
\subfile{Ch14/14.4 Overloading the IO operators}
\subfile{Ch14/14.5 Overloading operators using member functions}
% \subfile{Ch14/}

\newpage

\section{Move Semantics and Smart Pointers}

\subfile{Ch15/15.1 Introduction to smart pointers and move semantics}
\subfile{Ch15/15.2 R-value references}
\subfile{Ch15/15.3 Move constructors and move assignment}
\subfile{Ch15/15.4 std::move}
\subfile{Ch15/15.5 std::move_if_noexcept}
\subfile{Ch15/15.6 std::unique_ptr}
\subfile{Ch15/15.7 std::shared_ptr}
\subfile{Ch15/15.8 Circular dependency issues with std::shared_ptr, and std::weak_ptr}

\newpage

\section{An Introduction to Object Relationships}

\subfile{Ch16/16.2 Composition}
\subfile{Ch16/16.3 Aggregation}
\subfile{Ch16/16.4 Association}
\subfile{Ch16/16.5 Dependencies}
\subfile{Ch16/16.6 Container classes}
\subfile{Ch16/16.7 std::initializer_list}

\newpage

\section{Inheritance}

\subfile{Ch17/17.2 Basic inheritance in C++}
\subfile{Ch17/17.3 Order of construction of derived classes}
\subfile{Ch17/17.4 Constructors and initialization of derived classes}
\subfile{Ch17/17.5 Inheritance and access specifiers}
\subfile{Ch17/17.6 Adding new functionality to a derived class}
\subfile{Ch17/17.7 Calling inherited functions and overriding behavior}
\subfile{Ch17/17.8 Hiding inherited functionality}
\subfile{Ch17/17.9 Multiple inheritance}

\newpage

\section{Virtual Functions}

\subfile{Ch18/18.1 Pointers and references to the base class of derived objects}
\subfile{Ch18/18.2 Virtual functions and polymorphism}
\subfile{Ch18/18.3 The override and final specifiers, and covariant return types}
\subfile{Ch18/18.4 Virtual destructors, virtual assignment, and overriding virtualization}
\subfile{Ch18/18.5 Early binding and late binding}
\subfile{Ch18/18.6 The virtual table}
\subfile{Ch18/18.7 Pure virtual functions, abstract base classes, and interface classes}
\subfile{Ch18/18.8 Virtual base classes}
\subfile{Ch18/18.9 Object slicing}
\subfile{Ch18/18.10 Dynamic casting}
\subfile{Ch18/18.11 Printing inherited classes using operator<<}

\newpage

\section{Templates and Classes}

\subfile{Ch19/19.1 Template classes}
\subfile{Ch19/19.2 Template non-type parameters}
\subfile{Ch19/19.3 Function template specialization}
\subfile{Ch19/19.4 Class template specialization}
\subfile{Ch19/19.5 Partial template specialization}
\subfile{Ch19/19.6 Partial template specialization for pointers}

\newpage

\section{Exceptions}

\subfile{Ch20/20.1 The need for exceptions}
\subfile{Ch20/20.2 Basic exception handling}
\subfile{Ch20/20.3 Exceptions, functions, and stack unwinding}
\subfile{Ch20/20.4 Uncaught exceptions and catch-all handlers}
\subfile{Ch20/20.5 Exceptions, classes, and inheritance}
\subfile{Ch20/20.6 Rethrowing exceptions}
\subfile{Ch20/20.7 Function try blocks}
\subfile{Ch20/20.8 Exception dangers and downsides}
\subfile{Ch20/20.9 Exception specifications and noexcept}

\newpage

\section{The Standard Template Library}

\subfile{Ch21/21.1 The Standard Library}
\subfile{Ch21/21.2 STL containers overview}
\subfile{Ch21/21.3 STL iterators overview}
\subfile{Ch21/21.4 STL algorithms overview}

\newpage

\section{std::string}

\subfile{Ch22/22.1 std::string and std::wstring}
\subfile{Ch22/22.2 std::string construction and destruction}
\subfile{Ch22/22.3 std::string length and capacity}
\subfile{Ch22/22.4 std::string character access and conversion to C-style arrays}
\subfile{Ch22/22.5 std::string assignment and swapping}
\subfile{Ch22/22.6 std::string appending}
\subfile{Ch22/22.7 std::string inserting}

\newpage

\section{Input and Output (I/O)}

\newpage

\section{Miscellaneous Subjects}

\end{document}
