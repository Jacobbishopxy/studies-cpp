\documentclass[../../LearnCpp.tex]{subfiles}

\begin{document}

\asubsection{12}{Header guards}

通过\textbf{头文件保护符}这样一种机制可以避免重复定义这样的问题(也可被称为 \textbf{include guard})。
通过以下形式,头文件保护符可以条件编译指令:

\begin{lstlisting}[language=C++]
#ifndef SOME_UNIQUE_NAME_HERE
#define SOME_UNIQUE_NAME_HERE

// 用户定义(以及某些类型的定义)于此

#endif
\end{lstlisting}

当该头文件被 \#included,预处理器检查 \textit{SOME\_UNIQUE\_NAME\_HERE} 是否之前就被定义了。
如果是第一次引入该头文件,此时 \textit{SOME\_UNIQUE\_NAME\_HERE} 还未被定义。
因此 \#define \textit{SOME\_UNIQUE\_NAME\_HERE} 并且引入文件内容。
如果该头文件再次被引入至同一个文件,\textit{SOME\_UNIQUE\_NAME\_HERE} 已经被定义了,该头文件则被忽略(多亏了 \#ifndef)。

举个列子:

square.h:

\begin{lstlisting}[language=C++]
#ifndef SQUARE_H
#define SQUARE_H

int getSquareSides()
{
   return 4;
}

#endif
\end{lstlisting}

geometry.h:

\begin{lstlisting}[language=C++]
#ifndef GEOMETRY_H
#define GEOMETRY_H

#include "square.h"

#endif
\end{lstlisting}

main.cpp:

\begin{lstlisting}[language=C++]
#include "square.h"
#include "geometry.h"

int main()
{
   return 0;
}
\end{lstlisting}

在预处理器解决完所有的 \#include 指令后,程序会类似于这样:

main.cpp:

\begin{lstlisting}[language=C++]
// Square.h 被 main.cpp include
#ifndef SQUARE_H // square.h 被 main.cpp include
#define SQUARE_H // SQUARE_H 获得定义

// 其所有的内容在此被 include
int getSquareSides()
{
   return 4;
}

#endif // SQUARE_H

#ifndef GEOMETRY_H   // geometry.h 被 main.cpp include
#define GEOMETRY_H
#ifndef SQUARE_H     // square.h 被 geometry.h include, SQUARE_H 之前已经被定义过了
#define SQUARE_H     // 因此没有内容在此被 include

int getSquareSides()
{
   return 4;
}

#endif // SQUARE_H
#endif // GEOMETRY_H

int main()
{
   return 0;
}
\end{lstlisting}

\subsubsection*{头文件保护符不阻止头文件被不同的代码文件引用多次}

square.h:

\begin{lstlisting}[language=C++]
#ifndef SQUARE_H
#define SQUARE_H

int getSquareSides()
{
   return 4;
}

int getSquarePerimeter(int sideLength); // getSquarePerimeter 的前置定义

#endif
\end{lstlisting}

square.cpp:

\begin{lstlisting}[language=C++]
#include "square.h"  // square.h 在这里被引入一次

int getSquarePerimeter(int sideLength)
{
   return sideLength * getSquareSides();
}
\end{lstlisting}

main.cpp:

\begin{lstlisting}[language=C++]
#include "square.h" // square.h 在这里又被引入一次
#include <iostream>

int main()
{
   std::cout << "a square has " << getSquareSides() << " sides\n";
   std::cout << "a square of length 5 has perimeter length " << getSquarePerimeter(5) << '\n';

   return 0;
}
\end{lstlisting}

注意 \textit{square.h} 同时被 \textit{main.cpp} 和 \textit{square.cpp} 引入了,
这就意味着 \textit{square.h} 的内容也被分别引入进了两个代码文件。
最终结果就是项目可以被编译但是 linker 会告知有多个 \textit{getSquareSides} 标识符的定义。

避免这种情况最好的办法是简单的放置函数定义进一个 .cpp 文件,因此头文件仅需包含前置定义:

square.h:

\begin{lstlisting}[language=C++]
#ifndef SQUARE_H
#define SQUARE_H

int getSquareSides(); // getSquareSides 的前置定义
int getSquarePerimeter(int sideLength); // getSquarePerimeter 的前置定义

#endif
\end{lstlisting}

square.cpp:

\begin{lstlisting}[language=C++]
#include "square.h"

int getSquareSides() // getSquareSides 的实际定义
{
return 4;
}

int getSquarePerimeter(int sideLength)
{
   return sideLength * getSquareSides();
}
\end{lstlisting}

main.cpp:

\begin{lstlisting}[language=C++]
#include "square.h" // square.h 在此也被引用
#include <iostream>

int main()
{
   std::cout << "a square has " << getSquareSides() << "sides\n";
   std::cout << "a square of length 5 has perimeter length " << getSquarePerimeter(5) << '\n';

   return 0;
}
\end{lstlisting}

\subsubsection*{\#pragma once}

现代编译器提供了另外一个更简单的头文件保护符,那就是使用 \textit{\#pragma} 指令:

\begin{lstlisting}[language=C++]
#pragma once

// 用户代码
\end{lstlisting}

\acode{\#pragma once} 更加简短以及不易出错。
对于大多数项目而言 \acode{\#pragma once} 就可以很好的工作了。
然而 \acode{\#pragma once} 并不属于 C++ 的官方语言(并且将永远不是,因为它不能在所有情况下都提供可靠性)。

\end{document}
