\documentclass[../../LearnCpp.tex]{subfiles}

\begin{document}

\asubsection{10}{Introduction to the preprocessor}

\subsubsection*{翻译与预处理器}

当编译器编译代码时,用户可能期望编译器能完全按照写好的代码来进行编译。实际上并非如此。

比编译更提前的是,代码文件会通过一个\textbf{翻译}阶段。很多事情会发生在该阶段。
一个被应用的代码文件可以被称为一个\textbf{翻译单元}。

其中翻译阶段中最值得注意的是预处理器。
预处理器最好被视为一个独立的程序用于操作每个代码文件。

当预处理器运行时,它扫描整个代码文件(从上至下),寻找预处理器指令。
\textbf{预处理器指令}(通常被称为\textit{指令} directives)开始与 \# 号结束于新的一行(而不是分号)。
这些指令告诉预处理器执行指定的文本操作任务。
注意预处理器不能理解 C++ 的语法,相反的是,
其拥有自身的语法(在某些情况下与 C++ 语法相似,其他情况下则不同)。

预处理器的输出通过若干的翻译阶段,然后才被编译。
注意预处理器无论如何也不会修改原始的代码文件,更确切来说,
所有通过预处理器的文本变化只会发生在每次编译代码文件时的临时内存里或是临时文件上。

\subsubsection*{Includes}

当 \acode{\#include} 一个文件,预处理器用引用的文件内容替换掉 \#include 指令。
该引用的内容则被预处理(相对于文件的其他部分)并被编译。

\subsubsection*{宏定义}

\acode{\#define} 指令可以用于创建一个宏。
在 C++ 中,一个\textbf{宏}是一个定义了如何输入文本转为替换的输出文本的规则。

有两种基本的宏类型:\textit{类对象宏} object-like macros 以及 \textit{类函数宏} function-like macros。

\textit{类函数宏}类似于函数,服务于相同的目的。
这里不做讨论,因为它们的使用通常被视为有风险的,同时它们能完成的东西,普通函数也能胜任。

\textit{类对象宏}可以被定义为以下两种方式中的一种:

\begin{lstlisting}[language=C++]
#define identifier
#define identifier substitution_text
\end{lstlisting}

前者没有替换文本,而后者有。它们都是预处理器指令(非语句),因此不需要以分号结束。

\subsubsection*{有替代文本的类对象宏}

当预处理器遇到这类指令,任何之后出现这个的标识符将会被替换为\textit{替换文本} substitution\_text。
标识符通常使用全大写,并使用下划线来表示空格。

\begin{lstlisting}[language=C++]
#include <iostream>

#define MY_NAME "Alex"

int main()
{
    std::cout << "My name is: " << MY_NAME;

    return 0;
}
\end{lstlisting}

预处理器转换上述代码为:

\begin{lstlisting}[language=C++]
// iostream 的内容在此被插入

int main()
{
    std::cout << "My name is: " << "Alex";

    return 0;
}
\end{lstlisting}

带有替代文本的类对象被使用为(在 C 中)一种指定名称为字面值的方式。
这不再必要了,因为 C++ 拥有更好的方法了。此类类对象通常只在历史代码中出现。

推荐避免该类宏,因为有更适合的做法(将会在 4.13 中提到)。

\subsubsection*{无替代文本的类对象宏}

\begin{lstlisting}
#define USE_YEN
\end{lstlisting}

这种形式的宏正如所见:在之后出现该标识符的地方,用无做替换。

这看起来很没用,实际上确实对于文本替换而言是无用的。
然而,这种形式的指令通常不是这么用的。稍后会提到用法。

有别于有替代文本的类对象宏,这种形式的宏通常是可以被接受的。

\subsubsection*{条件编译}

\textit{条件编译}预处理指令允许用户有条件的选择编译与否。
有不同类型的条件编译指令,这里只覆盖三种最常用的:\acode{\#ifdef},\acode{\#ifndef} 以及 \acode{\#endif}。

\acode{\#ifdef} 预处理器指令允许预处理器检查标识符在之前是否被 \acode{\#define} 了。
如果是,则介于 \acode{\#ifdef} 与 \acode{\#endif} 之间的代码将被编译。如果否,则代码被忽略。

\begin{lstlisting}[language=C++]
#include <iostream>

#define PRINT_JOE

int main()
{
#ifdef PRINT_JOE
    std::cout << "Joe\n"; // 将会被编译,因为 PRINT_JOE 被定义了
#endif

#ifdef PRINT_BOB
    std::cout << "Bob\n"; // 将会被忽略,因为 PRINT_BOB 没有被定义
#endif

    return 0;
}
\end{lstlisting}

\acode{\#ifndef} 则与 \acode{\#ifdef} 相反。

\subsubsection*{\#if 0}

\begin{lstlisting}[language=C++]
#include <iostream>

int main()
{
    std::cout << "Joe\n";

#if 0 // 从这里开始不做编译
    std::cout << "Bob\n";
    std::cout << "Steve\n";
#endif // 直到这里

    return 0;
}
\end{lstlisting}

\subsubsection*{定义的域}

指令处理是先于编译的,根据文件顺序而从上至下执行的。

观察一下程序:

\begin{lstlisting}[language=C++]
#include <iostream>

void foo()
{
#define MY_NAME "Alex"
}

int main()
{
    std::cout << "My name is: " << MY_NAME;

    return 0;
}
\end{lstlisting}

尽管 \acode{\#define MY\_NAME "Alex"} 看起来像是定义在 \textit{foo} 函数中,
预处理器并不会察觉,因为它不理解 C++ 概念例如函数。
因此,该程序与无论在函数前后定义 \acode{\#define MY\_NAME "Alex"} 的程序无异。
对于普遍的阅读性而言,应该让 \acode{\#define} 标识符与函数外。

一旦预处理器结束时,文件中所有的标识符会被丢弃。
这意味着指令仅仅作用于其定义位置至文件结束。
同一个项目中,一个代码文件中的指令不会影响其他的代码文件。

\end{document}
