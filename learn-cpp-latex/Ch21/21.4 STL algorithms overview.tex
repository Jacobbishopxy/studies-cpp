\documentclass[../../LearnCpp.tex]{subfiles}

\begin{document}

\asubsection{4}{STL algorithms overview}

除了容器类与迭代器,STL 也提供了很多泛型算法用于处理容器类中的元素。
这些算法提供了比如查询,排序,插入,重排,移除,以及拷贝容器类中的元素。

注意算法被实现为用来操作迭代器的函数。
这就意味着每个算法仅需要被实现一次,它便会自动的对所有提供了迭代器的容器生效(包括自定义容器类)。
虽然这是非常强大的且带来了快速编写复杂代码的能力,但是它同样也有黑暗面:
一些算法的组合与容器类型可能是无效的,可能导致无限循环,或者是有效但性能极差。
因此使用它们需要承担风险。

STL 提供不少的算法 -- 这里只介绍一些常用的,其它的将留在新的章节讲述(暂未实现)。

要使用 STL 算法仅需引入 \acode{<algorithm>} 头文件。

\subsubsection*{min\_element 与 max\_element}

\acode{std::min\_element} 与 \acode{std::max\_element} 算法查询容器类中最小和最大的元素。\acode{std::iota} 生成一个连续序列的值。

\begin{lstlisting}[language=C++]
#include <algorithm> // std::min_element 与 std::max_element
#include <iostream>
#include <list>
#include <numeric> // std::iota

int main()
{
    std::list<int> li(6);
    // 从 0 开始填充 li
    std::iota(li.begin(), li.end(), 0);

    std::cout << *std::min_element(li.begin(), li.end()) << ' '
              << *std::max_element(li.begin(), li.end()) << '\n';

    return 0;
}
\end{lstlisting}

\subsubsection*{find (list::insert)}

下面的例子中使用 \acode{std::find()} 算法来查找 list 类中的值,
接着使用 \acode{list::insert()} 函数在该值的位置添加新值。

\begin{lstlisting}[language=C++]
#include <algorithm>
#include <iostream>
#include <list>
#include <numeric>

int main()
{
    std::list<int> li(6);
    std::iota(li.begin(), li.end(), 0);

    // 查找 list 中的值 3
    auto it{ std::find(li.begin(), li.end(), 3) };

    // 在 3 的位置之前插入 8
    li.insert(it, 8);

    for (int i : li) // 迭代器的 for 循环
        std::cout << i << ' ';

    std::cout << '\n';

    return 0;
}
\end{lstlisting}

打印:

\begin{lstlisting}
0 1 2 8 3 4 5
\end{lstlisting}

\subsubsection*{排序与翻转}

\begin{lstlisting}[language=C++]
#include <iostream>
#include <vector>
#include <algorithm>

int main()
{
    std::vector<int> vect{ 7, -3, 6, 2, -5, 0, 4 };

    // vector 排序
    std::sort(vect.begin(), vect.end());

    for (int i : vect)
    {
        std::cout << i << ' ';
    }

    std::cout << '\n';

    // vector 翻转
    std::reverse(vect.begin(), vect.end());

    for (int i : vect)
    {
        std::cout << i << ' ';
    }

    std::cout << '\n';

    return 0;
}
\end{lstlisting}

打印:

\begin{lstlisting}
-5 -3 0 2 4 6 7
7 6 4 2 0 -3 -5
\end{lstlisting}

另外,用户可以提供自定义的比较函数在 \acode{std::sort} 的第三个参数的位置。
\acode{functional} 头文件中拥有若干比较函数可供使用。
例如可以传递 \acode{std::greater} 给 \acode{std::sort} 接着移除 \acode{std::reverse}。
vector 也会从高到底进行排序。

注意 \acode{std::sort} 对于 list 容器类不起作用 --
list 类提供了自身的 \acode{sort()} 成员函数,其比通用的版本更加高效。

\end{document}
