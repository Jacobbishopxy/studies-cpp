\documentclass[../../LearnCpp.tex]{subfiles}

\begin{document}

\asubsection{11}{Class template argument deduction (CTAD) and deduction guides}

\subsubsection*{Class 模板参数推导(CTAD)(C++17)}

从 C++17 开始,当一个对象从 class 模板中实例化时,编译器可以根据对象的初始化类型中推导模板类型(CTAD):

\begin{lstlisting}[language=C++]
#include <utility> // std::pair

int main()
{
    std::pair<int, int> p1{ 1, 2 }; // 显式指定 std::pair<int, int> (C++11 onward)
    std::pair p2{ 1, 2 };           // CTAD 用于推导 std::pair<int, int> (C++17)

    return 0;
}
\end{lstlisting}

CTAD 仅作用于没有模板参数时。因此以下两种做法都是错误的:

\begin{lstlisting}[language=C++]
#include <utility> // std::pair

int main()
{
    std::pair<> p1 { 1, 2 };    // 错误:太少的模板参数,两个参数都没有被推导
    std::pair<int> p2 { 3, 4 }; // 错误:太少的模板参数,第二个参数没有被推导

    return 0;
}
\end{lstlisting}

\subsubsection*{模板参数推导指南(C++17)}

多数情况下,CTAD 都可以正常工作。
然而在特定情况下,编译器可能会需要一些额外的帮助用于理解如何正确的推导出模板类型。

以下的代码在 C++17 中不能被编译:

\begin{lstlisting}[language=C++]
// 自定义 Pair 类型
template <typename T, typename U>
struct Pair
{
    T first{};
    U second{};
};

int main()
{
    Pair<int, int> p1{ 1, 2 }; // 可以:显式的指定模板参数
    Pair p2{ 1, 2 };           // C++17 中编译错误

    return 0;
}
\end{lstlisting}

这是因为在 C++17 中,CTAD 并不知道如何为聚合类型进行模板参数推导。
为了解决这个问题,可以提供编译器一个\textbf{推导指南},
这样可以提示编译器如何在给定的 class 模板中进行正确的模板参数推导。

\begin{lstlisting}[language=C++]
template <typename T, typename U>
struct Pair
{
    T first{};
    U second{};
};

// 这里是 Pair 的推导指南
// Pair 对象通过参数类型 T 与 U 实例化成为 Pair<T, U>
template <typename T, typename U>
Pair(T, U) -> Pair<T, U>;

int main()
{
    Pair<int, int> p1{ 1, 2 }; // 显式指定模板参数 Pair<int, int>
    Pair p2{ 1, 2 };     // CTAD 推导出 Pair<int, int>

    return 0;
}
\end{lstlisting}

\acode{Pair} class 的推导指南很简单,现在看一看它是如何工作的。

\begin{lstlisting}[language=C++]
template <typename T, typename U>
Pair(T, U) -> Pair<T, U>;
\end{lstlisting}

首先,使用相同的模板类型定义 \acode{Pair} 类。
这很容易理解,因为如果推导指南告诉编译器如何为 \acode{Pair<T, U>} 进行类型推导,
需要同时定义 \acode{T} 和 \acode{U}。
其次,在箭头的右侧是帮助编译器推导的类型。
这里就是希望编译器可以为对象推导出类型 \acode{Pair<T, U>}。
最后再箭头的左侧告诉编译器什么类型的声明是它需要知道的。

\end{document}
