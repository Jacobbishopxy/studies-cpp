\documentclass[../../LearnCpp.tex]{subfiles}

\begin{document}

\asubsection{2}{Unscoped enumerations}

\subsubsection*{枚举}

\textbf{枚举} enumeration(也被称为 \textbf{enumerated type} 或是 \textbf{enum})
是一种包含了所有可能值并且每个值被定义为符号常量(称为 \textbf{enumerator})的组合型数据类型。

C++ 支持两种类型的枚举:无范围枚举 unscoped enumerations(本章讲解),
以及有范围枚举 scoped enumerations(下一章讲解)。

\subsubsection*{无范围枚举}

通过 \acode{enum} 关键字定义无范围枚举。

\begin{lstlisting}[language=C++]
// 定义一个名为 Color 的无范围枚举
enum Color
{
    // 这些为成员
    // 这些象征性的常量定义了所有该枚举可能出现的值,通过逗号分隔而不是分号
    red,
    green,
    blue, // 最后一个逗号是可选的,推荐使用
}; // 枚举定义的最后必须由分号结束

int main()
{
    // 定义一些枚举类类型 Color 的变量
    Color apple { red };   // apple 为 red
    Color shirt { green }; // shirt 为 green
    Color cup { blue };    // cup 为 blue

    Color socks { white }; // 错误:white 不是 Color 枚举的成员
    Color hat { 2 };       // 错误:2 不是 Color 枚举的成员

    return 0;
}
\end{lstlisting}

\subsubsection*{枚举以及其成员的命名}

为了方便枚举类都由大写字母开头(与成语定义类型相同)。

警告:枚举不需要被命名,但是未命名的枚举在现代 C++ 中应该避免。

\subsubsection*{枚举的类型是独特类型}

每个被创建的枚举的类型被视作\textbf{独特类型} distinct type,
意味着编译器可以从其他类型中做出区分(不同于 typedefs 或者类型别名,
它们被视为与其所别名的类型为非独特类型)。

因为枚举类型是独特类型,枚举成员为枚举中的一部分,因此成员不可以用于其他枚举类型的对象:

\begin{lstlisting}[language=C++]
enum Pet
{
    cat,
    dog,
    pig,
    whale,
};

enum Color
{
    black,
    red,
    blue,
};

int main()
{
    Pet myPet { black }; // 编译错误:black 不是 Pet 的成员
    Color shirt { pig }; // 编译错误:pig 不是 Color 的成员

    return 0;
}
\end{lstlisting}

\subsubsection*{使用枚举}

因为枚举成员是具有描述性质的,它们对于增强代码可读性有非常大的作用。
枚举类型最好是用于有一小个集合而又有关联的常量,对象只需要一次涵盖这些值即可。

\begin{lstlisting}[language=C++]
enum DaysOfWeek
{
    sunday,
    monday,
    tuesday,
    wednesday,
    thursday,
    friday,
    saturday,
};

enum CardinalDirections
{
    north,
    east,
    south,
    west,
};

enum CardSuits
{
    clubs,
    diamonds,
    hearts,
    spades,
};
\end{lstlisting}

有时函数可以返回状态码给调用者来只是函数的执行是否顺利或是遇到何种错误。
传统的做法是用一些小的负值来代表何种错误:

\begin{lstlisting}[language=C++]
int readFileContents()
{
    if (!openFile())
        return -1;
    if (!readFile())
        return -2;
    if (!parseFile())
        return -3;

    return 0; // success
}
\end{lstlisting}

然而直接使用值并没有任何描述性,因此使用枚举类型将会是另一种更好的方式:

\begin{lstlisting}[language=C++]
enum FileReadResult
{
    readResultSuccess,
    readResultErrorFileOpen,
    readResultErrorFileRead,
    readResultErrorFileParse,
};

FileReadResult readFileContents()
{
    if (!openFile())
        return readResultErrorFileOpen;
    if (!readFile())
        return readResultErrorFileRead;
    if (!parseFile())
        return readResultErrorFileParse;

    return readResultSuccess;
}
\end{lstlisting}

接着调用者可以测试函数的返回值,相比于返回数值,这样便于更好的理解:

\begin{lstlisting}[language=C++]
if (readFileContents() == readResultSuccess)
{
    // do something
}
else
{
    // print error message
}
\end{lstlisting}

枚举类型也可以作为函数入参:

\begin{lstlisting}[language=C++]
enum SortOrder
{
    alphabetical,
    alphabeticalReverse,
    numerical,
};

void sortData(SortOrder order)
{
    if (order == alphabetical)
        // 根据字母顺序排序
    else if (order == alphabeticalReverse)
        // 根据字母反序顺序排序
    else if (order == numerical)
        // 根据数值排序
}
\end{lstlisting}

\subsubsection*{无范围枚举的作用域}

无范围枚举之所以有这么个名字是因为它们把成员的名称放入了与枚举其自身所定义的作用域中
(而不是创建一个新的作用域,类似于命名空间那样)。

\begin{lstlisting}[language=C++]
enum Color // enum 定义在全局命名空间
{
    red, // 因此 red 也定义在全局命名空间
    green,
    blue,
};

int main()
{
    Color apple { red };
。
    return 0;
}
\end{lstlisting}

这就会导致拥有相同成员的若干枚举会发生命名冲突:

\begin{lstlisting}[language=C++]
enum Color
{
    red,
    green,
    blue, // blue 放入了全局命名空间
};

enum Feeling
{
    happy,
    tired,
    blue, // 错误:命名冲突
};

int main()
{
    Color apple { red };
    Feeling me { happy };

    return 0;
}
\end{lstlisting}

\subsubsection*{避免命名冲突}

\begin{lstlisting}[language=C++]
enum Color
{
    color_red,
    color_blue,
    color_green,
};

enum Feeling
{
    feeling_happy,
    feeling_tired,
    feeling_blue, // 不再有冲突
};

int main()
{
    Color paint { color_blue };
    Feeling me { feeling_blue };

    return 0;
}
\end{lstlisting}

而更好的做法是给它们提供各自的命名空间:

\begin{lstlisting}[language=C++]
namespace color
{
    // Color 枚举与成员都定义在 color 命名空间内
    enum Color
    {
        red,
        green,
        blue,
    };
}

namespace feeling
{
    enum Feeling
    {
        happy,
        tired,
        blue, // feeling::blue 不会与 color::blue 冲突
    };
}

int main()
{
    color::Color paint { color::blue };
    feeling::Feeling me { feeling::blue };

    return 0;
}
\end{lstlisting}

最佳实践:推荐枚举放入有名称的空间(例如命名空间或者类),这样就不会污染全局的命名空间了。

\subsubsection*{枚举的对比}

\begin{lstlisting}[language=C++]
#include <iostream>

enum Color
{
    red,
    green,
    blue,
};

int main()
{
    Color shirt{ blue };

    if (shirt == blue)
        std::cout << "Your shirt is blue!";
    else
        std::cout << "Your shirt is not blue!";

    return 0;
}
\end{lstlisting}

\end{document}
