\documentclass[../../LearnCpp.tex]{subfiles}

\begin{document}

\asubsection{4}{Scoped enumerations (enum classes)}

尽管无范围枚举在 C++ 中是独立类型,它们并不是类型安全的,有时候甚至会允许用户做没有意义的行为。

\begin{lstlisting}[language=C++]
#include <iostream>

int main()
{
    enum Color
    {
        red,
        blue,
    };

    enum Fruit
    {
        banana,
        apple,
    };

    Color color { red };
    Fruit fruit { banana };

    if (color == fruit) // 编译器将会作整数来比较 color 和 fruit
        std::cout << "color and fruit are equal\n"; // 并得出它们相等!
    else
        std::cout << "color and fruit are not equal\n";

    return 0;
}
\end{lstlisting}

\subsubsection*{有范围枚举}

上述问题的解决方案是使用\textbf{有范围枚举} scoped enumeration(C++ 中通常被称为 \textbf{enum class})。

有范围枚举类似于无范围枚举,不过主要有两个最大的不同:它们是强类型的(即不会隐式转换成整数),
以及强作用域的(枚举成员\textit{仅仅}放置在枚举自身域内)。

创建有范围枚举可以使用 \acode{enum class},其余的部分则跟无范围枚举的定义一致:

\begin{lstlisting}[language=C++]
#include <iostream>
int main()
{
    enum class Color // "enum class" 定义为有范围的枚举
    {
        red, // red 被视为 Color 作用域的一部分
        blue,
    };

    enum class Fruit
    {
        banana, // banana 被视为 Fruit 作用域的一部分
        apple,
    };

    Color color { Color::red }; // 注意:red 不再是可以直接被访问到的,必须使用 Color::red
    Fruit fruit { Fruit::banana }; // 注意:banana 不再是可以直接被访问到的,必须使用 Fruit::banana

    if (color == fruit) // 编译错误:编译器不知道如何比较两个不同的类型
        std::cout << "color and fruit are equal\n";
    else
        std::cout << "color and fruit are not equal\n";

    return 0;
}
\end{lstlisting}

额外话:\acode{class} 关键字(同样也有 \acode{static} 关键字),在 C++ 中是一个最超载的关键字,
根据不同的语境会有不同的意思。尽管有范围枚举使用了 \acode{class} 关键字,
其也不被视为“class 类”(即为 structs,classes,以及 unions 所保留的)。

\subsubsection*{有范围枚举定义自己的作用域}

有别于无范围枚举,即枚举成员与枚举在同一个作用域中,
有范围枚举使其成员\textit{仅仅}处于其自身定义的作用域中。
换言之,有范围枚举作用类似于命名空间中的枚举成员。
内建的命名空间减少了全局命名空间污染,以及潜在的命名冲突问题。

\subsubsection*{有范围枚举不会隐式转换成整数}

不同于无范围枚举,有范围枚举不会隐式转换成整数。

最佳实践:更加推荐有范围枚举而不是无范围枚举,除非是有令人信服的原因。

\acode{using enum} 声明(C++20)

在 C++20 中,\acode{using enum} 声明可以引入所有的枚举成员至当前作用域。
当使用的是 enum class 类型,这使得用户可以在无需添加前缀的情况下访问 enum class 的成员。

这在拥有很多标识符,以及重复的前缀(例如在 switch 声明中),用起来非常的方便:

\begin{lstlisting}[language=C++]
#include <iostream>
#include <string_view>

enum class Color
{
    black,
    red,
    blue,
};

constexpr std::string_view getColor(Color color)
{
    using enum Color; // 将所有的 Color 枚举成员引入当今作用域(C++20)
    // 现在可以不再使用 Color:: 作为前缀来访问枚举成员了

    switch (color)
    {
    case black: return "black"; // 注意:black 而不是 Color::black
    case red:   return "red";
    case blue:  return "blue";
    default:    return "???";
    }
}

int main()
{
    Color shirt{ Color::blue };

    std::cout << "Your shirt is " << getColor(shirt) << '\n';

    return 0;
}
\end{lstlisting}

\end{document}
