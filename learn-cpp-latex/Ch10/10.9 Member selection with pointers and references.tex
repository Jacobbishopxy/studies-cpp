\documentclass[../../LearnCpp.tex]{subfiles}

\begin{document}

\asubsection{9}{Member selection with pointers and references}

\subsubsection*{struct 指针的成员选择}

然而,使用成员选择操作符(.)在 struct 指针上不起作用:

\begin{lstlisting}[language=C++]
#include <iostream>

struct Employee
{
    int id{};
    int age{};
    double wage{};
};

int main()
{
    Employee joe{ 1, 34, 65000.0 };

    ++joe.age;
    joe.wage = 68000.0;

    Employee* ptr{ &joe };
    std::cout << ptr.id << '\n'; // 编译错误:不能在指针上使用操作符 \acode{.}

    return 0;
}
\end{lstlisting}

对于普通变量或者引用而言,可以直接访问对象。然而因为指针存储的是地址,
因此需要先对指针进行解引用拿到对象后才能访问。所以访问 struct 指针的成员可以这样做:

\begin{lstlisting}[language=C++]
#include <iostream>

struct Employee
{
    int id{};
    int age{};
    double wage{};
};

int main()
{
    Employee joe{ 1, 34, 65000.0 };

    ++joe.age;
    joe.wage = 68000.0;

    Employee* ptr{ &joe };
    std::cout << (*ptr).id << '\n'; // 不太合适,但是至少工作:先解引用再使用成员选择

    return 0;
}
\end{lstlisting}

然而这有点丑陋,特别是还需要用括号包住解引用操作符以便使解引用优先于成员选择符。

为了有更清晰的语法,C++ 提供了\textbf{从指针选择成员符(->)}(有时称为 \textbf{arrow operator})
用于从指向对象的指针中选择成员:

\begin{lstlisting}[language=C++]
#include <iostream>

struct Employee
{
    int id{};
    int age{};
    double wage{};
};

int main()
{
    Employee joe{ 1, 34, 65000.0 };

    ++joe.age;
    joe.wage = 68000.0;

    Employee* ptr{ &joe };
    std::cout << ptr->id << '\n'; // 更好:使用 -> 选择指向对象指针的成员

    return 0;
}
\end{lstlisting}

\subsubsection*{成员混合指针与非指针}

\begin{lstlisting}[language=C++]
#include <iostream>
#include <string>

struct Paw
{
    int claws{};
};

struct Animal
{
    std::string name{};
    Paw paw{};
};

int main()
{
    Animal puma{ "Puma", { 5 } };

    Animal* ptr{ &puma };

    // ptr 是指针,使用 ->
    // paw 非指针,使用 .

    std::cout << (ptr->paw).claws << '\n';

    return 0;
}
\end{lstlisting}

注意 \acode{(ptr -> paw).claws} 的圆括号不是必须的,
因为 \acode{->} 和 \acode{.} 两个符号都是从左至右计算的,
加上括号可以增加可读性。

\end{document}
