\documentclass[../../LearnCpp.tex]{subfiles}

\begin{document}

\asubsection{3}{Unscoped enumeration input and output}

上一章讲到了枚举成员是符号常量,但是没有提到它们是\textit{整数型}符号常量。
也就是说,枚举可以存储整数值。

\begin{lstlisting}[language=C++]
#include <iostream>

enum Color
{
    black, // 赋值 0
    red, // 赋值 1
    blue, // 赋值 2
    green, // 赋值 3
    white, // 赋值 4
    cyan, // 赋值 5
    yellow, // 赋值 6
    magenta, // 赋值 7
};

int main()
{
    Color shirt{ blue }; // 这里实际上存储的是整数 2

    return 0;
}
\end{lstlisting}

显式的定义成员值也是可以的。这里的整数值可以为正为负,也可以共享同一个值。
任何没定义的成员将被赋予一个较前成员的值增加 1 的整数:

\begin{lstlisting}[language=C++]
enum Animal
{
    cat = -3,
    dog,         // 赋值 -2
    pig,         // 赋值 -1
    horse = 5,
    giraffe = 5, // 与 horse 值相同
    chicken,      // 赋值 6
};
\end{lstlisting}

注意这里的 \acode{horse} 与 \acode{giraffe} 共享同一个值。当这种情况出现时,枚举成员变不再是独立的了 -- \acode{horse} 与 \acode{giraffe} 变成可以交换的了。尽管 C++ 允许这样的情况发生,但是最好还是避免。

最佳实践:避免显式赋值枚举成员,除非有令人信服的原因。

\subsubsection*{无范围枚举隐式转换整数值}

\begin{lstlisting}[language=C++]
#include <iostream>

enum Color
{
    black, // 赋值 0
    red, // 赋值 1
    blue, // 赋值 2
    green, // 赋值 3
    white, // 赋值 4
    cyan, // 赋值 5
    yellow, // 赋值 6
    magenta, // 赋值 7
};

int main()
{
    Color shirt{ blue };

    std::cout << "Your shirt is " << shirt; // 这是做什么呢?

    return 0;
}
\end{lstlisting}

因为枚举类型存储整数值,那么打印内容:\acode{Your shirt is 2}。

\subsubsection*{打印枚举名称}

通常的做法:

\begin{lstlisting}[language=C++]
// Using if-else for this is inefficient
void printColor(Color color)
{
    if (color == black) std::cout << "black";
    else if (color == red) std::cout << "red";
    else if (color == blue) std::cout << "blue";
    else std::cout << "???";
}
\end{lstlisting}

然而使用一系列的 if-else 声明效率低下,因为在匹配找到之前还需要若干比较。
一个高效的做法是使用 switch 声明:

\begin{lstlisting}[language=C++]
#include <iostream>
#include <string>

enum Color
{
    black,
    red,
    blue,
};


// 后面会展示 C++17 的更好的版本
std::string getColor(Color color)
{
    switch (color)
    {
    case black: return "black";
    case red:   return "red";
    case blue:  return "blue";
    default:    return "???";
    }
}

int main()
{
    Color shirt { blue };

    std::cout << "Your shirt is " << getColor(shirt) << '\n';

    return 0;
}
\end{lstlisting}

这样打印出 \acode{Your shirt is blue}。

然而这个版本仍然没有效率,因为每次函数被调用时都需要创建并返回 \acode{std::string}(非常的昂贵)。

C++17 中,一个更好的方法是用 \acode{std::string_view} 来代替 \acode{std::string}。
\acode{std::string_view} 允许用户返回字符串字面量并且拷贝起来没有那么的昂贵。

\begin{lstlisting}[language=C++]
#include <iostream>
#include <string_view> // C++17

enum Color
{
    black,
    red,
    blue,
};

constexpr std::string_view getColor(Color color) // C++17
{
    switch (color)
    {
    case black: return "black";
    case red:   return "red";
    case blue:  return "blue";
    default:    return "???";
    }
}

int main()
{
    Color shirt{ blue };

    std::cout << "Your shirt is " << getColor(shirt) << '\n';

    return 0;
}
\end{lstlisting}

\end{document}
