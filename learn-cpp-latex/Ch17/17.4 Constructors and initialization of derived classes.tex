\documentclass[../../LearnCpp.tex]{subfiles}

\begin{document}

\asubsection{4}{Constructors and initialization of derived classes}

继续使用上一章的例子。

\begin{lstlisting}[language=C++]
class Base
{
public:
    int m_id {};

    Base(int id=0)
        : m_id{ id }
    {
    }

    int getId() const { return m_id; }
};

class Derived: public Base
{
public:
    double m_cost {};

    Derived(double cost=0.0)
        : m_cost{ cost }
    {
    }

    double getCost() const { return m_cost; }
};
\end{lstlisting}

不使用派生类时,构造函数只需要考虑自身的成员:

\begin{lstlisting}[language=C++]
int main()
{
    Base base{ 5 }; // 使用 Base(int) 构造函数

    return 0;
}
\end{lstlisting}

实例化时的真实步骤:

\begin{enumerate}
    \item \acode{base} 的内存被分配
    \item 合适的 \acode{Base} 构造函数被调用
    \item 成员初始化列表初始化变量
    \item 构造函数的函数体被执行
    \item 调用者拿回控制权
\end{enumerate}

对于派生类的实例化过程而言,也是非常直接的:

\begin{lstlisting}[language=C++]
int main()
{
    Derived derived{ 1.3 }; // 使用 Derived(double) 构造函数

    return 0;
}
\end{lstlisting}

\begin{enumerate}
    \item 派生类的内存被分配(足够同时用于 \acode{Base} 与 \acode{Derived} 部分)
    \item 合适的 \acode{Derived} 构造函数被调用
    \item \acode{Base} 对象首先使用合适的 \acode{Base} 构造函数被构建。
          如果没有合适的,则默认构造函数被调用
    \item 成员初始化列表初始化变量
    \item 构造函数的函数体被执行
    \item 调用者拿回控制权
\end{enumerate}

\subsubsection*{初始化基类成员}

\begin{lstlisting}[language=C++]
class Derived: public Base
{
public:
    double m_cost {};

    Derived(double cost=0.0, int id=0)
        : Base{ id } // 调用 Base(int) 构造函数并传入 id 值!
        , m_cost{ cost }
    {
    }

    double getCost() const { return m_cost; }
};
\end{lstlisting}

\subsubsection*{另一个例子}

优化之前章节的例子:

\begin{lstlisting}[language=C++]
#include <iostream>
#include <string>
#include <string_view>

class Person
{
private:
    std::string m_name;
    int m_age {};

public:
    Person(const std::string_view name = "", int age = 0)
        : m_name{ name }, m_age{ age }
    {
    }

    const std::string& getName() const { return m_name; }
    int getAge() const { return m_age; }

};
// BaseballPlayer 公开继承 Person
class BaseballPlayer : public Person
{
private:
    double m_battingAverage {};
    int m_homeRuns {};

public:
    BaseballPlayer(const std::string_view name = "", int age = 0,
        double battingAverage = 0.0, int homeRuns = 0)
        : Person{ name, age } // 调用 Person(const std::string_view, int) 来初始化这些字段
        , m_battingAverage{ battingAverage }, m_homeRuns{ homeRuns }
    {
    }

    double getBattingAverage() const { return m_battingAverage; }
    int getHomeRuns() const { return m_homeRuns; }
};

int main()
{
    BaseballPlayer pedro{ "Pedro Cerrano", 32, 0.342, 42 };

    std::cout << pedro.getName() << '\n';
    std::cout << pedro.getAge() << '\n';
    std::cout << pedro.getBattingAverage() << '\n';
    std::cout << pedro.getHomeRuns() << '\n';

    return 0;
}
\end{lstlisting}

打印:

\begin{lstlisting}
Pedro Cerrano
32
0.342
42
\end{lstlisting}

\subsubsection*{继承链}

\begin{lstlisting}[language=C++]
#include <iostream>

class A
{
public:
    A(int a)
    {
        std::cout << "A: " << a << '\n';
    }
};

class B: public A
{
public:
    B(int a, double b)
    : A{ a }
    {
        std::cout << "B: " << b << '\n';
    }
};

class C: public B
{
public:
    C(int a, double b, char c)
    : B{ a, b }
    {
        std::cout << "C: " << c << '\n';
    }
};

int main()
{
    C c{ 5, 4.3, 'R' };

    return 0;
}
\end{lstlisting}

打印:

\begin{lstlisting}
A: 5
B: 4.3
C: R
\end{lstlisting}

值得注意的是构造函数仅可以调用它们的父类构造函数。
也就是说 \acode{C} 的构造函数不能直接调用或者传递参数给 \acode{A} 的构造函数。

\subsubsection*{析构函数}

当一个派生类被销毁,每个析构函数也会与构造函数\textit{相反}的顺序被调用。
上述例子中 \acode{c} 被销毁,\acode{C} 的析构函数先被调用,
然后是 \acode{B} 的析构函数,最后是 \acode{A} 的析构函数。

\end{document}
