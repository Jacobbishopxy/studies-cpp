\documentclass[../../LearnCpp.tex]{subfiles}

\begin{document}

\asubsection{6}{Adding new functionality to a derived class}

之前的章节里讲过使用派生类最大的好处就是其拥有复用代码的能力。
可以从基类中继承功能,接着添加新的功能,修改现有功能,或者是隐藏不需要的功能。
从这一章节开始,开始演示上述能力。

一个简单的基类:

\begin{lstlisting}[language=C++]
#include <iostream>

class Base
{
protected:
    int m_value {};

public:
    Base(int value)
        : m_value { value }
    {
    }

    void identify() const { std::cout << "I am a Base\n"; }
};
\end{lstlisting}

\subsubsection*{添加新功能至派生类}

\begin{lstlisting}[language=C++]
class Derived: public Base
{
public:
    Derived(int value)
        : Base { value }
    {
    }

    int getValue() const { return m_value; }
};
\end{lstlisting}

\end{document}
