\documentclass[../../LearnCpp.tex]{subfiles}

\begin{document}

\asubsection{5}{Inheritance and access specifiers}

\subsubsection*{protected 访问限定符}

C++ 有第三种暂未被提及的访问限定符,因为其仅在继承的上下文中有用。
\textbf{protected} 访问限定符运允许类成员属于,友元,以及派生类访问成员,
而 protected 成员不能被类的外部访问。

\begin{lstlisting}[language=C++]
class Base
{
public:
    int m_public {};    // 可以被任何人访问
protected:
    int m_protected {}; // 可以被 Base 成员,友元,以及派生类访问
private:
    int m_private {};   // 仅可被 Base 成员以及友元(派生类访问不了)访问
};

class Derived: public Base
{
public:
    Derived()
    {
        m_public = 1;     // 允许
        m_protected = 2;  // 允许
        m_private = 3;    // 不允许
    }
};

int main()
{
    Base base;
    base.m_public = 1;    // 允许
    base.m_protected = 2; // 不允许
    base.m_private = 3;   // 不允许

    return 0;
}
\end{lstlisting}

\subsubsection*{何时使用 protected 访问限定符}

基类带有 protected 属性的成员可以在派生类中直接访问。
这就意味着之后任何关于该成员的修改(类型,值的含义,等等)需要同时去修改基类以及派生类。

隐藏使用 protected 访问限定符在开发者(或组)编写自定义类与派生类时很有用。
这样的话修改基类的实现,并更新作为结果的派生类时,可以自主的进行更新(不会消耗大量时间,毕竟派生类的数量是有限的)。

将成员设为私有则意味着公共与派生类都不能直接访问基类。
这么做隔离了公共与派生类进行修改,并确保了变体能够正确的被维护。
然而这也意味着可能需要大量 public(或 protected)结构来支持所有用于公共或派生类操作的函数,
它们都有各自的构建,测试以及维护成本。

通常来说,尽可能的让成员私有,仅当计划需要有派生类的构建与维护接口,
而私有成员的成本又太高时,才使用 protected。

最佳实践:更加推荐私有成员而不是 protected 成员。

\subsubsection*{不同类型的继承,以及它们在访问时的影响}

\begin{lstlisting}[language=C++]
// Inherit from Base publicly
class Pub: public Base
{
};

// Inherit from Base protectedly
class Pro: protected Base
{
};

// Inherit from Base privately
class Pri: private Base
{
};

class Def: Base // Defaults to private inheritance
{
};
\end{lstlisting}

如果不选择继承类型,C++ 默认使用 private 继承(正如成员也是默认为 private 访问)。

这就带来了 9 种组合:3 种成员访问限定符(public,private,protected),
以及 3 种继承类型(public,private,protected)。

\subsubsection*{Public 继承}

Public 继承是继承中最为广泛使用的。实际上,通常很难看到其他类型的继承。

\begin{center}
    \begin{tiny}
        \begin{tabularx}{ 1\textwidth}{
                | >{\raggedright\arraybackslash}X
                | >{\raggedright\arraybackslash}X |
            }
            \hline
            Access specifier in base class & Access specifier when inherited publicly \\
            \hline
            Public                         & Public                                   \\
            Protected                      & Protected                                \\
            Private                        & Inaccessible                             \\
            \hline
        \end{tabularx}
    \end{tiny}
\end{center}

\begin{lstlisting}[language=C++]
class Base
{
public:
    int m_public {};
protected:
    int m_protected {};
private:
    int m_private {};
};

class Pub: public Base // note: public inheritance
{
    // Public inheritance means:
    // Public inherited members stay public (so m_public is treated as public)
    // Protected inherited members stay protected (so m_protected is treated as protected)
    // Private inherited members stay inaccessible (so m_private is inaccessible)
public:
    Pub()
    {
        m_public = 1; // okay: m_public was inherited as public
        m_protected = 2; // okay: m_protected was inherited as protected
        m_private = 3; // not okay: m_private is inaccessible from derived class
    }
};

int main()
{
    // Outside access uses the access specifiers of the class being accessed.
    Base base;
    base.m_public = 1; // okay: m_public is public in Base
    base.m_protected = 2; // not okay: m_protected is protected in Base
    base.m_private = 3; // not okay: m_private is private in Base

    Pub pub;
    pub.m_public = 1; // okay: m_public is public in Pub
    pub.m_protected = 2; // not okay: m_protected is protected in Pub
    pub.m_private = 3; // not okay: m_private is inaccessible in Pub

    return 0;
}
\end{lstlisting}

最佳实践:使用 public 继承除非有非常特殊的原因。

\subsubsection*{Protected 继承}

Protected 继承是最少见的一种继承方式。
可以说是几乎不怎么使用它,除非是特别的案例。
通过 protected 继承,public 与 protected 成员都变为 protected,同时 private 保持不可访问。

\begin{center}
    \begin{tiny}
        \begin{tabularx}{ 1\textwidth}{
                | >{\raggedright\arraybackslash}X
                | >{\raggedright\arraybackslash}X |
            }
            \hline
            Access specifier in base class & Access specifier when inherited protectedly \\
            \hline
            Public                         & Protected                                   \\
            Protected                      & Protected                                   \\
            Private                        & Inaccessible                                \\
            \hline
        \end{tabularx}
    \end{tiny}
\end{center}

\subsubsection*{Private 继承}

通过 private 继承,所有基类的成员都被继承为私有。

注意着并不影响派生类访问基类成员!这仅仅影响的是派生来外部想要访问这些基类成员的代码。

\begin{lstlisting}[language=C++]
class Base
{
public:
    int m_public {};
protected:
    int m_protected {};
private:
    int m_private {};
};

class Pri: private Base // note: private inheritance
{
    // Private inheritance means:
    // Public inherited members become private (so m_public is treated as private)
    // Protected inherited members become private (so m_protected is treated as private)
    // Private inherited members stay inaccessible (so m_private is inaccessible)
public:
    Pri()
    {
        m_public = 1; // okay: m_public is now private in Pri
        m_protected = 2; // okay: m_protected is now private in Pri
        m_private = 3; // not okay: derived classes can't access private members in the base class
    }
};

int main()
{
    // Outside access uses the access specifiers of the class being accessed.
    // In this case, the access specifiers of base.
    Base base;
    base.m_public = 1; // okay: m_public is public in Base
    base.m_protected = 2; // not okay: m_protected is protected in Base
    base.m_private = 3; // not okay: m_private is private in Base

    Pri pri;
    pri.m_public = 1; // not okay: m_public is now private in Pri
    pri.m_protected = 2; // not okay: m_protected is now private in Pri
    pri.m_private = 3; // not okay: m_private is inaccessible in Pri

    return 0;
}
\end{lstlisting}

\begin{center}
    \begin{tiny}
        \begin{tabularx}{ 1\textwidth}{
                | >{\raggedright\arraybackslash}X
                | >{\raggedright\arraybackslash}X |
            }
            \hline
            Access specifier in base class & Access specifier when inherited privately \\
            \hline
            Public                         & Private                                   \\
            Protected                      & Private                                   \\
            Private                        & Inaccessible                              \\
            \hline
        \end{tabularx}
    \end{tiny}
\end{center}

\subsubsection*{总结}

\begin{center}
    \begin{tiny}
        \begin{tabularx}{ 1\textwidth}{
                | >{\raggedright\arraybackslash}X
                | >{\raggedright\arraybackslash}X
                | >{\raggedright\arraybackslash}X
                | >{\raggedright\arraybackslash}X |
            }
            \hline
            Access specifier in base class & Access specifier when inherited publicly & Access specifier when inherited privately & Access specifier when inherited protectedly \\
            \hline
            Public                         & Public                                   & Private                                   & Protected                                   \\
            Protected                      & Protected                                & Private                                   & Protected                                   \\
            Private                        & Inaccessible                             & Inaccessible                              & Inaccessible                                \\
            \hline
        \end{tabularx}
    \end{tiny}
\end{center}

\end{document}
