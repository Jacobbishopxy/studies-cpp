\documentclass[../../LearnCpp.tex]{subfiles}

\begin{document}

\asubsection{5}{Early binding and late binding}

这一章节与下一章节将会学习到虚函数是如何实现的。
虽然这个信息不是使用虚函数严格需求的,它很有趣。可以认为这两个章节为可选阅读。

当 C++ 程序被执行时,它会顺序执行,开始于 main() 函数。当遇到函数调用时,
执行点跳转到被调用函数的起始处。那么 CPU 是怎么知道的呢?

当一个程序被编译,编译器转换每个 C++ 程序中的声明成为一行或多行的机器语言。
每一行的机器语言都会被给予唯一的序列地址。
函数也是同样的 -- 当遇到一个函数,它被转换为机器语言并给与下一个可用的地址。
因此,每个函数都会带有一个唯一的地址。

\textbf{绑定} Binding 指的是用于转换标识符(例如变量和函数名)成为地址的过程。
尽管绑定是用于变量和函数的,这一节中着重谈论函数绑定。

\subsubsection*{Early binding}

编译期遇到的大多数函数调用都是直接函数调用。直接的函数调用是一个声明:

\begin{lstlisting}[language=C++]
#include <iostream>

void printValue(int value)
{
    std::cout << value;
}

int main()
{
    printValue(5); // 直接的函数调用
    return 0;
}
\end{lstlisting}

直接函数调用可以使用 early binding 的过程进行解析。\textbf{Early binding}(也被称为 static binding)意为编译器(或 linker)可以通过机器地址直接关联标识符名称(例如函数或变量名)。鉴于所有函数都有唯一的地址。因此当编译器(或 linker)遇到了函数调用,替换函数调用为机器语言指令,并告诉 CPU 跳转的函数的地址。

\begin{lstlisting}[language=C++]
#include <iostream>

int add(int x, int y)
{
    return x + y;
}

int subtract(int x, int y)
{
    return x - y;
}

int multiply(int x, int y)
{
    return x * y;
}

int main()
{
    int x{};
    std::cout << "Enter a number: ";
    std::cin >> x;

    int y{};
    std::cout << "Enter another number: ";
    std::cin >> y;

    int op{};
    do
    {
        std::cout << "Enter an operation (0=add, 1=subtract, 2=multiply): ";
        std::cin >> op;
    } while (op < 0 || op > 2);

    int result {};
    switch (op)
    {
        // 直接使用 early binging 来调用目标函数
        case 0: result = add(x, y); break;
        case 1: result = subtract(x, y); break;
        case 2: result = multiply(x, y); break;
    }

    std::cout << "The answer is: " << result << '\n';

    return 0;
}
\end{lstlisting}

\subsubsection*{Late binding}

在有些程序中,函数调用只能在运行时才可以解析。
在 C++ 中,有时被称为 \textbf{late binding} (或者使用虚函数方案 \textbf{dynamic binding})。

注:通常的编程术语中,“late binding”的概念意为函数被调用时在运行时查找函数名称,而 C++ 并不支持这么做。
C++ 中的”late binding“通常用于函数被调用时,不被编译期或 linker 所知。
相反,函数调用(在运行时)在此时被决定。

C++ 中,使用 late binding 的一种方式是通过函数指针。

通过函数指针调用函数同样也被认作是间接函数调用。下面使用函数指针改造上面的代码:

\begin{lstlisting}[language=C++]
#include <iostream>

int add(int x, int y)
{
    return x + y;
}

int subtract(int x, int y)
{
    return x - y;
}

int multiply(int x, int y)
{
    return x * y;
}

int main()
{
    int x{};
    std::cout << "Enter a number: ";
    std::cin >> x;

    int y{};
    std::cout << "Enter another number: ";
    std::cin >> y;

    int op{};
    do
    {
        std::cout << "Enter an operation (0=add, 1=subtract, 2=multiply): ";
        std::cin >> op;
    } while (op < 0 || op > 2);

    // 创建一个名为 pFcn 的函数指针(语法很丑陋)
    int (*pFcn)(int, int) { nullptr };

    // 设置 pFcn 指向用户所选的函数
    switch (op)
    {
        case 0: pFcn = add; break;
        case 1: pFcn = subtract; break;
        case 2: pFcn = multiply; break;
    }

    // 调用 pFcn 指向的函数,x 与 y 作为入参,这里使用了 late binding
    std::cout << "The answer is: " << pFcn(x, y) << '\n';

    return 0;
}
\end{lstlisting}

Late binding 的性能稍微差一些这是因为它涉及到了额外一层的间接行为。
使用 early binding,CPU 可以直接跳转到函数地址;
使用 late binding,程序需要读取指针中的地址,接着再跳转去改地址。
这里就牵扯到了额外的一步,因此稍微变慢了。
然而 late binding 的优势是拥有了更多的灵活性,因为决定函数调用可以在运行时做。

\end{document}
