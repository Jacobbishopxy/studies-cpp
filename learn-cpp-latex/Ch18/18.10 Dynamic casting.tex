\documentclass[../../LearnCpp.tex]{subfiles}

\begin{document}

\asubsection{10}{Dynamic casting}

8.5 章节讲到了显式类型转换(casting)以及 static\_cast,
本章节继续测试另一种类型的 cast:dynamic\_cast。

\subsubsection*{dynamic\_cast 的需求}

在处理多态时,经常会遇到指针指向的是基类,但是又希望能访问派生类中的信息。

考虑以下代码(稍微有些做作):

\begin{lstlisting}[language=C++]
#include <iostream>
#include <string>

class Base
{
protected:
  int m_value{};

public:
  Base(int value)
    : m_value{value}
  {
  }

  virtual ~Base() = default;
};

class Derived : public Base
{
protected:
  std::string m_name{};

public:
  Derived(int value, const std::string& name)
    : Base{value}, m_name{name}
  {
  }

  const std::string& getName() const { return m_name; }
};

Base* getObject(bool returnDerived)
{
  if (returnDerived)
    return new Derived{1, "Apple"};
  else
    return new Base{2};
}

int main()
{
  Base* b{ getObject(true) };

  // 在仅有基类指针的情况下,如何打印派生类对象的名字?

  delete b;

  return 0;
}
\end{lstlisting}

这个程序中,函数 \acode{getObject()} 总是返回一个 \acode{Base} 指针,
但是该指针很有可能本身是指向 \acode{Base} 或 \acode{Derived} 的对象的。
这个例子中的指针指向的是 \acode{Derived} 的对象,
那么该如何调用 \acode{Derived::getName()} 呢?

一种方法是添加名为 \acode{getName()} 的虚函数至 \acode{Base} 中(这样可以通过 \acode{Base} 的指针/引用来进行调用,
同时使其动态解析成 \acode{Derived::getName()})。
但是如果通过 \acode{Base} 指针/引用而实际上其指向的是 \acode{Base} 对象时,
调用这个函数返回的是什么呢?这个时候其实没有任何值具有意义。
此外,有些应该只在派生类中考虑的东西将会污染 \acode{Base} 类。

众所周知 C++ 将隐式转换 \acode{Derived} 指针称为 \acode{Base} 指针。
这个过程有时被称为\textbf{向上转型} upcasting。
然而是否有一种方法使得 \acode{Base} 指针转换回 \acode{Derived} 指针呢?
这样就可以使用该指针直接调用 \acode{Derived::getName()},且不需要担心虚函数的解析了。

\subsubsection*{dynamic\_cast}

C++ 提供了一种名为 \textbf{dynamic\_cast} 的转型操作符以完成上述目的。
尽管动态转型只有很少的功能,现在位置最常用的地方是转化基类指针称为派生类指针。
这个过程被称为\textbf{向下转型} downcasting。

dynamic\_cast 的使用与 static\_cast 类似。

\begin{lstlisting}[language=C++]
int main()
{
  Base* b{ getObject(true) };

  Derived* d{ dynamic_cast<Derived*>(b) }; // 使用动态转型转换 Base 指针成为 Derived 指针

  std::cout << "The name of the Derived is: " << d->getName() << '\n';

  delete b;

  return 0;
}
\end{lstlisting}

打印:

\begin{lstlisting}
The name of the Derived is: Apple
\end{lstlisting}

\subsubsection*{dynamic\_cast 失败}

上述例子可以工作是因为 \acode{b} 实际上指向的是 \acode{Derived} 对象,
因此转换 \acode{b} 成为 \acode{Derived} 指针可以成功。

然而我们这里下定了一个危险的假设:\acode{b} 指向的是 \acode{Derived} 对象,
那么如果 \acode{b} 指向的不是 \acode{Derived} 对象呢?
通过改变 \acode{getObject()} 的入参从 true 变为 false 可以很容易的测试出来。
这种情况下,\acode{getObject()} 将返回一个指向 \acode{Base} 对象的 \acode{Base} 指针。
这时尝试 dynamic\_cast 转为 \acode{Derived} 将会失败,因为转换是不可能的。

如果 dynamic\_cast 失败了,装换的结果会是一个空指针。

因为没有对结果进行空指针检查,访问 \acode{d->getName()} 时将会对空指针进行解引用,
从而导致未定义行为(可能是程序崩溃)。

为了让程序安全,可以确保 dynamic\_cast 的结果是真的成功:

\begin{lstlisting}[language=C++]
int main()
{
  Base* b{ getObject(true) };

  Derived* d{ dynamic_cast<Derived*>(b) }; // 使用动态转型转换 Base 指针成为 Derived 指针

  if (d) // 确保 d 非空
    std::cout << "The name of the Derived is: " << d->getName() << '\n';

  delete b;

  return 0;
}
\end{lstlisting}

规则:总是确保动态转型之后进行空指针检查。

注意因为 dynamic\_cast 在运行时做了一些连贯性检测(用来确保可以进行转换),
使用 dynamic\_cast 会带来性能下降的惩罚。

同时注意有以下几种情况使用 dynamic\_cast 进行向下转型不能工作:

\begin{enumerate}
  \item 带有 protected 或 private 的继承
  \item 没有声明或者继承任何虚函数的类(因为没有虚表)
  \item 一些特定的情况涉及了虚基类(详见\href{https://learn.microsoft.com/en-us/cpp/cpp/dynamic-cast-operator?redirectedfrom=MSDN&view=msvc-170}{这里})
\end{enumerate}

\subsubsection*{通过 static\_cast 进行向下转型}

事实上向下转型同样也可以通过 static\_cast 完成。
这里主要的区别是 static\_cast 没有运行时检查来确保转型是否有意义。
这就让 static\_cast 更快了,但是也更危险了。
如果转换一个 \acode{Base*} 成为 \acode{Derived*},
可以“成功”即使 \acode{Base} 指针并没有指向 \acode{Derived} 对象。
当尝试访问 \acode{Derived} 指针时将会导致未定义行为(因为实际上指向的是 \acode{Base} 对象)。

如果对于向下转型的指针完全有把握成功,那么使用 static\_cast 是可以的。
其中一种保证指向的对象类型的方式,是使用虚函数。这里是一个方法(并不是那么好):

\begin{lstlisting}[language=C++]
#include <iostream>
#include <string>

// Class 标识符
enum class ClassID
{
  base,
  derived
  // 其它可以之后在这里添加
};

class Base
{
protected:
int m_value{};

public:
  Base(int value)
    : m_value{value}
  {
  }

  virtual ~Base() = default;
  virtual ClassID getClassID() const { return ClassID::base; }
};

class Derived : public Base
{
protected:
  std::string m_name{};

public:
  Derived(int value, const std::string& name)
    : Base{value}, m_name{name}
  {
  }

  const std::string& getName() const { return m_name; }
  virtual ClassID getClassID() const { return ClassID::derived; }
};

Base* getObject(bool bReturnDerived)
{
  if (bReturnDerived)
    return new Derived{1, "Apple"};
  else
    return new Base{2};
}

int main()
{
  Base* b{ getObject(true) };

  if (b->getClassID() == ClassID::derived)
  {
    // 已经证明了 b 是指向 Derived 对象的,因此这里总是会成功
    Derived* d{ static_cast<Derived*>(b) };
    std::cout << "The name of the Derived is: " << d->getName() << '\n';
  }

  delete b;

  return 0;
}
\end{lstlisting}

但是如果通过这些麻烦的步骤来实现(花费调用虚函数的成本,以及处理整个过程),
或许直接使用 dynamic\_cast 更好。

\subsubsection*{dynamic\_cast 与引用}

尽管上述所有的例子展示的都是指针的动态转型(也是最常见的),dynamic\_cast 同样也可以作用于引用。

\begin{lstlisting}[language=C++]
#include <iostream>
#include <string>

class Base
{
protected:
  int m_value;

public:
  Base(int value)
    : m_value{value}
  {
  }

  virtual ~Base() = default;
};

class Derived : public Base
{
protected:
  std::string m_name;

public:
  Derived(int value, const std::string& name)
    : Base{value}, m_name{name}
  {
  }

  const std::string& getName() const { return m_name; }
};

int main()
{
  Derived apple{1, "Apple"};
  Base& b{ apple };
  Derived& d{ dynamic_cast<Derived&>(b) }; // 动态转型,使用引用而不是指针

  std::cout << "The name of the Derived is: " << d.getName() << '\n'; // 可以通过 d 来访问 Derived::getName

  return 0;
}
\end{lstlisting}

由于 C++ 并没有一个“空引用”,失败的时候 dynamic\_cast 并不能返回一个空引用。
反而如果 dynamic\_cast 对于引用失败时,\acode{std::bad_cast} 的异常会被抛出。

\subsubsection*{dynamic\_cast vs static\_cast}

新手程序员有时会迷惑何时使用 dynamic\_cast 或是 static\_cast。
答案非常简单:使用 static\_cast 除非使用向下转型,这种情况下 dynamic\_cast 会是更好的选择。
然而同样也需要考虑避免使用转型,而是去使用虚函数。

\subsubsection*{向下转型 vs 虚函数}

有一些开发者认为 dynamic\_cast 是邪恶的,并且是坏的类设计。相反这些程序员会建议使用虚函数。

通常来说,使用虚函数*应该*是比向下转型更为推荐的。然而有时候向下转型却是更好的选择:

\begin{itemize}
  \item 当不可以修改基类添加一个虚函数时(例如因为基类是标准库的一部分)
  \item 当需要访问一些只有派生类的东西(例如访问仅存在于派生类中的函数)
  \item 当添加虚函数给基类没有任何意义时(例如没有合适的值作为基类的返回)。
        这里使用存虚函数可能是一个选项,只要不需要实例化基类。
\end{itemize}

\subsubsection*{dynamic\_cast 与 RTTI 的警告}

运行时类型信息(Run-time type information (RTTI))是 C++ 的一个特征,
用来在运行时暴露对象的数据类型。这个能力是由 dynamic\_cast 完成的。
因为 RTTI 拥有很大的性能成本,一些编译器允许用户关闭 RTTI 来进行优化。
无需多言,如果这么做,dynamic\_cast 不能正确工作。

\end{document}
