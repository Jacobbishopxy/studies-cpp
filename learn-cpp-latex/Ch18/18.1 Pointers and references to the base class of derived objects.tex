\documentclass[../../LearnCpp.tex]{subfiles}

\begin{document}

\asubsection{1}{Pointers and references to the base class of derived objects}

在之前的章节中学习了如何使用继承从现有类派生出新的类。
这一章将会集中学习最重要也是最强大的继承 -- 虚函数。

\begin{lstlisting}[language=C++]
#include <string_view>

class Base
{
protected:
    int m_value {};

public:
    Base(int value)
        : m_value{ value }
    {
    }

    std::string_view getName() const { return "Base"; }
    int getValue() const { return m_value; }
};

class Derived: public Base
{
public:
    Derived(int value)
        : Base{ value }
    {
    }

    std::string_view getName() const { return "Derived"; }
    int getValueDoubled() const { return m_value * 2; }
};
\end{lstlisting}

当创建一个派生对象,它包含了 \acode{Base} 部分(首先被构造出来),
以及 \acode{Derived} 部分(其次被构造)。注意继承表示两个对象之间是 is-a 关系。
因为 \acode{Derived} is-a \acode{Base},那么 \acode{Derived} 包含了 \acode{Base} 部分是合理的。

\subsubsection*{指针,引用,以及派生类}

对 \acode{Derived} 对象设置指针和引用应该是符合直觉的。

\begin{lstlisting}[language=C++]
#include <iostream>

int main()
{
    Derived derived{ 5 };
    std::cout << "derived is a " << derived.getName() << " and has value " << derived.getValue() << '\n';

    Derived& rDerived{ derived };
    std::cout << "rDerived is a " << rDerived.getName() << " and has value " << rDerived.getValue() << '\n';

    Derived* pDerived{ &derived };
    std::cout << "pDerived is a " << pDerived->getName() << " and has value " << pDerived->getValue() << '\n';

    return 0;
}
\end{lstlisting}

打印:

\begin{lstlisting}
derived is a Derived and has value 5
rDerived is a Derived and has value 5
pDerived is a Derived and has value 5
\end{lstlisting}

然而因为 \acode{Derived} 拥有 \acode{Base} 部分,
一个更加有趣的问题就是 C++ 是否让用户设置 \acode{Base} 指针或者引用给 \acode{Derived} 对象。
实际上这是可以的!

\begin{lstlisting}[language=C++]
#include <iostream>

int main()
{
    Derived derived{ 5 };

    // 它们都是合法的!
    Base& rBase{ derived };
    Base* pBase{ &derived };

    std::cout << "derived is a " << derived.getName() << " and has value " << derived.getValue() << '\n';
    std::cout << "rBase is a " << rBase.getName() << " and has value " << rBase.getValue() << '\n';
    std::cout << "pBase is a " << pBase->getName() << " and has value " << pBase->getValue() << '\n';

    return 0;
}
\end{lstlisting}

打印:

\begin{lstlisting}
derived is a Derived and has value 5
rBase is a Base and has value 5
pBase is a Base and has value 5
\end{lstlisting}

这个结果可能并不像一开始预期那样!

这是因为 \acode{rBase} 与 \acode{pBase} 是 \acode{Base} 的引用与指针,
它们只能看到 \acode{Base} 的成员(或者是任何 \acode{Base} 所继承的类的成员)。
因此即使 \acode{Derived::getName()} 覆盖 shadows 了 \acode{Derived} 对象的 \acode{Base::getName()},
\acode{Base} 指针/引用并不能看到 \acode{Derived::getName()}。
结果而言,它们调用的是 \acode{Base::getName()}。

注意这也同样意味着不能使用 \acode{rBase} 或 \acode{pBase} 来调用 \acode{Derived::getValueDoubled()}。
它们不能看到任何 \acode{Derived} 的内容。

下面是一个将在下一章构建的更为复杂的例子:

\begin{lstlisting}[language=C++]
#include <iostream>
#include <string_view>
#include <string>

class Animal
{
protected:
    std::string m_name;

    // 使该构造函数为 protected,因为不希望任何人可以直接创建 Animal 对象,
    // 但是仍然希望派生类可以使用它。
    Animal(std::string_view name)
        : m_name{ name }
    {
    }

    // 用于阻止切片(稍后讲解)
    Animal(const Animal&) = default;
    Animal& operator=(const Animal&) = default;

public:
    std::string_view getName() const { return m_name; }
    std::string_view speak() const { return "???"; }
};

class Cat: public Animal
{
public:
    Cat(std::string_view name)
        : Animal{ name }
    {
    }

    std::string_view speak() const { return "Meow"; }
};

class Dog: public Animal
{
public:
    Dog(std::string_view name)
        : Animal{ name }
    {
    }

    std::string_view speak() const { return "Woof"; }
};

int main()
{
    const Cat cat{ "Fred" };
    std::cout << "cat is named " << cat.getName() << ", and it says " << cat.speak() << '\n';

    const Dog dog{ "Garbo" };
    std::cout << "dog is named " << dog.getName() << ", and it says " << dog.speak() << '\n';

    const Animal* pAnimal{ &cat };
    std::cout << "pAnimal is named " << pAnimal->getName() << ", and it says " << pAnimal->speak() << '\n';

    pAnimal = &dog;
    std::cout << "pAnimal is named " << pAnimal->getName() << ", and it says " << pAnimal->speak() << '\n';

    return 0;
}
\end{lstlisting}

打印:

\begin{lstlisting}
cat is named Fred, and it says Meow
dog is named Garbo, and it says Woof
pAnimal is named Fred, and it says ???
pAnimal is named Garbo, and it says ???
\end{lstlisting}

这里也有同样的问题。
因为 \acode{pAnimal} 是 \acode{Animal} 指针,
它只可以看见作为 \acode{Animal} 类的部分。
因此,\acode{pAnimal->speak()} 调用了 \acode{Animal::speak()} 而不是 \acode{Dog::speak()} 或 \acode{Cat::speak()} 函数。

\subsubsection*{对基类使用指针与引用}

这个时候有人可能会说“上述的例子看起来有点蠢。
为什么不直接设置派生对象的指针或者引用?”这有几个很好的原因。

首先,假设要编写一个函数打印动物名以及声音。不使用基类指针的话则需要使用重载函数,例如:

\begin{lstlisting}[language=C++]
void report(const Cat& cat)
{
    std::cout << cat.getName() << " says " << cat.speak() << '\n';
}

void report(const Dog& dog)
{
    std::cout << dog.getName() << " says " << dog.speak() << '\n';
}
\end{lstlisting}

这么看起来并不难,但是如果有 30 种不同的动物类型呢?
需要编写 30 个几乎相同的函数!另外每次新增一个动物类型都需要为其编写一个新函数。

然而因为 Cat 和 Dog 都是 Animal 的派生类,Cat 和 Dog 都拥有 Animal 部分。
因此以下的做法是符合常理的:

\begin{lstlisting}[language=C++]
void report(const Animal& rAnimal)
{
    std::cout << rAnimal.getName() << " says " << rAnimal.speak() << '\n';
}
\end{lstlisting}

当然了,这样的问题还是在于引用还是调用基类的函数而不是派生类的函数。

其次,假设存在 3 个 cat 与 3 个 dog,并希望使它们在同一个数组中便于访问。
因为数组仅可以保存一种类型的对象,在不使用基类的指针或者引用的情况下,
需要为每个派生类型都去创建各自的数组。例如:

\begin{lstlisting}[language=C++]
#include <array>
#include <iostream>

// Cat and Dog from the example above

int main()
{
    const auto& cats{ std::to_array<Cat>({{ "Fred" }, { "Misty" }, { "Zeke" }}) };
    const auto& dogs{ std::to_array<Dog>({{ "Garbo" }, { "Pooky" }, { "Truffle" }}) };

    // Before C++20
    // const std::array<Cat, 3> cats{{ { "Fred" }, { "Misty" }, { "Zeke" } }};
    // const std::array<Dog, 3> dogs{{ { "Garbo" }, { "Pooky" }, { "Truffle" } }};

    for (const auto& cat : cats)
    {
        std::cout << cat.getName() << " says " << cat.speak() << '\n';
    }

    for (const auto& dog : dogs)
    {
        std::cout << dog.getName() << " says " << dog.speak() << '\n';
    }

    return 0;
}
\end{lstlisting}

那么如果是 30 种不同类型的动物呢?需要 30 个数组!

然而因为 Cat 和 Dog 都是 Animal 的派生类,以下的做法也是符合常理的:

\begin{lstlisting}[language=C++]
#include <array>
#include <iostream>

// Cat and Dog from the example above

int main()
{
    const Cat fred{ "Fred" };
    const Cat misty{ "Misty" };
    const Cat zeke{ "Zeke" };

    const Dog garbo{ "Garbo" };
    const Dog pooky{ "Pooky" };
    const Dog truffle{ "Truffle" };

    // Set up an array of pointers to animals, and set those pointers to our Cat and Dog objects
    // Note: to_array requires C++20 support (and at the time of writing, Visual Studio 2022 still doesn't support it correctly)
    const auto animals{ std::to_array<const Animal*>({&fred, &garbo, &misty, &pooky, &truffle, &zeke }) };

    // Before C++20, with the array size being explicitly specified
    // const std::array<const Animal*, 6> animals{ &fred, &garbo, &misty, &pooky, &truffle, &zeke };

    for (const auto animal : animals)
    {
        std::cout << animal->getName() << " says " << animal->speak() << '\n';
    }

    return 0;
}
\end{lstlisting}

然而这里遇到的问题还是与上述案例一样。现在可以猜一猜虚函数是做什么用的呢?

\end{document}
