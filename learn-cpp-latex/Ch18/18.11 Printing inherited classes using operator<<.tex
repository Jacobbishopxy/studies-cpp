\documentclass[../../LearnCpp.tex]{subfiles}

\begin{document}

\asubsection{11}{Printing inherited classes using operator<<}

考虑下面使用虚函数的程序:

\begin{lstlisting}[language=C++]
#include <iostream>

class Base
{
public:
  virtual void print() const { std::cout << "Base";  }
};

class Derived : public Base
{
public:
  void print() const override { std::cout << "Derived"; }
};

int main()
{
  Derived d{};
  Base& b{ d };
  b.print(); // 将调用 Derived::print()

  return 0;
}
\end{lstlisting}

虽然像是上面这样调用成员函数是没问题的,但是这种风格的函数并不能与 \acode{std::cout} 混用:

\begin{lstlisting}[language=C++]
#include <iostream>

int main()
{
  Derived d{};
  Base& b{ d };

  std::cout << "b is a ";
  b.print(); // 凌乱,调用该函数打破了打印声明
  std::cout << '\n';

  return 0;
}
\end{lstlisting}

本章将会详解如何使用继承来为类进行重写操作符 \acode{<<},使得可以使用 \acode{<<} 如:

\begin{lstlisting}[language=C++]
std::cout << "b is a " << b << '\n'; // 更好了
\end{lstlisting}

\subsubsection*{操作符<<的挑战}

用典型的方法来重载操作符<<:

\begin{lstlisting}[language=C++]
#include <iostream>

class Base
{
public:
  virtual void print() const { std::cout << "Base"; }

  friend std::ostream& operator<<(std::ostream& out, const Base& b)
  {
    out << "Base";
    return out;
  }
};

class Derived : public Base
{
public:
  void print() const override { std::cout << "Derived"; }

  friend std::ostream& operator<<(std::ostream& out, const Derived& d)
  {
    out << "Derived";
    return out;
  }
};

int main()
{
  Base b{};
  std::cout << b << '\n';

  Derived d{};
  std::cout << d << '\n';

  return 0;
}
\end{lstlisting}

这里不需要虚函数解析,程序正如预期,打印:

\begin{lstlisting}
Base
Derived
\end{lstlisting}

现在来看看以下的 \acode{main()} 函数:

\begin{lstlisting}[language=C++]
int main()
{
    Derived d{};
    Base& bref{ d };
    std::cout << bref << '\n';

    return 0;
}
\end{lstlisting}

打印:

\begin{lstlisting}
Base
\end{lstlisting}

这并不是预期的结果,这是因为操作符 \acode{<<} 处理的 \acode{Base} 对象并不是虚化的,
因此 \acode{std::cout << bref} 调用的是 \acode{Base} 对象的 \acode{<<} 而不是 \acode{Derived} 对象的。

\subsubsection*{能让操作符 \acode{<<} 虚化吗?}

答案是不。这里有一系列的原因。

首先,仅有成员函数可以被虚化 --
这很合理,因为只有类可以继承其他的类,并且没有办法在类外部去重写一个函数(可以重载非成员函数,而不是重写它们)。
由于通常实现的操作符<<是作为友元,友元并不被视为成员函数,那么友元的操作符<<并不没有被虚化的资格。

其次,就算是可以虚化操作符<<,
问题在于 \acode{Base::operator<<} 的函数入参以及 \acode{Derived::operator<<} 是不同的(\acode{Base} 的入参是 \acode{Base},
\acode{Derived} 的则是 \acode{Derived})。
因此 \acode{Derived} 并不能视为重写了 \acode{Base},因此也不没有虚化函数解析的资格。

\subsubsection*{解决方案}

答案其实惊人的简单。

首先,与通常一样设置操作符<<为基类的友元。
但是不让操作符<<打印其自身,而是委派这个责任给与一个普通的成员函数,即\textit{可以}被虚化!

\begin{lstlisting}[language=C++]
#include <iostream>

class Base
{
public:
  // 此处为重载操作符<<
  friend std::ostream& operator<<(std::ostream& out, const Base& b)
  {
    // 委派打印功能给函数 print()
    return b.print(out);
  }

  // 依赖成员函数 print() 来完成真正的打印工作。
  // 因为 print() 是一个普通的成员函数,它可以被虚化
  virtual std::ostream& print(std::ostream& out) const
  {
    out << "Base";
    return out;
  }
};

class Derived : public Base
{
public:
  // 这里是 print() 函数的重写,用来处理 Derived 的情况
  std::ostream& print(std::ostream& out) const override
  {
    out << "Derived";
    return out;
  }
};

int main()
{
  Base b{};
  std::cout << b << '\n';

  Derived d{};
  std::cout << d << '\n'; // 注意这可以工作,即使没有操作符<< 显示的处理 Derived 对象

  Base& bref{ d };
  std::cout << bref << '\n';

  return 0;
}
\end{lstlisting}

上述程序打印:

\begin{lstlisting}
Base
Derived
Derived
\end{lstlisting}

现在来看一下细节。

首先,在 \acode{Base} 里,调用操作符<<后调用虚函数 \acode{print()}。
因为引用入参指向的是 \acode{Base} 对象,\acode{b.print()} 解析为 \acode{Base::print()},即可以打印。
这里没有特别的。

在 \acode{Derived} 里,编译器首先检查操作符<<是否获得一个 \acode{Derived} 对象。
而并没有这样的重载,因为并没有被定义。
接着编译器检查操作符<<是否获得一个 \acode{Base} 对象。
有,因此编译器完成嘞一个隐式的向上转型,使得 \acode{Derived} 对象成为一个 \acode{Base\&},
并调用函数(这里可以人工进行向上转型,但是编译器可以完成)。
该函数调用虚函数 \acode{print()},并被解析为 \acode{Derived::print()}。

注意这里不需要在每个派生类中定义操作符<<!
\acode{Base} 类中定义的完全可以胜任任何 \acode{Base} 对象以及其派生的类!

在第三个打印过程中混合了前两种。
首先,编译器通过操作符<<匹配变量 \acode{bref} 从未获取 \acode{Base}。
其调用了虚函数 \acode{print()}。
因为 \acode{Base} 引用实际上指向的是 \acode{Derived} 对象,因此被解析为 \acode{Derived::print()},正如预期。

\end{document}
