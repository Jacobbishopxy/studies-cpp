\documentclass[../../LearnCpp.tex]{subfiles}

\begin{document}

\asubsection{7}{Pure virtual functions, abstract base classes, and interface classes}

迄今为止所有的虚函数都有函数体(定义)。
然而 C++ 允许创建一种名为\textbf{纯虚函数} pure virtual function
(或\textbf{抽象函数} abstract function)的没有函数体的特殊虚函数!
纯虚函数类似于占位符一样,意为在派生类中被重新定义。

创建一个纯虚函数仅需要对函数赋值为 0。

\begin{lstlisting}[language=C++]
class Base
{
public:
    const char* sayHi() const { return "Hi"; } // 普通的非虚函数

    virtual const char* getName() const { return "Base"; } // 普通的虚函数

    virtual int getValue() const = 0; // 纯虚函数

    int doSomething() = 0; // 编译错误:不能设置非虚函数为 0
};
\end{lstlisting}

在类中添加一个纯虚函数,等同于是在说,“由派生类来实现这个函数”。

使用纯虚函数有两个主要的结果:
首先,任何带有纯虚函数的类会变成\textbf{抽象基类} abstract base class,
这就意味着它不能被实例化!考虑以下代码:

\begin{lstlisting}[language=C++]
int main()
{
    Base base; // 不能实例化一个抽象基类,但是为了演示,假设这样可以
    base.getValue(); // 那么这会是什么呢?

    return 0;
}
\end{lstlisting}

其次,任何派生类必须为此函数定义,否则该派生类同样也会被视为抽象基类。

\subsubsection*{纯虚函数案例}

\begin{lstlisting}[language=C++]
#include <string>

class Animal
{
protected:
    std::string m_name;

    // 使该构造函数为 protected,因为不希望其他人直接创建 Animal 对象,
    // 但是派生类还是可以使用它。
    Animal(const std::string& name)
        : m_name{ name }
    {
    }

public:
    std::string getName() const { return m_name; }
    virtual const char* speak() const { return "???"; }

    virtual ~Animal() = default;
};

class Cat: public Animal
{
public:
    Cat(const std::string& name)
        : Animal{ name }
    {
    }

    const char* speak() const override { return "Meow"; }
};

class Dog: public Animal
{
public:
    Dog(const std::string& name)
        : Animal{ name }
    {
    }

    const char* speak() const override { return "Woof"; }
};
\end{lstlisting}

上述代码通过 protected 构造函数,阻止了 Animal 对象被直接创建。
然而,还是可以创建派生类可以不用重新定义函数 \acode{speak()}。

例如:

\begin{lstlisting}[language=C++]
#include <iostream>
#include <string>

class Cow : public Animal
{
public:
    Cow(const std::string& name)
        : Animal{ name }
    {
    }

    // 这里忘记了去重新定义 speak
};

int main()
{
    Cow cow{"Betsy"};
    std::cout << cow.getName() << " says " << cow.speak() << '\n';

    return 0;
}
\end{lstlisting}

该问题的一个更好的解决方案是使用纯虚函数:

\begin{lstlisting}[language=C++]
#include <string>

class Animal // 该 Animal 是一个抽象基类
{
protected:
    std::string m_name;

public:
    Animal(const std::string& name)
        : m_name{ name }
    {
    }

    const std::string& getName() const { return m_name; }
    virtual const char* speak() const = 0; // 注意 speak 现在是一个纯虚函数

    virtual ~Animal() = default;
};
\end{lstlisting}

这里有几点需要注意的。首先,\acode{speak()} 现在是一个纯虚函数。
这就意味着 \acode{Animal} 现在是一个抽象基类,并不可再被实例化。
结果就是不再需要让构造函数 protected(尽管也没别的影响)。
其次,因为 \acode{Cow} 类是由 \acode{Animal} 派生而来,
但是并没有定义 \acode{Cow::speak()},\acode{Cow} 仍然是一个抽象基类。
如果尝试编译代码:

\begin{lstlisting}[language=C++]
#include <iostream>

class Cow: public Animal
{
public:
    Cow(const std::string& name)
        : Animal{ name }
    {
    }

    // 忘记重新定义 speak
};

int main()
{
    Cow cow{ "Betsy" };
    std::cout << cow.getName() << " says " << cow.speak() << '\n';

    return 0;
}
\end{lstlisting}

编译器会报错,因为 \acode{Cow} 是一个抽象基类,不能直接实例化它。
也就是告诉我们除非提供了 \acode{speak()} 函数体给 \acode{Cow},否则不能实例化它。

\begin{lstlisting}[language=C++]
#include <iostream>
#include <string>

class Animal
{
protected:
    std::string m_name;

public:
    Animal(const std::string& name)
        : m_name{ name }
    {
    }

    const std::string& getName() const { return m_name; }
    virtual const char* speak() const = 0;

    virtual ~Animal() = default;
};

class Cow: public Animal
{
public:
    Cow(const std::string& name)
        : Animal(name)
    {
    }

    const char* speak() const override { return "Moo"; }
};

int main()
{
    Cow cow{ "Betsy" };
    std::cout << cow.getName() << " says " << cow.speak() << '\n';

    return 0;
}
\end{lstlisting}

现在打印:

\begin{lstlisting}
Betsy says Moo
\end{lstlisting}

\subsubsection*{带有定义的纯虚函数}

结果就是可以创建带有定义的纯虚函数:

\begin{lstlisting}[language=C++]
#include <string>

class Animal
{
protected:
    std::string m_name;

public:
    Animal(const std::string& name)
        : m_name{ name }
    {
    }

    std::string getName() { return m_name; }
    virtual const char* speak() const = 0;

    virtual ~Animal() = default;
};

const char* Animal::speak() const  // 尽管它有一个定义
{
    return "buzz";
}
\end{lstlisting}

上述例子中,\acode{speak()} 仍然被视为纯虚函数因为“=0”(尽管被给与了一个定义),
同时 \acode{Animal} 仍然被视为抽象基类(因此还是不能被实例化)。
任何继承了 \acode{Animal} 的类需要提供自身的 \acode{speak()} 定义,否则被视为抽象基类。

当为纯虚函数提供了定义,该定义必须是分开定义的(而不是内联的方式)。

这个范式非常的有用:当用户需要为函数提供一个默认实现,但是仍然可以强制派生类提供自身的实现。
同时如果派生类愿意使用有基类提供的默认实现,可以简单的直接调用基类的实现。例如:

\begin{lstlisting}[language=C++]
#include <string>
#include <iostream>

class Animal // 抽象基类
{
protected:
    std::string m_name;

public:
    Animal(const std::string& name)
        : m_name(name)
    {
    }

    const std::string& getName() const { return m_name; }
    virtual const char* speak() const = 0; // 注意 speak 是一个纯虚函数

    virtual ~Animal() = default;
};

const char* Animal::speak() const
{
    return "buzz"; // 默认实现
}

class Dragonfly: public Animal
{

public:
    Dragonfly(const std::string& name)
        : Animal{name}
    {
    }

    const char* speak() const override// 该类不再是抽象基类,因为提供了函数定义
    {
        return Animal::speak(); // 使用 Animal 的默认实现
    }
};

int main()
{
    Dragonfly dfly{"Sally"};
    std::cout << dfly.getName() << " says " << dfly.speak() << '\n';

    return 0;
}
\end{lstlisting}

打印:

\begin{lstlisting}
Sally says buzz
\end{lstlisting}

析构函数也可以是纯虚函数,但是必须给予定义,这样可以在派生对象销毁时调用它。

\subsubsection*{接口类}

\textbf{接口类} interface class 是一种没有成员变量的类,同时*所有*函数都是纯虚函数!
换言之,这种类是纯粹的定义,并没有实际的实现。
希望在派生类中必须定义所有功能时,并且将所有实现的细节完全交托给派生类时,接口非常的有用。

\begin{lstlisting}[language=C++]
class IErrorLog
{
public:
    virtual bool openLog(const char* filename) = 0;

    virtual bool closeLog() = 0;

    virtual bool writeError(const char* errorMessage) = 0;

    virtual ~IErrorLog() {} // 使析构函数为虚函数,以防删除了 IErrorLog 指针
};
\end{lstlisting}

任何继承 \acode{IErrorLog} 的类必须提供上述的三个函数实现才可以进行实例化。
可以派生一个 \acode{FileErrorLog} 的类,其中的 \acode{openLog()} 打开一个文件,
\acode{closeLog()} 关闭文件,\acode{writeError()} 向文件写入信息。
也可以派生一个 \acode{ScreenErrorLog} 类,其中的 \acode{openLog()} 与 \acode{closeLog()} 不做任何事,
\acode{writeError()} 向显示器打印信息。

假设需要使用日志,如果直接引入 \acode{FileErrorLog} 或是 \acode{ScreenErrorLog},
那么很容易就被卡住。例如入参的选择:

\begin{lstlisting}[language=C++]
#include <cmath> // for sqrt()

double mySqrt(double value, FileErrorLog& log)
{
    if (value < 0.0)
    {
        log.writeError("Tried to take square root of value less than 0");
        return 0.0;
    }
    else
    {
        return std::sqrt(value);
    }
}
\end{lstlisting}

一个更好的做法是使用 \acode{IErrorLog} :

\begin{lstlisting}[language=C++]
#include <cmath> // for sqrt()
double mySqrt(double value, IErrorLog& log)
{
    if (value < 0.0)
    {
        log.writeError("Tried to take square root of value less than 0");
        return 0.0;
    }
    else
    {
        return std::sqrt(value);
    }
}
\end{lstlisting}

现在调用者可以传递\textit{任何}符合 \acode{IErrorLog} 接口的类。

不要忘记为接口类包含一个虚化的析构函数,这样的话如果一个该接口的指针被删除时,正确的析构函数才能够被调用。

接口类变得非常的流行,因为它们的使用简单,扩展方便,以及易于维护。
实际上一些现代语言,例如 Java 和 C\# 都添加了“interface”关键字,便于程序员直接定义接口类而不需要标记所有成员函数为抽象。

此外,尽管 Java(版本 8 之前)以及 C\# 不让普通类使用若干继承,它们允许使用若干的接口。
因为接口没有数据,也没有函数本体,它们避免了很多由多重继承带来的传统问题的同时提供了灵活性。

\subsubsection*{纯虚函数和虚表}

抽象类仍然有虚表,它们仍然可以被使用,如果抽象类拥有一个指针或者引用。
对于一个类而言虚表的入口是纯虚函数的情况下,会生成空指针或者指向普通打印错误的函数(有时这个函数名为 \acode{\_\_purecall} )。

\end{document}
