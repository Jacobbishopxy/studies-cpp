\documentclass[../../LearnCpp.tex]{subfiles}

\begin{document}

\asubsection{11}{Return by reference and return by address}

\subsubsection*{返回引用}

\textbf{返回引用}可以避免返回类型的拷贝。

\begin{lstlisting}[language=C++]
std::string&       returnByReference(); // 返回一个已存在的 std::string 的引用(廉价)
const std::string& returnByReferenceToConst(); // 返回一个已存在的 std::string 的 const 引用(廉价)
\end{lstlisting}

\begin{lstlisting}[language=C++]
#include <iostream>
#include <string>

const std::string& getProgramName() // 返回一个 const 引用
{
    static const std::string s_programName { "Calculator" }; // static duration,在程序结束后销毁

    return s_programName;
}

int main()
{
    std::cout << "This program is named " << getProgramName();

    return 0;
}
\end{lstlisting}

\subsubsection*{返回引用的对象必须在函数返回后存在}

使用引用作为返回时有一个严峻的警告:程序员\textit{必须}确保被引用的对象存在的时间比返回引用时要长。
否则,返回的引用会变成悬垂引用(对象被销毁了),使用这样的引用将导致未定义行为。

现代编译器在遇到尝试返回本地变量的引用时,会报警或者返回编译错误,
但是编译器有时在遇到更复杂的情况下可能会遇到问题。

\subsubsection*{不要返回 non-const local static 变量的引用}

\begin{lstlisting}[language=C++]
#include <iostream>
#include <string>

const int& getNextId()
{
    static int s_x{ 0 }; // 注意:变量为 non-const
    ++s_x; // 生成下一个 id
    return s_x; // 返回其引用
}

int main()
{
    const int& id1 { getNextId() }; // id1 为引用
    const int& id2 { getNextId() }; // id2 为引用

    std::cout << id1 << id2 << '\n';

    return 0;
}
\end{lstlisting}

上述代码返回 \acode{22}。这是因为 \acode{id1} 与 \acode{id2} 引用的是同一个对象(static 变量 \acode{s_x}),因此当任何(例如 \acode{getNextId()})修改了值,所有的引用访问了已修改的值。另一个常见的问题是当返回一个 static local 的 const 引用时,没有标准化的方法重置 \acode{s_x} 置默认值。这样的程序要么使用非自然的方法(例如重置参数),或者只能由退出后重启来进行重置。

最佳实践:避免返回 non-const local static 变量的引用。

\subsubsection*{通过拷贝返回的引用进行赋值/初始化普通变量}

如果函数返回引用,且引用用于初始化或者给非引用变量进行赋值,那么返回值将被拷贝(就像是直接返回值)。

\begin{lstlisting}[language=C++]
#include <iostream>
#include <string>

const int& getNextId()
{
    static int s_x{ 0 };
    ++s_x;
    return s_x;
}

int main()
{
    const int id1 { getNextId() }; // id1 现在是普通变量,其获取 getNextId() 返回的引用并进行值拷贝
    const int id2 { getNextId() }; // id2 现在是普通变量,其获取 getNextId() 返回的引用并进行值拷贝

    std::cout << id1 << id2 << '\n';

    return 0;
}
\end{lstlisting}

上述代码中,\acode{getNextId()} 返回引用,但是 \acode{id1} 与 \acode{id2} 为非引用变量。
这种情况下,返回的引用值被拷贝成正常变量。因此打印 \acode{12}。

另外也要注意如果一个程序返回悬垂引用,引用在拷贝之前变为悬垂,也会带来未定义行为:

\begin{lstlisting}[language=C++]
#include <iostream>
#include <string>

const std::string& getProgramName() // 返回 const 引用
{
    const std::string programName{ "Calculator" };

    return programName;
}

int main()
{
    std::string name { getProgramName() }; // 拷贝悬垂引用
    std::cout << "This program is named " << name << '\n'; // 未定义行为

    return 0;
}
\end{lstlisting}

\subsubsection*{返回引用参数的引用是可以的}

有几种情况下返回对象的引用是有意义的。

如果入参就是引用,那么返回入参引用是安全的,例如:

\begin{lstlisting}[language=C++]
#include <iostream>
#include <string>

const std::string& firstAlphabetical(const std::string& a, const std::string& b)
{
  return (a < b) ? a : b;
}

int main()
{
  std::string hello { "Hello" };
  std::string world { "World" };

  std::cout << firstAlphabetical(hello, world) << '\n';

  return 0;
}
\end{lstlisting}

\subsubsection*{调用者可以通过引用修改值}

当 non-const 引用作为参数被传递至函数,函数则可以使用该引用进行值修改。

同样的,当 non-const 引用被返回,调用者可以使用该引用进行值修改。

\begin{lstlisting}[language=C++]
#include <iostream>

// 接收两个 non-const 引用,返回最大的那个
int& max(int& x, int& y)
{
    return (x > y) ? x : y;
}

int main()
{
    int a{ 5 };
    int b{ 6 };

    max(a, b) = 7; // 修改最大的 a 或 b 至 7

    std::cout << a << b << '\n';

    return 0;
}
\end{lstlisting}

\subsubsection*{返回地址}

\textbf{返回地址}与返回引用,功能基本一致,前者返回的是对象的指针,后者返回的则是对象的引用。
返回地址与返回引用一样需要额外的警惕 -- 返回地址的这个对象必须在函数返回后存活着,不然调用者则会接收到一个悬垂指针。

相比于返回引用,返回地址最大的优势是,如果没有合适的对象返回时,可以让函数返回 \acode{nullptr}。

而返回地址最大的劣势是,调用者必须记得在解引用之前进行 \acode{nullptr} 的检查,
否则空指针解引用则会带来未定义行为。正因为这样的危险,才更加推荐返回引用而不是返回地址,
除非返回“无对象”是真实需要的。

最佳实践:推荐返回引用而不是返回地址,除非需求“无对象”(使用 \acode{nullptr})的返回。

\end{document}
