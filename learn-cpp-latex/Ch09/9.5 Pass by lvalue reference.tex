\documentclass[../../LearnCpp.tex]{subfiles}

\begin{document}

\asubsection{5}{Pass by lvalue reference}

上一章介绍了左值引用以及 const 左值引用。
单独来看它们似乎不是很有用 -- 当可以使用变量本身的时候为什么还给变量创建别名呢?

首先在一些上下文中,函数参数是被拷贝成为函数的入参:

\begin{lstlisting}[language=C++]
#include <iostream>

void printValue(int y)
{
    std::cout << y << '\n';
} // y 在此销毁

int main()
{
    int x { 2 };

    printValue(x); // x 被值传递(拷贝)进入参数 y (廉价)

    return 0;
}
\end{lstlisting}

\subsubsection*{一些对象的拷贝很昂贵}

大多数由标准库提供的类型(例如 \acode{std::string})为 \acode{class types}。
类类型的拷贝通常是很昂贵的。可能的话,应该避免不必要的对象拷贝,特别是拷贝之后又被立刻销毁。

考虑以下代码:

\begin{lstlisting}[language=C++]
#include <iostream>
#include <string>

void printValue(std::string y)
{
    std::cout << y << '\n';
} // y 在此销毁

int main()
{
    std::string x { "Hello, world!" }; // x 为 std::string

    printValue(x); // x 被值传递(拷贝)进入参数 y(昂贵)

    return 0;
}
\end{lstlisting}

\subsubsection*{引用传递}

调用函数时,避免昂贵拷贝的一种方式是使用 \acode{pass by reference}。当使用**传递引用**时,需要定义函数的入参为引用类型(或者 const 引用类型)而不是普通类型。当函数被调用时,每个引用入参绑定合理的参数。因为引用充当参数的别名,那么就不再需要进行参数拷贝。

\begin{lstlisting}[language=C++]
#include <iostream>
#include <string>

void printValue(std::string& y) // 类型改为 std::string&
{
    std::cout << y << '\n';
} // y 在此销毁

int main()
{
    std::string x { "Hello, world!" };

    printValue(x); // x 被引用传递进入参数 y(廉价)

    return 0;
}
\end{lstlisting}

\subsubsection*{引用传递进行参数的值修改}

\begin{lstlisting}[language=C++]
#include <iostream>

void addOne(int y) // y 为 x 的拷贝
{
    ++y; // 这里修改 x 的拷贝,而不是 x 对象
}

int main()
{
    int x { 5 };

    std::cout << "value = " << x << '\n';

    addOne(x);

    std::cout << "value = " << x << '\n'; // x 没有被改变

    return 0;
}
\end{lstlisting}

\begin{lstlisting}[language=C++]
#include <iostream>

void addOne(int& y) // y 绑定对象 x
{
    ++y; // 这里修改对象 x
}

int main()
{
    int x { 5 };

    std::cout << "value = " << x << '\n';

    addOne(x);

    std::cout << "value = " << x << '\n'; // x 被改变了

    return 0;
}
\end{lstlisting}

\subsubsection*{非 const 的引用传递仅接受可变左值引用}

\begin{lstlisting}[language=C++]
#include <iostream>
#include <string>

void printValue(int& y) // y 进接受可变左值
{
    std::cout << y << '\n';
}

int main()
{
    int x { 5 };
    printValue(x); // 正确:x 为可变左值

    const int z { 5 };
    printValue(z); // 错误:z 为不可变左值
    printValue(5); // 错误:5 为右值

    return 0;
}
\end{lstlisting}

传递 const 引用提供同样的便利(避免参数拷贝),同时也保证函数*不会*改变其引用的值。

\begin{lstlisting}[language=C++]
void addOne(const int& ref)
{
    ++ref; // 不被允许:ref 是 const
}
\end{lstlisting}

最佳实践:推荐传递 const 引用而不是非 const 引用,除非有特别的理由需要做额外的事情(例如函数需要修改入参的值)。

\subsubsection*{混合值传递与引用传递}

\begin{lstlisting}[language=C++]
#include <string>

void foo(int a, int& b, const std::string& c)
{
}

int main()
{
    int x { 5 };
    const std::string s { "Hello, world!" };

    foo(5, x, s);

    return 0;
}
\end{lstlisting}

\subsubsection*{何时引用传递}

最佳实践:基础类型使用值传递,类(或结构体)类型使用 const 引用传递。

\subsubsection*{值传递与引用传递的开销}

不是所有的类类型都需要引用传递。

有两个关键点可以帮助我们理解何时需要值传递 vs 引用传递:

首先,拷贝对象的开销普遍与两点成正比:

\begin{itemize}
    \item 对象的大小。对象使用越多的内存就需要更长的时间拷贝。
    \item 任何其它的准备开销。一些类类型需要额外准备的当它们被实例化时(例如打开文件或者数据库,
          或者分配特定的动态内存用以存储对象中的变量)。这些准备开销必须在每个对象拷贝时花费时间。
\end{itemize}

另一方面,为对象绑定引用总是迅速的(与拷贝基础类型的速度相同)。

其次,通过引用访问对象对比直接访问对象会稍微昂贵一些。
通过变量的标识符,编译器可以直接获取分配给变量的内存地址并访问值。
通过引用时,通常会有额外的一步:编译器必须先判定被引用的是哪个对象。
编译器有时也会优化代码使用对象传递值。
这就意味着,对象被引用传递生成的代码通常会慢于值传递所生成的代码。

现在可以回答为什么不在所有的地方进行引用传递:

\begin{itemize}
    \item 对于廉价的对象拷贝,拷贝的开销与绑定的开销差不多,因此推荐值传递因为代码生成会快一些。
    \item 对于昂贵的对象拷贝,拷贝占据了大部分开销,因此推荐传递(const)引用来避免拷贝。
\end{itemize}

最佳实践:当对象拷贝廉价时,推荐值传递;对象拷贝昂贵时,推荐 const 引用传递;如果不确定,推荐后者。

最后一个问题是,如何定义“廉价拷贝”?
这里没有绝对的答案,因为这随着编译器,使用案例,架构等变化。
然而可以设立一个简单的规则:如果一个对象使用两个或更少的“单词”的内存(一个“单词”约等于一个内存地址大小)
以及没有准备开销,视为廉价拷贝。

下面代码定义了一个宏用于判断一个类型(或对象)是有两个或者更少的内存地址:

\begin{lstlisting}[language=C++]
#include <iostream>

// 返回 true 如果类型(或者对象)使用两个或更少的内存地址
#define isSmall(T) (sizeof(T) <= 2 * sizeof(void*))

struct S
{
    double a, b, c;
};

int main()
{
    std::cout << std::boolalpha; // 打印 true 或 false 而不是 0 或 1
    std::cout << isSmall(int) << '\n'; // true
    std::cout << isSmall(double) << '\n'; // true
    std::cout << isSmall(S) << '\n'; // false

    return 0;
}
\end{lstlisting}

然而很难知道一个类类型的对象是否有准备开销。最好是假设大部分的标准库都有准备开销。

\end{document}
