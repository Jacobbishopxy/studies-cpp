\documentclass[../../LearnCpp.tex]{subfiles}

\begin{document}

\asubsection{9}{Pass by address}

\begin{lstlisting}[language=C++]
#include <iostream>
#include <string>

void printByValue(std::string val) // 函数参数为 str 的拷贝
{
    std::cout << val << '\n'; // 通过拷贝进行打印
}

void printByReference(const std::string& ref) // 函数参数为绑定 str 的引用
{
    std::cout << ref << '\n'; // 通过引用进行打印
}

void printByAddress(const std::string* ptr) // 函数参数为存储 str 地址的指针
{
    std::cout << *ptr << '\n'; // 通过解引用指针进行打印
}

int main()
{
    std::string str{ "Hello, world!" };

    printByValue(str); // 值传递,拷贝 str
    printByReference(str); // 引用传递,不会拷贝 str
    printByAddress(&str); // 地址传递,不会拷贝 str

    std::string* ptr { &str }; // 定义一个存储 str 地址的指针变量
    printByAddress(ptr); // 地址传递,不会拷贝 str

    return 0;
}
\end{lstlisting}

\subsubsection*{地址传递允许函数修改入参值}

当通过对象的地址进行传递,函数获取对象的地址,可以通过解引用进行访问。
因为这个地址属于真实的对象(而不是对象的拷贝),如果函数入参的指针非 const,
函数可以通过指针修改入参的值:

\begin{lstlisting}[language=C++]
#include <iostream>

void changeValue(int* ptr) // 注意:ptr 在本示例中指向非 const
{
    *ptr = 6; // 修改值为 6
}

int main()
{
    int x{ 5 };

    std::cout << "x = " << x << '\n'; // 打印 5

    changeValue(&x); // 传递 x 地址给函数

    std::cout << "x = " << x << '\n'; // 打印 6

    return 0;
}
\end{lstlisting}

\subsubsection*{Null 检查}

\begin{lstlisting}[language=C++]
#include <iostream>

void print(int* ptr)
{
    if (ptr) // 如果 ptr 不是一个空指针
    {
        std::cout << *ptr;
    }
}

int main()
{
  int x{ 5 };

  print(&x);
  print(nullptr);

  return 0;
}
\end{lstlisting}

大多数情况下,与上述例子相反的判断更有效:测试函数入参是否为空作为先决条件:

\begin{lstlisting}[language=C++]
#include <iostream>

void print(int* ptr)
{
  if (!ptr) // 如果 ptr 为空指针,提前结束函数
      return;

  // 到了这里可以认为 ptr 是有效的,因此不在需要测试或嵌套了

  std::cout << *ptr;
}

int main()
{
  int x{ 5 };

  print(&x);
  print(nullptr);

  return 0;
}
\end{lstlisting}

如果空指针根本不可以传递至函数,则加上 \acode{assert}(这里 asserts 用于记录永远不会出现):

\begin{lstlisting}[language=C++]
#include <iostream>
#include <cassert>

void print(const int* ptr) // const int 指针
{
  assert(ptr); // debug 模式下如果传递的是空指针则程序失败(因为这永远不会出现)

  //(可选)在生产模式下处理错误,如果真实发生了也不会导致程序崩溃
  if (!ptr)
    return;

  std::cout << *ptr;
}

int main()
{
  int x{ 5 };

  print(&x);
  print(nullptr);

  return 0;
}
\end{lstlisting}

\subsubsection*{推荐传递(const)引用}

注意上述例子中的 \acode{print} 函数处理空值不是很理想 --
即直接退出函数。那么为什么允许用户传递空值呢?
传递引用有着与传递指针相同的便利,而不用承担解引用空指针的风险。

相较于传递地址,传递 const 引用还有一些额外的优势。

首先,因为传递的对象必须拥有地址,仅左值可以被传递地址(右值没有地址)。
传递 const 引用更加的灵活,既可以接受左值也可以接受右值:

\begin{lstlisting}[language=C++]

// ... 与最开头的函数一致

int main()
{
    printByValue(5);     // 有效,但是需要拷贝
    printByReference(5); // 有效,因为入参是 const 引用
    printByAddress(&5);  // 错误,不可以获取 address-of 右值

    return 0;
}
\end{lstlisting}

其次,传递引用的语法更加自然,仅需要传递字面值或者对象。
传递地址却需要加上 \& 符号(调用时)以及 * 符号(函数编写时)。

最佳实践:现代 C++ 中,大部分的事务都可以通过传递地址完成。
遵循这条规则:“尽可能使用传递引用,必须时才传递地址”。

\subsubsection*{“可选”入参的地址传递}

\begin{lstlisting}[language=C++]
#include <iostream>
#include <string>

void greet(std::string* name=nullptr)
{
    std::cout << "Hello ";
    std::cout << (name ? *name : "guest") << '\n';
}

int main()
{
    greet(); // 还不知道是哪个用户

    std::string joe{ "Joe" };
    greet(&joe); // 现在知道了用户是 joe

    return 0;
}
\end{lstlisting}

不过大多数情况下,使用重载函数会更好一些:

\begin{lstlisting}[language=C++]
#include <iostream>
#include <string>
#include <string_view>

void greet(std::string_view name)
{
    std::cout << "Hello " << name << '\n';
}

void greet()
{
    greet("guest");
}

int main()
{
    greet(); // 还不知道是哪个用户

    std::string joe{ "Joe" };
    greet(joe); // 现在知道了用户是 joe

    return 0;
}
\end{lstlisting}

这样做有更多的好处:不再需要担心解引用空值,以及可以直接传递字符串字面值。

\subsubsection*{修改指针参数的指向}

当传递地址给函数时,地址从入参中被拷贝进指针参数(没有问题,因为拷贝地址很快)。现在考虑以下案例:

\begin{lstlisting}[language=C++]
#include <iostream>

// [[maybe_unused]] 去除编译器警告:ptr2 设置了但是没有被使用
void nullify([[maybe_unused]] int* ptr2)
{
    ptr2 = nullptr; // 使函数参数变为空指针
}

int main()
{
    int x{ 5 };
    int* ptr{ &x }; // ptr 指向 x

    std::cout << "ptr is " << (ptr ? "non-null\n" : "null\n"); // 打印 ptr is non-null

    nullify(ptr);

    std::cout << "ptr is " << (ptr ? "non-null\n" : "null\n"); // 打印 ptr is non-null
    return 0;
}
\end{lstlisting}

如上所见,在函数内修改函数入参指针的地址不会影响参数的地址(\acode{ptr} 仍然指向 \acode{x})。
当函数 \acode{nullify()} 被调用时,\acode{ptr2} 获取了传递进来的地址的拷贝(即 \acode{ptr} 所存储的地址,即 \acode{x} 的地址)。
当函数修改 \acode{ptr2} 的指向时,仅仅影响的是拷贝后的 \acode{ptr2}。

那么想要函数修改指针所存储的地址需要怎么做呢?

\subsubsection*{通过引用传递地址?}

正如可以通过引用传递普通变量一样,可以传递指针的引用:

\begin{lstlisting}[language=C++]
#include <iostream>

void nullify(int*& refptr) // refptr 现在是指针的引用
{
    refptr = nullptr; // 使函数参数变为空指针
}

int main()
{
    int x{ 5 };
    int* ptr{ &x }; // ptr 指向 x

    std::cout << "ptr is " << (ptr ? "non-null\n" : "null\n"); // 打印 ptr is non-null

    nullify(ptr);

    std::cout << "ptr is " << (ptr ? "non-null\n" : "null\n"); // 打印 ptr is null
    return 0;
}
\end{lstlisting}

因为 \acode{refptr} 现在是指针的引用,当 \acode{ptr} 作为参数时,\acode{refptr} 绑定 \acode{ptr}。
这就意味着 \acode{refptr} 的任何改动都将作用于 \acode{ptr} 上。

\subsubsection*{其实只有值传递}

通过了引用传递,地址传递,值传递的学习,那么现在来看看如何做减法。

因为编译器总是能优化整个引用,很多时候实际需要引用的情况是不存在的。
引用通常被编译器实现成指针。这就意味着在这个场景下,传递引用实际上仅仅传递的是地址(通过隐式的解引用来访问引用)。

同时之前的课程中提到了传递地址仅仅是拷贝调用者的地址给到被调用的函数 -- 即以值传递的方式传递地址。

因此,可以总结 C++ 真正传递的是值!传递地址(和引用)仅存在于需要解引用修改值的时候,
这种时候是不可以用通常的值作为入参。

\end{document}
