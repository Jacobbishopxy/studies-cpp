\documentclass[../../LearnCpp.tex]{subfiles}

\begin{document}

\asubsection{12}{Type deduction with pointers, references, and const}

\acode{auto} 关键字推导时会丢弃 \acode{const} 限定符:

\begin{lstlisting}[language=C++]
const double foo()
{
    return 5.6;
}

int main()
{
    const double cd{ 7.8 };

    auto x{ cd };    // double(const 被丢弃)
    auto y{ foo() }; // double(const 被丢弃)

    return 0;
}
\end{lstlisting}

通过添加 \acode{const} 限定符可以重新应用 const:

\begin{lstlisting}[language=C++]
const double foo()
{
    return 5.6;
}

int main()
{
    const double cd{ 7.8 };

    const auto x{ cd };    // const double(const 重新应用)
    const auto y{ foo() }; // const double(const 重新应用)

    return 0;
}
\end{lstlisting}

\subsubsection*{类型推导丢弃引用}

除了丢弃 const 限定符,类型推导也会丢弃引用:

\begin{lstlisting}[language=C++]
#include <string>

std::string& getRef(); // 某返回引用的函数

int main()
{
    auto ref { getRef() }; // 类型推导为 std::string(而不是 std::string&)

    return 0;
}
\end{lstlisting}

与丢弃 const 限定符一样,如果需要推导的类型成为一个引用,可以进行重新应用:

\begin{lstlisting}[language=C++]
#include <string>

std::string& getRef(); // 某返回引用的函数

int main()
{
    auto ref1 { getRef() };  // std::string(引用被丢弃)
    auto& ref2 { getRef() }; // std::string&(引用重新应用)

    return 0;
}
\end{lstlisting}

\subsubsection*{高等 const 与 低等 const}

\textbf{高等 const}是 const 限定符直接应用于对象本身:

\begin{lstlisting}[language=C++]
const int x;    // 该 const 应用于 x,因此它是高等 const
int* const ptr; // 该 const 应用于 ptr,因此它是高等 const
\end{lstlisting}

相反,\textbf{低等 const}则是 const 限定符应用于对象的引用或者指针:

\begin{lstlisting}[language=C++]
const int& ref; // 该 const 应用于被引用的对象,因此它是低等 const
const int* ptr; // 该 const 应用于被指向的对象,因此它是低等 const
\end{lstlisting}

const 值的引用永远是低等 const,而指针可以拥有高等,低等,或者两者兼具的 const:

\begin{lstlisting}[language=C++]
const int* const ptr; // 左 const 低等,右 const 高等
\end{lstlisting}

当提到类型推导丢弃 const 限定符,它仅仅丢弃高等 const,而低等 const 是不会被丢弃的。接下来看几个例子。

\subsubsection*{类型推导与 const 引用}

如果初始化的是 const 的引用,引用首先被丢弃(再重新应用,如果合适),接着任何高等 const 从结果中丢弃。

\begin{lstlisting}[language=C++]
#include <string>

const std::string& getRef(); // 某返回 const 引用的函数

int main()
{
    auto ref1{ getRef() }; // std::string(引用被丢弃,接着高等 const 在结果中被丢弃)

    return 0;
}
\end{lstlisting}

上述例子中,因为 \acode{getRef()} 返回 \acode{const std::string&},引用会先被丢弃,留下 \acode{const std::string}。该 const 为高等,因此也被丢弃,留下推导类型 \acode{std::string}。

可以进行重新应用:

\begin{lstlisting}[language=C++]
#include <string>

const std::string& getRef(); // 某返回 const 引用的函数

int main()
{
    auto ref1{ getRef() };        // std::string(高等 const 与引用被丢弃了)
    const auto ref2{ getRef() };  // const std::string(高等 const 被重新应用,引用被丢弃了)

    auto& ref3{ getRef() };       // const std::string&(引用被重新应用,低等 const 未被丢弃)
    const auto& ref4{ getRef() }; // const std::string&(引用与 const 都被重新应用)

    return 0;
}
\end{lstlisting}

\acode{ref1} 和 \acode{ref2} 没有问题。问题在于 \acode{ref3},通常来说引用会被丢弃,但是因为重新应用了引用,它没有被丢弃。这意味着类型仍然是 \acode{const std::string&}。同时因为 const 为低等 const,没有被丢弃。因此这里的推导类型是 \acode{const std::string&}。

\acode{ref4} 类似于 \acode{ref3},除了重新应用了 \acode{const} 限定符。因为类型已经被推导成 const 的引用,重新应用 \acode{const} 在这里是多余的。在此使用 \acode{const} 可以显式的清楚知道结果为 const(而 \acode{ref3} 的案例中推导的 const,其结果是隐式的且不明显的)。

最佳实践:如果需要 const 引用,重新应用 \acode{const} 即便它不是被严格的需求,这可以使代码更加清晰并且防止错误。

\subsubsection*{类型推导与指针}

不同于引用,类型推导不会丢弃指针:

\begin{lstlisting}[language=C++]
#include <string>

std::string* getPtr(); // 某返回指针的函数

int main()
{
    auto ptr1{ getPtr() }; // std::string*

    return 0;
}
\end{lstlisting}

也可以用星号来表示类型推导:

\begin{lstlisting}[language=C++]
#include <string>

std::string* getPtr(); // 某返回指针的函数

int main()
{
    auto ptr1{ getPtr() };  // std::string*
    auto* ptr2{ getPtr() }; // std::string*

    return 0;
}
\end{lstlisting}

\subsubsection*{auto 与 auto* 的不同之处}

当使用 \acode{auto} 作为指针类型的初始化,\acode{auto} 的类型推导包含了指针。因此上述例子中的 \acode{ptr1},替换 \acode{auto} 的类型是 \acode{std::string*}。

当使用 \acode{auto*} 作为指针类型的初始化,类型推导则*不会*包含指针 -- 指针再之后被重新应用在推导的类型上。因此 \acode{ptr2},替换 \acode{auto} 的类型是 \acode{std::string},指针再重新应用。

多数情况下,两者的效果是一致的(\acode{ptr1} 与 \acode{ptr2} 都被推导为 \acode{std::string*})。

然而它们之间还有一些区别。首先,\acode{auto*} 必须被指定为指针初始化,否则编译器会报错:

\begin{lstlisting}[language=C++]
#include <string>

std::string* getPtr(); // 某返回指针的函数

int main()
{
    auto ptr3{ *getPtr() };      // std::string(因为解引用了 getPtr())
    auto* ptr4{ *getPtr() };     // 不能被编译(initializer 不是一个指针)

    return 0;
}
\end{lstlisting}

\subsubsection*{类型推导与 const 指针}

由于指针不会被丢弃,不需要担心它。但是对于指针,还有 const 指针以及指向 const 的指针需要考虑,同样也有 \acode{auto} vs \acode{auto*}。与引用类似,只有高等 const 在指针类型推导时会被丢弃。

来看一个简单的例子:

\begin{lstlisting}[language=C++]
#include <string>

std::string* getPtr(); // 某返回指针的函数

int main()
{
    const auto ptr1{ getPtr() };  // std::string* const
    auto const ptr2 { getPtr() }; // std::string* const

    const auto* ptr3{ getPtr() }; // const std::string*
    auto* const ptr4{ getPtr() }; // std::string* const

    return 0;
}
\end{lstlisting}

当使用 \acode{auto const} 或 \acode{const auto} 时,可以说,“让任何推导类型都为 const”。因此 \acode{ptr1} 与 \acode{ptr2} 的推导类型是 \acode{std::string* const}。类似于 \acode{const int} 和 \acode{int const} 表明的是同样一件事。

然而当使用 \acode{auto*} 时,const 限定符的顺序则是有作用的。\acode{const} 在左侧意为“让推导的指针类型指向 const”,而 \acode{const} 在右侧意为“让推导的指针类型为 const 指针”。因此 \acode{ptr3} 成为了指向 const 的指针,\acode{ptr4} 成为了 const 指针。

接下来看一个例子 initializer 作为指向 const 的 const 指针:

\begin{lstlisting}[language=C++]
#include <string>

const std::string* const getConstPtr(); // 某返回指向 const 的 const 指针的函数

int main()
{
    auto ptr1{ getConstPtr() };  // const std::string*
    auto* ptr2{ getConstPtr() }; // const std::string*

    auto const ptr3{ getConstPtr() };  // const std::string* const
    const auto ptr4{ getConstPtr() };  // const std::string* const

    auto* const ptr5{ getConstPtr() }; // const std::string* const
    const auto* ptr6{ getConstPtr() }; // const std::string*

    const auto const ptr7{ getConstPtr() };  // 错误:const 限定符不可以被二次应用
    const auto* const ptr8{ getConstPtr() }; // const std::string* const

    return 0;
}
\end{lstlisting}

\acode{ptr1} 与 \acode{ptr2} 很直接。高等 const 被丢弃,低等 const 保留,因此这两种情况的最终类型皆为 \acode{const std::string*}。

\acode{ptr3} 与 \acode{ptr4} 同样很直接。高等 const 被丢弃,但是被重新应用,因此这两种情况皆为 \acode{const std::string* const}。

\acode{ptr5} 与 \acode{ptr6} 类似于之前的例子。高等 const 被丢弃。对于 \acode{ptr5} 而言,\acode{auto* const} 重新应用了高等 const,其最终类型为 \acode{const std::string* const}。而对于 \acode{ptr6} 而言,\acode{const auto*} 重新应用的 const 为其所指向的(在这里已经是 const 了),因此最终类型为 \acode{const std::string*}。

\acode{ptr7} 应用了 const 两次,这是不被允许的,因此导致编译错误。

\acode{ptr8} 则是在指针的两侧都重新应用了 const(这是允许的因为 \acode{auto*} 必须为一个指针类型),因此其返回的类型是 \acode{const std::string* const}。

最佳实践:如果需要一个 const 指针,重新应用 \acode{const} 限定符即使不是真正严格的需求,这样使得意图更为明确以及可以防止错误。

\end{document}
