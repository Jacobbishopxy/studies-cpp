\documentclass[../../LearnCpp.tex]{subfiles}

\begin{document}

\asubsection{3}{lvalue references}

C++ 中\textbf{引用} references 是已有对象的别称。
一旦一个引用被定义了,任何应用于引用的操作则会应用到被引用的对象其本身。

这就意味着用户可以使用引用来读取或者修改被引用的对象。
尽管引用刚开始看起来有点笨拙无用或者多余的,引用却在 C++ 中的任何地方都有用到。

也可以对函数创建引用,尽管这不常见。

现代 C++ 包含两种类型的引用:\acode{lvalue references} 以及 \acode{rvalue references}。本章讲解前者。

相关内容:Rvalue references 则会在 \acode{move semantics} 章节中讲到。

\subsubsection*{左值引用类型}

\textbf{lvalue} 引用(通常称为 \textbf{引用} 因为 C++11 之前只有一种引用)充当已有对象(例如变量)的别名。

\begin{lstlisting}
int      // 一个普通的 int 类型
int&     // 一个对于 int 对象的左值引用
double&  // 一个对于 double 对象的左值引用
\end{lstlisting}

\subsubsection*{左值引用变量}

使用左值引用类型,用户可以创建一个左值引用变量。
\textbf{左值引用变量}是一种充当左值(通常是别的变量)引用的变量。

创建一个左值引用变量,只需要简单的定义一个带有左值引用类型的变量:

\begin{lstlisting}[language=C++]
#include <iostream>

int main()
{
    int x { 5 };    // x 是一个普通的整数变量
    int& ref { x }; // ref 是一个左值引用变量,现在用作于 x 变量的别名

    std::cout << x << '\n';  // 打印 x (5)
    std::cout << ref << '\n'; // 打印 x 通过 ref (5)

    return 0;
}
\end{lstlisting}

上述例子中,\acode{int&} 类型定义 \acode{ref} 作为 int 的左值引用,接着初始化左值表达式 \acode{x}。
这样 \acode{ref} 和 \acode{x} 可以同义的被使用。

在编译器的角度,无论 \& 符号是放在类型名称上(\acode{int\& ref}还是变量名称上(\acode{int \&ref}都没有差别,
选择何种写法与风格有关。
现代 C++ 程序员更加偏向前者,即 \& 与类型绑定,
因为这样可以更清楚知道引用是类型部分的信息,而不是标识符的信息。

最佳实践:定义引用时将 \& 符号放在类型边上(而不是引用变量名字)。

进阶:对于已经知道指针的用户而言,\& 符号在这个上下文中不代表“什么的地址”,而是“什么的左值引用”。

\subsubsection*{通过左值引用修改值}

\begin{lstlisting}[language=C++]
#include <iostream>

int main()
{
    int x { 5 }; // 普通整数变量
    int& ref { x }; // ref 是左值引用变量,现在用作于于变量 x 的别名

    std::cout << x << ref << '\n'; // 打印 55

    x = 6; // x 现在为 6

    std::cout << x << ref << '\n'; // 打印 66

    ref = 7; // 被引用的对象 (x) 现在为 7

    std::cout << x << ref << '\n'; // 打印 77

    return 0;
}
\end{lstlisting}

上述例子中,\acode{ref} 为 \acode{x} 的别名,因此可以通过 \acode{x} 或 \acode{ref} 修改 \acode{x} 的值。

\subsubsection*{左值引用的初始化}

与常量相同,所有引用都必须被初始化。

当一个引用被初始化了一个对象(或函数),可以说其\textbf{绑定}了该对象(或函数)。
这个绑定的过程被称为\textbf{引用绑定}。被引用的对象(或函数)有时被称为\textbf{指示物} referent。

左值引用必须绑定至一个\textit{可变}左值。

\begin{lstlisting}[language=C++]
int main()
{
    int x { 5 };
    int& ref { x }; // 可行:左值引用绑定至一个可变的左值

    const int y { 5 };
    int& invalidRef { y };  // 不可行: int 引用不可被绑定至 double 变量
    int& invalidRef2 { 0 }; // 不可行: double 引用不可被绑定至 int 变量

    return 0;
}
\end{lstlisting}

\acode{void} 的左值引用是不被允许的(它指向哪里?)。

\subsubsection*{引用不可以被再复位 reseated(改变引用的对象)}

C++ 中的引用一旦初始化后不可以\textbf{再复位},意为不可以改变其引用的对象。

新手 C++ 程序员经常通过赋值让引用指向另一个变量从而 reseat 引用。
这样可以被编译和运行 -- 只不过跟预期不一样。考虑以下程序:

\begin{lstlisting}[language=C++]
#include <iostream>

int main()
{
    int x { 5 };
    int y { 6 };

    int& ref { x }; // ref 现在为 x 的别名

    ref = y; // 赋值 6(y 的值)至 x(被 ref 引用)
    // 上述代码并不会改变 ref 的引用至变量 y !!!

    std::cout << x << '\n'; // 用户预期这里打印 5,实际上打印 6

    return 0;
}
\end{lstlisting}

当一个引用在表达式中计算,它处理的是其引用的对象。所以 \acode{ref = y} 不会改变 \acode{ref} 引用到 \acode{y}。
相反,因为 \acode{ref} 是 \acode{x} 的别名,表达式则视为 \acode{x = y} 的计算,
因为 \acode{y} 表示的是值 \acode{6},\acode{x} 则被赋值了 \acode{6}。

\subsubsection*{左值引用的作用域与持续性}

与普通变量规则相同:

\begin{lstlisting}[language=C++]
#include <iostream>

int main()
{
    int x { 5 }; // 普通整数
    int& ref { x }; // 变量的引用

     return 0;
} // x 和 ref 在此消亡
\end{lstlisting}

\subsubsection*{引用与被引用者拥有独立的生命周期}

除了一种情况(下一章详解)以外,引用的生命周期与其被引用的生命周期是独立的。
换言之,下面两种情况都是对的:

\begin{itemize}
    \item 在对象被引用前,引用可以被销毁。
    \item 被引用的对象可以在被引用时被销毁。
\end{itemize}

\subsubsection*{悬垂引用}

当一个被引用的对象在被引用之前销毁了,那么引用指向的对象不再存在,这种引用被成为\textbf{悬垂引用}。
访问悬垂引用会导致未被定义的行为。

悬垂引用很容易避免,不过之后的章节中会讲到在实际代码中何时容易出现。

\subsubsection*{引用非对象}

在 C++ 中引用不是对象。一个引用不需要存在或者占用存储。
如果可能的话,编译器将会通过替换被引用者至引用者来优化所有的引用。
然而这也不总是可能的,因为如果可能,引用则会需要存储。

这也意为着“引用变量”这个词并不准确,因为变量都是带有名称的对象,而引用并不是对象。

\end{document}
