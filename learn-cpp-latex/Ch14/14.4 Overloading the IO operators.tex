\documentclass[../../LearnCpp.tex]{subfiles}

\begin{document}

\asubsection{4}{Overloading the IO operators}

对于拥有若干成员变量的类而言,打印每个独立的变量在屏幕上令人厌倦。例如考虑以下类:

\begin{lstlisting}[language=C++]
class Point
{
private:
    double m_x{};
    double m_y{};
    double m_z{};

public:
    Point(double x=0.0, double y=0.0, double z=0.0)
      : m_x{x}, m_y{y}, m_z{z}
    {
    }

    double getX() const { return m_x; }
    double getY() const { return m_y; }
    double getZ() const { return m_z; }
};
\end{lstlisting}

如果希望打印这个类的实力在屏幕上,需要如下:

\begin{lstlisting}[language=C++]
Point point{5.0, 6.0, 7.0};

std::cout << "Point(" << point.getX() << ", " <<
    point.getY() << ", " <<
    point.getZ() << ')';
\end{lstlisting}

当然使用一个可复用的函数更有意义:

\begin{lstlisting}[language=C++]
class Point
{
private:
    double m_x{};
    double m_y{};
    double m_z{};

public:
    Point(double x=0.0, double y=0.0, double z=0.0)
      : m_x{x}, m_y{y}, m_z{z}
    {
    }

    double getX() const { return m_x; }
    double getY() const { return m_y; }
    double getZ() const { return m_z; }

    void print() const
    {
        std::cout << "Point(" << m_x << ", " << m_y << ", " << m_z << ')';
    }
};
\end{lstlisting}

这样更好一点,但是仍然有缺陷。因为 \acode{print()} 返回 \acode{void},它不能在输出声明中调用,因此需要这么做:

\begin{lstlisting}[language=C++]
int main()
{
    const Point point{5.0, 6.0, 7.0};

    std::cout << "My point is: ";
    point.print();
    std::cout << " in Cartesian space.\n";
}
\end{lstlisting}

那如果是以下这样会更简单:

\begin{lstlisting}[language=C++]
Point point{5.0, 6.0, 7.0};
cout << "My point is: " << point << " in Cartesian space.\n";
\end{lstlisting}

返回同样的结果,且不会破坏若干声明的输出,也不需要记住打印函数。

幸运的是,重载 << 操作符可以达到这样的效果!

\subsubsection*{重载操作符<<}

重载操作符<< 类似于重载操作符+(它们皆为二元操作符),只不过参数类型不同。

考虑这样一个表达式 \acode{std::cout << point}。如果操作符为 <<,那么运算对象呢?
左侧的运算对象是 std::cout 对象,而右侧的运算对象则是 Point 类对象。
std::cout 实际上是一个类型为 std::ostream 的对象。因此重载函数应该像这样:

\begin{lstlisting}[language=C++]
// std::ostream is the type for object std::cout
friend std::ostream& operator<< (std::ostream& out, const Point& point);
\end{lstlisting}

为 Point 类实现操作符<< 非常的直接 -- 因为 C++ 已经知道了如何使用操作符<< 输出 doubles,
且所有成员也都是 doubles,那么可以简单的使用操作符<< 来输出 Point 的成员变量。

\begin{lstlisting}[language=C++]
#include <iostream>

class Point
{
private:
    double m_x{};
    double m_y{};
    double m_z{};

public:
    Point(double x=0.0, double y=0.0, double z=0.0)
      : m_x{x}, m_y{y}, m_z{z}
    {
    }

    friend std::ostream& operator<< (std::ostream& out, const Point& point);
};

std::ostream& operator<< (std::ostream& out, const Point& point)
{
    // 因为 operator<<  是 Point 类的友元,可以直接访问 Point 的成员。
    out << "Point(" << point.m_x << ", " << point.m_y << ", " << point.m_z << ')'; // 真实输出在这里完成

    return out; // 返回 std::ostream 使得可以串联调用 operator<<
}

int main()
{
    const Point point1{2.0, 3.0, 4.0};

    std::cout << point1 << '\n';

    return 0;
}
\end{lstlisting}

打印:

\begin{lstlisting}
Point(2, 3.5, 4) Point(6, 7.5, 8)
\end{lstlisting}

\subsubsection*{重载操作符>>}

同样的重载输入操作符也是可以的。其实现与输出操作符相似,
重点在于了解 \acode{std::cin} 是 \acode{std::istream} 类型的对象。

\begin{lstlisting}[language=C++]
#include <iostream>

class Point
{
private:
    double m_x{};
    double m_y{};
    double m_z{};

public:
    Point(double x=0.0, double y=0.0, double z=0.0)
      : m_x{x}, m_y{y}, m_z{z}
    {
    }

    friend std::ostream& operator<< (std::ostream& out, const Point& point);
    friend std::istream& operator>> (std::istream& in, Point& point);
};

std::ostream& operator<< (std::ostream& out, const Point& point)
{
    // Since operator<< is a friend of the Point class, we can access Point's members directly.
    out << "Point(" << point.m_x << ", " << point.m_y << ", " << point.m_z << ')';

    return out;
}

std::istream& operator>> (std::istream& in, Point& point)
{
    // 因为 operator<<  是 Point 类的友元,可以直接访问 Point 的成员。
    // 注意参数 point 必须是非 const 的,因此可以使用输入值修改其类成员
    in >> point.m_x;
    in >> point.m_y;
    in >> point.m_z;

    return in;
}

int main()
{
    std::cout << "Enter a point: ";

    Point point;
    std::cin >> point;

    std::cout << "You entered: " << point << '\n';

    return 0;
}
\end{lstlisting}

假设用户输入了 \acode{3.0 4.5 7.26},则打印:

\begin{lstlisting}
You entered: Point(3, 4.5, 7.26)
\end{lstlisting}

\end{document}
