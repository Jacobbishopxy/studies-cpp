\documentclass[../../LearnCpp.tex]{subfiles}

\begin{document}

\asubsection{3}{std::vector capacity and stack behavior}

之前的讨论过 std::vector 可以作为一个兼具记住长度以及有需要是动态调整大小的数组。
尽管这些是 std::vector 最有用最通用的地方,然而它还有一些额外的属性与能力可以用作他出。

\begin{lstlisting}[language=C++]
int* array{ new int[10] { 1, 2, 3, 4, 5 } };
\end{lstlisting}

上述代码表示数组长度为 10,即使只使用了 5 个元素进行分配。
然而如果希望遍历这些初始化后的元素,并保留未使用的位置呢?
这种情况下需要分开追踪到底有多少元素从被分配的元素中被“使用”了。
有别于内建数组或者 std::array,仅记住长度,std::vector 包含两种分开的属性:长度与容量。
在 std::vector 的上下文中,\textbf{长度} length是数组中有多少元素被使用了,
而\textbf{容量} capacity是有多少元素被分配进了内存。

\begin{lstlisting}[language=C++]
#include <vector>
#include <iostream>

int main()
{
    std::vector array { 0, 1, 2 };
    array.resize(5); // 设定长度为 5

    std::cout << "The length is: " << array.size() << '\n';

    for (auto element: array)
        std::cout << element << ' ';

    return 0;
};
\end{lstlisting}

打印:

\begin{lstlisting}
The length is: 5
0 1 2 0 0
\end{lstlisting}

上述代码使用 \acode{resize()} 函数设定向量长度为 5,也就是告诉数组只使用数组的前五位元素。
然而有趣的问题是:该数组的容量是多少?

可以通过 \acode{capacity()} 函数进行查询:

\begin{lstlisting}[language=C++]
#include <vector>
#include <iostream>

int main()
{
    std::vector array { 0, 1, 2 };
    array.resize(5); // set length to 5

    std::cout << "The length is: " << array.size() << '\n';
    std::cout << "The capacity is: " << array.capacity() << '\n';
}
\end{lstlisting}

打印:

\begin{lstlisting}
The length is: 5
The capacity is: 5
\end{lstlisting}

为什么要区分长度和容量呢?std::vector 在需要的时候会释放其内存,不过它倾向于这么做,因为数组的重设大小从计算上而言是昂贵的。考虑以下代码:

\begin{lstlisting}[language=C++]
#include <vector>
#include <iostream>

int main()
{
  std::vector array{};
  array = { 0, 1, 2, 3, 4 }; // okay, array length = 5
  std::cout << "length: " << array.size() << "  capacity: " << array.capacity() << '\n';

  array = { 9, 8, 7 }; // okay, array length is now 3!
  std::cout << "length: " << array.size() << "  capacity: " << array.capacity() << '\n';

  return 0;
}
\end{lstlisting}

打印:

\begin{lstlisting}
length: 5  capacity: 5
length: 3  capacity: 5
\end{lstlisting}

注意尽管赋值了更新的数组给向量,向量并没有释放内存(容量仍然是 5)。修改长度很容易,因为当前只有 3 个元素可用。

数组下标和 \acode{at()} 都是基于长度而言的,而不是容量。

\subsubsection*{std::vector 栈的行为}

如果下标操作符和 \acode{at()} 函数是基于数组长度的,同时容量也总是大于长度的,为什么还需要担心容量呢?尽管 std::vector 可以被用作为动态数组,也可以用作为栈。只需要使用 3 个函数匹配栈的操作:

- \acode{push\_back()} 推入元素进栈
- \acode{back()} 返回栈的顶层元素
- \acode{pop\_back()} 移除栈的顶层元素

\begin{lstlisting}[language=C++]
#include <iostream>
#include <vector>

void printStack(const std::vector<int>& stack)
{
  for (auto element : stack)
    std::cout << element << ' ';
  std::cout << "(cap " << stack.capacity() << " length " << stack.size() << ")\n";
}

int main()
{
  std::vector<int> stack{};

  printStack(stack);

  stack.push_back(5); // push_back() 推入元素
  printStack(stack);

  stack.push_back(3);
  printStack(stack);

  stack.push_back(2);
  printStack(stack);

  std::cout << "top: " << stack.back() << '\n'; // back() 返回最后一个元素

  stack.pop_back(); // pop_back() 移除元素
  printStack(stack);

  stack.pop_back();
  printStack(stack);

  stack.pop_back();
  printStack(stack);

  return 0;
}
\end{lstlisting}

打印:

\begin{lstlisting}
(cap 0 length 0)
5 (cap 1 length 1)
5 3 (cap 2 length 2)
5 3 2 (cap 3 length 3)
top: 2
5 3 (cap 3 length 2)
5 (cap 3 length 1)
(cap 3 length 0)
\end{lstlisting}

因为向量的重设大小非常的昂贵,可以提前使用 \acode{reserve()} 函数告诉向量分配特定大小的空间:

\begin{lstlisting}[language=C++]
#include <vector>
#include <iostream>

void printStack(const std::vector<int>& stack)
{
  for (auto element : stack)
    std::cout << element << ' ';
  std::cout << "(cap " << stack.capacity() << " length " << stack.size() << ")\n";
}

int main()
{
  std::vector<int> stack{};

  stack.reserve(5); // 设定容量(至少)为 5

  printStack(stack);

  stack.push_back(5);
  printStack(stack);

  stack.push_back(3);
  printStack(stack);

  stack.push_back(2);
  printStack(stack);

  std::cout << "top: " << stack.back() << '\n';

  stack.pop_back();
  printStack(stack);

  stack.pop_back();
  printStack(stack);

  stack.pop_back();
  printStack(stack);

  return 0;
}
\end{lstlisting}

打印:

\begin{lstlisting}
(cap 5 length 0)
5 (cap 5 length 1)
5 3 (cap 5 length 2)
5 3 2 (cap 5 length 3)
top: 2
5 3 (cap 5 length 2)
5 (cap 5 length 1)
(cap 5 length 0)
\end{lstlisting}

\subsubsection*{向量可能分配额外容量}

当向量重设大小时,有可能会分配更多的容量。这是为了让额外的元素有“喘息空间”,来减少重设空间操作的次数:

\begin{lstlisting}[language=C++]
#include <vector>
#include <iostream>

int main()
{
  std::vector v{ 0, 1, 2, 3, 4 };
  std::cout << "size: " << v.size() << "  cap: " << v.capacity() << '\n';

  v.push_back(5); // 增加另一元素
  std::cout << "size: " << v.size() << "  cap: " << v.capacity() << '\n';

  return 0;
}
\end{lstlisting}

有可能打印:

\begin{lstlisting}[language=C++]
size: 5  cap: 5
size: 6  cap: 7
\end{lstlisting}

如何分配额外容量是基于编译器的实现所决定的。

\end{document}
