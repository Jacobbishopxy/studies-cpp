\documentclass[../../LearnCpp.tex]{subfiles}

\begin{document}

\asubsection{2}{The stack and the heap}

程序使用的内存通常被分隔成不同的区域,被称为段 segments:

\begin{itemize}
  \item 代码段 code segment(也被称为文本段 text segment),是编译好的项目的存储。代码段通常只能可读。
  \item bss 段(也被称为非初始化数据段),是零初始化全局以及静态变量的存储。
  \item 数据段(也被称为初始化数据段),是初始化全局以及静态变量的存储。
  \item 堆,是动态分配变量的存储。
  \item 调用栈,是函数参数,本地变量,以及其它函数相关信息的存储。
\end{itemize}

\subsubsection*{堆 segment}

堆 segment(也被熟知为“free store”)追踪动态内存分配的内存使用。

在 C++ 中,当使用 new 操作符分配内存,内存会被分配在应用程序的堆 segment 中。

\begin{lstlisting}[language=C++]
int* ptr { new int };       // ptr 在堆中被指定了 4 字节
int* array { new int[10] }; // array 在堆中被指定了 40 字节
\end{lstlisting}

当一个动态分配变量被删除时,内存被“返回”给堆,接着可以被重新分配给未来的内存请求。
需要留意的是删除一个指针并没有删除变量,只是仅仅将关联的内存还给操作系统。

堆有好处也有坏处:

\begin{itemize}
  \item 堆上分配的内存相对来说比较慢。
  \item 被分配到的内存一直占用直到被指定释放内存(小心内存泄漏),或是程序终止(操作系统进行清理)。
  \item 动态分配内存必须通过指针来进行访问。解引用指针相比于直接访问变量会更慢一些的。
  \item 因为对是一个很大的内存池,大数组,结构体,或者类可以被分配到这里。
\end{itemize}

\subsubsection*{调用栈}

\textbf{调用栈} call stack(通常指代“the stack”)扮演更加有趣的角色。
调用栈在程序启动时就持续追踪所有被激活的函数(那些已经被调用但是还没有结束),
并且处理所有函数参数以及本地变量的分配。

调用栈被实现为栈数据结构。因此在谈到它们是如何工作之前,必须要理解清楚什么是一个栈数据结构。

\subsubsection*{栈数据结构}

\textbf{数据结构}是程序用来管理数据的机制,使得数据可以被有效的利用。
我们已经学习了若干数据结构,例如数组和结构体。
它们都提供了便利的存储数据以及访问数据的机制。
还有很多别的数据结构在程序中普遍使用,它们之间相当多的都已经被标准库实现了,栈就是其中之一。

在计算机程序中,栈是一种可以存储若干变量(如同数组)的容器数据类型。
然而数组允许用户无需顺序的方式(\textbf{随机访问} random access)访问以及修改元素,栈却有其限制。
栈有下列三种操作方式:

\begin{enumerate}
  \item 查看栈顶层元素(通常由函数 \acode{top()} 进行调用,有时称为 \acode{peek()})
  \item 移除栈顶层元素(通过 \acode{pop()} 函数)
  \item 放置新元素在栈顶层(通过 \acode{push()} 函数)
\end{enumerate}

栈是后进 last-in 先出 first-out(LIFO)结构。

\subsubsection*{调用栈 segment}

调用栈 segment 是用于存储调用栈的内存。
在应用启动时,main() 函数由操作系统推入到调用栈中,接着程序继续执行。

当遇到函数调用时,函数推入调用栈。
当当前函数结束时,函数从调用栈上移除(这个过程有时被称为 \textbf{unwinding the stack})。
因此,观察当前调用栈的函数,可以看到所用被调用的函数来获取当前真正执行的点位。

\subsubsection*{栈溢出}

栈的大小是有限的,因此只能存储有限的信息。
如果程序尝试放入超过栈所能承受的信息时,\textbf{栈溢出} stack overflow 则会发生。
即所有的栈都被分配了 -- 后续的分配会溢出到别的部分的内存中。

\end{document}
