\documentclass[../../LearnCpp.tex]{subfiles}

\begin{document}

\asubsection{7}{Introduction to lambdas (anonymous functions)}

\begin{lstlisting}[language=C++]
#include <algorithm>
#include <array>
#include <iostream>
#include <string_view>

// 如果元素匹配,函数返回 true
bool containsNut(std::string_view str)
{
    // 如果找不到子字符串,std::string_view::find 返回 std::string_view::npos,
    // 否则返回子字符串的索引。
    return (str.find("nut") != std::string_view::npos);
}

int main()
{
    std::array<std::string_view, 4> arr{ "apple", "banana", "walnut", "lemon" };

    // 检索 array 查看元素是否包含”nut“子字符串
    auto found{ std::find_if(arr.begin(), arr.end(), containsNut) };

    if (found == arr.end())
    {
        std::cout << "No nuts\n";
    }
    else
    {
        std::cout << "Found " << *found << '\n';
    }

    return 0;
}
\end{lstlisting}

这里根本的问题是 \acode{std::find_if} 需要用户传递一个函数指针。
正因如此,需要强制定义一个只会使用一次的函数,且必须要有函数名,
且必须放入全局作用域中(因为函数不可以嵌套!)。

\subsubsection*{Lambdas 来拯救}

一个**lambda 表达式**(也称**lambda**或者**闭包 closure**)允许用户在函数中定义匿名函数。
嵌套是很重要的,因为它允许用户既可以避免命名空间的污染,还可以定义与使用位置更近的函数(提供额外的内容)。

C++ 中 lambdas 的语法是一个奇怪的东西,需要花一点时间习惯它:

\begin{lstlisting}
[ captureClause ] ( parameters ) -> returnType
{
    statements;
}
\end{lstlisting}

- 如果不需要捕获变量,捕获从句可以为空。
- 如果不需要参数,参数列表可以为空也可以省略。
- 返回类型是可选地,且如果省略了,则视为 \acode{auto}(这样会使用类型推导来决定返回类型)。

那么使用 lambda 改写后的代码如下:

\begin{lstlisting}[language=C++]
#include <algorithm>
#include <array>
#include <iostream>
#include <string_view>

int main()
{
  constexpr std::array<std::string_view, 4> arr{ "apple", "banana", "walnut", "lemon" };

  const auto found{ std::find_if(arr.begin(), arr.end(),
                           [](std::string_view str) // 此为 lambda,无捕获从句
                           {
                             return (str.find("nut") != std::string_view::npos);
                           }) };

  if (found == arr.end())
  {
    std::cout << "No nuts\n";
  }
  else
  {
    std::cout << "Found " << *found << '\n';
  }

  return 0;
}
\end{lstlisting}

\subsubsection*{lambda 的类型}

上述例子中,定义了一个 lambda。那样的使用方式有时被称为**函数字面量 function literal**。

然而,在其使用的地方编写的 lambda,有时会让代码变得难以阅读。命名的 lambda 可以让代码便于阅读。

例如下面的代码切片中使用 \acode{std::all_of} 来检查所有元素是否为偶数:

\begin{lstlisting}[language=C++]
// 坏:需要阅读 lambda 才能理解发生了什么
return std::all_of(array.begin(), array.end(), [](int i){ return ((i % 2) == 0); });
\end{lstlisting}

这样可以提高可读性:

\begin{lstlisting}[language=C++]
// 好:可以存储 lambda 为一个命名变量,并传递其至函数
auto isEven{
  [](int i)
  {
    return ((i % 2) == 0);
  }
};

return std::all_of(array.begin(), array.end(), isEven);
\end{lstlisting}

进阶:实际上,lambdas 并不是函数(这也是为什么它们可以避开 C++ 不支持嵌套函数的限制)。
它们是特殊的名为函子 functor 的对象。
函子是一种包含了重载 \acode{operator()} 的对象,使得它们可以像函数那样被调用。

有几种存储 lambda 的方式。如果 lambda 有空的捕获从句(即 [] 里没有任何东西),
可以使用通常的函数指针。\acode{std::function} 或者通过 \acode{auto} 关键字的类型推导同样也可用(即使 lambda 拥有捕获从句)。

\begin{lstlisting}[language=C++]
#include <functional>

int main()
{
  // 一个普通函数指针。仅作用在空捕获从句
  double (*addNumbers1)(double, double){
    [](double a, double b) {
      return (a + b);
    }
  };

  addNumbers1(1, 2);

  // 使用 std::function。lambda 可以拥有非空的捕获从句
  std::function addNumbers2{ // 注意:在 C++17 之前,使用 std::function<double(double, double)>
    [](double a, double b) {
      return (a + b);
    }
  };

  addNumbers2(3, 4);

  // 使用 auto。存储带有其真实类型的 lambda
  auto addNumbers3{
    [](double a, double b) {
      return (a + b);
    }
  };

  addNumbers3(5, 6);

  return 0;
}
\end{lstlisting}

仅有的使用 lambda 真实类型的方法是 \acode{auto}。\acode{auto} 拥有着不需要比较 \acode{std::function} 的便利。

不幸的是,在 C++20 之前,用户不能总是使用 \acode{auto}。以防真实的 lambda 是未知的(例如因为传递 lambda 给一个函数作为入参,且由调用者决定传递什么样的 lambda),不可以在无比较的情况下使用 \acode{auto}。这种情况下 \acode{std::function} 反而可以使用。

\begin{lstlisting}[language=C++]
#include <functional>
#include <iostream>

// 不知道 fn 会是什么。std::function 可用作于通常函数以及 lambdas。
void repeat(int repetitions, const std::function<void(int)>& fn)
{
  for (int i{ 0 }; i < repetitions; ++i)
  {
    fn(i);
  }
}

int main()
{
  repeat(3, [](int i) {
    std::cout << i << '\n';
  });

  return 0;
}
\end{lstlisting}

如果给 \acode{fn} 使用 \acode{auto},函数的调用者不会知道 \acode{fn} 的入参以及返回类型。当缩写函数模版在 C++20 里添加了,这个限制才被打破。

另外,因为它们本质是模板,拥有 \acode{auto} 参数的函数不再可以被分隔进头文件与源文件。

规则:当初始化 lambda 变量时,使用 \acode{auto};而不可通过 lambda 进行初始化的时候,使用 \acode{std::function} 。

\subsubsection*{泛型 lambdas}

大多数情况下,lambda 参数与普通函数参数无异。

一个显著的区别是自从 C++14 之后可以使用 \acode{auto} 的参数(注意:C++20,普通函数也可以使用 \acode{auto} 参数了)。当一个 lambda 拥有一个或以上的 \acode{auto} 参数,编译器将从调用 lambda 处推断参数所需要的类型。

因为带有一个或以上 \acode{auto} 的 lambdas 可以用作于广泛的类型,它们通常被称为**泛型 lambdas**。

进阶:在 lambda 上下文中使用时,\acode{auto} 是模版参数的缩写。

\begin{lstlisting}[language=C++]
#include <algorithm>
#include <array>
#include <iostream>
#include <string_view>

int main()
{
  constexpr std::array months{ // C++17 之前使用 std::array<const char*, 12>
    "January",
    "February",
    "March",
    "April",
    "May",
    "June",
    "July",
    "August",
    "September",
    "October",
    "November",
    "December"
  };

  // 查询两个连续的拥有相同开头字母的月份
  const auto sameLetter{ std::adjacent_find(months.begin(), months.end(),
                                      [](const auto& a, const auto& b) {
                                        return (a[0] == b[0]);
                                      }) };

  // 确保两个月份被找到
  if (sameLetter != months.end())
  {
    // std::next 返回 sameLetter 的下一个遍历器
    std::cout << *sameLetter << " and " << *std::next(sameLetter)
              << " start with the same letter\n";
  }

  return 0;
}
\end{lstlisting}

上述例子中使用 \acode{auto} 参数来记录字符串的 \acode{const} 引用。因为所有字符串类型允许通过 \acode{operator[]} 访问它们的独立字符,所以不需要关心使用者是否传达的是一个 \acode{std::string},C-style 字符串,或者是其它。这允许用户编写一个可以接受上述任何类型的 lambda,意味着如果之后替换了 \acode{months} 的类型,不需要再重写 lambda。

然而,\acode{auto} 并不总是最好的选择:

\begin{lstlisting}[language=C++]
#include <algorithm>
#include <array>
#include <iostream>
#include <string_view>

int main()
{
  constexpr std::array months{ // C++17 之前使用 std::array<const char*, 12>
    "January",
    "February",
    "March",
    "April",
    "May",
    "June",
    "July",
    "August",
    "September",
    "October",
    "November",
    "December"
  };

  // 统计多少个月份是 5 个字母组成的
  const auto fiveLetterMonths{ std::count_if(months.begin(), months.end(),
                                       [](std::string_view str) {
                                         return (str.length() == 5);
                                       }) };

  std::cout << "There are " << fiveLetterMonths << " months with 5 letters\n";

  return 0;
}
\end{lstlisting}

这个例子中,使用 \acode{auto} 将推断一个 \acode{const char*} 的类型。
C-style 字符串并不容易处理(除了使用 \acode{operator[]})。
这种情况下,推荐显示定义参数为 \acode{std::string\_view},
这样允许用户更方便的操作(例如查询 string\_view 其长度,
即使用户传递的是 C-style 数组)。

\subsubsection*{泛型 lambdas 与静态变量}

有一件值得注意的是唯一的 lambda 会因为 \acode{auto} 的解释,被生成不同类型:

\begin{lstlisting}[language=C++]
#include <algorithm>
#include <array>
#include <iostream>
#include <string_view>

int main()
{
  // 打印值并统计 @print 被调用了多少次
  auto print{
    [](auto value) {
      static int callCount{ 0 };
      std::cout << callCount++ << ": " << value << '\n';
    }
  };

  print("hello"); // 0: hello
  print("world"); // 1: world

  print(1); // 0: 1
  print(2); // 1: 2

  print("ding dong"); // 2: ding dong

  return 0;
}
\end{lstlisting}

\subsubsection*{返回类型推导以及尾随返回类型}

如果使用了类型推导,一个 lambda 的返回类型会由自身内部的 \acode{return} 声明推导出来,lambda 中的所有返回声明必须返回同样类型(否则编译器无法知道使用哪个)。

\begin{lstlisting}[language=C++]
#include <iostream>

int main()
{
  auto divide{ [](int x, int y, bool intDivision) { // 注意:没有值得返回类型
    if (intDivision)
      return x / y; // 返回类型为 int
    else
      return static_cast<double>(x) / y; // 错误:返回类型与前一个返回类型不同
  } };

  std::cout << divide(3, 2, true) << '\n';
  std::cout << divide(3, 2, false) << '\n';

  return 0;
}
\end{lstlisting}

这会产生编译错误,因为第一个返回声明(int)的返回类型鱼第二个返回声明(double)的返回类型不匹配。

针对返回不同类型的问题有两种方法:

1. 显式的强制转换类型使得所有返回类型匹配
1. 显式指定 lambda 的返回类型,并让编译器进行隐式转换

第二个方法通常来说更好一些:

\begin{lstlisting}[language=C++]
#include <iostream>

int main()
{
  // 注意:显式指定这里的返回为 double
  auto divide{ [](int x, int y, bool intDivision) -> double {
    if (intDivision)
      return x / y; // 将会进行隐式转换,将结果转换成 double
    else
      return static_cast<double>(x) / y;
  } };

  std::cout << divide(3, 2, true) << '\n';
  std::cout << divide(3, 2, false) << '\n';

  return 0;
}
\end{lstlisting}

\subsubsection*{标准库函数对象}

对于常用的操作(例如加法,否定,或者比较),用户不需要编写 lambdas,因为标准库已经有很多可调用对象,它们都定义于 \acode{<functional>} 头文件。

\begin{lstlisting}[language=C++]
#include <algorithm>
#include <array>
#include <iostream>
#include <functional> // std::greater

int main()
{
  std::array arr{ 13, 90, 99, 5, 40, 80 };

  // 传递 std::greater 给 std::sort
  std::sort(arr.begin(), arr.end(), std::greater{}); // 注意:需要花括号进行实例化

  for (int i : arr)
  {
    std::cout << i << ' ';
  }

  std::cout << '\n';

  return 0;
}
\end{lstlisting}

\end{document}
