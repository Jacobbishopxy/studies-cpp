\documentclass[../../LearnCpp.tex]{subfiles}

\begin{document}

\asubsection{4}{Recursion}

C++ 中的\textbf{递归函数} recursive function 是一种调用自身的函数。

\subsubsection*{递归终结条件}

\textbf{递归终结} recursive termination 是当条件达满足时,控制函数停止调用自己。

\begin{lstlisting}[language=C++]
#include <iostream>

void countDown(int count)
{
    std::cout << "push " << count << '\n';

    if (count > 1) // 终结条件
        countDown(count-1);

    std::cout << "pop " << count << '\n';
}

int main()
{
    countDown(5);
    return 0;
}
\end{lstlisting}

\subsubsection*{记忆化算法}

\textbf{记忆化} memoization 缓存所有昂贵函数调用,在当同样输入再次发生时,其结果可以直接被返回。

\begin{lstlisting}[language=C++]
#include <iostream>
#include <vector>

int fibonacci(int count)
{
  // 使用 static std::vector 缓存计算结果
  static std::vector results{ 0, 1 };

  // 如果该结果已被计算过,则直接使用缓存中的结果
  if (count < static_cast<int>(std::size(results)))
    return results[count];

  // 否则结算新值并添加至缓存
  results.push_back(fibonacci(count - 1) + fibonacci(count - 2));
  return results[count];
}

int main()
{
  for (int count { 0 }; count < 13; ++count)
    std::cout << fibonacci(count) << ' ';

  return 0;
}
\end{lstlisting}

\end{document}
