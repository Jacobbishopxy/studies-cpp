\documentclass[../../LearnCpp.tex]{subfiles}

\begin{document}

\asubsection{1}{Function Pointers}

创建一个非 const 函数指针的语法:

\begin{lstlisting}[language=C++]
// fcnPtr 是一个无入参并返回整数的函数指针
int (*fcnPtr)();
\end{lstlisting}

fcnPtr 可以指向任何匹配其类型的函数。

*fcnPtr 的圆括号是必须的,因为 \acode{int* fcnPtr()}
会被解释为一个无入参并返回整数且名为 fcnPtr 函数的前向声明。

如果是一个 const 函数指针,只需要将 const 放在星号后:

\begin{lstlisting}[language=C++]
int (*const fcnPtr)();
\end{lstlisting}

如果把 const 放在 int 之前,则会解释为指向的函数其返回值为 const int 类型。

\subsubsection*{赋值函数指针}

函数指针可以通过函数进行初始化(同时非 const 函数指针还可以被赋值函数)。

\begin{lstlisting}[language=C++]
int foo()
{
    return 5;
}

int goo()
{
    return 6;
}

int main()
{
    int (*fcnPtr)(){ &foo };  // fcnPtr 指向函数 foo
    fcnPtr = &goo;            // fcnPtr 现在指向了函数 goo

    return 0;
}
\end{lstlisting}

注意函数指针的类型(入参与返回类型)必须匹配函数的类型:

\begin{lstlisting}[language=C++]
// 函数原型
int foo();
double goo();
int hoo(int x);

// 函数指针赋值
int (*fcnPtr1)(){ &foo };     // 可以
int (*fcnPtr2)(){ &goo };     // 错误 -- 返回类型不匹配!
double (*fcnPtr4)(){ &goo };  // 可以
fcnPtr1 = &hoo;               // 错误 -- fcnPtr1 没有入参,但是 hoo() 有入参
int (*fcnPtr3)(int){ &hoo };  // 可以
\end{lstlisting}

不同于基础类型,如果需要 C++ \textit{将}隐式转换一个函数成为一个函数指针(因此用户不需要使用 address-of 操作符(\&)来获取函数地址)。
然而,并不会隐式转换函数指针成为 void 指针,或者反过来。

函数指针可以被初始化或赋值 nullptr:

\begin{lstlisting}[language=C++]
int (*fcnptr)() { nullptr };  // 可以
\end{lstlisting}

\subsubsection*{使用函数指针调用函数}

首先是通过显式解引用:

\begin{lstlisting}[language=C++]
int foo(int x)
{
    return x;
}

int main()
{
    int (*fcnPtr)(int){ &foo }; // 使用函数 foo 初始化 fcnPtr
    (*fcnPtr)(5); // 通过 fcnPtr 调用函数 foo(5)

    return 0;
}
\end{lstlisting}

第二种方法是通过隐式解引用:

\begin{lstlisting}[language=C++]
int foo(int x)
{
    return x;
}

int main()
{
    int (*fcnPtr)(int){ &foo }; // 使用函数 foo 初始化 fcnPtr
    fcnPtr(5); // 通过 fcnPtr 调用函数 foo(5)

    return 0;
}
\end{lstlisting}

可以看到隐式解引用的方法与普通函数调用一致,因为普通函数名称就是函数的指针!

有一个有趣的地方值得注意:默认参数并不会通过函数指针调用而工作。
默认参数在编译时决定(也就是说在不提供参数时,编译器在编译时替换为默认参数),而函数指针作用于运行时。
因此在调用函数指针时,默认参数不能被决定。
这种情况下,用户必须显式传递任何默认参数。

另外注意因为函数指针可以被设为 nullptr,那么在调用之前进行断言或者条件测试是一个好主意。

\begin{lstlisting}[language=C++]
int foo(int x)
{
    return x;
}

int main()
{
    int (*fcnPtr)(int){ &foo }; // 使用函数 foo 初始化 fcnPtr
    if (fcnPtr) // 确保 fcnPtr 不是空指针
        fcnPtr(5); // 否则这会导致未定义行为

    return 0;
}
\end{lstlisting}

\subsubsection*{函数作为参数传递至其它函数}

函数指针一个最有用的地方在于它可以作为参数传递进其它函数中。
作为参数的函数有时候会被称为\textbf{回调函数} callback functions。

这里是之前章节中的选择排序:

\begin{lstlisting}[language=C++]
#include <utility> // std::swap

void SelectionSort(int* array, int size)
{
    // 遍历数组中的每一个元素
    for (int startIndex{ 0 }; startIndex < (size - 1); ++startIndex)
    {
        // 最小数值的索引
        int smallestIndex{ startIndex };

        // 查询数组中剩下元素的最小值
        for (int currentIndex{ startIndex + 1 }; currentIndex < size; ++currentIndex)
        {
            // 如果当前元素小于之前找到的元素(由 startIndex + 1 开始)
            if (array[smallestIndex] > array[currentIndex]) // 比较在这里完成
            {
                // 本次遍历中最新的最小值
                smallestIndex = currentIndex;
            }
        }

        // 交换起始元素与最小的元素
        std::swap(array[startIndex], array[smallestIndex]);
    }
}
\end{lstlisting}

现在替换比较函数,比较函数可以是这样:

\begin{lstlisting}[language=C++]
bool ascending(int x, int y)
{
    return x > y; // 如果第一个元素大于第二个元素则进行交换
}
\end{lstlisting}

那么就可以这样替换:

\begin{lstlisting}[language=C++]
#include <utility>

void SelectionSort(int* array, int size)
{
    for (int startIndex{ 0 }; startIndex < (size - 1); ++startIndex)
    {
        int smallestIndex{ startIndex };

        for (int currentIndex{ startIndex + 1 }; currentIndex < size; ++currentIndex)
        {
            if (ascending(array[smallestIndex], array[currentIndex]))
            {
                smallestIndex = currentIndex;
            }
        }

        std::swap(array[startIndex], array[smallestIndex]);
    }
}
\end{lstlisting}

函数指针类似于:

\begin{lstlisting}[language=C++]
bool (*comparisonFcn)(int, int);
\end{lstlisting}

最终完整的代码:

\begin{lstlisting}[language=C++]
#include <utility> // std::swap
#include <iostream>

// 注意用户自定义比较函数位于第三个参数
void selectionSort(int* array, int size, bool (*comparisonFcn)(int, int)) {
  // 遍历数组中的每一个元素
  for (int startIndex{0}; startIndex < (size - 1); ++startIndex) {
    // 最小/大数值的索引
    int bestIndex{startIndex};

    // 查询数组中剩下元素的最小/大值
    for (int currentIndex{startIndex + 1}; currentIndex < size; ++currentIndex) {
      // 如果当前元素小于/大于之前找到的元素(由 startIndex + 1 开始)
      if (comparisonFcn(array[bestIndex], array[currentIndex])) // 比较在这里完成
      {
        // 本次遍历中最新的最小/大值
        bestIndex = currentIndex;
      }
    }

    // 交换起始元素与最小/大的元素
    std::swap(array[startIndex], array[bestIndex]);
  }
}

// 升序的比较函数
bool ascending(int x, int y)
{
    return x > y;
}

// 降序的比较函数
bool descending(int x, int y)
{
    return x < y;
}

// 用于打印元素
void printArray(int* array, int size)
{
    for (int index{ 0 }; index < size; ++index)
    {
        std::cout << array[index] << ' ';
    }

    std::cout << '\n';
}

int main()
{
    int array[9]{ 3, 7, 9, 5, 6, 1, 8, 2, 4 };

    // 根据 descending() 进行降序排序
    selectionSort(array, 9, descending);
    printArray(array, 9);

    // 根据 ascending() 进行升序排序
    selectionSort(array, 9, ascending);
    printArray(array, 9);

    return 0;
}
\end{lstlisting}

\subsubsection*{让函数指针拥有类型别名}

函数指针的语法非常的丑陋,然而可以使用类型别名使其看起来更像是普通变量:

\begin{lstlisting}[language=C++]
using ValidateFunction = bool(*)(int, int);
\end{lstlisting}

那么不需要再像这样的使用:

\begin{lstlisting}[language=C++]
bool validate(int x, int y, bool (*fcnPtr)(int, int)); // 丑陋
\end{lstlisting}

可以这么做:

\begin{lstlisting}[language=C++]
bool validate(int x, int y, ValidateFunction pfcn) // 简洁
\end{lstlisting}

\subsubsection*{使用 std::function}

另一种定义并存储函数指针的方式是使用 \acode{std::function},
其位于 \acode{<functional>} 头文件。

\begin{lstlisting}[language=C++]
#include <functional>

// std::function 方法返回布尔值且接受两个整数参数
bool validate(int x, int y, std::function<bool(int, int)> fcn);
\end{lstlisting}

更新之前的例子:

\begin{lstlisting}[language=C++]
#include <functional>
#include <iostream>

int foo()
{
    return 5;
}

int goo()
{
    return 6;
}

int main()
{
    std::function<int()> fcnPtr{ &foo }; // 声明函数指针
    fcnPtr = &goo; // fcnPtr 现在指向函数 goo
    std::cout << fcnPtr() << '\n'; // 与通常一样调用函数

    return 0;
}
\end{lstlisting}

std::function 的类型别名有利于可读性:

\begin{lstlisting}[language=C++]
using ValidateFunctionRaw = bool(*)(int, int);          // 原始函数指针的别名
using ValidateFunction = std::function<bool(int, int)>; // std::function 的别名
\end{lstlisting}

从 C++17 开始,CTAD 可以由初始化器推导 std::function 的模板参数,
上述例子中可以将 \acode{std::function<int()> fcnPtr{ &foo };}
修改为 \acode{std::function fcnPtr{ &foo };},并让编译器计算出模板类型。
然而 CTAD 不能作用于类型别名,这是因为没有提供初始化器。

\subsubsection*{函数指针的类型接口}

与推断普通变量的类型一样,\acode{auto} 关键字也可以推断函数指针的类型:

\begin{lstlisting}[language=C++]
#include <iostream>

int foo(int x)
{
    return x;
}

int main()
{
    auto fcnPtr{ &foo };
    std::cout << fcnPtr(5) << '\n';

    return 0;
}
\end{lstlisting}

不过缺陷就是,所有关于函数参数以及返回值的细节都被隐藏了,
这样在调用的时候,或者使用其返回值时,很容易造成错误。

\end{document}
