\documentclass[../../LearnCpp.tex]{subfiles}

\begin{document}

\asubsection{8}{Lambda captures}

\textbf{捕获从句} capture clause 用于(间接的)给予 lambda 访问周边作用域的可用变量。

\begin{lstlisting}[language=C++]
#include <algorithm>
#include <array>
#include <iostream>
#include <string_view>
#include <string>

int main()
{
  std::array<std::string_view, 4> arr{ "apple", "banana", "walnut", "lemon" };

  std::cout << "search for: ";

  std::string search{};
  std::cin >> search;

  // 捕获 @search                                   vvvvvvv
  auto found{ std::find_if(arr.begin(), arr.end(), [search](std::string_view str) {
    return (str.find(search) != std::string_view::npos);
  }) };

  if (found == arr.end())
  {
    std::cout << "Not found\n";
  }
  else
  {
    std::cout << "Found " << *found << '\n';
  }

  return 0;
}
\end{lstlisting}

\subsubsection*{捕获是如何工作的?}

上述代码中,lambda 看起来像是直接访问 \acode{main} 中的 \acode{search} 变量,实际上并不是这样。
lambdas 可能看起来像是嵌套代码块,但是实际上有细微的差别(而这细微的差别却很重要)。

当一个 lambda 定义被执行时,对于每个其需要捕获的变量会在 lambda 中进行拷贝。
这些被拷贝的变量皆在外部作用域进行初始化。

因此上述例子中,当 lambda 对象被创建,lambda 获取自身所需变量 \acode{search} 的拷贝。
该 \acode{search} 拥有与 \acode{main} 中的 \acode{search} 相同的值,
所以 lambda 的行为像是在访问 \acode{main} 中的 \acode{search} 而实际上并不是。

而拷贝的变量拥有相同的名称,却不需要拥有与原始变量相同的类型(这在之后章节中会讲到)。

重点:被 lambda 捕获的变量是外部作用域变量的拷贝,而不是真实变量。

进阶:尽管 lambdas 看起来像是函数,它们实际上是可以像函数那样调用的对象
(名为\textbf{函子} functors -- 之后的章节会讲解如何创建用户自定义函子)。
当编译器遇到 lambda 定义,lambda 对象被实例化,lambda 的成员也被初始化。

\subsubsection*{捕获默认为 const 值}

默认情况下,变量是被 \acode{const value} 所捕获。
这就意味着当 lambda 创建时,lambda 捕获外部作用域中变量的常数拷贝,也就是说 lambda 不允许修改变量。

\begin{lstlisting}[language=C++]
#include <iostream>

int main()
{
  int ammo{ 10 };

  // 定义一个 lambda 并由 “shoot” 变量进行存储
  auto shoot{
    [ammo]() {
      // 非法,ammo 作为 const 拷贝被捕获
      --ammo;

      std::cout << "Pew! " << ammo << " shot(s) left.\n";
    }
  };

  // 调用 lambda
  shoot();

  std::cout << ammo << " shot(s) left\n";

  return 0;
}
\end{lstlisting}

\subsubsection*{可变捕获}

\textbf{mutable} 关键字从\textit{所有}被捕获的变量上移除 \acode{const} 限制。

\begin{lstlisting}[language=C++]
#include <iostream>

int main()
{
  int ammo{ 10 };

  auto shoot{
    // 在参数列表后添加 mutable
    [ammo]() mutable {
      // 现在可以修改 ammo 了
      --ammo;

      std::cout << "Pew! " << ammo << " shot(s) left.\n";
    }
  };

  shoot();
  shoot();

  std::cout << ammo << " shot(s) left\n";

  return 0;
}
\end{lstlisting}

警告:因为被捕获的变量是 lambda 对象的成员,它们的值会持续存在于若干 lambda 调用!

\subsubsection*{引用捕获}

\begin{lstlisting}[language=C++]
#include <iostream>

int main()
{
  int ammo{ 10 };

  auto shoot{
    // 这里不再需要 mutable 了
    [&ammo]() { // &ammo 意味着 ammo 是被引用捕获的
      // 修改 ammo 将会影响 main 的 ammo
      --ammo;

      std::cout << "Pew! " << ammo << " shot(s) left.\n";
    }
  };

  shoot();

  std::cout << ammo << " shot(s) left\n";

  return 0;
}
\end{lstlisting}

使用引用捕获统计 \acode{std::sort} 进行了多少次比较:

\begin{lstlisting}[language=C++]
#include <algorithm>
#include <array>
#include <iostream>
#include <string>

struct Car
{
  std::string make{};
  std::string model{};
};

int main()
{
  std::array<Car, 3> cars{ { { "Volkswagen", "Golf" },
                             { "Toyota", "Corolla" },
                             { "Honda", "Civic" } } };

  int comparisons{ 0 };

  std::sort(cars.begin(), cars.end(),
    // 引用捕获 @comparisons
    [&comparisons](const auto& a, const auto& b) {
      // 引用捕获,可以不使用 "mutable" 关键字修改 comparisons
      ++comparisons;

      // 排序 make
      return (a.make < b.make);
  });

  std::cout << "Comparisons: " << comparisons << '\n';

  for (const auto& car : cars)
  {
    std::cout << car.make << ' ' << car.model << '\n';
  }

  return 0;
}
\end{lstlisting}

\subsubsection*{捕获若干变量}

由逗号分隔捕获若干变量,这样可以混合包含值捕获或是引用捕获:

\begin{lstlisting}[language=C++]
int health{ 33 };
int armor{ 100 };
std::vector<CEnemy> enemies{};

// 值捕获 health 与 armor,引用捕获 enemies
[health, armor, &enemies](){};
\end{lstlisting}

\subsubsection*{默认捕获}

有时显式列举捕获的变量很麻烦,如果修改了 lambda,有可能会忘记添加或移除被捕获的变量。
幸运的是,可以授权编译器帮助用户自动生成需要捕获的变量列表。

\textbf{默认捕获} default capture(也称为\textbf{capture-default})捕获所有在 lambda 中提及的变量,
而没有提及的变量则不会捕获。

使用 \acode{=} ,进行值默认捕获。

使用 \acode{&} ,进行引用默认捕获。

\begin{lstlisting}[language=C++]
#include <algorithm>
#include <array>
#include <iostream>

int main()
{
  std::array areas{ 100, 25, 121, 40, 56 };

  int width{};
  int height{};

  std::cout << "Enter width and height: ";
  std::cin >> width >> height;

  auto found{ std::find_if(areas.begin(), areas.end(),
                           [=](int knownArea) { // 值默认捕获 width 与 height
                             return (width * height == knownArea); // 因为它们在这里被提及
                           }) };

  if (found == areas.end())
  {
    std::cout << "I don't know this area :(\n";
  }
  else
  {
    std::cout << "Area found :)\n";
  }

  return 0;
}
\end{lstlisting}

默认捕获可以与普通捕获混合。可以值捕获一些变量并引用捕获其它,但是每个变量只能被捕获一次。

\begin{lstlisting}[language=C++]
int health{ 33 };
int armor{ 100 };
std::vector<CEnemy> enemies{};

// 值捕获 health 与 armor,引用捕获 enemies
[health, armor, &enemies](){};

// 引用捕获 enemies,其余的值默认捕获
[=, &enemies](){};

// 值捕获 armor,其余的引用默认捕获
[&, armor](){};

// 非法,已经说了引用默认捕获所有变量
[&, &armor](){};

// 非法,已经说了值默认捕获所有变量
[=, armor](){};

// 非法,armor 出现了两次
[armor, &health, &armor](){};

// 非法,默认捕获必须位于捕获组的第一个元素
[armor, &](){};
\end{lstlisting}

\subsubsection*{在 lambda 捕获中定义新的变量}

有时候用户希望在捕获变量时允许微小的修改,或者是声明一个新的只在 lambda 中可见的变量。
那么可以在 lambda 捕获时,定义一个没有指定类型的变量。

\begin{lstlisting}[language=C++]
#include <array>
#include <iostream>
#include <algorithm>

int main()
{
  std::array areas{ 100, 25, 121, 40, 56 };

  int width{};
  int height{};

  std::cout << "Enter width and height: ";
  std::cin >> width >> height;

  // 这里存储了 areas,但是用户输入 width 与 height
  // 需要在进行查询之前,计算 area
  auto found{ std::find_if(areas.begin(), areas.end(),
                           // 声明一个新的只在 lambda 中可见的变量
                           // userArea 的类型自动被推导为 int
                           [userArea{ width * height }](int knownArea) {
                             return (userArea == knownArea);
                           }) };

  if (found == areas.end())
  {
    std::cout << "I don't know this area :(\n";
  }
  else
  {
    std::cout << "Area found :)\n";
  }

  return 0;
}
\end{lstlisting}

\acode{userArea} 只会在 lambda 定义时计算一次。
被计算的值会存储在 lambda 对象内,该值在每次调用时相同。
如果一个 lambda 是可变的,同时修改了定义在捕获的变量,那么原始值则被覆盖。

最佳实践:仅在值很短并且类型显而易见时,在捕获时初始化变量。
否则最好将变量定义在 lambda 外,然后捕获它。

\subsubsection*{悬垂捕获变量}

变量在 lambda 定义时被捕获。如果是引用捕获在 lambda 之前消亡了,那么 lambda 持有的则变成悬垂引用。

\begin{lstlisting}[language=C++]
#include <iostream>
#include <string>

// 返回一个 lambda
auto makeWalrus(const std::string& name)
{
  // 引用捕获 name,返回 lambda
  return [&]() {
    std::cout << "I am a walrus, my name is " << name << '\n'; // 未定义行为
  };
}

int main()
{
  // sayName 是一个由 makeWalrus 返回的 lambda
  auto sayName{ makeWalrus("Roofus") };

  // 调用 lambda 函数
  sayName();

  return 0;
}
\end{lstlisting}

调用 \acode{makeWalrus} 时,从字符串字面量 “Roofus” 创建了临时的 \acode{std::string} 。
\acode{makeWalrus} 中的 lambda 引用捕获了字符串。
该临时字符串在 \acode{makeWalrus} 返回时消亡,但是 lambda 仍然指向它。
接着调用 \acode{sayName} 时,悬垂引用被访问了,导致了未定义行为。

注意这同样也发生在 \acode{name} 值传递给 \acode{makeWalrus} 时,
因为 \acode{name} 还是会在 \acode{makeWalrus} 结束时消亡,同样的留下悬垂引用。

警告:引用捕获变量时要额外的小心,特别是引用默认捕获。
被捕获变量的存活时间必须比 lambda 长。

如果希望在 lambda 使用时,被捕获的 \acode{name} 有效,
需要通过值捕获的方式(要么显式声明要么使用默认的值捕获)。

\subsubsection*{可变 lambdas 的无意识的拷贝}

因为 lambdas 是对象,它们可以被拷贝。在某些情况下会导致一些问题。

\begin{lstlisting}[language=C++]
#include <iostream>

int main()
{
  int i{ 0 };

  // 创建一个名为 count 的 lambda
  auto count{ [i]() mutable {
    std::cout << ++i << '\n';
  } };

  count(); // 唤起 count

  auto otherCount{ count }; // 创建一个 count 的拷贝

  // 唤起 count 与其拷贝
  count();
  otherCount();

  return 0;
}
\end{lstlisting}

打印:

\begin{lstlisting}
1
2
2
\end{lstlisting}

上述代码没有打印 1,2,3,而是打印 2 两次。
当作为 \acode{count} 的拷贝,创建 \acode{otherCount} 时,
用户创建了带有当前状态的 \acode{count} 的拷贝。
因此 \acode{otherCount} 拷贝 \acode{count} 时,它们拥有自己的 \acode{i} 。

再来看一个不是那么明显的例子:

\begin{lstlisting}[language=C++]
#include <iostream>
#include <functional>

void myInvoke(const std::function<void()>& fn)
{
    fn();
}

int main()
{
    int i{ 0 };

    // 自增并打印 @i 的本地拷贝
    auto count{ [i]() mutable {
      std::cout << ++i << '\n';
    } };

    myInvoke(count);
    myInvoke(count);
    myInvoke(count);

    return 0;
}
\end{lstlisting}

打印:

\begin{lstlisting}
1
1
1
\end{lstlisting}

这展示了与之前代码同样的问题。
当 \acode{std::function} 创建了 lambda,\acode{std::function} 内在拷贝了 lambda 对象。
因此调用 \acode{fn()} 实际上是在执行 lambda 的拷贝,而不是真实的 lambda。

如果传递的是可变 lambda,同时希望避免拷贝发生,那么有两种选择。
第一个选择是使用非捕获 lambda -- 上述例子中,移除捕获并使用静态本地变量进行状态追踪。
但是静态本地变量很难被追踪,同时导致代码可读性降低。
另一个更好的选择是在创建 lambda 第一时间就阻止其拷贝的发生。
但是由于用户不能改变 \acode{std::function} (或者其它标准库函数与对象)的实现,那么该怎么做呢?

幸运的是,C++ 提供了一个名为 \acode{std::reference\_wrapper} 的便捷类型(同样也定义在 \acode{<functional>} 头文件中),
它允许用户传递普通类型而实际上是引用。
更方便的是,\acode{std::reference_wrapper} 可以通过 \acode{std::ref()} 函数创建。
通过包裹 lambda 在一个 \acode{std::reference_wrapper} 中,任何时候任何人尝试拷贝该 lambda,
都会拷贝 lambda 的引用而不是真实的对象。

下面是通过 \acode{std::ref} 进行代码升级:

\begin{lstlisting}[language=C++]
#include <iostream>
#include <functional>

void myInvoke(const std::function<void()>& fn)
{
    fn();
}

int main()
{
    int i{ 0 };

    // 自增并打印 @i 的本地拷贝
    auto count{ [i]() mutable {
      std::cout << ++i << '\n';
    } };

    // std::ref(count) 确保 count 被视为引用
    // 因此,任何对 count 的拷贝实际上都是拷贝引用
    // 确保只有一个 count 存在
    myInvoke(std::ref(count));
    myInvoke(std::ref(count));
    myInvoke(std::ref(count));

    return 0;
}
\end{lstlisting}

现在的打印符合预期了:

\begin{lstlisting}
1
2
3
\end{lstlisting}

注意即使 \acode{invoke} 值获取 \acode{fn} 时,输出也不会有变化,
因为通过 \acode{std::ref} 创建的 \acode{std::function} 并不创建 lambda 的拷贝。

规则:标准库的函数可能会拷贝函数对象(提醒:lambdas 是函数对象)。
如果用户希望提供带有可变捕获变量的 lambdas 时,使用 \acode{std::ref} 传递它们的引用。

最佳实践:尝试不使用可变 lambdas。不可变的 lambdas 更容易理解也不需要面对上述问题,
而且在增加并行执行时会增加更多危险的问题。

\end{document}
