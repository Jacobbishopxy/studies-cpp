\documentclass[../../LearnCpp.tex]{subfiles}

\begin{document}

\asubsection{4}{std::move}

一旦开始普遍的使用移动语义,会发现有时希望唤起移动语义,但是处理的对象却是左值而不是右值。
考虑下面代码:

\begin{lstlisting}[language=C++]
#include <iostream>
#include <string>

template<class T>
void myswapCopy(T& a, T& b)
{
  T tmp { a };  // 唤起拷贝构造函数
  a = b;        // 唤起拷贝赋值
  b = tmp;      // 唤起拷贝赋值
}

int main()
{
  std::string x{ "abc" };
  std::string y{ "de" };

  std::cout << "x: " << x << '\n';
  std::cout << "y: " << y << '\n';

  myswapCopy(x, y);

  std::cout << "x: " << x << '\n';
  std::cout << "y: " << y << '\n';

  return 0;
}
\end{lstlisting}

打印:

\begin{lstlisting}[language=C++]
x: abc
y: de
x: de
y: abc
\end{lstlisting}

上一节讲到过拷贝性能低,而这个版本的 swap 做了三次拷贝。

C++11 中 \acode{std::move} 作为一个标准库函数转型(使用 \acode{static_cast})它的参数成为右值引用,
因此移动语义可以被唤起。
所以可以使用 \acode{std::move} 来转换左值成为一个想要被移动而不是拷贝的类型,
\acode{std::move} 定义于 \acode{<utility>} 头文件。

以下是使用了 \acode{std::move} 转换左值成为右值并唤起移动语义的例子:

\begin{lstlisting}[language=C++]
#include <iostream>
#include <string>
#include <utility> // std::move

template<class T>
void myswapMove(T& a, T& b)
{
  T tmp { std::move(a) }; // 唤起移动构造函数
  a = std::move(b);       // 唤起移动赋值
  b = std::move(tmp);     // 唤起移动赋值
}

int main()
{
  std::string x{ "abc" };
  std::string y{ "de" };

  std::cout << "x: " << x << '\n';
  std::cout << "y: " << y << '\n';

  myswapMove(x, y);

  std::cout << "x: " << x << '\n';
  std::cout << "y: " << y << '\n';

  return 0;
}
\end{lstlisting}

打印:

\begin{lstlisting}
x: abc
y: de
x: de
y: abc
\end{lstlisting}

\subsubsection*{另一个例子}

在填充容器的元素时,可以使用 \acode{std::move},例如通过左值填充 \acode{std::vector}。

下列的例子中,首先通过拷贝语义添加一个元素给向量。接着通过移动语义再添加一个元素。

\begin{lstlisting}[language=C++]
#include <iostream>
#include <string>
#include <utility> // std::move
#include <vector>

int main()
{
  std::vector<std::string> v;

  // 这里使用 std::string,因为它是可移动的(std::string_view 不可移动)
  std::string str { "Knock" };

  std::cout << "Copying str\n";
  v.push_back(str); // 调用左值版本的 push_back,即拷贝 str 至数组元素

  std::cout << "str: " << str << '\n';
  std::cout << "vector: " << v[0] << '\n';

  std::cout << "\nMoving str\n";

  v.push_back(std::move(str)); // 调用右值版本的 push_back,即移动 str 至数组元素

  std::cout << "str: " << str << '\n'; // 这个结果是难以确定的
  std::cout << "vector:" << v[0] << ' ' << v[1] << '\n';

  return 0;
}
\end{lstlisting}

可能打印:

\begin{lstlisting}
Copying str
str: Knock
vector: Knock

Moving str
str:
vector: Knock Knock
\end{lstlisting}

第一种情况中,传递给 \acode{push_back()} 的是一个左值,
因此它使用了拷贝语义来添加元素至向量。也正是这个原因,str 仍然还在。

第二种情况中,传递给 \acode{push_back()} 的是一个右值
(事实上是左值通过 \acode{std::move} 转换的),
因此它使用了移动语义来添加元素至向量。
这更高效,因为向量元素可以窃取字符串的值而不是拷贝值。

\subsubsection*{被移动后的对象将变为无效,不过也可能处于不确定的状态}

当从一个临时对象中移动值,无需在意被移动的对象会是什么样子,因为临时对象会被立刻销毁。
但是如果是使用了 \acode{std::move} 之后的左值对象呢?

这里有两派的学说。
第一种学说认为被移动后的对象应该被重置成为默认/零值状态,该对象再也不拥有任何资源了;
另一种学说认为应该怎么简单怎么来,在不便利的情况下,不需要限制清除移动后的对象。

那么标准库是怎么处理的呢?
关于这个问题 C++ 标准说“除非另外指定,
被移动后的对象(其类型是定义于 C++ 标准库中的)应该放置在一个有效但是未指明的状态”。

本章的例子中,在调用 \acode{std::move} 之后再打印 \acode{str} 的值,得出的是空字符串。
然而这不是必须得,它可以打印任何有效的字符串,包括空字符串、原始字符串、或者其他有效字符串。
因此,用户应该避免使用被移动过后的对象,因为其结果是根据实现制定的。

关键点:\acode{std::move()} 提示编译器程序员不再需要改值的对象了。
仅对持久对象使用 \acode{std::move()},并且不要对移动后对象的值做任何的假设。
当值被移动后,该对象再次被赋值时可以的(使用操作符=)。

\subsubsection*{std::move() 还在那里有用?}

当排列数组元素时,\acode{std::move()} 同样也非常的有用。
很多排序算法(例如选择排序和冒泡排序)都是由元素之前进行交换完成的。

同样有用的地方在于从一个智能指针中移动内容至另一个智能指针。

\end{document}
