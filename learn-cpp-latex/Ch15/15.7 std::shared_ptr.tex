\documentclass[../../LearnCpp.tex]{subfiles}

\begin{document}

\asubsection{7}{std::shared\_ptr}

不同于 \acode{std::unique_ptr} 设计用来各自拥有并管理资源,
\acode{std::shared\_ptr} 意为解决需要若干智能指针共同拥有同一个资源。

这就意味着可以有若干个 \acode{std::shared\_ptr} 指向同一个资源。
在其内部持续追踪总共有多少个 \acode{std::shared\_ptr} 正在共享资源。
只要至少有一个 \acode{std::shared\_ptr} 指向资源,那么该资源就不会被释放。
一旦最后一个 \acode{std::shared\_ptr} 离开了作用域(或者重新赋值指向其它资源),
那么资源则被释放。

与 \acode{std::unique_ptr} 一样,\acode{std::shared\_ptr} 定义于 \acode{<memory>} 头文件。

\begin{lstlisting}[language=C++]
#include <iostream>
#include <memory> // std::shared_ptr

class Resource
{
public:
  Resource() { std::cout << "Resource acquired\n"; }
  ~Resource() { std::cout << "Resource destroyed\n"; }
};

int main()
{
  // 分配 Resource 对象,且使其被 std::shared_ptr 所有
  Resource* res { new Resource };
  std::shared_ptr<Resource> ptr1{ res };
  {
    std::shared_ptr<Resource> ptr2 { ptr1 }; // 使另一个 std::shared_ptr 指向同样的东西

    std::cout << "Killing one shared pointer\n";
  } // ptr2 离开作用域,但是无事发生

  std::cout << "Killing another shared pointer\n";

  return 0;
} // ptr1 在此离开作用域,被分配的 Resource 被销毁
\end{lstlisting}

打印:

\begin{lstlisting}
Resource acquired
Killing one shared pointer
Killing another shared pointer
Resource destroyed
\end{lstlisting}

上述代码中创建了一个动态 \acode{Resource} 对象,
并设置一个名为 \acode{ptr1} 的 \acode{std::shared\_ptr} 来管理它。
在嵌套的代码块中使用了拷贝构造函数来创建第二个 \acode{std::shared\_ptr}(\acode{ptr2})
指向同样的 \acode{Resource}。
当 \acode{ptr2} 离开作用域,\acode{Resource} 没有被释放,
因为 \acode{ptr1} 仍然指向 \acode{Resource}。
当 \acode{ptr1} 离开作用域,\acode{ptr1} 发现没有任何 \acode{std::shared\_ptr} 在管理 \acode{Resource},
所以 \acode{Resource} 被释放了。

注意第二个共享指针是由第一个共享指针创建而来的。这很重要。考虑下面类似的代码:

\begin{lstlisting}[language=C++]
#include <iostream>
#include <memory> // std::shared_ptr

class Resource
{
public:
  Resource() { std::cout << "Resource acquired\n"; }
  ~Resource() { std::cout << "Resource destroyed\n"; }
};

int main()
{
  Resource* res { new Resource };
  std::shared_ptr<Resource> ptr1 { res };
  {
    std::shared_ptr<Resource> ptr2 { res }; // 直接从 res 创建 ptr2(而不是从 ptr1)

    std::cout << "Killing one shared pointer\n";
  } // ptr2 离开嘞作用域,被分配的 Resource 被销毁

  std::cout << "Killing another shared pointer\n";

  return 0;
} // ptr1 在此离开作用域,被分配的 Resource 再次被销毁
\end{lstlisting}

打印:

\begin{lstlisting}
Resource acquired
Killing one shared pointer
Resource destroyed
Killing another shared pointer
Resource destroyed
\end{lstlisting}

接着程序崩溃。

上述两份代码唯一的区别在于第二份代码中创建了两个独立的 \acode{std::shared\_ptr}。
结果就是尽管它们都指向了 \acode{Resource},它们并不知晓对方。
当 \acode{ptr2} 离开作用域,它认为它是 \acode{Resource} 的唯一拥有者,接着被释放。
当 \acode{ptr1} 离开作用域时考虑同样的事情,尝试再次删除 \acode{Resource}。

最佳实践:如果需要多个 \acode{std::shared\_ptr} 指向给定资源,
拷贝现有的 \acode{std::shared\_ptr} 即可。

\subsubsection*{std::make\_shared}

类似于在 C++14 中使用 \acode{std::make_unique()} 来创建 \acode{std::unique\_ptr},
\acode{std::make\_shared()} 可以用于创建 \acode{std::shared\_ptr}(C++11 可用)。

\begin{lstlisting}[language=C++]
#include <iostream>
#include <memory> // std::shared_ptr

class Resource
{
public:
  Resource() { std::cout << "Resource acquired\n"; }
  ~Resource() { std::cout << "Resource destroyed\n"; }
};

int main()
{
  // 分配一个 Resource 对象,且使其被 std::shared_ptr 所有
  auto ptr1 { std::make_shared<Resource>() };
  {
    auto ptr2 { ptr1 }; // 使用拷贝 ptr1 创建 ptr2

    std::cout << "Killing one shared pointer\n";
  } // ptr2 在此离开作用域,无事发生

  std::cout << "Killing another shared pointer\n";

  return 0;
} // ptr1 离开作用域,分配的 Resource 被销毁
\end{lstlisting}

\subsubsection*{深挖 std::shared\_ptr}

不同于 \acode{std::unique\_ptr} 内部使用单个指针,
\acode{std::shared\_ptr} 内部使用两个指针。
一个指向了其所管理的资源,另一个指向一个“控制块”,即一个动态分配的对象用于追踪一些东西,
包含了有多少个 \acode{std::shared\_ptr} 指向了资源。
当一个通过 \acode{std::shared\_ptr} 构造函数创建的 \acode{std::shared\_ptr},
被管理对象(通常为传入的)的内存和控制块(由构造函数创造的)是分开的内存分配。
然而当使用 \acode{std::make\_shared()},这可以被优化成一个单独的内存分配,这带来了更好的性能。

这同时解释了为什么独立创建两个 \acode{std::shared\_ptr} 指向同一个资源会带来麻烦。
每个 \acode{std::shared\_ptr} 将拥有一个指针指向资源。
然而每个 \acode{std::shared\_ptr} 将独立分配其自己的控制块,即表示它是唯一拥有该资源的指针。
因此当 \acode{std::shared\_ptr} 离开作用域,释放资源,
并不会察觉还有其他的 \acode{std::shared\_ptr} 同样也尝试管理该资源。

然而当 \acode{std::shared\_ptr} 是由拷贝赋值克隆,
数据在控制块中可以被恰当的更新来表明现在有额外一个 \acode{std::shared\_ptr} 协同管理该资源。

\subsubsection*{std::shared\_ptr 的事故}

\acode{std::shared\_ptr} 有着与 \acode{std::unique\_ptr} 相同的挑战 --
如果 \acode{std::shared\_ptr} 没有被正确的销毁
(要么是因为它是动态分配的且永远没有删除,要么它是某个动态分配且永远没有删除的对象中的一部分),
那么其管理的资源也不会被释放。\acode{std::unique\_ptr} 仅需要担心一个智能指针是否被正确的销毁;
而 \acode{std::shared\_ptr} 需要担心所有的智能指针。
如果有任何一个管理资源的 \acode{std::shared\_ptr} 没有被正确的销毁,那么资源将不能被正确的释放。

\subsubsection*{std::shared\_ptr 与数组}

C++17 或更早,\acode{std::shared\_ptr} 并不能正确的管理数组,
同时也不允许用来管理 C-style 数组。C++20 开始 \acode{std::shared\_ptr} 支持数组。

\end{document}
