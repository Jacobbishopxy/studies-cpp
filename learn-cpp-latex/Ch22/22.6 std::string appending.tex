\documentclass[../../LearnCpp.tex]{subfiles}

\begin{document}

\asubsection{6}{std::string appending}

在字符串末尾 appending 字符串可以简单的由操作符+=, \acode{append()} 或者 \acode{push_back()} 来实现。

\begin{quotation}
  \textbf{string\& string::operator+=(const string\& str)}

  \textbf{string\& string::append(const string\& str)}

  \begin{itemize}
    \item 俩函数可在字符串末尾添加 str 字符串
    \item 俩函数返回的都是 *this 以便被串联
    \item 如果返回超过了字符串的最大数量则抛出 length\_error 异常
  \end{itemize}

  \begin{lstlisting}[language=C++]
  std::string sString{"one"};

  sString += std::string{" two"};

  std::string sThree{" three"};
  sString.append(sThree);

  std::cout << sString << '\n';
  \end{lstlisting}

  输出:

  \begin{lstlisting}
  one two three
  \end{lstlisting}
\end{quotation}

还有一种风格的 \acode{append()} 可以 append 子字符串:

\begin{quotation}
  \textbf{string\& string::append(const string\& str, size\_type index, size\_type num)}

  \begin{itemize}
    \item 函数 append 起始于 index 共 num 长的字符给字符串
    \item 俩函数返回的都是 *this 以便被串联
    \item 如果 index 超出边界抛出 out\_of\_range 异常
    \item 如果返回超过了字符串的最大数量则抛出 length\_error 异常
  \end{itemize}

  \begin{lstlisting}[language=C++]
  std::string sString{"one "};

  const std::string sTemp{"twothreefour"};
  sString.append(sTemp, 3, 5); // append substring of sTemp starting at index 3 of length 5
  std::cout << sString << '\n';
  \end{lstlisting}

  输出:

  \begin{lstlisting}
  one three
  \end{lstlisting}
\end{quotation}

操作符+= 与 \acode{append()} 也有 C-style 字符串的版本:

\begin{quotation}
  \textbf{string\& string::operator+= (const char* str)} \

  \textbf{string\& string::append (const char* str)}

  \begin{itemize}
    \item 俩函数可在字符串末尾添加 str 字符串
    \item 俩函数返回的都是 *this 以便被串联
    \item 如果返回超过了字符串的最大数量则抛出 length\_error 异常
    \item str 不能为 NULL
  \end{itemize}

  \begin{lstlisting}[language=C++]
  std::string sString{"one"};

  sString += " two";
  sString.append(" three");
  std::cout << sString << '\n';
  \end{lstlisting}

  输出:

  \begin{lstlisting}
  one two three
  \end{lstlisting}
\end{quotation}

同样风格的 \acode{append()} 也对 C-style 字符串起作用:

\begin{quotation}
  \textbf{string\& string::append(const char* str, size\_type len)}

  \begin{itemize}
    \item append 起始 len 个字符至字符串
    \item 俩函数返回的都是 *this 以便被串联
    \item 如果返回超过了字符串的最大数量则抛出 length\_error 异常
    \item 忽略特殊字符(包括")
  \end{itemize}

  \begin{lstlisting}[language=C++]
  std::string sString{"one "};

  sString.append("threefour", 5);
  std::cout << sString << '\n';
  \end{lstlisting}

  输出:

  \begin{lstlisting}
  one three
  \end{lstlisting}

  该函数很危险,并不推荐使用。
\end{quotation}

同样还有一系列的函数用于 append 字符。
注意用于 append 字符的非操作符函数的名称为 \acode{push\_back()},而不是 \acode{append()}!

\begin{quotation}
  \textbf{string\& string::operator+=(char c)}

  \textbf{void string::push\_back(char c)}

  \begin{itemize}
    \item 俩函数 append 字符 c 至字符串
    \item 操作符+= 返回 *this,以便被串联
    \item 如果返回超过了字符串的最大数量则俩函数都会抛出 length\_error 异常
  \end{itemize}

  \begin{lstlisting}[language=C++]
  std::string sString{"one"};

  sString += ' ';
  sString.push_back('2');
  std::cout << sString << '\n';
  \end{lstlisting}

  输出:

  \begin{lstlisting}
  one 2
  \end{lstlisting}
\end{quotation}

现在或许会疑惑为什么函数是 \acode{push_back()} 而不是 \acode{append()}。
这是因为栈的命名习惯,\acode{push_back()} 是添加单个项至栈的函数。
如果预想一个字符串是一个字符的栈,使用 \acode{push_back()} 在字符串末尾添加一个字符是有道理的。
然而缺少 \acode{append()} 函数确是与习俗不一致的!

实际上对于字符的 \acode{append()} 函数,看上去像是这样:

\begin{quotation}
  \textbf{string\& string::append(size\_type num, char c)}

  \begin{itemize}
    \item 添加字符 c 出现的 size\_type 次数
    \item 俩函数返回的都是 *this 以便被串联
    \item 如果返回超过了字符串的最大数量则抛出 length\_error 异常
  \end{itemize}

  \begin{lstlisting}[language=C++]
  std::string sString{"aaa"};

  sString.append(4, 'b');
  std::cout << sString << '\n';
  \end{lstlisting}

  输出:

  \begin{lstlisting}
  aaabbbb
  \end{lstlisting}
\end{quotation}

最后一种风格的 \acode{append()} 用作于迭代器:

\begin{quotation}
  \textbf{string\& string::append(InputIterator start, InputIterator end)}

  \begin{itemize}
    \item 从 [start, end) 范围中 append 所有的字符
    \item 俩函数返回的都是 *this 以便被串联
    \item 如果返回超过了字符串的最大数量则抛出 length\_error 异常
  \end{itemize}
\end{quotation}

\end{document}
