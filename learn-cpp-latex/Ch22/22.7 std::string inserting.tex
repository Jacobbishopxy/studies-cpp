\documentclass[../../LearnCpp.tex]{subfiles}

\begin{document}

\asubsection{7}{std::string inserting}

插入字符至已有的字符串可以通过 \acode{insert()} 函数实现:

\begin{quotation}
  \textbf{string\& string::insert(size\_type index, const string\& str)}

  \textbf{string\& string::insert(size\_type index, const char* str)}

  \begin{itemize}
    \item 俩函数插入字符 str 在字符串的 index 位置
    \item 俩函数都返回 *this 以便被串联
    \item 如果 index 无效,俩函数都会抛出 out\_of\_range
    \item 如果结果超过了字符串最大数量,俩函数都会抛出 length\_error 异常
    \item C-style 版本中 str 不可为 NULL
  \end{itemize}

  \begin{lstlisting}[language=C++]
  string sString("aaaa");
  cout << sString << endl;

  sString.insert(2, string("bbbb"));
  cout << sString << endl;

  sString.insert(4, "cccc");
  cout << sString << endl;
  \end{lstlisting}

  输出:

  \begin{lstlisting}
  aaaa
  aabbbbaa
  aabbccccbbaa
  \end{lstlisting}
\end{quotation}

以下版本的 \acode{insert()} 允许用户在任意位置插入子字符串:

\begin{quotation}
  \textbf{string\& string::insert(size\_type index, const string\& str, size\_type startindex, size\_type num)}

  \begin{itemize}
    \item 该函数插入 num 字符的 str,始于 startindex,插入进 string 的 index 位置
    \item 返回 *this 以便被串联
    \item 如果 index 或 startindex 超出边界则抛出 out\_of\_range
    \item 如果结果超出字符串的最大长度则抛出 length\_error 异常
  \end{itemize}

  \begin{lstlisting}[language=C++]
  string sString("aaaa");

  const string sInsert("01234567");
  sString.insert(2, sInsert, 3, 4); // insert substring of sInsert from index [3,7) into sString at index 2
  cout << sString << endl;
  \end{lstlisting}

  输出:

  \begin{lstlisting}
  aa3456aa
  \end{lstlisting}
\end{quotation}

另外还有 \acode{insert()} 多次插入同样的字符:

\begin{quotation}
  \textbf{string\& string::insert(size\_type index, size\_type num, char c)}

  \begin{itemize}
    \item 插入 num 数量的 c 字符至 index 位置
    \item 返回 *this 以便被串联
    \item 如果 index 无效,则抛出 out\_of\_range 异常
    \item 如果结果超出字符串的最大长度则抛出 length\_error 异常
  \end{itemize}

  \begin{lstlisting}[language=C++]
  string sString("aaaa");

  sString.insert(2, 4, 'c');
  cout << sString << endl;
  \end{lstlisting}

  输出:

  \begin{lstlisting}
  aaccccaa
  \end{lstlisting}
\end{quotation}

最后还有使用迭代器的三个不同版本的 \acode{insert()}:

\begin{quotation}
  \textbf{void insert(iterator it, size\_type num, char c)}

  \textbf{iterator insert(iterator it, char c)}

  \textbf{void string::insert(iterator it, InputIterator begin, InputIterator end)}

  \begin{itemize}
    \item 第一个函数在迭代之前,插入 num 数量的 c 字符
    \item 第二个函数在迭代之前,插入单个字符 c,并返回插入位置的迭代器
    \item 第三个函数在迭代之前,在 $\left[begin, end\right)$ 之间插入所有字符
    \item 如果结果超出字符串最大长度,所有函数抛出 length\_error 异常
  \end{itemize}
\end{quotation}

\end{document}
