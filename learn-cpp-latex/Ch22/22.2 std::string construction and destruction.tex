\documentclass[../../LearnCpp.tex]{subfiles}

\begin{document}

\asubsection{2}{std::string construction and destruction}

本节将会讲述如何构造 \acode{std::string} 对象,以及如何由数字创建 \acode{std::string},反之亦然。

\subsubsection*{字符串构造函数}

字符串类拥有很多构造函数,让我们一个一个来看看。

注意: \acode{string::size_type} 解析为 \acode{size_t},
由 \acode{sizeof} 操作符返回的结果等同于无符号整数类型。
\acode{size_t} 的真实大小取决于其环境。本教程中将其视为无符号整型。

\begin{quotation}
  \textbf{string::string()}

  \begin{itemize}
    \item 此为默认构造函数,其创建空字符串。
  \end{itemize}

  \begin{lstlisting}[language=C++]
    std::string sSource;
    std::cout << sSource;
    \end{lstlisting}

  输出:

  \begin{lstlisting}

    \end{lstlisting}
\end{quotation}

\begin{quotation}
  \textbf{string::string(const string\& strString)}

  \begin{itemize}
    \item 此为拷贝构造函数,由 strString 拷贝创建一个新字符串。
  \end{itemize}

  \begin{lstlisting}[language=C++]
    std::string sSource{ "my string" };
    std::string sOutput{ sSource };
    std::cout << sOutput;
    \end{lstlisting}

  输出:

  \begin{lstlisting}
    my string
    \end{lstlisting}
\end{quotation}

\begin{quotation}

  \textbf{string::string(const string\& strString, \\ size\_type unIndex)}

  \textbf{string::string(const string\& strString, size\_type unIndex, size\_type unLength)}

  \begin{itemize}
    \item 创建新字符串包含至多从 strString 而来的 unLength 大小的字符,由索引 unIndex 开始。
          如果遇到 NULL,字符串的拷贝则结束,即使没有触及 unLength。
    \item 如果没有提供 unLength,所有字符由 unIndex 开始使用。
    \item 如果 unIndex 大于字符串的长度,那么 out\_of\_range 异常将被抛出
  \end{itemize}

  \begin{lstlisting}[language=C++]
    std::string sSource{ "my string" };
    std::string sOutput{ sSource, 3 };
    std::cout << sOutput<< '\n';
    std::string sOutput2(sSource, 3, 4);
    std::cout << sOutput2 << '\n';
    \end{lstlisting}

  输出:

  \begin{lstlisting}
    string
    stri
    \end{lstlisting}
\end{quotation}

\begin{quotation}
  \textbf{string::string(const char* szCString)}

  \begin{itemize}
    \item 从 C-style 字符串 szCString 创建新字符串,但不包括空终止符。
    \item 如果返回的大小超过了最大字符串长度,length\_error 异常将被抛出。
    \item 警告:szCSting 必须不为空。
  \end{itemize}

  \begin{lstlisting}[language=C++]
    const char* szSource{ "my string" };
    std::string sOutput{ szSource };
    std::cout << sOutput << '\n';
    \end{lstlisting}

  输出:

  \begin{lstlisting}
    my string
    \end{lstlisting}
\end{quotation}

\begin{quotation}
  \textbf{string::string(const char* szCString, size\_type unLength)}

  \begin{itemize}
    \item 从 C-style 字符串 szCString 的首个 unLength 字符创建字符串。
    \item 如果返回的大小超过了最大字符串长度,length\_error 异常将被抛出。
    \item 警告:仅在这个函数,空值不被视为 szCString 中的终止字符串!
          这就意味着如果 unLength 足够大,可以读取字符串到最后!注意不要溢出字符串的缓存!
  \end{itemize}

  \begin{lstlisting}[language=C++]
    const char* szSource{ "my string" };
    std::string sOutput(szSource, 4);
    std::cout << sOutput << '\n';
    \end{lstlisting}

  输出:

  \begin{lstlisting}
    my s
    \end{lstlisting}
\end{quotation}

\begin{quotation}
  \textbf{string::string(size\_type nNum, char chChar)}

  \begin{itemize}
    \item 通过字符 chChar 出现的 nNum 次数创建字符串。
    \item 如果返回的大小超过了最大字符串长度,length\_error 异常将被抛出。
  \end{itemize}

  \begin{lstlisting}[language=C++]
    std::string sOutput(4, 'Q');
    std::cout << sOutput << '\n';
    \end{lstlisting}

  输出:

  \begin{lstlisting}
    QQQQ
    \end{lstlisting}
\end{quotation}

\begin{quotation}
  \textbf{template string::string(InputIterator itBeg, InputIterator itEnd)}

  \begin{itemize}
    \item 由初始化范围字符 [itBeg, itEnd] 创建新字符串。
    \item 如果返回的大小超过了最大字符串长度,length\_error 异常将被抛出。
  \end{itemize}

  没有示例代码,因为它的晦涩使得用户基本不会使用它。
\end{quotation}

\begin{quotation}
  \textbf{string::~string()}

  \textbf{字符串析构函数}

  \begin{itemize}
    \item 析构函数。销毁字符串并释放内存。
  \end{itemize}

  没有示例代码,因为不会显式调用析构函数。
\end{quotation}

\subsubsection*{数值的构造函数}

\begin{lstlisting}[language=C++]
#include <iostream>
#include <sstream>
#include <string>

template <typename T>
inline std::string ToString(T tX)
{
    std::ostringstream oStream;
    oStream << tX;
    return oStream.str();
}

int main()
{
    std::string sFour{ ToString(4) };
    std::string sSixPointSeven{ ToString(6.7) };
    std::string sA{ ToString('A') };
    std::cout << sFour << '\n';
    std::cout << sSixPointSeven << '\n';
    std::cout << sA << '\n';
}
\end{lstlisting}

打印:

\begin{lstlisting}
4
6.7
A
\end{lstlisting}

注意这个方法省略了任何错误检查。在 tX 插入 oStream 可能失败。合适的响应应该是在转换失败时抛出异常。

\subsubsection*{字符串转换成数值}

\begin{lstlisting}[language=C++]
#include <iostream>
#include <sstream>
#include <string>

template <typename T>
inline bool FromString(const std::string& sString, T& tX)
{
    std::istringstream iStream(sString);
    return !(iStream >> tX).fail(); // extract value into tX, return success or not
}

int main()
{
    double dX;
    if (FromString("3.4", dX))
        std::cout << dX << '\n';
    if (FromString("ABC", dX))
        std::cout << dX << '\n';
}
\end{lstlisting}

打印:

\begin{lstlisting}
3.4
\end{lstlisting}

注意第二个转换失败了并返回 false。

\end{document}
