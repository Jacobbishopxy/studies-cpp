\documentclass[../../LearnCpp.tex]{subfiles}

\begin{document}

\asubsection{3}{Exceptions, functions, and stack unwinding}

\subsubsection*{被调用的函数中抛出异常}

重点:try blocks 不仅可以从 try block 中的声明中捕获异常,
同样也可以捕获 try block 中被调用函数的异常。

\begin{lstlisting}[language=C++]
#include <cmath> // sqrt() 函数
#include <iostream>

// 一个组合式的开根号函数
double mySqrt(double x)
{
    // 如果用户输入的是负值,这是一个错误条件
    if (x < 0.0)
        throw "Can not take sqrt of negative number"; // 抛出 const char* 类型的异常

    return std::sqrt(x);
}

int main()
{
    std::cout << "Enter a number: ";
    double x {};
    std::cin >> x;

    try // 查看在 try block 中出现的异常,并导航到附属的 catch block(s)
    {
        double d = mySqrt(x);
        std::cout << "The sqrt of " << x << " is " << d << '\n';
    }
    catch (const char* exception) // 捕获异常类型 const char*
    {
        std::cerr << "Error: " << exception << std::endl;
    }

    return 0;
}
\end{lstlisting}

\subsubsection*{异常处理与栈展开}

当异常被抛出时,程序首先查看异常是否能立刻被当前函数中处理
(意为在当前函数的 try block 中抛出异常,且有相关的关联 catch block)。
如果可以处理那么便处理。

如果不能被处理,程序会检查函数调用者(下一个在栈上的函数)是否能处理异常。
为了使函数的调用者处理异常,调用当前函数必须时在一个 try block 中,
并与一个匹配的 catch block 关联。如果没有匹配的捕获,调用者的调用者再去进行检查,以此类推。

如果一个匹配的异常被捕获,则执行跳转到与之匹配的 catch block。
这就需要展开栈(在调用栈中移除当前函数)若干次直到处理异常的函数在调用栈的顶部。

如果没有匹配被发现,栈有可能会被展开。这会在下一章中详细讲解。

\subsubsection*{另一个栈展开的例子}

\begin{lstlisting}[language=C++]
#include <iostream>

void D() // 由 C() 调用
{
    std::cout << "Start D\n";
    std::cout << "D throwing int exception\n";

    throw - 1;

    std::cout << "End D\n"; // 跳过
}

void C() // 由 B() 调用
{
    std::cout << "Start C\n";
    D();
    std::cout << "End C\n";
}

void B() // 由 A() 调用
{
    std::cout << "Start B\n";

    try
    {
        C();
    }
    catch (double) // 不捕获:异常类型不匹配
    {
        std::cerr << "B caught double exception\n";
    }

    try
    {
    }
    catch (int) // 不捕获:异常类型不匹配
    {
        std::cerr << "B caught int exception\n";
    }

    std::cout << "End B\n";
}

void A() // 由 main() 调用
{
    std::cout << "Start A\n";

    try
    {
        B();
    }
    catch (int) // 这里捕获异常并处理
    {
        std::cerr << "A caught int exception\n";
    }

    std::cout << "End A\n";
}

int main()
{
    std::cout << "Start main\n";

    try
    {
        A();
    }
    catch (int) // 没有被调用,因为异常在 A 中被处理了
    {
        std::cerr << "main caught int exception\n";
    }
    std::cout << "End main\n";

    return 0;
}
\end{lstlisting}

打印:

\begin{lstlisting}
Start main
Start A
Start B
Start C
Start D
D throwing int exception
A caught int exception
End A
End main
\end{lstlisting}

\end{document}
