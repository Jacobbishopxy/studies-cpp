\documentclass[../../LearnCpp.tex]{subfiles}

\begin{document}

\asubsection{4}{Uncaught exceptions and catch-all handlers}

\subsubsection*{未捕获异常}

当一个函数抛出异常且并没有被自身处理,那么是在假设函数的调用栈上将会处理该异常。

\begin{lstlisting}[language=C++]
#include <iostream>
#include <cmath> // sqrt()

double mySqrt(double x)
{
    if (x < 0.0)
        throw "Can not take sqrt of negative number";

    return std::sqrt(x);
}

int main()
{
    std::cout << "Enter a number: ";
    double x;
    std::cin >> x;

    // 这里没有异常处理!
    std::cout << "The sqrt of " << x << " is " << mySqrt(x) << '\n';

    return 0;
}
\end{lstlisting}

当一个函数不能找到异常处理时,\acode{std::terminate()} 会被调用,整个程序会被终止。
这样的情况,调用栈有可能被展开!
如果没有被展开,本地变量则不会被销毁,那么任何预期的析构函数并不会出现!

警告:如果一个异常未被捕获,调用栈有可能不被展开。

\subsubsection*{catch-all 处理}

那么现在的难题是:

\begin{itemize}
  \item 函数可能潜在抛出任何类型异常(包括程序定义的数据类型),意味着异常会有无限种可能。
  \item 如果异常没有被捕获,程序则会立马停止(同时栈有可能不被展开,因此程序有可能不能正确的清除自身)。
  \item 为每一种可能得类型添加一个显式的捕获处理太过冗长,特别是一些预期就是仅有在特殊情况下才能达到!
\end{itemize}

幸运的是 C++ 同样也提供了一个异常全捕获的机制。
也就是熟知的 \textbf{catch-all handler}。

\begin{lstlisting}[language=C++]
#include <iostream>

int main()
{
  try
  {
    throw 5; // throw an int exception
  }
  catch (double x)
  {
    std::cout << "We caught an exception of type double: " << x << '\n';
  }
  catch (...) // catch-all handler
  {
    std::cout << "We caught an exception of an undetermined type\n";
  }
}
\end{lstlisting}

\subsubsection*{使用 catch-all handler 来包裹 main()}

\begin{lstlisting}[language=C++]
#include <iostream>

int main()
{

    try
    {
        runGame();
    }
    catch(...)
    {
        std::cerr << "Abnormal termination\n";
    }

    saveState();
    return 1;
}
\end{lstlisting}

最佳实践:如果程序使用了异常,考虑在 main 中使用 catch-all handler,
这样可以避免未捕获异常出现时而导致的极端情况发生。
同样在 debug 构建中不使用 catch-all handler,
这样可以更简单的定位到未捕获异常是如何发生的。

\end{document}
