\documentclass[../../LearnCpp.tex]{subfiles}

\begin{document}

\asubsection{14}{Constexpr and consteval functions}

\subsubsection*{Constexpr 函数可以在编译期被计算}

一个\textbf{constexpr} 函数其返回值可能在编译期被计算。
只需要在函数的返回值前加上 \acode{constexpr} 关键字就可以使一个函数成为 constexpr 函数。

\begin{lstlisting}[language=C++]
#include <iostream>

constexpr int greater(int x, int y) // constexpr 函数
{
    return (x > y ? x : y);
}

int main()
{
    constexpr int x{ 5 };
    constexpr int y{ 6 };

    // 稍后会解释这里为何使用变量 g
    constexpr int g { greater(x, y) }; // 将在编译期被计算

    std::cout << g << " is greater!\n";

    return 0;
}
\end{lstlisting}

具备编译期计算的能力,一个函数必须拥有一个 constexpr 的返回类型同时不调用任何非 constexpr 函数。
除此之外,调用函数者必须是 constexpr 参数(例如 constexpr 变量或者字面值)。

\subsubsection*{Constexpr 函数是隐式内联的}

因为 constexpr 函数有可能在编译期被计算,在其被调用处,编译器必须知道其所有的定义。

这就意味着在若干文件中调用的 constexpr 函数需要其定义也被引用至各个文件中 -- 这通常违反了单定义规则。为了避免这种情况,constexpr 函数是隐式内联的,使得它们从单独定义规则中豁免。

结果就是 constexpr 函数通常定义在头文件中,这样他们可以被 \#include 到任何需求全定义的 .cpp 文件中。

\subsubsection*{Constexpr 函数同样也可以在运行时被计算}

\begin{lstlisting}[language=C++]
#include <iostream>

constexpr int greater(int x, int y)
{
    return (x > y ? x : y);
}

int main()
{
    int x{ 5 }; // 非 constexpr
    int y{ 6 }; // 非 constexpr

    std::cout << greater(x, y) << " is greater!\n"; // 将在运行时被计算

    return 0;
}
\end{lstlisting}

\subsubsection*{Constexpr 函数什么时候会在编译期执行?}

\begin{lstlisting}[language=C++]
#include <iostream>

constexpr int greater(int x, int y)
{
    return (x > y ? x : y);
}

int main()
{
    constexpr int g { greater(5, 6) };              // case 1: 编译期计算
    std::cout << g << " is greater!\n";

    int x{ 5 }; // 非 constexpr
    std::cout << greater(x, 6) << " is greater!\n"; // case 2: 运行时计算

    std::cout << greater(5, 6) << " is greater!\n"; // case 3: 有可能是编译期计算,也有可能是运行时计算,根据编译器优化等级决定

    return 0;
}
\end{lstlisting}

\subsubsection*{判断 constexpr 函数是编译期计算还是运行时计算}

C++20 之前是没有合适的标准语言工具的。

在 C++20 中,\acode{std::is\_constant\_evaluated()}(定义在 <type\_traits> 头文件中)返回一个 \acode{bool} 表示当前函数调用是否执行在常数上下文中。
这可以与条件声明结合起来,使一个函数的行为根据编译期或运行时做出不同的行为。

\begin{lstlisting}[language=C++]
#include <type_traits> // 内含 std::is_constant_evaluated

constexpr int someFunction()
{
    if (std::is_constant_evaluated()) // 如果是编译期计算
        // 做什么
    else // 运行时计算
        // 做其他
}
\end{lstlisting}

\subsubsection*{Consteval C++20}

C++20 引入关键字 \textbf{consteval},用以表示一个函数\textit{必须}在编译时计算,否则会返回编译错误。
这种函数被称为 \textbf{immediate functions}。

\begin{lstlisting}[language=C++]
#include <iostream>

consteval int greater(int x, int y) // 函数为 consteval
{
    return (x > y ? x : y);
}

int main()
{
    constexpr int g { greater(5, 6) };              // ok: 编译期计算
    std::cout << greater(5, 6) << " is greater!\n"; // ok: 编译期计算

    int x{ 5 }; // 非 constexpr
    std::cout << greater(x, 6) << " is greater!\n"; // error: consteval 函数必须在编译期计算

    return 0;
}
\end{lstlisting}

最佳实践:如果因为某种原因(例如性能)函数必须在编译期执行,请使用 \textbf{consteval}。

\subsubsection*{使用 consteval 使 constexpr 在编译期执行 C++20}

consteval 这类函数的缺点就是不能在运行时计算,使得它们相较于 constexpr 函数少了一些灵活性。
因此,拥有一个方便的方法迫使 constexpr 函数在编译期计算(即使其返回值被用在不需要的常数表达式中),
那么可以尽可能的编译期计算,并且不可能时运行时计算。

consteval 函数提供了一种方式使其成为可能,使用一个干净的帮助函数:

\begin{lstlisting}[language=C++]
#include <iostream>

// 使用 abbreviated 函数模版 (C++20) 以及 `auto` 返回值使得该函数可以对任何类型值生效
consteval auto compileTime(auto value)
{
    return value;
}

constexpr int greater(int x, int y) // 函数为 constexpr
{
    return (x > y ? x : y);
}

int main()
{
    std::cout << greater(5, 6) << '\n';              // 编译期可能执行
    std::cout << compileTime(greater(5, 6)) << '\n'; // 编译期执行

    int x { 5 };
    std::cout << greater(x, 6) << '\n';              // 仍然可以在运行时调用 constexpr 版本

    return 0;
}
\end{lstlisting}

上述之所以行得通是因为 constexpr 函数需要常数表达式作为入参 --
因此如果我们使用 constexpr 函数的返回值作为 consteval 函数的入参,constexpr 函数就必须在编译期执行了!

注意 consteval 函数返回的是值,而这可能在运行时效率不高(如果值是一些拷贝起来很昂贵的类型,例如 std::string),
在编译期的上下文中倒是无所谓因为 consteval 函数的整个调用过程最终会由计算后的返回值替换掉。

相关内容:

\begin{itemize}
    \item 章节 8.8 中讲解 \acode{auto}。
    \item 章节 8.15 中讲解 abbreviated 函数模版(\acode{auto} 参数)。
\end{itemize}

\end{document}
