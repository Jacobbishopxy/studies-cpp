\documentclass[../../LearnCpp.tex]{subfiles}

\begin{document}

\asubsection{6}{Internal linkage}

本地变量,我们称“一个标识符的链接决定了其它对该名称的声明,是否指向同一个对象”,
那么本地变量是 \acode{no linkage} 的。

全局变量以及函数标识符既可以是 \acode{internal linkage} 也可以是 \acode{external linkage}。
本章将会讲解 \acode{internal linkage},下一章讲解 \acode{external linkage}。

一个 \textbf{internal linkage} 的标识符仅可以在单个文件中可视以及可用,
并不能被其它文件访问(也就是说,它没有被暴露给 linker)。
这就意味着如果两个文件拥有同名的标识符且是 internal linkage 的,那么它们则被视为独立。

\subsubsection*{带有 internal linkage 的全局变量}

带有 internal linkage 的全局变量有时候会被称为 \textbf{内部变量} internal variables。

使用 \acode{static} 关键字可以让非常数全局变量 internal。

\begin{lstlisting}[language=C++]
static int g_x; // 非常数全局变量默认是 external linkage 的,通过 static 关键字使其变为 internal linkage

const int g_y { 1 }; // const 全局变量默认是 internal linkage 的
constexpr int g_z { 2 }; // constexpr 全局变量默认是 internal linkage 的

int main()
{
    return 0;
}
\end{lstlisting}

\subsubsection*{带有 internal linkage 的函数}

由于 linkage 是一个标识符的属性(而不是变量),函数标识符也有同样的 linkage 属性。
函数默认是 external linkage(详见下一章),但是可以通过 \acode{static} 关键字设置成 internal linkage。

add.cpp:

\begin{lstlisting}[language=C++]
// 函数本声明为 static,现在只能在本文件中可用
// 在其他文件中通过前向声明尝试访问它则会失败
static int add(int x, int y)
{
    return x + y;
}
\end{lstlisting}

main.cpp:

\begin{lstlisting}[language=C++]
#include <iostream>

int add(int x, int y); // 函数 add 的前向声明

int main()
{
    std::cout << add(3, 4) << '\n';

    return 0;
}
\end{lstlisting}

\end{document}
