\documentclass[../../LearnCpp.tex]{subfiles}

\begin{document}

\asubsection{2}{User-defined namespaces and the scope resolution operator}

\subsubsection*{自定义命名空间}

C++ 允许用户通过 \acode{namespace} 关键字来定义自己的命名空间。
这样的命名空间被称为\textbf{用户定义命名空间} user-defined namespaces。

\begin{lstlisting}[language=C++]
#include <iostream>

namespace foo // 定义命名空间 foo
{
    // doSomething() 属于 foo 命名空间
    int doSomething(int x, int y)
    {
        return x + y;
    }
}

namespace goo // 定义命名空间 goo
{
    // doSomething() 属于 goo 命名空间
    int doSomething(int x, int y)
    {
        return x - y;
    }
}

int main()
{
    std::cout << foo::doSomething(4, 3) << '\n'; // 使用 foo 命名空间的 doSomething()
    std::cout << goo::doSomething(4, 3) << '\n'; // 使用 goo 命名空间的 doSomething()
    return 0;
}
\end{lstlisting}

\subsubsection*{若干命名空间(同名)也被允许}

在不同的地方(不同的文件下,或者是同一文件下的不同位置),声明若干个命名空间是合法的。
所有拥有相同名称的不同声明处,可被视为命名空间的组成部分。

circle.h:

\begin{lstlisting}[language=C++]
#ifndef CIRCLE_H
#define CIRCLE_H

namespace basicMath
{
    constexpr double pi{ 3.14 };
}

#endif
\end{lstlisting}

growth.h:

\begin{lstlisting}[language=C++]
#ifndef GROWTH_H
#define GROWTH_H

namespace basicMath
{
    // 常数 e 同样也是 basicMath 命名空间的一部分
    constexpr double e{ 2.7 };
}

#endif
\end{lstlisting}

main.cpp:

\begin{lstlisting}[language=C++]
#include "circle.h" // basicMath::pi
#include "growth.h" // basicMath::e

#include <iostream>

int main()
{
    std::cout << basicMath::pi << '\n';
    std::cout << basicMath::e << '\n';

    return 0;
}
\end{lstlisting}

当分隔代码至若干文件时,用户可以在头文件与源文件使用命名空间。

add.h:

\begin{lstlisting}[language=C++]
#ifndef ADD_H
#define ADD_H

namespace basicMath
{
    // 函数 add() 是 basicMath 命名空间的一部分
    int add(int x, int y);
}

#endif
\end{lstlisting}

add.cpp:

\begin{lstlisting}[language=C++]
#include "add.h"

namespace basicMath
{
    // 定义函数 add()
    int add(int x, int y)
    {
        return x + y;
    }
}
\end{lstlisting}

main.cpp:

\begin{lstlisting}[language=C++]
#include "add.h" // 使用 basicMath::add()

#include <iostream>

int main()
{
    std::cout << basicMath::add(4, 3) << '\n';

    return 0;
}
\end{lstlisting}

\subsubsection*{嵌套命名空间}

\begin{lstlisting}[language=C++]
#include <iostream>

namespace foo
{
    namespace goo // goo is a namespace inside the foo namespace
    {
        int add(int x, int y)
        {
            return x + y;
        }
    }
}

int main()
{
    std::cout << foo::goo::add(1, 2) << '\n';
    return 0;
}
\end{lstlisting}

C++17 之后还可以这样声明命名空间:

\begin{lstlisting}[language=C++]
#include <iostream>

namespace foo::goo // goo 是一个在 foo 命名空间里的命名空间(C++17 风格)
{
  int add(int x, int y)
  {
    return x + y;
  }
}

int main()
{
    std::cout << foo::goo::add(1, 2) << '\n';
    return 0;
}
\end{lstlisting}

\subsubsection*{命名空间别名}

C++ 允许用户创建\textbf{命名空间别名},即可以临时缩短很长一串的命名空间成为更短的名称:

\begin{lstlisting}[language=C++]
#include <iostream>

namespace foo::goo
{
    int add(int x, int y)
    {
        return x + y;
    }
}

int main()
{
    namespace active = foo::goo; // active 现在指向 foo::goo

    std::cout << active::add(1, 2) << '\n'; // 这即是 foo::goo::add()

    return 0;
} // active 的别名在此处结束
\end{lstlisting}

\end{document}
