\documentclass[../../LearnCpp.tex]{subfiles}

\begin{document}

\asubsection{9}{Sharing global constants across multiple files (using inline variables)}

在一些应用中,特定的有意义的常量会在整个项目代码中用到。
它们可能包含了物理学或是数学不变的常数(例如 π 或者阿佛加德罗常量),
或是应用的“调参”值(例如摩擦系数或者重力系数)。
不需要在每个文件中重新定义这些常数,而是仅集中声明它们一次并在任何需要的地方使用。
这样的话,如果需要修改的时候仅需要改一处地方。

\subsubsection*{作为内部变量的全局常数}

在 C++17 之前,以下步骤是最方便以及常用的解决方案:

\begin{itemize}
    \item 创建一个包含这些常量的头文件
    \item 头文件中,定义命名空间
    \item 在命名空间内添加所有需要的常量(确保它们都是 \textit{constexpr})
    \item 在需要的地方 \#include 头文件
\end{itemize}

constants.h:

\begin{lstlisting}[language=C++]
#ifndef CONSTANTS_H
#define CONSTANTS_H

// 自定义命名空间用于保存常量
namespace constants
{
    // 常量默认为 internal linkage
    constexpr double pi { 3.14159 };
    constexpr double avogadro { 6.0221413e23 };
    constexpr double myGravity { 9.2 }; // m/s^2 -- 这个星球的重力小一点
    // ... 其它相关常量
}
#endif
\end{lstlisting}

使用范围解析操作符 scope resolution operator (::) 与命名空间(位于操作符左侧)
以及定义的变量名称在右侧,访问定义好的常量:

main.cpp:

\begin{lstlisting}[language=C++]
#include "constants.h" // 本文件中 include 了这些常量的拷贝

#include <iostream>

int main()
{
    std::cout << "Enter a radius: ";
    int radius{};
    std::cin >> radius;

    std::cout << "The circumference is: " << 2.0 * radius * constants::pi << '\n';

    return 0;
}
\end{lstlisting}

\subsubsection*{作为外部变量的全局常数}

上述的方法有几个缺点。

如果是简单的(适合小型程序),每次 constant.h 被 \#include 到不同的代码文件,每个变量都被拷贝到代码文件中。
因此如果 constants.h 被 include 到了 20 个不同文件中,每个变量就被拷贝了 20 次。
头文件保护符并不会停止这事发生,它们仅仅阻止头文件在同一个文件中被 include 多次,
而不是阻止头文件被多个不同的文件 include。这就带来了两个挑战:

\begin{enumerate}
    \item 改变一个常数将需要每个引用其的文件重新编译,这就导致了大项目的编译时间大大延长。
    \item 如果常数所需空间很大并且无法被优化,则会使用大量的内存。
\end{enumerate}

避免上述问题的方法之一是使这些常量变为外部变量,因为我们可以拥有一个单变量(初始化一次)并在所有文件中共享。
这个方法,用户需要定义常量与一个 .cpp 文件中(为了确保定义只存在于一个处),并放置前向声明在头文件中(以便其它文件引入)。

注意:这里使用 const 而不是 constexpr,因为 constexpr 变量不可以被前向声明,
即便他们有 external linkage。这是因为编译器需要在编译时知道这个变量的值,而前向声明并不提供该信息。

constants.cpp:

\begin{lstlisting}[language=C++]
#include "constants.h"

namespace constants
{
    // 真实全局变量
    extern const double pi { 3.14159 };
    extern const double avogadro { 6.0221413e23 };
    extern const double myGravity { 9.2 }; // m/s^2
}
\end{lstlisting}

constants.h:

\begin{lstlisting}[language=C++]
#ifndef CONSTANTS_H
#define CONSTANTS_H

namespace constants
{
    // 由于真实的变量在命名空间中,前向声明同样也需要在命名空间内
    extern const double pi;
    extern const double avogadro;
    extern const double myGravity;
}

#endif
\end{lstlisting}

使用代码与之前的方法相同:

\begin{lstlisting}[language=C++]
#include "constants.h" // 引用所有的前向声明

#include <iostream>

int main()
{
    std::cout << "Enter a radius: ";
    int radius{};
    std::cin >> radius;

    std::cout << "The circumference is: " << 2.0 * radius * constants::pi << '\n';

    return 0;
}
\end{lstlisting}

因为全局标识的常数需要命名空间(避免与其他的全局命名空间中的标识符冲突),“g\_” 前缀的命名则不再需要了。

然而,这个方法也有一些缺陷。首先这些常数仅仅在它们定义处(\acode{constants.cpp})被视为编译时的常数。
其他文件中,编译器只能看到前向声明,即不定义常量(且必须有 linker 解决)。
这就意味着在其他文件中,它们被视为运行时常数而不是编译期常数。
因此在 \acode{constants.cpp} 之外,这些变量不能被用于任何需要编译期常数的地方。
其次,因为编译期常数通常会比运行时常数得到更多的优化,编译器对这些常数的优化支持就没有这么多了。

备注:变量为了被视为编译期的可用内容,例如数组大小,编译器必须知道变量的定义(而不仅仅是前向定义)。

因为编译器对每个源文件都是单独编译的,它仅可以通过源文件(也包含了任何引用的头文件)中所在的变量看到其定义。
例如,定义在 \acode{constants.cpp} 中的变量在编译器编译 \acode{main.cpp} 时是不可见的。
因此,\acode{constexpr} 变量不能被分离至头文件以及源文件中,它们必须被定义在头文件中。

\subsubsection*{作为内联变量的全局常数 C++17}

C++17 引入了一个新的概念称为 \acode{incline variables}。
C++中,称为 \acode{inline} 的以及进化到意为“若干定义皆被允许”。
因此,一个**\textbf{内联变量} inline variable 就是变量允许被定义在若干文件中且不危害唯一性规则。
内联全局变量默认拥有 external linkage。

linker 将会合并一个变量的所有内联定义成为单个变量定义(因此符合单个定义规则)。
这使得用户可以在头文件中定义变量,同时使变量被视为在某个 .cpp 文件中的唯一定义。
也就是说有一个常量被 \#include 到了 10 个代码文件中。
没有 inline,用户会得到 10 个定义。
有了 inline,编译器则会选择一个定义为典型定义 canonical definition,因此用户只会得到一个定义。
这就意味着省下了 9 个常量的内存。

这些常数在所有被引用的文件中,同样也保留了 constexpr 功能,
所以他们可以被用于任何 constexpr 值所需要的地方。
相比于运行时常数(或者非 const)变量,constexpr 值则可以被编译器高度优化。

内联变量拥有两个主要的约束需要被遵守:

\begin{enumerate}
    \item 所有的内联变量的定义需要时唯一的(否则未定义行为会出现)。
    \item 内联变量定义(不是前向声明)必须出现在任何文件中使用的变量。
\end{enumerate}

constants.h:

\begin{lstlisting}[language=C++]
#ifndef CONSTANTS_H
#define CONSTANTS_H

// 自定义命名空间用于保存常量
namespace constants
{
    inline constexpr double pi { 3.14159 }; // 注意:现在是 inline constexpr
    inline constexpr double avogadro { 6.0221413e23 };
    inline constexpr double myGravity { 9.2 }; // m/s^2
    // ... 其他相关变量
}
#endif
\end{lstlisting}

main.cpp:

\begin{lstlisting}[language=C++]
#include "constants.h"

#include <iostream>

int main()
{
    std::cout << "Enter a radius: ";
    int radius{};
    std::cin >> radius;

    std::cout << "The circumference is: " << 2.0 * radius * constants::pi << '\n';

    return 0;
}
\end{lstlisting}

用户可以引用 \acode{constants.h} 至多个代码文件,这些变量仅会被初始化一次并可在所有代码文件中共享。

这个方法仍然保留了在任何变量修改后,需要所有引用其的文件重新进行编译的缺陷。
如果需要总是改变常数(例如调参的目的)并且导致更长的编译时间,
移动常变的常数至其自己的头文件中(为了减少 \#include 的次数)可能会更有帮助。

最佳实践:如果需要全局常数,同时可以使用 C++17 编译器,那么推荐在头文件中定义内联变量。

提醒:使用 \acode{std::string_view} 为 \acode{constexpr} 字符串。

\end{document}
