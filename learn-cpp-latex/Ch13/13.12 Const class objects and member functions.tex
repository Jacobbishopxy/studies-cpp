\documentclass[../../LearnCpp.tex]{subfiles}

\begin{document}

\asubsection{12}{Const class objects and member functions}

\subsubsection*{Const 类}

实例化对象的时候也可以使用 const 关键字。初始化由类的构造函数完成。

\begin{lstlisting}[language=C++]
const Date date1;                   // 使用默认构造函数进行初始化
const Date date2(2020, 10, 16);     // 带参数初始化
const Date date3 { 2020, 10, 16 };  // 带参数初始化 (C++11)
\end{lstlisting}

一旦一个 const 类对象通过构造函数被初始化,尝试修改成员变量是不被允许的,因为这违背了 const 对象规则。
这里包含了直接修改成员变量(如果它们是共有的),或者是调用成员函数设置成员变量。

\begin{lstlisting}[language=C++]
class Something
{
public:
    int m_value {};

    Something(): m_value{0} { }

    void setValue(int value) { m_value = value; }
    int getValue() { return m_value ; }
};

int main()
{
    const Something something{}; // 调用默认构造函数

    something.m_value = 5; // 编译错误:违背 const
    something.setValue(5); // 编译错误:违背 const

    return 0;
}
\end{lstlisting}

\subsubsection*{Const 成员函数}

一个\textbf{const 成员函数}确保不会修改对象或者调用任何非 const 成员函数(因为它们有可能修改对象)。

\begin{lstlisting}[language=C++]
class Something
{
public:
    int m_value {};

    Something(): m_value{0} { }

    void resetValue() { m_value = 0; }
    void setValue(int value) { m_value = value; }

    int getValue() const { return m_value; } // 注意 const 关键字在参数列表后,函数体前
};
\end{lstlisting}

对于成员函数定义在类定义外的情况,const 关键字必须同时定义在类定义的函数协议上,以及函数定义上。

\begin{lstlisting}[language=C++]
class Something
{
public:
    int m_value {};

    Something(): m_value{0} { }

    void resetValue() { m_value = 0; }
    void setValue(int value) { m_value = value; }

    int getValue() const; // 注意这里额外的 const 关键字
};

int Something::getValue() const // 以及这里
{
    return m_value;
}
\end{lstlisting}

另外,任何 const 成员函数尝试修改一个成员变量或者调用非 const 成员函数会导致变异错误。

\begin{lstlisting}[language=C++]
class Something
{
public:
    int m_value {};

    void resetValue() const { m_value = 0; } // 编译错误,const 函数不能修改成员变量
};
\end{lstlisting}

最佳实践:令任何不修改类变量的成员函数为 const,这样它们可以被调用为 const 对象。

\subsubsection*{传递 const 引用用于创建 const 对象}

尽管实例化 const 类对象是一种创造 const 对象的方法,另一种常用的方法是传递 const 引用对象用于创建 const 对象。

\begin{lstlisting}[language=C++]
class Date
{
private:
    int m_year {};
    int m_month {};
    int m_day {};

public:
    Date(int year, int month, int day)
    {
        setDate(year, month, day);
    }

    // setDate() cannot be const, modifies member variables
    void setDate(int year, int month, int day)
    {
        m_year = year;
        m_month = month;
        m_day = day;
    }

    // The following getters can all be made const
    int getYear() const { return m_year; }
    int getMonth() const { return m_month; }
    int getDay() const { return m_day; }
};

// note: We're passing date by const reference here to avoid making a copy of date
void printDate(const Date& date)
{
    std::cout << date.getYear() << '/' << date.getMonth() << '/' << date.getDay() << '\n';
}

int main()
{
    Date date{2016, 10, 16};
    printDate(date);

    return 0;
}
\end{lstlisting}

\subsubsection*{const 成员不可以返回非 const 成员的引用}

当一个成员函数是 const 时,隐藏的 \*this 指针同样也是 const 的,
这就意味着所有在函数内的成员都被视为 const。
因此,一个 const 成员函数不可以返回一个非 const 成员的引用,
因为这样允许了调用者使用非 const 访问该 const 成员。
const 成员函数可以返回 const 成员的引用。详见下一章节。

\subsubsection*{重载 const 与非 const 函数}

最后,尽管这么做不是很常见,还是有可能会重载一个函数变成带有 const 与非 const 版本的同一函数。
这可以工作是因为 const 限定符被视为函数签名的一部分,因此两个函数的区别只在于只因 const 视为不同。

\begin{lstlisting}[language=C++]
#include <string>

class Something
{
private:
    std::string m_value {};

public:
    Something(const std::string& value=""): m_value{ value } {}

    const std::string& getValue() const { return m_value; } // getValue() 的 const 对象(返回 const 引用)
    std::string& getValue() { return m_value; } // getValue() 的非 const 对象(返回非 const 引用)
};
\end{lstlisting}

那么 const 版本的函数将会被调用于任何 const 对象,而非 const 版本将会被调用与任何非 const 对象:

\begin{lstlisting}[language=C++]
int main()
{
  Something something;
  something.getValue() = "Hi"; // 调用非 const getValue()

  const Something something2;
  something2.getValue(); // 调用 const getValue()

  return 0;
}
\end{lstlisting}

\end{document}
