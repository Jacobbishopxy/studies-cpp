\documentclass[../../LearnCpp.tex]{subfiles}

\begin{document}

\asubsection{9}{Destructors}

在类中,\textbf{析构函数} destructor 是另一种在类对象被销毁时调用的特殊成员函数。
构造函数被设计用于初始化一个类,而析构函数被设计用于帮助清理。

当一个对象普通的离开作用域,或者使用 \acode{delete} 关键字删除动态分配的对象时,
类的析构函数会自动被调用(如果存在的话),用于在对象从内存中被移除之前,清理任何必要的内容。
对于简单类而言(那些仅初始化普通成员变量的),结构函数是不需要的,因为 C++ 将自动清除内存。

然而如果类对象存储了任何资源(例如动态内存,或者文件,或者数据库连接),
或者用户需要在对象被摧毁之前做任何的维护,那么结构函数正是最佳的地方,
因为它通常是对象被摧毁前最后的一件事。

\subsubsection*{析构函数命名}

与构造函数类似,结构函数也有特定的规则:

\begin{itemize}
  \item 必须与类名称相同,并在前加上波浪号(\acode{\~})。
  \item 不能带有参数。
  \item 没有返回值。
\end{itemize}

通常来说用户不应该直接显式的调用结构函数(因为它在对象销毁之前会被自动调用),
因为只有很罕见的情况需要对一个对象进行若干次清理。
然而,结构函数可能安全的调用其它成员函数,因为对象并没有被摧毁,直到析构函数被执行完毕。

\subsubsection*{析构函数案例}

\begin{lstlisting}[language=C++]
#include <iostream>
#include <cassert>
#include <cstddef>

class IntArray
{
private:
  int* m_array{};
  int m_length{};

public:
  IntArray(int length) // 构造函数
  {
    assert(length > 0);

    m_array = new int[static_cast<std::size_t>(length)]{};
    m_length = length;
  }

  ~IntArray() // 析构函数
  {
    // 动态删除之前分配过的数组
    delete[] m_array;
  }

  void setValue(int index, int value) { m_array[index] = value; }
  int getValue(int index) { return m_array[index]; }

int getLength() { return m_length; }
};

int main()
{
  IntArray ar ( 10 ); // 分配 10 个整数
  for (int count{ 0 }; count < ar.getLength(); ++count)
    ar.setValue(count, count+1);

  std::cout << "The value of element 5 is: " << ar.getValue(5) << '\n';

  return 0;
} // ar 在这里被摧毁,因此 ~IntArray() 析构函数在这里被调用
\end{lstlisting}

\subsubsection*{构造函数与析构函数的时机}

正如之前提到的,构造函数在对象创建时被调用,析构函数在对象被摧毁时调用。
下面例子中,在它们中使用 \acode{cout} 声明来验证:

\begin{lstlisting}[language=C++]
#include <iostream>

class Simple
{
private:
    int m_nID{};

public:
    Simple(int nID)
        : m_nID{ nID }
    {
        std::cout << "Constructing Simple " << nID << '\n';
    }

    ~Simple()
    {
        std::cout << "Destructing Simple" << m_nID << '\n';
    }

    int getID() { return m_nID; }
};

int main()
{
    // 在栈上分配一个 Simple
    Simple simple{ 1 };
    std::cout << simple.getID() << '\n';

    // 动态分配一个 Simple
    Simple* pSimple{ new Simple{ 2 } };

    std::cout << pSimple->getID() << '\n';

    // 之前动态分配的 pSimple,因此需要删除它
    delete pSimple;

    return 0;
} // simple 在这里离开作用域
\end{lstlisting}

上述代码打印:

\begin{lstlisting}
Constructing Simple 1
1
Constructing Simple 2
2
Destructing Simple 2
Destructing Simple 1
\end{lstlisting}

\subsubsection*{RAII}

RAII (Resource Acquisition Is Initialization) 是一种编程技术,
凭此技术资源的使用与对象的生命周期自动绑定(例如,非动态分配对象)。
在 C++ 中,RAII 通过类的构造函数与析构函数来实现。
一个资源(例如内存,文件或者数据库连接)通常在对象的构造函数中是需要的
(通过构造函数,在对象创建后生成是符合逻辑的)。
该资源在对象存活期间是可使用的,在对象被销毁时释放于析构函数中。
RAII 最关键的优点就是在拥有资源的对象自动进行清理,防止资源泄漏(例如内存没有被释放)。

\end{document}
