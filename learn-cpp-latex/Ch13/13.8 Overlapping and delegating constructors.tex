\documentclass[../../LearnCpp.tex]{subfiles}

\begin{document}

\asubsection{8}{Overlapping and delegating constructors}

\subsubsection*{带有重复功能的构造函数}

当实例化一个新对象时,对象的构造函数是隐式调用的。
对于一个类拥有若干重复功能的构造函数是很常见的。考虑以下代码:

\begin{lstlisting}[language=C++]
class Foo
{
public:
    Foo()
    {
        // code to do A
    }

    Foo(int value)
    {
        // code to do A
        // code to do B
    }
};
\end{lstlisting}

该类拥有两个构造函数:一个默认构造函数,一个带有 int 入参的构造函数。
因为“code to do A”的构造函数在两个构造函数中都被需要,这里代码就重复了。

\subsubsection*{显而易见的方案并不能工作}

\begin{lstlisting}[language=C++]
class Foo
{
public:
    Foo()
    {
        // code to do A
    }

    Foo(int value)
    {
        Foo(); // 使用上述构造函数完成 A(不工作)
        // code to do B
    }
};
\end{lstlisting}

这是因为 \acode{Foo()} 实例化了一个新的 \acode{Foo} 对象,接着立刻被销毁。

\subsubsection*{委托构造函数}

构造函数允许调用同一个类下的构造函数。
这个过程被称为\textbf{委托构造函数} delegating constructors(或者 \textbf{constructor chaining})。

要让一个构造函数调用同一个类下的另一个构造函数,只需要简单的在成员初始化列表中调用即可。
这是直接调用另一个构造函数的可以接受的方案:

\begin{lstlisting}[language=C++]
class Foo
{
private:

public:
    Foo()
    {
        // code to do A
    }

    Foo(int value): Foo{} // 使用 Foo() 默认构造函数完成 A
    {
        // code to do B
    }

};
\end{lstlisting}

另一个使用委托构造函数的例子:

\begin{lstlisting}[language=C++]
#include <iostream>
#include <string>
#include <string_view>

class Employee
{
private:
    int m_id{};
    std::string m_name{};

public:
    Employee(int id=0, std::string_view name=""):
        m_id{ id }, m_name{ name }
    {
        std::cout << "Employee " << m_name << " created.\n";
    }

    // 使用委托构造函数来最小化冗余代码
    Employee(std::string_view name) : Employee{ 0, name }
    { }
};
\end{lstlisting}

这里有几点关于委托构造函数需要注意的地方。
首先,使用委托构造函数的构造函数自身不允许做任何成员初始化。
因此构造函数可以是委托或者是初始化,而不能是同时两者。

其次一个构造函数委托另一个构造函数,可以被委托回到第一个构造函数。
这就形成了无限循环,将导致程序用尽栈空间并崩溃。
确保所有委托构造函数使用的都是非委托构造函数,可以避免这种情况。

最佳实践:如果有若干拥有相同功能的构造函数,使用委托构造函数来避免重复代码。

\subsubsection*{使用普通成员函数进行设置}

由于构造函数仅可以初始化或者委托,那么如果默认构造函数做了一些共同的初始化,这就带来了挑战。
考虑以下代码:

\begin{lstlisting}[language=C++]
class Foo
{
private:
    const int m_value { 0 };

public:
    Foo()
    {
         // 代码用于做一些共同的设置任务(例如打开文件夹或者连接数据库)
    }

    Foo(int value) : m_value { value } // 必须初始化 m_value 因为它是 const
    {
        // 那么如何获取 Foo() 中共同的初始化代码呢 ?
    }

};
\end{lstlisting}

\acode{Foo(int)} 构造函数仅可以初始化 \acode{m_value},或者是委托 \acode{Foo()} 访问设置代码

\begin{lstlisting}[language=C++]
#include <iostream>

class Foo
{
private:
    const int m_value { 0 };

    void setup() // setup 是私有的,因此仅可在构造函数中使用
    {
        // 代码用于做一些共同的设置任务(例如打开文件夹或者连接数据库)
        std::cout << "Setting things up...\n";
    }

public:
    Foo()
    {
        setup();
    }

    Foo(int value) : m_value { value } // 必须初始化 m_value 因为它是 const
    {
        setup();
    }

};

int main()
{
    Foo a;
    Foo b{ 5 };

    return 0;
}
\end{lstlisting}

\subsubsection*{重置类对象}

有时候可能会发现编写一个成员函数(例如名为 \acode{reset()})用来重置一个类对象至初始状态。

由于有可能已经拥有了默认构造函数用于初始化,用户可能会尝试直接在 \acode{reset()} 函数中调用该默认构造函数。
然而直接调用会带来未定义行为,正如之前所示那样,因此这是行不通的。

如果一个类是可赋值的(意味着它拥有一个可用的赋值操作符),那么可以创建一个新的类对象,然后覆盖原有值:

\begin{lstlisting}[language=C++]
#include <iostream>

class Foo
{
private:
    int m_a{ 5 };
    int m_b{ 6 };


public:
    Foo()
    {
    }

    Foo(int a, int b)
        : m_a{ a }, m_b{ b }
    {
    }

    void print()
    {
        std::cout << m_a << ' ' << m_b << '\n';
    }

    void reset()
    {
        *this = Foo(); // 创建新 Foo 对象,接着使用赋值来覆盖隐式对象
    }
};

int main()
{
    Foo a{ 1, 2 };
    a.reset();

    a.print();

    return 0;
}
\end{lstlisting}

关联内容:\acode{this} 指针将在 13.10 中覆盖,对类进行赋值将在 14.15 中覆盖。

\end{document}
