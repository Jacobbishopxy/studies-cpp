\documentclass[../../LearnCpp.tex]{subfiles}

\begin{document}

\asubsection{4}{Access functions and encapsulation}

\subsubsection*{封装}

面向对象编程里,\textbf{封装} encapsulation(也称为\textbf{信息隐藏} Information hiding)
是一种将实现的细节隐藏的步骤。用户通过公有接口访问对象,而不需要知道其内部是如何实现的。

\begin{itemize}
    \item 既可以便于使用也可以减少程序的复杂度。
    \item 帮助保护数据以及防止滥用。
    \item 利于调试
\end{itemize}

\subsubsection*{访问函数}

\textbf{访问函数}是简短的公有函数,其职责是获取或修改私有成员变量。

访问函数通常有两种:getters 以及 setters。
\textbf{Getters}(有时也称为 \textbf{accessors})是用于返回私有成员变量的函数;
\textbf{Setters}(有时也称为 \textbf{mutators})是用于设置私有成员变量的函数。

\begin{lstlisting}[language=C++]
class Date
  {
    private:
    int m_month;
    int m_day;
    int m_year;

    public:
    int getMonth() { return m_month; } // month 的 getter
    void setMonth(int month) { m_month = month; } // month 的 setter

    int getDay() { return m_day; } // day 的 getter
    void setDay(int day) { m_day = day; } // day 的 setter

    int getYear() { return m_year; } // year 的 getter
    void setYear(int year) { m_year = year; } // year 的 setter
  };
\end{lstlisting}

一个简单的例子,getter 返回非 const 引用:

\begin{lstlisting}[language=C++]
#include <iostream>

class Foo
{
private:
int m_value{ 4 };

public:
int& getValue() { return m_value; } // 返回非 const 引用
};

int main()
{
    Foo f;                     // f.m_value 初始化为 4
    f.getValue() = 5;          // 使用非 const 引用赋值 m_value 为 5
    std::cout << f.getValue(); // 打印 5

    return 0;
  }
\end{lstlisting}

最佳实践:getters 应该返回值或者 const 引用。

\subsubsection*{访问函数的关注点}

创建类的时候考虑以下几点:

\begin{itemize}
    \item 如果类的成员没有任何的外部访问需求,则不要提供该成员的访问函数
    \item 如果类的成员需要被外部访问,则需要考虑是否暴露 behavior 还是 action
          (例如是一个 \acode{setAlive(bool)} 的 setter,还是实现 \acode{kill()} 函数)
    \item 如果都不是,考虑是否仅可以提供一个 getter
\end{itemize}

\end{document}
