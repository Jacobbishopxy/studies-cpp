\documentclass[../../LearnCpp.tex]{subfiles}

\begin{document}

\asubsection{17}{Nested types in classes}

考虑以下代码:

\begin{lstlisting}[language=C++]
#include <iostream>

enum class FruitType
{
  apple,
  banana,
  cherry
};

class Fruit
{
private:
  FruitType m_type {};
  int m_percentageEaten { 0 };

public:
  Fruit(FruitType type) :
    m_type { type }
  {
  }

  FruitType getType() const { return m_type; }
  int getPercentageEaten() const { return m_percentageEaten; }
};

int main()
{
  Fruit apple { FruitType::apple };

  if (apple.getType() == FruitType::apple)
    std::cout << "I am an apple";
  else
    std::cout << "I am not an apple";

  return 0;
}
\end{lstlisting}

代码没有错。但是因为枚举 FruitType 用于连接 Fruit 类,独立存在于类本身其实是有点怪异的。

\subsubsection*{嵌套类型}

类似于函数和数据可以作为类的成员,C++ 中,类型也同样可以在类中被定义。
只需要在类中定义类型并使用 public 访问说明符即可。

\begin{lstlisting}[language=C++]
#include <iostream>

class Fruit
{
public:
  // 注意:移动 FruitType 进类中并位于 public 访问说明符之下
  // 同样也从 enum class 转换为 enum
  enum FruitType
  {
    apple,
    banana,
    cherry
  };

private:
  FruitType m_type {};
  int m_percentageEaten { 0 };

public:
  Fruit(FruitType type) :
    m_type { type }
  {
  }

  FruitType getType() const { return m_type; }
  int getPercentageEaten() const { return m_percentageEaten; }
};

int main()
{
  // 注意:可以通过 Fruit 来访问 FruitType 了
  Fruit apple { Fruit::apple };

  if (apple.getType() == Fruit::apple)
    std::cout << "I am an apple";
  else
    std::cout << "I am not an apple";

  return 0;
}
\end{lstlisting}

\end{document}
