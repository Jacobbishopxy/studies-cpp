\documentclass[../../LearnCpp.tex]{subfiles}

\begin{document}

\asubsection{6}{Constructor member initializer lists}

上一章节中,为了简化,在构造函数中使用了分配操作符用于初始化类成员。

\begin{lstlisting}[language=C++]
class Something
{
private:
    int m_value1 {};
    double m_value2 {};
    char m_value3 {};

public:
    Something()
    {
        // 这些都是赋值,不是初始化
        m_value1 = 1;
        m_value2 = 2.2;
        m_value3 = 'c';
    }
};
\end{lstlisting}

当类的构造函数被执行时,m\_value1,m\_value2 以及 m\_value3 被创建。
接着构造函数的函数体运行使成员变量被赋值。这就类似于以下代码的非面向对象 C++ 的流程:

\begin{lstlisting}[language=C++]
int m_value1 {};
double m_value2 {};
char m_value3 {};

m_value1 = 1;
m_value2 = 2.2;
m_value3 = 'c';
\end{lstlisting}

虽然 C++ 中是合法的语法,但是它并不是一个好的风格(同时比起初始化要低性能)。

然而在上一章节中,一些数据类型(例如 const 与引用变量)必须在其声明处进行初始化,考虑以下代码:

\begin{lstlisting}[language=C++]
class Something
{
private:
    const int m_value;

public:
    Something()
    {
        m_value = 1; // 错误:const 变量不能被赋值
    }
};
\end{lstlisting}

在构造函数中,对 const 或者引用成员变量赋值是不可能的。

\subsubsection*{成员初始化列表}

为了解决这个问题,C++ 提供了一种名为\textit{成员初始化列表} member initializer list(通常被称为“member initialization list”)的初始化类成员变量的方法(与其在变量创建后赋值)。不要把它们与用于对数组赋值的初始化列表搞混淆。

在 1.4 章节中讲解了可以通过三种方法初始化变量:拷贝,直接,以及标准化初始化。

\begin{lstlisting}[language=C++]
int value1 = 1;     // copy initialization
double value2(2.2); // direct initialization
char value3 {'c'};  // uniform initialization
\end{lstlisting}

使用一个初始化列表几乎与直接初始化或标准化初始化相同。

使用初始化列表改写上面的代码:

\begin{lstlisting}[language=C++]
#include <iostream>

class Something
{
private:
    int m_value1 {};
    double m_value2 {};
    char m_value3 {};

public:
    Something() : m_value1{ 1 }, m_value2{ 2.2 }, m_value3{ 'c' } // 初始化成员变量
    {
    // 这里不再需要赋值
    }

    void print()
    {
         std::cout << "Something(" << m_value1 << ", " << m_value2 << ", " << m_value3 << ")\n";
    }
};

int main()
{
    Something something{};
    something.print();
    return 0;
}
\end{lstlisting}

成员初始化列表在构造函数参数后参数插入。
由冒号(\acode{:})开始,接着列出每个初始化的变量,且由逗号分隔。

注意在构造函数体内不再需要赋值,因为初始化列表替换了这个功能。同时注意初始化列表没有分号结尾。

当然了,当允许调用者传递初始化值时,构造函数更有用了:

\begin{lstlisting}[language=C++]
#include <iostream>

class Something
{
private:
    int m_value1 {};
    double m_value2 {};
    char m_value3 {};

public:
    Something(int value1, double value2, char value3='c')
        : m_value1{ value1 }, m_value2{ value2 }, m_value3{ value3 } // 直接初始化成员变量
    {
    // 这里不再需要赋值
    }

    void print()
    {
         std::cout << "Something(" << m_value1 << ", " << m_value2 << ", " << m_value3 << ")\n";
    }

};

int main()
{
    Something something{ 1, 2.2 }; // value1 = 1, value2=2.2, value3 为默认值 'c'
    something.print();
    return 0;
}
\end{lstlisting}

最佳实践:使用成员初始化列表来初始化类成员变量,而不是赋值。

\subsubsection*{初始化 const 成员变量}

类可以包含 const 成员变量,它们与普通的 const 变量表现一致 -- 必须被初始化,
同时它们的值之后不能再被改变。

可以使用构造函数的成员初始化列表来初始化 const 成员(与非 const 成员相似),
同时初始化值可以为常量或非常量。

\begin{lstlisting}[language=C++]
#include <iostream>

class Something
{
private:
    const int m_value;

public:
    Something(int x) : m_value{ x } // 直接初始化 const 成员
    {
    }

    void print()
    {
        std::cout << "Something(" << m_value << ")\n";
    }
};

int main()
{
    std::cout << "Enter an integer: ";
    int x{};
    std::cin >> x;

    Something s{ x };
    s.print();

    return 0;
}
\end{lstlisting}

规则:const 成员变量必须被初始化。

\subsubsection*{通过成员初始化列表初始化数组成员}

考虑一个带有数字成员的类:

\begin{lstlisting}[language=C++]
class Something
{
private:
    const int m_array[5];

};
\end{lstlisting}

在 C++11 之前,用户只能通过成员初始化列表来零初始化一个数组成员:

\begin{lstlisting}[language=C++]
class Something
{
private:
    const int m_array[5];

public:
    Something(): m_array {} // 零初始化数组成员
    {
    }

};
\end{lstlisting}

而从 C++11 开始,用户可以使用标准化初始化来初始化一个数组成员:

\begin{lstlisting}[language=C++]
class Something
{
private:
    const int m_array[5];

public:
    Something(): m_array { 1, 2, 3, 4, 5 } // 使用标准化初始化来初始化一个数组成员
    {
    }

};
\end{lstlisting}

\subsubsection*{初始化成员变量为类}

成员初始化列表同样也可以用于初始化类成员:

\begin{lstlisting}[language=C++]
#include <iostream>

class A
{
public:
    A(int x = 0) { std::cout << "A " << x << '\n'; }
};

class B
{
private:
    A m_a {};
public:
    B(int y)
        : m_a{ y - 1 } // 调用 A(int) 构造函数来初始化成员 m_a
    {
        std::cout << "B " << y << '\n';
    }
};

int main()
{
    B b{ 5 };
    return 0;
}
\end{lstlisting}

\subsubsection*{格式化初始化列表}

C++ 提供给用户了很多关于如何格式化初始化列表的灵活性,并且可以完全随用户的喜好。

如果初始化列表与函数在同一列,那么把所有东西放在一行是可以的:

\begin{lstlisting}[language=C++]
class Something
{
private:
    int m_value1 {};
    double m_value2 {};
    char m_value3 {};

public:
    Something() : m_value1{ 1 }, m_value2{ 2.2 }, m_value3{ 'c' } // 所有东西在一行
    {
    }
};
\end{lstlisting}

如果初始化列表没有与函数名同一行,那么应该缩进在下一行:

\begin{lstlisting}[language=C++]
class Something
{
private:
    int m_value1;
    double m_value2;
    char m_value3;

public:
    Something(int value1, double value2, char value3='c') // 这一行已经有很多东西了
        : m_value1{ value1 }, m_value2{ value2 }, m_value3{ value3 } // 所以可以把所有东西缩进再下一行
    {
    }

};
\end{lstlisting}

如果初始化不在单独一行(或者初始化很重要),则可以每个都占据一行:

\begin{lstlisting}[language=C++]
class Something
{
private:
    int m_value1 {};
    double m_value2 {};
    char m_value3 {};
    float m_value4 {};

public:
    Something(int value1, double value2, char value3='c', float value4=34.6f) // 这一行已经有很多东西了
        : m_value1{ value1 } // 每个都占据一行
        , m_value2{ value2 }
        , m_value3{ value3 }
        , m_value4{ value4 }
    {
    }

};
\end{lstlisting}

\subsubsection*{初始化列表顺序}

出乎意料的是,在初始化列表中的变量初始化的顺序与指定的顺序并不相同。
而是根据它们如何在类中声明的顺序一致。

为了更好的结果,下列建议应该被遵守:

\begin{enumerate}
    \item 在成员变量依赖其他成员变量先初始化时,不要使用初始化列表
          (换言之,确保成员变量能正确的初始化即使初始化顺序是不同的)。
    \item 初始化列表中的初始化变量顺序应该与类中声明的顺序一致。
          这并非是严格需要的,只要前一条建议被遵循,但在所有警告都打开的情况下,编译器可能会给出顺序不一致的警告。
\end{enumerate}

\subsubsection*{总结}

成员初始化列表允许用户初始化成员而不是赋值成员。
这是唯一初始化成员需要值的方法,例如 const 或者引用成员,
同时相较于在构造函数体内赋值有更良好的性能。
成员初始化列表可以同时作用于基础类型以及类成员。

\end{document}
