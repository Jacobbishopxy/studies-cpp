\documentclass[../../LearnCpp.tex]{subfiles}

\begin{document}

\asubsection{15}{Friend functions and classes}

从这一章开始就一直在宣传数据私有的概念。
然而还是偶尔发现在某些情况下,有某类的外部进行更紧密的操作的操作需求。
例如,希望一个类存储数据,一个函数(或者其它类)用于展示该数据。
尽管存储类和展示部分的代码为了维护方便被分隔,后者与前者的细节联系的却非常紧密。
就结果而言,展示部分的代码并没有从隐藏存储类细节中获益。

这种情况下有两种选择:

\begin{enumerate}
  \item 展示部分的代码使用存储类暴露出来的公有函数。然而这有若干潜在的缺陷。
        首先公有成员函数必须被定义,这就耗费了时间,并且使得存储类的接口杂乱无章。
        其次,存储类在暴露给展示部分的代码时,可能需要一些根本不想暴露给其他使用者的函数。
        也就是没有办法做到“这个函数值暴露给展示类”。
  \item 第二种选择就是使用友元类以及友元函数给与展示部分的代码访问存储类私有的细节。
        这样可以直接访问存储类的所有私有成员和函数,同时隔离其它的使用者!
\end{enumerate}

\subsubsection*{友元函数}

\textbf{友元函数}是一种被视为类成员并可以访问私有成员的函数。
在所有其他方面,友元函数就像是普通函数那样。
一个友元函数可以是一个普通函数,或者是另一个类的成员函数。
要声明一个友元函数,仅需在函数原型前使用 \textit{friend} 关键字。
在类的私有或者公有部分声明友元函数并没差异。

\begin{lstlisting}[language=C++]
class Accumulator
{
private:
    int m_value { 0 };

public:
    void add(int value) { m_value += value; }

    // 让 reset() 函数成为该类的友元函数
    friend void reset(Accumulator& accumulator);
};

// reset() 现在是 Accumulator 类的友元函数
void reset(Accumulator& accumulator)
{
    // 可以访问 Accumulator 对象的私有数据
    accumulator.m_value = 0;
}

int main()
{
    Accumulator acc;
    acc.add(5); // accumulator 加 5
    reset(acc); // 重置 accumulator 至 0

    return 0;
}
\end{lstlisting}

这个例子中,声明了 \acode{reset()} 函数获取 Accumulator 类的对象,设置 \acode{m_value} 值为 \acode{0}。
因为 \acode{reset()} 不是 Accumulator 类的成员,普通的 \acode{reset()} 不能访问 Accumulator 的私有成员。
然而因为 Accumulator 特别声明了 \acode{reset()} 函数为该类的友元函数,\acode{reset()} 函数被给予了访问 Accumulator 私有成员的权限。

注意这里传递了一个 Accumulator 对象给 \acode{reset()}。
这是因为 \acode{reset()} 不是一个成员函数。
它没有 *this 指针,也没有一个可用的 Accumulator 对象,除非提供了。

这里有另一个例子:

\begin{lstlisting}[language=C++]
#include <iostream>

class Value
{
private:
    int m_value{};

public:
    Value(int value)
        : m_value{ value }
    {
    }

    friend bool isEqual(const Value& value1, const Value& value2);
};

bool isEqual(const Value& value1, const Value& value2)
{
    return (value1.m_value == value2.m_value);
}

int main()
{
    Value v1{ 5 };
    Value v2{ 6 };
    std::cout << std::boolalpha << isEqual(v1, v2);

    return 0;
}
\end{lstlisting}

这个例子中,声明了 \acode{isEqual()} 函数为 Value 类的友元函数。
\acode{isEqual()} 获取两个 Value 对象作为参数。
因为 \acode{isEqual()} 是 Value 类的友元函数,它可以访问 Value 对象的所有私有成员。
这个情况下,比较两个对象,返回 true 如果它们相等。

\subsubsection*{若干友元}

一个函数还可以同时做若干类的友元函数:

\begin{lstlisting}[language=C++]
#include <iostream>

class Humidity;

class Temperature
{
private:
    int m_temp {};

public:
    Temperature(int temp=0)
        : m_temp { temp }
    {
    }

    friend void printWeather(const Temperature& temperature, const Humidity& humidity);
};

class Humidity
{
private:
    int m_humidity {};

public:
    Humidity(int humidity=0)
        : m_humidity { humidity }
    {
    }

    friend void printWeather(const Temperature& temperature, const Humidity& humidity);
};

void printWeather(const Temperature& temperature, const Humidity& humidity)
{
    std::cout << "The temperature is " << temperature.m_temp <<
       " and the humidity is " << humidity.m_humidity << '\n';
}

int main()
{
    Humidity hum{10};
    Temperature temp{12};

    printWeather(temp, hum);

    return 0;
}
\end{lstlisting}

这里有两点需要注意的是:第一,因为 \acode{printWeather} 是两个类的友元函数,
它可以访问两者对象的私有数据。其次,注意第一行:

\begin{lstlisting}[language=C++]
class Humidity;
\end{lstlisting}

这是一个类原型用于告诉编译器之后要定义一个 Humidity 的类。
没有这一行的话编译器会说不知道什么是 Humidity。
类的原型与函数原型的作用一致 -- 告诉编译器有一个这样的东西并在之后会定义。
然而不同于函数的是,类没有返回类型或者参数,所以类原型总是简单的 \acode{class ClassName} 即可。

\subsubsection*{友元类}

同样的,一整个类是另一个类的友元也是可以的。这样让所有友元类的成员都能访问另一个类的私有成员。

\begin{lstlisting}[language=C++]
#include <iostream>

class Storage
{
private:
    int m_nValue {};
    double m_dValue {};
public:
    Storage(int nValue, double dValue)
       : m_nValue { nValue }, m_dValue { dValue }
    {
    }

    // 使 Display 类成为 Storage 类的友元
    friend class Display;
};

class Display
{
private:
    bool m_displayIntFirst;

public:
    Display(bool displayIntFirst)
         : m_displayIntFirst { displayIntFirst }
    {
    }

    void displayItem(const Storage& storage)
    {
        if (m_displayIntFirst)
            std::cout << storage.m_nValue << ' ' << storage.m_dValue << '\n';
        else // 先展示 double
            std::cout << storage.m_dValue << ' ' << storage.m_nValue << '\n';
    }
};

int main()
{
    Storage storage{5, 6.7};
    Display display{false};

    display.displayItem(storage);

    return 0;
}
\end{lstlisting}

关于友元类还有额外的几点需要注意。
首先,即使 \acode{Display} 是 \acode{Storage} 的友元类,\acode{Display} 并没有直接访问 \acode{Storage} 对象的 \*this 指针。
其次,并不意味着 \acode{Storage} 是 \acode{Display} 的友元。
如果想要两个类互为友元,那么需要互相声明对方为友元。
最后,如果类 A 是类 B 的友元,类 B 是 C 的友元,那么也不意味着 A 是 C 的友元。

使用友元函数与类的时候要注意,因为它允许友元函数或类违反封装。
如果类的细节被改变了,那么友元的细节也将被强制改变。
最后,应该最小程度的限制使用友元函数或类。

\subsubsection*{友元成员函数}

与其让整个类成为友元,可以让单个成员函数成为友元。
这与让一个普通函数成为友元类似,不过使用的是带有 className::prefix 的成员函数的名称。

然而实际上这比预期的样子要奇怪一些。上述例子用 Display::displayItem 一个友元函数:

\begin{lstlisting}[language=C++]
#include <iostream>

class Display; // Display 类的前向声明

class Storage
{
private:
  int m_nValue {};
  double m_dValue {};
public:
  Storage(int nValue, double dValue)
    : m_nValue { nValue }, m_dValue { dValue }
  {
  }

  // 使 Display::displayItem 成员函数作为 Storage 类的友元
  friend void Display::displayItem(const Storage& storage); // 错误:Storage 还未看到 Display 类的完整定义
};

class Display
{
private:
  bool m_displayIntFirst {};

public:
  Display(bool displayIntFirst)
    : m_displayIntFirst { displayIntFirst }
  {
  }

  void displayItem(const Storage& storage)
  {
    if (m_displayIntFirst)
      std::cout << storage.m_nValue << ' ' << storage.m_dValue << '\n';
    else
      std::cout << storage.m_dValue << ' ' << storage.m_nValue << '\n';
  }
};
\end{lstlisting}

为了使一个成员函数成为友元,编译器必须看到作为友元成员函数的那个类的所有定义(而不仅仅只是前向声明)。

幸运的是,解决这个问题最简单的做法就是将整个 \acode{Display} 类的定义移动到 \acode{Storage} 类的前方。

\begin{lstlisting}[language=C++]
#include <iostream>

class Display
{
private:
  bool m_displayIntFirst {};

public:
  Display(bool displayIntFirst)
    : m_displayIntFirst { displayIntFirst }
  {
  }

  void displayItem(const Storage& storage) // 错误:编译器并不知道什么是 Storage
  {
    if (m_displayIntFirst)
      std::cout << storage.m_nValue << ' ' << storage.m_dValue << '\n';
    else
      std::cout << storage.m_dValue << ' ' << storage.m_nValue << '\n';
}
};

class Storage
{
private:
  int m_nValue {};
  double m_dValue {};
public:
  Storage(int nValue, double dValue)
    : m_nValue { nValue }, m_dValue { dValue }
  {
  }

  // 使 Display::displayItem 成员函数作为 Storage 类的友元
  friend void Display::displayItem(const Storage& storage); // 现在可以了
};
\end{lstlisting}

然而这有另一个问题。
因为成员函数 \acode{Display::displayItem()} 使用 \acode{Storage} 作为引用参数,
而刚刚将 \acode{Storage} 的定义移动到了 \acode{Display} 的后面。

幸运的是,只需要几步就可以解决这个问题。首先添加 \acode{Storage} 类的前向声明。
其次,移动 \acode{Display::displayItem()} 的定义至类的外部,放在 \acode{Storage} 类的完整定义之后。

\begin{lstlisting}[language=C++]
#include <iostream>

class Storage; // Storage 类的前向声明

class Display
{
private:
  bool m_displayIntFirst {};

public:
  Display(bool displayIntFirst)
    : m_displayIntFirst { displayIntFirst }
  {
  }

  void displayItem(const Storage& storage); // 之前的前向声明在这里的使用需要
};

class Storage // Storage 类的全部定义
{
private:
  int m_nValue {};
  double m_dValue {};
public:
  Storage(int nValue, double dValue)
    : m_nValue { nValue }, m_dValue { dValue }
  {
  }

  // 使 Display::displayItem 成员函数成为 Storage 类的友元(需要看到 Display 类的所有定义)
  friend void Display::displayItem(const Storage& storage);
};

// 现在可以定义 Display::displayItem 了,它需要看到 Storage 类的所有定义
void Display::displayItem(const Storage& storage)
{
  if (m_displayIntFirst)
    std::cout << storage.m_nValue << ' ' << storage.m_dValue << '\n';
  else
    std::cout << storage.m_dValue << ' ' << storage.m_nValue << '\n';
}

int main()
{
  Storage storage(5, 6.7);
  Display display(false);

  display.displayItem(storage);

  return 0;
}
\end{lstlisting}

现在整个思路就清晰了:\acode{Storage} 类的前向定义满足了声明 \acode{Display::displayItem()} 的条件,
接着 \acode{Display} 的完整定义满足了声明 \acode{Display::displayItem()} 作为 \acode{Storage} 友元声明的条件,
接着 \acode{Storage} 类的完整定义满足了 \acode{Display::displayItem()} 作为 \acode{Storage} 成员函数的条件。

这看起来很痛苦 -- 确实如此。
幸运的是,需要这样的操作仅仅是因为想要定义所有的内容在单个文件中。
一个更好的办法是将所有类定义都放在头文件中,而成员函数定义在相应的 .cpp 文件中。这种方式,
所有的类定义都讲可以立刻在 .cpp 文件中看到,也就不再需要重新排列类或者函数了!

\subsubsection*{总结}

友元函数/类是一种函数/类,被视为其他类成员,用于访问其他类私有成员的方式。
这使得友元函数/类可以更加紧密的与其他类工作,而不需要其他类暴露自身的私有成员(例如通过访问函数)。

友元经常在定义重载操作符时使用(详见下一章),或者是两个或以上的类需要以一种更紧密的方式工作。

注意,在让指定的成员函数成为友元时,需要知道所有被友元的类的定义。

\end{document}
