\documentclass[../../LearnCpp.tex]{subfiles}

\begin{document}

\asubsection{16}{Anonymous objects}

某些情况下,用户会需要一个临时变量,例如:

\begin{lstlisting}[language=C++]
#include <iostream>

int add(int x, int y)
{
    int sum{ x + y };
    return sum;
}

int main()
{
    std::cout << add(5, 3) << '\n';

    return 0;
}
\end{lstlisting}

在 \acode{add()} 函数中,注意 \acode{sum} 变量仅仅用作于临时占位变量。
它并没有多大贡献 -- 相反,它仅作用于传递表达式的返回值。

其实还有一种更简单的方式,那就是使用匿名对象来编写\acode{add()} 函数。
\textbf{匿名对象}本质上是一个无名的值。
因为它们没有名字,所以没有办法在它们创建地之外引用它们。
结果而言,它们拥有“表达式作用域”,意为它们被创建,计算,并摧毁在单个表达式中。

以下是使用匿名对象重写的 \acode{add()} 函数:

\begin{lstlisting}[language=C++]
#include <iostream>

int add(int x, int y)
{
    return x + y; // 匿名对象被创建用于存储并返回 x + y 的结果
}

int main()
{
    std::cout << add(5, 3) << '\n';

    return 0;
}
\end{lstlisting}

当 \acode{x + y} 被计算,其结果被存储于一个匿名对象中。
匿名对象的拷贝接着将值返回给调用者,接着匿名对象被销毁。

这不仅仅作用于返回值,同样也作用于函数参数。

\begin{lstlisting}[language=C++]
#include <iostream>

void printValue(int value)
{
    std::cout << value;
}

int main()
{
    int sum{ 5 + 3 };
    printValue(sum);

    return 0;
}
\end{lstlisting}

可以写成这样:

\begin{lstlisting}[language=C++]
#include <iostream>

void printValue(int value)
{
    std::cout << value;
}

int main()
{
    printValue(5 + 3);

    return 0;
}
\end{lstlisting}

\subsubsection*{匿名类对象}

尽管之前的例子使用的都是内建的数据类型,但是为自定义的类构建匿名对象也是可行的。
与普通创建对象类似,但是省略变量名。

\begin{lstlisting}[language=C++]
Cents cents{ 5 }; // 普通变量
Cents{ 7 };       // 匿名对象
\end{lstlisting}

现在来看一个更好的例子:

\begin{lstlisting}[language=C++]
#include <iostream>

class Cents
{
private:
    int m_cents{};

public:
    Cents(int cents)
        : m_cents { cents }
    {}

    int getCents() const { return m_cents; }
};

void print(const Cents& cents)
{
   std::cout << cents.getCents() << " cents\n";
}

int main()
{
    Cents cents{ 6 };
    print(cents);

    return 0;
}
\end{lstlisting}

可以使用匿名对象简化上述代码:

\begin{lstlisting}[language=C++]
#include <iostream>

class Cents
{
private:
    int m_cents{};

public:
    Cents(int cents)
        : m_cents { cents }
    {}

    int getCents() const { return m_cents; }
};

void print(const Cents& cents)
{
   std::cout << cents.getCents() << " cents\n";
}

int main()
{
    print(Cents{ 6 }); // 注意:传递一个匿名 Cents 值

    return 0;
}
\end{lstlisting}

现在来看一个更复杂的例子:

\begin{lstlisting}[language=C++]
#include <iostream>

class Cents
{
private:
    int m_cents{};

public:
    Cents(int cents)
        : m_cents { cents }
    {}

    int getCents() const { return m_cents; }
};

Cents add(const Cents& c1, const Cents& c2)
{
    Cents sum{ c1.getCents() + c2.getCents() };
    return sum;
}

int main()
{
    Cents cents1{ 6 };
    Cents cents2{ 8 };
    Cents sum{ add(cents1, cents2) };
    std::cout << "I have " << sum.getCents() << " cents.\n";

    return 0;
}
\end{lstlisting}

使用匿名值:

\begin{lstlisting}[language=C++]
#include <iostream>

class Cents
{
private:
    int m_cents{};

public:
    Cents(int cents)
        : m_cents { cents }
    {}

    int getCents() const { return m_cents; }
};

Cents add(const Cents& c1, const Cents& c2)
{
    // 列表初始化查看函数的返回类型并根据它们创建正确的对象
    return { c1.getCents() + c2.getCents() }; // 返回匿名 Cents 值
}

int main()
{
    Cents cents1{ 6 };
    Cents cents2{ 8 };
    std::cout << "I have " << add(cents1, cents2).getCents() << " cents.\n";

    return 0;
}
\end{lstlisting}

实际上,因为 \acode{cents1} 与 \acode{cents2} 只被用在一处,可以更进一步的匿名:

\begin{lstlisting}[language=C++]
#include <iostream>

class Cents
{
private:
    int m_cents{};

public:
    Cents(int cents)
        : m_cents { cents }
    {}

    int getCents() const { return m_cents; }
};

Cents add(const Cents& c1, const Cents& c2)
{
    return { c1.getCents() + c2.getCents() }; // 返回 Cents 值
}

int main()
{
    std::cout << "I have " << add(Cents{ 6 }, Cents{ 8 }).getCents() << " cents.\n"; // 打印 Cents 值

    return 0;
}
\end{lstlisting}

\subsubsection*{总结}

C++ 中,匿名对象主要用于传递或者返回值,避免了创建大量的临时变量。
动态的内存分配也是如此匿名(这也是为什么它们的地址必须分配给指针,否则没有办法执行它们)。

同样值得注意的是因为匿名对象拥有表达式作用域,它们仅可以被使用一次
(除非被绑定到左值引用,即延长临时对象的声明周期来匹配引用的生命周期)。
如果需要多次引用一个表达式,那么则需要命名变量。

\end{document}
