\documentclass[../../LearnCpp.tex]{subfiles}

\begin{document}

\asubsection{2}{Classes and class members}

\subsubsection*{类}

面向对象编程的世界中,既希望自定义类型不仅仅存储数据,同样也希望提供处理数据的函数。
C++ 中通过关键字 \textbf{class} 可以定义一个新的 program-defined 类型。

在 C++ 中,类与结构体本质上相同,实际上,下述代码中的结构体与类完全一样:

\begin{lstlisting}[language=C++]
struct DateStruct
{
    int year {};
    int month {};
    int day {};
};

class DateClass
{
public:
    int m_year {};
    int m_month {};
    int m_day {};
};
\end{lstlisting}

注意仅有一个明显的差别是 \textit{public}:类中的关键字。下一章讨论其功能。

类似于结构体的定义,类的定义也不会分配任何内存,它仅仅定义着是类的模样。

\subsubsection*{成员函数}

除了存储数据,类(结构体)可以包含函数!
在类中定义的函数被称为\textbf{成员函数} member functions。
成员函数既可以定义在类定义的内部,也可以是外部。

\begin{lstlisting}[language=C++]
class DateClass
{
public:
    int m_year {};
    int m_month {};
    int m_day {};

    void print() // 定义一个名为 print() 的成员函数
    {
        std::cout << m_year << '/' << m_month << '/' << m_day;
    }
};

int main()
{
    DateClass today { 2020, 10, 14 };

    today.m_day = 16;
    today.print();

    return 0;
}
\end{lstlisting}

当调用“today.print()”时,编译器解释 \acode{m_day} 为 \acode{today.m_day},
\acode{m_month} 为 \acode{today.m_month} 以及 \acode{m_year} 为 \acode{today.m_year}。

通过这样的方式,关联对象被隐式传递到成员函数。
正因如此,它通常被称为 \textbf{the implicit object}。

为成员变量添加“m\_”前缀可以帮助其从函数参数或者成员函数中的本地变量中区分开来。

最佳实践:类命名以大写字母开头。

\begin{lstlisting}[language=C++]
#include <iostream>
#include <string>

class Employee
{
public:
    std::string m_name {};
    int m_id {};
    double m_wage {};

    // 打印员工信息
    void print()
    {
        std::cout << "Name: " << m_name <<
                "  Id: " << m_id <<
                "  Wage: $" << m_wage << '\n';
    }
};

int main()
{
    // 定义两个员工
    Employee alex { "Alex", 1, 25.00 };
    Employee joe { "Joe", 2, 22.25 };

    // 打印员工信息
    alex.print();
    joe.print();

    return 0;
}
\end{lstlisting}

对于非成员函数而言,一个函数不能调用在其“之后”定义的函数(除非有前向声明):

\begin{lstlisting}[language=C++]
void x()
{
// 不可以在此调用 y() 除非编译器已经看到了 y() 的前向声明
}

void y()
{
}
\end{lstlisting}

对于成员函数而言,这个限制不再存在:

\begin{lstlisting}[language=C++]
class foo
{
public:
     void x() { y(); } // 可以在此调用 y(),尽管 y() 定义在类的尾部
     void y() { };
};
\end{lstlisting}

\subsubsection*{成员类型}

除了成员变量以及成员函数,类还可以拥有\textbf{成员类型}或者\textbf{嵌套类型}(包括类型别称)。

\begin{lstlisting}[language=C++]
class Employee
{
public:
    using IDType = int;

    std::string m_name{};
    IDType m_id{};
    double m_wage{};

    void print()
    {
        std::cout << "Name: " << m_name <<
            "  Id: " << m_id <<
            "  Wage: $" << m_wage << '\n';
    }
};
\end{lstlisting}

最佳实践:面对只有数据的结构时,使用 struct 关键字;面对既有数据又有函数的对象而言,使用 class 关键字。

\subsubsection*{已经使用过类的例子}

\begin{lstlisting}[language=C++]
#include <string>
#include <array>
#include <vector>
#include <iostream>

int main()
{
    std::string s { "Hello, world!" }; // 实例化一个字符串类对象
    std::array<int, 3> a { 1, 2, 3 }; // 实例化一个数组类对象
    std::vector<double> v { 1.1, 2.2, 3.3 }; // 实例化一个向量类对象

    std::cout << "length: " << s.length() << '\n'; // 调用成员函数

    return 0;
}
\end{lstlisting}

\end{document}
