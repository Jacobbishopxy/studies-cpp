\documentclass[../../LearnCpp.tex]{subfiles}

\begin{document}

\asubsection{13}{Static member variables}

之前的章节中讲到过 static 变量会一直保留其值,即使在离开作用域后也不会被销毁。

static 关键之应用在全局变量时拥有了另一个含义 -- 给予它们 internal linkage
(即限制它们在所定义的文件范围外可视/使用)属性。
因为全局变量通常来说是需要避免的,因此 static 关键之很少用作这里。

\subsubsection*{static 成员变量}

在应用于类的时候,static 关键字多了两种用法:static 成员变量,以及 static 成员函数。
幸运的是这俩用法都很直接。在这一章节里将讲解 static 成员变量,下一章节讲解 static 成员函数。

\begin{lstlisting}[language=C++]
#include <iostream>

class Something
{
public:
    static int s_value;
};

int Something::s_value{ 1 };

int main()
{
    Something first;
    Something second;

    first.s_value = 2;

    std::cout << first.s_value << '\n';
    std::cout << second.s_value << '\n';
    return 0;
}
\end{lstlisting}

打印:

\begin{lstlisting}
2
2
\end{lstlisting}

\subsubsection*{static 成员并不与类对象关联}

尽管用户可以通过类的对象来访问 static 成员,
实际上在没有任何该类的对象实例化时 static 成员依然存在!
类似于全局变量,它们在程序启动时就被创建,在程序结束时被销毁。

因此,最好是将 static 成员视为属于类的本身,而不是该类的对象。
因为 `s\_value` 独立于任何类对象,
它可以直接使用类名称以及作用域解析操作符来进行访问(即 `Something::s\_value`):

\begin{lstlisting}[language=C++]
#include <iostream>

class Something
{
public:
    static int s_value; // 声明 static 成员变量
};

int Something::s_value{ 1 }; // 定义 static 成员变量(稍后讨论)

int main()
{
    // 注意:这里没有实例化任何 Something 类型的对象

    Something::s_value = 2;
    std::cout << Something::s_value << '\n';
    return 0;
}
\end{lstlisting}

最佳实践:通过类名称访问 static 成员(使用作用域解析操作符)而不是通过类对象来进行访问(使用成员选择操作符)。

\subsubsection*{定义以及初始化 static 成员变量}

当在类的内部定义一个 static 成员变量,那么就是在告诉编译器 static 成员变量的存在,
而不是真正的定义它(更像是前向声明)。
因为 static 成员变量不是类对象的一部分(而是类似于被视为全局变量,并在程序启动时被初始化),
用户必须显式的在类的外部定义 static 成员,即全局作用域。

上述的列子中,通过这一行:

\begin{lstlisting}[language=C++]
int Something::s_value{ 1 }; // 定义 static 成员变量
\end{lstlisting}

这一行代码有两个目的:实例化 static 成员变量(如同全局变量),并可选的初始化它。
这个例子中,提供了初始化值 1.如果没有提供初始化,C++ 将初始化值为 0。

注意这个 static 成员定义不属于访问控制:用户可以定义并初始化变量即使其定义在类中是私有的(或者受保护的)。

如果类是定义在 .h 文件中,static 成员定义通常是放在该类关联的代码文件中(例 Something.cpp)。
如果类是定义在 .cpp 文件中,static 成员定义通常直接放在类的下方。
不要将 static 成员定义在头文件中(类似于全局变量,如果头文件被多次引入,则会导致若干定义而产生 linker 错误)。

\subsubsection*{static 成员变量的内联初始化}

上述方法存在一些捷径。

首先,当 static 成员是一个 const 整数类型(包括 char 与 bool)或者是一个 const 枚举,
那么 static 成员可以在类定义内部初始化:

\begin{lstlisting}[language=C++]
class Whatever
{
public:
    static const int s_value{ 4 }; // 一个 static const int 可以直接被定义以及初始化
};
\end{lstlisting}

其次,从 C++17 开始,可以通过内联声明的方法,初始化非 const static 成员在类定义中:

\begin{lstlisting}[language=C++]
class Whatever
{
public:
    static inline int s_value{ 4 }; // 一个 static inline int 可以直接被定义以及初始化(C++17)
};
\end{lstlisting}

\subsubsection*{案例}

为什么在类内部使用 static 变量呢?一个有用的案例是分配一个唯一 ID 给每个类的实例:

\begin{lstlisting}[language=C++]
#include <iostream>

class Something
{
private:
    static inline int s_idGenerator { 1 };  // C++17
    // static int s_idGenerator;            // C++14 或更早
    int m_id { };

public:
    Something()
    : m_id { s_idGenerator++ } // 从 id 生成器中获取下一个值
    {}

    int getID() const { return m_id; }
};

// 如果是 C++14 或更早版本,用户需要在类定义的外部,初始化非 const static 成员。
// 注意这里定义并初始化了 s_idGenerator,即使上述的声明是私有的。
// 这样做是可以的,因为定义并不属于访问控制。
// int Something::s_idGenerator { 1 }; // 通过 ID 生成器获取值 1(C++14 或更早版本)

int main()
{
    Something first;
    Something second;
    Something third;

    std::cout << first.getID() << '\n';
    std::cout << second.getID() << '\n';
    std::cout << third.getID() << '\n';
    return 0;
}
\end{lstlisting}

打印:

\begin{lstlisting}
1
2
3
\end{lstlisting}

因为 `s\_idGenerator` 被所有的 `Something` 对象所共享,当 `Something` 对象被创建时,
构造函数获取 `s\_idGenerator` 当前值并为下一个对象增加值。
这就保证了每个实例化的 `Something` 对象都获取了唯一 id(顺序创建的)。
这样可以方便数组中包含若干项的调试,因为提供了一种用来分辨同样类型对象的方法!

static 成员变量在类需要利用内部检索表时也很有用(例如用于存储一系列预计算值的数组)。
通过 static 可以使检索表在所有对象中只存在一份,而不是每个对象都实例化一份检索表。
这样可以显著的减少内存使用。

\end{document}
