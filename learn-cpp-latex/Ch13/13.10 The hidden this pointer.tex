\documentclass[../../LearnCpp.tex]{subfiles}

\begin{document}

\asubsection{10}{The hidden "this" pointer}

一个经常被面向对象新手问到的问题,“当一个成员函数被调用时,C++ 是如何追踪是哪个对象被调用了呢?”。
答案就是 C++ 利用了一个隐藏的名为 “this” 的指针!

\subsubsection*{被隐藏的 *this 指针}

\begin{lstlisting}[language=C++]
simple.setID(2);
\end{lstlisting}

尽管调用 \acode{setID()} 函数看起来像是只有一个参数,实际上有两个!
当被编译时,编译器转换 \acode{simple.setID(2);} 成为:

\begin{lstlisting}[language=C++]
setID(&simple, 2); // 注意 simple 从前置的一个对象变为了一个函数的入参!
\end{lstlisting}

注意这仅仅是一个标准的函数调用,同时 simple 对象现在被地址传递成为函数的一个入参。

不过这也只是一半答案。
因为函数调用现在有了额外的参数,成员函数定义需要进行修改来接受(并使用)这个参数。
结果就是以下的成员函数:

\begin{lstlisting}[language=C++]
void setID(int id) { m_id = id; }
\end{lstlisting}

被转换为:

\begin{lstlisting}[language=C++]
void setID(Simple* const this, int id) { this->m_id = id; }
\end{lstlisting}

当编译器编译普通成员函数时,它隐式添加名为“this”的新参数给函数。
**this 指针**是一个隐藏的 const 指针用于存储被调用函数的对象的地址。

这里还有一个细节需要关注。那就是在成员函数内部,任何类成员(函数与变量)同样也需要更新,
这样它们才能指向被调用成员函数的对象。
在它们每个前添加“this ->”前缀很容易。因此函数 \acode{setID()},\acode{m_id}(类成员变量)
可以被转换为 \acode{this->m_id}。
因此当“this”指向 simple 的地址,this->m\_id 成为 shimple.m\_id。

把它们全部联系起来就是:

\begin{enumerate}
    \item 当调用 \acode{simple.setID(2)} 时,编译器实际上调用的是 \acode{setID(&simple, 2)}。
    \item 在 \acode{setID()} 中,“this”指针存储了 simple 对象的地址。
    \item 任何在 \acode{setID()} 中的成员需要“this->”前缀。因此当 \acode{m_id = id} 时,
          编译器实际上执行的是 \acode{this->m\_id = id},这种情况下将 simple.m\_id 更新至 id。
\end{enumerate}

\subsubsection*{“this” 总是指向其被操作的对象}

\begin{lstlisting}[language=C++]
int main()
{
    Simple A{1};  // this = &A 在 Simple 的构造函数内
    Simple B{2};  // this = &B 在 Simple 的构造函数内
    A.setID(3);   // this = &A 在成员函数 setID 内
    B.setID(4);   // this = &B 在成员函数 setID 内

    return 0;
}
\end{lstlisting}

\subsubsection*{显式引用“this”}

\begin{lstlisting}[language=C++]
class Something
{
private:
    int data;

public:
    Something(int data)
    {
        this->data = data; // this->data 为成员,data 为本地参数
    }
};
\end{lstlisting}

\subsubsection*{串联成员函数}

\begin{lstlisting}[language=C++]
#include <iostream>

class Calc
{
private:
    int m_value{};

public:
    Calc& add(int value) { m_value += value; return *this; }
    Calc& sub(int value) { m_value -= value; return *this; }
    Calc& mult(int value) { m_value *= value; return *this; }

    int getValue() { return m_value; }
};

int main()
{
    Calc calc{};
    calc.add(5).sub(3).mult(4);

    std::cout << calc.getValue() << '\n';
    return 0;
}
\end{lstlisting}

\end{document}
