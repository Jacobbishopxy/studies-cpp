\documentclass[../../LearnCpp.tex]{subfiles}

\begin{document}

\asubsection{14}{Static member functions}

上一章讲解了 static 成员变量,知道了它们是属于类的而不是该类对象的。
如果 static 成员变量是公有的,则可以通过类名以及作用域解析操作符直接访问它们。
那么如果 static 成员变量是私有的呢?

考虑以下例子:

\begin{lstlisting}[language=C++]
class Something
{
private:
    static int s_value;

};

int Something::s_value{ 1 }; // 初始化,这是可以的,即使 s_value 的声明是私有的

int main()
{
    // 那么这里该如何访问私有的 Something::s_value 呢?
}
\end{lstlisting}

上述情况在 main 中不可以直接访问 \acode{Something::s\_value},因为它是私有的。
通常而言可以通过公有成员函数来访问私有成员。
而创建一个普通的公有成员函数来访问 \acode{s\_value} 首先需要初始化一个该类的对象才能使用该公有函数!
可以有更好的办法,那就是使函数 static。

类似于 static 成员变量,static 成员函数也没有依附于任何对象。

\begin{lstlisting}[language=C++]
#include <iostream>

class Something
{
private:
    static int s_value;
public:
    static int getValue() { return s_value; } // static 成员函数
};

int Something::s_value{ 1 }; // 初始化

int main()
{
    std::cout << Something::getValue() << '\n';
}
\end{lstlisting}

\subsubsection*{static 成员函数没有 *this 指针}

static 成员函数有两个有趣的怪异之处值得注意。
首先,因为 static 成员函数没有依附于任何对象,它们没有 \textit{this} 指针!
想象一下也很合理 -- \textit{this} 指针指向的总是对象。
static 成员函数并不作用于对象,因此 \textit{this} 指针是不需要的。

其次,static 成员函数可以直接访问其它的 static 成员(变量或函数),
但是不能访问非 static 成员。
这是因为非 static 成员必须属于一个类对象,而 static 成员函数与对象无关!

\subsubsection*{另一个案例}

static 成员函数可以定义在类声明的外边。这与普通的成员函数一样:

\begin{lstlisting}[language=C++]
#include <iostream>

class IDGenerator
{
private:
    static int s_nextID; // static 成员的声明

public:
     static int getNextID(); // static 函数的声明
};

// 在类的外边定义 static 成员。注意这里不要使用 static 关键字。
// 在 1 生成 IDs
int IDGenerator::s_nextID{ 1 };

// 在类的外边定义 static 函数。注意这里不要使用 static 关键字。
int IDGenerator::getNextID() { return s_nextID++; }

int main()
{
    for (int count{ 0 }; count < 5; ++count)
        std::cout << "The next ID is: " << IDGenerator::getNextID() << '\n';

    return 0;
}
\end{lstlisting}

打印:

\begin{lstlisting}
The next ID is: 1
The next ID is: 2
The next ID is: 3
The next ID is: 4
The next ID is: 5
\end{lstlisting}

\subsubsection*{一个关于所有带有 static 成员的类的警告}

注意当编写带有 static 成员的类时,尽管“pure static classes”(也被称为“monostates”)很有用,但是同时也带来了一些缺点。

首先,因为所有的 static 成员只会实例化一次,因此没有办法获得多个纯 static 类(不进行拷贝以及重命名类的情况下)。
例如如果需要两个独立的 IDGenerator 对象,在一个纯 static 类的情况下,这是不可能的。

其次,全局变量是危险的,因为任何微小的代码可以修改其值,并破坏了另一处看似无关的代码。
这同样也是纯 static 类的问题,因为所有成员都属于类(而不是类的对象),
而类的声明通常拥有全局作用域,一个纯 static 类基本上等于把函数以及变量声明在了全局可访问的命名空间,
也就带来了全局变量所拥有的缺点。

\subsubsection*{C++ 不支持 static 构造函数}

如果通过构造函数可以初始换普通成员,那么对其延伸就意味着可以通过 static 构造函数来初始化 static 成员函数。
虽然有一些现代语言确实支持 static 构造函数来达到这个目的,但是不幸的是 C++ 并不是它们这些现代语言其中之一。

如果 static 变量可以被直接初始化,则不需要构造函数:可以在定义的地方直接初始化 static 成员变量(即使是私有的)。
上面的例子就是这么做的。以下是另一个例子:

\begin{lstlisting}[language=C++]
class MyClass
{
public:
  static std::vector<char> s_mychars;
};

std::vector<char> MyClass::s_mychars{ 'a', 'e', 'i', 'o', 'u' }; // 在定义处初始化 static 变量
\end{lstlisting}

如果初始化 static 成员变量需要执行代码(例如循环),则有很多不同的笨拙的方法来完成它。
一种对所有变量,无论是否 static 的方法是使用 lambda 并立刻调用它。

\begin{lstlisting}[language=C++]
class MyClass
{
public:
    static std::vector<char> s_mychars;
};

std::vector<char> MyClass::s_mychars{
  []{ // lambdas 在没有参数时可以省略参数列表
      // 在 lambda 内可以声明另一个 vector 并使用循环
      std::vector<char> v{};

      for (char ch{ 'a' }; ch <= 'z'; ++ch)
      {
          v.push_back(ch);
      }

      return v;
  }() // 立马调用 lambda
};
\end{lstlisting}

下列代码看起来更像是通常的构造函数。然而,它有一点诡异,且应当永远不需要它。

\begin{lstlisting}[language=C++]
class MyClass
{
public:
  static std::vector<char> s_mychars;

  class init_static // 定义一个名为 init_static 的嵌套类
  {
  public:
    init_static() // 初始化构造函数初始化 static 变量
    {
      for (char ch{ 'a' }; ch <= 'z'; ++ch)
      {
        s_mychars.push_back(ch);
      }
    }
  };

private:
  static init_static s_initializer; // 使用这个 static 对象来确保 init_static 构造函数被调用
};

std::vector<char> MyClass::s_mychars{}; // 定义 static 成员变量
MyClass::init_static MyClass::s_initializer{}; // 定义 static 初始化,其调用 init_static 构造函数并初始化 s_mychars
\end{lstlisting}

当 static 成员 s\_initializer 被定义,init\_static() 默认构造函数会被调用(因为 s\_initializer 类型为 init\_static)。可以使用这个构造函数来初始化任何 static 成员变量。这个方法的好处是所有的初始化代码通过 static 成员都被隐藏在原始类中。

\end{document}
