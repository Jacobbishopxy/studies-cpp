\documentclass[../../LearnCpp.tex]{subfiles}

\begin{document}

\asubsection{3}{Introduction to objects and variables}

\subsubsection*{对象与变量}

所有的电脑都有内存,被称为 \textbf{RAM} (random access memory 的简称) 供程序使用。
可以吧 RAM 看作是一系列的数字标记后的信箱,其在程序运行时可以存储一部分数据。
被存储于内存的数据单个部分,被称为\textbf{值}。

C++ 中,直接的内存访问是不被孤立的。
相对的,用户通过一个对象间接的访问内存。
一个\textbf{对象}是一块存储(通常为内存)包含了值或者其他关联的属性。
编译器与操作系统是如何为对象分配内存的问题超出了这个课程。
但是关键点在于,与其说\textit{获取编号 7532 信箱存储的值},不如说\textit{获取该对象的值}。
这意味着用户可以专注于使用对象来存储与获取值,并不用再担心哪块的内存被使用了。

对象可以被命名或者匿名。
一个命名的对象为一个\textbf{变量},该对象的名称叫做一个标识符。

\end{document}
