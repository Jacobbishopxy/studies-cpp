\documentclass[../../LearnCpp.tex]{subfiles}

\begin{document}

\asubsection{3}{Function template specialization}

当用给定类型实例化一个函数模板时,编译器拓印模板函数的副本并用真实类型替换掉模板类型参数。
这意味着对于每个实例化类型(使用不同的类型)特定的函数将拥有同样的实现细节。
大多数时候这正是用户预期的,而在偶尔一些情况下,根据特定数据类型实现一个差异化的模板函数却非常有用。

模板特化就是一种手段。

现在来看一个非常简单的模板类:

\begin{lstlisting}[language=C++]
#include <iostream>

template <typename T>
class Storage
{
private:
    T m_value {};
public:
    Storage(T value)
      : m_value { value }
    {
    }

    void print()
    {
        std::cout << m_value << '\n';
    }
};
\end{lstlisting}

上述代码可以服务于很多种数据类型:

\begin{lstlisting}[language=C++]
int main()
{
    // 定义一些存储单元
    Storage<int> nValue { 5 };
    Storage<double> dValue { 6.7 };

    // 打印一些值
    nValue.print();
    dValue.print();
}
\end{lstlisting}

现在希望浮点值(且仅浮点值)输出的是科学计数法,
这个时候可以使用\textbf{函数模板特化} function template specialization
(有时被称为完全或显式函数模板特化)来为浮点类型创建一个特化版本的 \acode{print()} 函数。
做法相当的简单:仅需定义一个特化函数(如果函数是成员函数,在类定义外同样可以定义)替换模板类型为特定希望重新定义的类型。
这里是一个浮点特化的 \acode{print()} 函数:

\begin{lstlisting}[language=C++]
template <>
void Storage<double>::print()
{
    std::cout << std::scientific << m_value << '\n';
}
\end{lstlisting}

当编译器实例化 \acode{Storage<double>::print()} 时,会看到已经显式定义的函数,并使用它而不是拓印通用模板类。

模板 \acode{<>} 告诉编译器这是一个模板函数,但是没有模板变量(这个例子中已经显式的指定了类型)。

\subsubsection*{另一个例子}

现在来看另一个模板特化的案例。如果尝试使用 \acode{const char*} 作为模板 \acode{Storage} 类的类型。

\begin{lstlisting}[language=C++]
#include <iostream>
#include <string>

template <typename T>
class Storage
{
private:
    T m_value {};
public:
    Storage(T value)
      : m_value { value }
    {
    }

    void print()
    {
        std::cout << m_value << '\n';
    }
};

int main()
{
    // 动态分配临时字符串
    std::string s;

    // 询问名称
    std::cout << "Enter your name: ";
    std::cin >> s;

    // 存储名称
    Storage<char*> storage(s.data());

    storage.print(); // 打印名称

    s.clear(); // 清除 std::string

    storage.print(); // 打印空
}
\end{lstlisting}

第二次的 \acode{storage.print()} 什么都没有打印!那这是为什么呢?

当 \acode{Storage} 为 \acode{char*} 类型实例化时,\acode{Storage<char*>} 的构造函数类似于:

\begin{lstlisting}[language=C++]
template <>
Storage<char*>::Storage(char* value)
      : m_value { value }
{
}
\end{lstlisting}

换言之,这里完成了一个指针赋值(浅拷贝)!
结果 \acode{m_value} 指向了与 \acode{string} 相同的地址。
当 \acode{main()} 中删除 \acode{string} 时,也是删除了 \acode{m_value} 指向的值!
因此尝试打印值的时候得到的是垃圾。

幸运的是,可以通过模板特化来解决这个问题。
相比于使用一个指针拷贝,更加希望的是构造函数获取的是输入字符串的拷贝。

\begin{lstlisting}[language=C++]
template <>
Storage<char*>::Storage(char* const value)
{
    if (!value)
        return;

    // 弄清楚字符串的长度
    int length { 0 };
    while (value[length] != '\0')
        ++length;
    ++length; // +1 用于解释 null 终止符

    // 分配内存用于存储值字符串
    m_value = new char[length];

    // 拷贝实际值字符串给 m_value 刚分配好的内存
    for (int count=0; count < length; ++count)
        m_value[count] = value[count];
}
\end{lstlisting}

现在当分配 \acode{Storage<char*>} 类型变量时,这个构造函数将使用非默认版本的。
接着 \acode{m_value} 将获取字符串的拷贝。
因此当删除字符串时,\acode{m_value} 并不会受到影响。

然而这个类现在对于 \acode{char*} 类型会产生内存泄露,
因为 \acode{m_value} 在 \acode{Storage} 离开作用域时并不会被删除。
如果想着可以由指定的 \acode{Storage<char*>} 析构函数来解决:

\begin{lstlisting}[language=C++]
template <>
Storage<char*>::~Storage()
{
    delete[] m_value;
}
\end{lstlisting}

这样的话当 \acode{Storage<char*>} 类型的变量离开作用域时,
由构造函数产生的内存分配会被特化的析构函数删除。

然而,可能出乎意料,上述特化的析构函数并不会通过编译。
这是因为特化函数必须特化一个显式函数(而不是编译器提供的默认函数)。
由于并没有为 \acode{Storage<T>} 定义一个析构函数,编译器提供了一个默认析构函数,因此不能对其进行特化。
为了解决这个问题,就必须为 \acode{Storage<T>} 定义一个结构函数。

\begin{lstlisting}[language=C++]
#include <iostream>
#include <string>

template <typename T>
class Storage
{
private:
    T m_value{};
public:
    Storage(T value)
        : m_value{ value }
    {
    }
    ~Storage() {}; // 为了特化,需要一个显式定义的析构函数

    void print()
    {
        std::cout << m_value << '\n';
    }
};

template <>
Storage<char*>::Storage(char* const value)
{
    if (!value)
        return;

    // 弄清楚字符串的长度
    int length{ 0 };
    while (value[length] != '\0')
        ++length;
    ++length; // +1 用于解释 null 终止符

    // 分配内存用于存储值字符串
    m_value = new char[length];

    // 拷贝实际值字符串给 m_value 刚分配好的内存
    for (int count = 0; count < length; ++count)
        m_value[count] = value[count];
}

template <>
Storage<char*>::~Storage()
{
    delete[] m_value;
}

int main()
{
    // 动态分配一个临时的字符串
    std::string s;

    // 询问名称
    std::cout << "Enter your name: ";
    std::cin >> s;

    // 保存名称
    Storage<char*> storage(s.data());

    storage.print(); // 打印名称

    s.clear(); // 清除 std::string

    storage.print(); // 打印名称
}
\end{lstlisting}

尽管上述代码中包含的都是成员函数,不过还是可以用同样的方式特化非成员函数。

\end{document}
