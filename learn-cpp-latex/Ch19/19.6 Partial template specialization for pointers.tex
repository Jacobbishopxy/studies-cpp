
\documentclass[../../LearnCpp.tex]{subfiles}

\begin{document}

\asubsection{6}{Partial template specialization for pointers}

19.3 函数模板特化的章节中提到的 \acode{Storage} 类:

\begin{lstlisting}[language=C++]
#include <iostream>

template <typename T>
class Storage
{
private:
    T m_value;
public:
    Storage(T value)
        : m_value { value }
    {
    }

    ~Storage()
    {
    }

    void print() const
    {
        std::cout << m_value << '\n';
    }
};
\end{lstlisting}

之前展示了在 \acode{T} 为 \acode{char*} 类型的时候遇到的问题,因为浅拷贝/指针赋值会发生在构造函数中。
在之前的章节中,用到全模板特化来为 \acode{char*} 类型创建一个特殊版本的构造函数,
使其可以分配内存并对 \acode{m\_value} 进行真正的深拷贝。
作为参考,以下是全特化的 \acode{char*} \acode{Storage} 的构造函数与析构函数:

\begin{lstlisting}[language=C++]
// 需要 include Storage<T> 类

template <>
Storage<char*>::Storage(char* value)
{
  int length { 0 };

  while (value[length] != '\0')
    ++length;
  ++length;

  m_value = new char[length];

  for (int count=0; count < length; ++count)
    m_value[count] = value[count];
}

template<>
Storage<char*>::~Storage()
{
  delete[] m_value;
}
\end{lstlisting}

对于 \acode{Storage<char*>} 而言效果非常的好,那么其他的指针类型(例如 \acode{int*})该怎么办呢?
很容易看得出来如果 \acode{T} 是任何指针类型,之前讨论过的问题仍然都会存在。

由于全模板特化强制全解析了模板类型,为了解决这个问题,所有的指针类型都必须定义一个新的特化构造函数(以及析构函数)!
这又带来了很多重复的代码,那么现在可以用偏特化模板来避免这个问题。

\begin{lstlisting}[language=C++]
#include <iostream>

template <typename T>
class Storage<T*> // 这是一个偏特化的 Storage 仅作用于指针类型
{
private:
    T* m_value;
public:
  Storage(T* value) // 指针类型的 T
    : m_value { new T { *value } } // 拷贝单个值,而不是数组
  {
  }

  ~Storage()
  {
    delete m_value; // 因此这里用的是标量的删除,而不是数组删除
  }

  void print() const
  {
    std::cout << *m_value << '\n';
  }
};
\end{lstlisting}

测试:

\begin{lstlisting}[language=C++]
int main()
{
  Storage<int> myint { 5 };
  myint.print();

  int x { 7 };
  Storage<int*> myintptr { &x };

  // 展示 myintptr 与 x 无关。
  // 如果修改了 x,myintptr 不应该改变。
  x = 9;
  myintptr.print();

  return 0;
}
\end{lstlisting}

打印:

\begin{lstlisting}
5
7
\end{lstlisting}

当 \acode{myintptr} 的定义带有 \acode{int*} 模板参数时,
编译器看到定义得到偏特化模板类可用作于任何指针类型,接着通过模板实例化一个 \acode{Storage},
其构造函数会对参数 \acode{x} 进行深拷贝,当修改 \acode{x} 成 \acode{9} 时,
\acode{myintptr.m_value} 并不收到影响,因为其指向的是自身独立的值。

值得注意的是因为偏特化的 \acode{Storage} 类仅分配单独一个值,
对于 C-style 字符串而言,仅会拷贝第一个字符。
如果需要拷贝整个字符串,则需要为 \acode{char*} 类型全特化一个构造函数(以及析构函数)。
全特化的版本优先级高于偏特化的版本。
以下是一个为指针使用的偏特化,以及为 \acode{char*} 全特化的例子:

\begin{lstlisting}[language=C++]
#include <iostream>
#include <cstring>

// 非指针的 Storage 模板类
template <typename T>
class Storage
{
private:
  T m_value;
public:
  Storage(T value)
    : m_value { value }
  {
  }

  ~Storage()
  {
  }

  void print() const
  {
    std::cout << m_value << '\n';
  }
};

// 指针的 Storage 偏特化模板类
template <typename T>
class Storage<T*>
{
private:
  T* m_value;
public:
  Storage(T* value)
    : m_value { new T { *value } } // 这里拷贝单独一个值,而不是一个数组
  {
  }

  ~Storage()
  {
    delete m_value;
  }

  void print() const
  {
    std::cout << *m_value << '\n';
  }
};

// char* 类型的全特化的构造函数
template <>
Storage<char*>::Storage(char* value)
{
  // 弄清楚 string 的长度
  int length { 0 };
  while (value[length] != '\0')
    ++length;
  ++length; // +1 用于解释 null 终止符

  // 分配内存用于存储 string 值
  m_value = new char[length];

  // 拷贝实际的 string 值到刚分配的 m_value 内存
  for (int count = 0; count < length; ++count)
    m_value[count] = value[count];
}

// char* 类型的全特化的析构函数
template<>
Storage<char*>::~Storage()
{
  delete[] m_value;
}

// char* 类型的全特化的打印函数
// 没有这个函数打印 Storage<char*> 将会调用 Storage<T*>::print(),则只会打印第一个字符
template<>
void Storage<char*>::print() const
{
  std::cout << m_value;
}

int main()
{
  // 声明一个非指针 Storage 来证明可以工作
  Storage<int> myint { 5 };
  myint.print();

  // 声明一个指针 Storage 来证明可以工作
  int x { 7 };
  Storage<int*> myintptr { &x };

  // 如果 myintptr 对 x 实施的是指针赋值,那么修改 x 也会修改 myintptr
  x = 9;
  myintptr.print();

  // 动态分配一个临时 string
  char* name { new char[40]{ "Alex" } };

  // 存储名称
  Storage<char*> myname { name };

  // 删除临时 string
  delete[] name;

  // 打印名称来证明构造时进行了拷贝
  myname.print();

  return 0;
}
\end{lstlisting}

打印:

\begin{lstlisting}
5
7
Alex
\end{lstlisting}

使用偏函数模板类特化来分别创建指针与非指针的实现非常的有用,
尤其是希望一个类能够分别处理这两种情况,但是某种层度上对端点使用者是完全透明的。

\end{document}
