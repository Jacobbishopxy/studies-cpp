\documentclass[../../LearnCpp.tex]{subfiles}

\begin{document}

\asubsection{4}{Class template specialization}

上一节讲述了如何特化函数来为特定的数据类型提供不同的功能。
事实证明,不仅可以特化函数,还可以特化整个类!

这里有一个简化的类:

\begin{lstlisting}[language=C++]
template <typename T>
class Storage8
{
private:
    T m_array[8];

public:
    void set(int index, const T& value)
    {
        m_array[index] = value;
    }

    const T& get(int index) const
    {
        return m_array[index];
    }
};
\end{lstlisting}

因为这是一个模板类,它可以对任意给定类型生效:

\begin{lstlisting}[language=C++]
#include <iostream>

int main()
{
    // Define a Storage8 for integers
    Storage8<int> intStorage;

    for (int count{ 0 }; count < 8; ++count)
        intStorage.set(count, count);

    for (int count{ 0 }; count < 8; ++count)
        std::cout << intStorage.get(count) << '\n';

    // Define a Storage8 for bool
    Storage8<bool> boolStorage;
    for (int count{ 0 }; count < 8; ++count)
        boolStorage.set(count, count & 3);

    std::cout << std::boolalpha;

    for (int count{ 0 }; count < 8; ++count)
    {
        std::cout << boolStorage.get(count) << '\n';
    }

    return 0;
}
\end{lstlisting}

尽管这个类完全有效,但是 \acode{Storage8<bool>} 的实现非常的低效。
因为所有变量都有地址,同时 CPU 无法处理任何小于一字节的东西,所有变量都必须是至少一个字节的大小。
因此,布尔型的值使用了整个字节,即使技术上而言它只需要一个 bit 来存储 true 与 false 值!
也就是说 1 bit 是有用的信息而其他 7 bit 浪费了空间。

为了消除浪费的空间,需要对类进行修改以满足布尔型,替换 8 个布尔值的数组为单个字节大小。
虽然可以创建一个完全新的类来完成目标,但是却会有一个最大的缺点:
需要给其一个不同的名字。
那么程序员就必须记住 \acode{Storage8<T>} 是用于非布尔类型的,而 \acode{Storage8Bool}(或者其他名字)是用于布尔类型的。
幸运的是,C++ 提供了一个更好的方法:类模板特化。

\subsubsection*{类模板特化}

类模板特化允许用户为了特定数据类型(若干模板参数时同样成立)特化一个模板类。
这个例子中使用类模板特化编写一个自定义版本的 \acode{Storage8<bool>} 优先级高于通用的 \acode{Storage8<T>}。
其工作方式类似于特化函数的优先级高于通用的模板函数。

类模板特化被视为完全独立的类,即使它们的内存分配方式与模板类相同。
这就意味这可以修改特化类中的所有的东西,包括其实现,甚至于公有化函数,正如一个独立的类一般。

\begin{lstlisting}[language=C++]
// 需要上面 Storage8 的类型定义

template <> // 以下是一个不带有模板参数的模板类
class Storage8<bool> // 开始为 bool 特化 Storage8
{
// 依照标准的类实现细节
private:
    unsigned char m_data{};

public:
    void set(int index, bool value)
    {
        // 弄清楚那个 bit 是 setting/unsetting
        // 放置 1 进 bit 用于 on/off
        auto mask{ 1 << index };

        if (value)  // 如果是 setting bit
            m_data |= mask;   // 使用 bitwise-or 使 bit on
        else  // 如果是 bit off
            m_data &= ~mask;  // 使用 bitwise-and 以及反转 mask 使 bit off
    }

    bool get(int index)
    {
        // 弄清楚获取的是哪个 bit
        auto mask{ 1 << index };
        // bitwise-and 获取 bit 值,接着隐式转换为布尔值
        return (m_data & mask);
    }
};
\end{lstlisting}

首先,注意由 \acode{template<>} 开始。
template 关键字告诉编译器下面都是模板,同时空的尖括号意味着没有任何模板参数。
本例中没有任何模板参数是因为替换了仅有的参数(T)为指定的类型(bool)。

接着,添加 \acode{<bool>} 给类名来表示正在特化一个 bool 版本的 \acode{Storage8} 类。

所有其他的修改都是类的实现细节。现在暂时不需要理解 bit 逻辑是如何工作的。

注意特化类使用了一个 \acode{unsigned char}(1 bit)而不是 8 个 bool 的数组(8 字节)。

现在当声明一个 \acode{Storage8<T>} 而 \acode{T} 不为布尔使,
获取的是拓印通用模板 \acode{Storage8<T>} 类,
而当声明类型为 \acode{Storage8<bool>} 时,获取的是刚创建好的特化版本。
注意在两种模板中都用的是共有暴露接口 -- 而 C++ 给与了全力来添加,移除,
或者修改 \acode{Storage8<bool>},保持一致的接口意味着程序员可以使用拥有完全相同行为的类。

\end{document}
