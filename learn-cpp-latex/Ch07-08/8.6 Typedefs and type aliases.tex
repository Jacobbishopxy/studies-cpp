\documentclass[../../LearnCpp.tex]{subfiles}

\begin{document}

\asubsection{6}{Typedefs and type aliases}

\subsubsection*{Type 别名}

C++ 中,\textbf{using} 是一个用于对现有数据类型创建别名的关键词。

\begin{lstlisting}[language=C++]
using Distance = double; // 定义 Distance 实则为 double 类型

Distance milesToDestination{ 3.4 }; // 定义变量
\end{lstlisting}

\subsubsection*{Naming type aliases}

历史原因,没有关于类型如何命名的一致性。通常有三种常见的命名转换:

\begin{itemize}
    \item 类型别名带有 “\_t” 后缀(“type” 的缩写)。这类转换通常用于标准库的全局域类型名称(例如 \acode{size\_t} 以及 \acode{nullptr\_t})。
          这类转换有 C 继承而来,在用户自定义类型(有时别的类型)别名是尤为常见,但是这不太符合现代 C++ 的习惯。
    \item 类型别名带有 “\_type” 后缀。这类转换用于一些标准库类型(例如 \acode{std::string})用于嵌套类型的别名(例如 \acode{std::string::size\_type})。
          但是还有一些嵌套类型完全没有后缀(例如 \acode{std::string::iterator}),因此这种用法最好是非连贯的。
    \item 不带后缀的类型别名。现代 C++ 中使用无后缀。大写字母可以帮助从变量以及函数名称(小写字母开头)中辨别类型名称,并防止它们命名冲突。
\end{itemize}

\begin{lstlisting}[language=C++]
void printDistance(Distance distance); // Distance 在别处被定义
\end{lstlisting}

最佳实践:自定义类型使用大写的方式并且不带后缀(除非有非常特别的原因)。

\subsubsection*{类型别名的作用域}

类型别名标识符与变量标识符的作用域相同。如果需要跨文件使用,可以在头文件中定义:

\begin{lstlisting}[language=C++]
#ifndef MYTYPES
#define MYTYPES

    using Miles = long;
    using Speed = long;

#endif
\end{lstlisting}

\subsubsection*{Typedefs}

\textbf{typedef}(“type definition” 的缩写)是创建类型别名的旧方法。

\begin{lstlisting}[language=C++]
// The following aliases are identical
typedef long Miles;
using Miles = long;
\end{lstlisting}

因为向后兼容的原因 Typedefs 仍然在 C++ 中存在,不过在现代 C++ 中,它们已经大量的被类型别名给替换了。

最佳实践:使用类型别名而不是 typedefs。

\end{document}
