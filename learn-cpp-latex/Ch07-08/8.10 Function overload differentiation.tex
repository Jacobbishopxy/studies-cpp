\documentclass[../../LearnCpp.tex]{subfiles}

\begin{document}

\asubsection{10}{Function overload differentiation}

\subsubsection*{重载还是是如何区分的}

\begin{center}
    \begin{tiny}
        \begin{tabularx}{ 1\textwidth}{
                | >{\raggedright\arraybackslash}X
                | >{\raggedright\arraybackslash}X
                | >{\raggedright\arraybackslash}X |
            }
            \hline
            Function property    & Used for differentiation & Notes                                                                                        \\
            \hline
            Number of parameters & Yes                      &                                                                                              \\
            Type of parameters   & Yes                      & Excludes typedefs, type aliases, and const qualifier on value parameters. Includes ellipses. \\
            Return type          & No                       &                                                                                              \\
            \hline
        \end{tabularx}
    \end{tiny}
\end{center}

对于成员函数,额外的函数等级限定符也会被考虑到:

\begin{center}
    \begin{tiny}
        \begin{tabularx}{ 1\textwidth}{
                | >{\raggedright\arraybackslash}X
                | >{\raggedright\arraybackslash}X |
            }
            \hline
            Function-level qualifier & Used for overloading \\
            \hline
            const or volatile        & Yes                  \\
            Ref-qualifiers           & Yes                  \\
            \hline
        \end{tabularx}
    \end{tiny}
\end{center}

\subsubsection*{根据入参数的重载}

\begin{lstlisting}[language=C++]
int add(int x, int y)
{
    return x + y;
}

int add(int x, int y, int z)
{
    return x + y + z;
}
\end{lstlisting}

\subsubsection*{根据参数类型的重载}

\begin{lstlisting}[language=C++]
int add(int x, int y); // 整数
double add(double x, double y); // 浮点
double add(int x, double y); // 混合
double add(double x, int y); // 混合
\end{lstlisting}

暂时还没有覆盖到省略号的内容,但是省略号参数被视为唯一类型的参数:

\begin{lstlisting}[language=C++]
void foo(int x, int y);
void foo(int x, ...); // 与 foo(int, int) 不同
\end{lstlisting}

\subsubsection*{函数返回值类型的不同不被认为是有区分}

\begin{lstlisting}[language=C++]
int getRandomValue();
double getRandomValue();
\end{lstlisting}

上述会导致编译器报错。

\end{document}
