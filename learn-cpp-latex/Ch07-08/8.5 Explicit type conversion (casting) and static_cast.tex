\documentclass[../../LearnCpp.tex]{subfiles}

\begin{document}

\asubsection{5}{Explicit type conversion (casting) and static\_cast}

8.1 中讨论了编译器可以隐式转换值类型通过一个 \acode{implicit type conversion} 系统。
当用户希望提升一个值类型至更宽泛的类型,使用隐式转换是可以的。

很多新手尝试这样做:

\begin{lstlisting}[language=C++]
double d = 10 /4; // 整数除法,d 初始值为 2.0
\end{lstlisting}

因为 \acode{10} 和 \acode{4} 的类型皆为 \acode{int} ,整数除法被执行,表达式计算 \acode{int} 值为 \acode{2} 。
该值再经历数值转换成 \acode{double} 值 \acode{2.0} 。这样很可能不是符合预期。

字面值的情况下用户可以手动改为带有小数的值,然而如果是变量,如下:

\begin{lstlisting}[language=C++]
int x { 10 };
int y { 4 };
double d = x / y; // 结果还是 2.0
\end{lstlisting}

好在 C++ 有不同的\textbf{类型转换操作符} type casting operators(通常被称为 \textbf{casts})使用户可以请求编译器进行类型转换。
因为转换是用户显式的请求,这些类型转换通常被称为\textbf{显式类型转换} explicit type conversion(与隐式类型转换相对)。

\subsubsection*{类型 casting}

C++ 提供五中类型的 casts:\acode{C-style casts} ,\acode{static casts} ,\acode{const casts} ,\acode{dynamic casts} 以及 \acode{reinterpret casts} 。
后面的四种方式有时被称为 \textbf{named casts}。

相关内容:18.10 中讲解 dynamic casts。

警告:除非有很好的理由,否则不要用 const casts 以及 reinterpret casts。

\subsubsection*{C-style casts}

在标准 C 编程里,casts 是通过 () 操作符,需要转换的类型放置在圆括号中。

\begin{lstlisting}[language=C++]
#include <iostream>

int main()
{
    int x { 10 };
    int y { 4 };


    double d { (double)x / y }; // 转换 x 为 double 使其可以做除法
    std::cout << d; // 打印 2.5

    return 0;
}
\end{lstlisting}

尽管 \acode{C-style cast} 看起来是单 cast,在不同的上下文中仍然会产生不同结果。
它可以包含 \acode{static casts} ,\acode{const casts} 或者 \acode{reinterpret casts}(后两者之前提过是要尽量避免的)。
因此 \acode{C-style cast} 会有使用不当的风险并产生非预期的行为。

最佳实践:避免使用 \acode{C-style cast} 。

\subsubsection*{static\_cast}

C++ 引入了一个 casting 操作符称为 \textbf{static\_cast},可以被用于值转换。

\begin{lstlisting}[language=C++]
#include <iostream>

int main()
{
    int x { 10 };
    int y { 4 };

    // static cast x 至 double 类型用于除法计算
    double d { static_cast<double>(x) / y };
    std::cout << d; // 打印 2.5

    return 0;
}
\end{lstlisting}

最佳实践:推荐 static\_cast。

\end{document}
