\documentclass[../../LearnCpp.tex]{subfiles}

\begin{document}

\asubsection{13}{Function templates}

\subsubsection*{C++ 模版简介}

C++ 中,模版系统用于简化创建可以用作于不同数据类型的函数(或类)的过程。

\textbf{模版} template 描述函数或者类的样子。
不同于普通的定义(即所有类型需要被定义),模版中可以使用多个占位符类型。
占位符类型代表着一些类型在编写模版时暂时未知,但是在之后会被提供。

一旦模版被定义,编译器可以使用模版来生成所需要的若干重载函数(或类),每个都使用不同的真实类型!

最后得到了几乎所有的函数或类(每组都有不同的类型),而仅仅只需创建和维护单个模版,编译器则做了其他所有的活。

重点:编译器可以使用单个模版生成一组相关函数或类,每个使用一组不同的类型。

重点:在编写模版时,其可以作用于暂时还不存在的类型。这使得模版代码兼具了灵活性以及未来的扩展性。

\subsubsection*{函数模板}

\textbf{函数模版}是类函数定义的,使用于生成一个或多个重载函数,并带有不同真实类型的集合。
也就是说可以让用户创建函数可以作用于不同的类型。

当创建函数模版时,使用占位符类型(同样也被称为\textbf{模版类型} template types)来对应任何参数类型,
返回类型,或是函数体中的类型。

最佳实践:使用单个大写字母(起始于 T)用于命名模版类型(例如 T,U,V,等等)。

\textbf{模版参数声明}:

\begin{lstlisting}[language=C++]
template <typename T> // 模版参数声明在此
T max(T x, T y) // 函数模版定义 max<T>
{
    return (x > y) ? x : y;
}
\end{lstlisting}

\end{document}
