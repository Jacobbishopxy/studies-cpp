\documentclass[../../LearnCpp.tex]{subfiles}

\begin{document}

\asubsection{3}{Aggregation}

\subsubsection*{聚合}

被视为\textbf{聚合} aggregation,一个对象以及其部分必须符合以下关系:

\begin{itemize}
  \item 部分(成员)是对象(类)的一部分
  \item 部分(成员)同一时间可以属于不止一个对象(类)
  \item 部分(成员)的存在\textit{没有}被对象(类)管理
  \item 部分(成员)不知道对象(类)的存在
\end{itemize}

与组合类似,聚合仍然是“部分-整体”的关系,“部分”包含在“整体”内,同时也是单向关系。
而与组合不同的是,“部分”在同一时间内可以属于多个对象,并且对象对其“部分”的存活并不负责。
当一个聚合被创建时,没有创建其部分的责任;当聚合被销毁时,也没有销毁其部分的责任。

可以说聚合构建了“has-a”模型关系。

类似于组合,聚合的“部分”可以是单个的或者多个的。

\subsubsection*{实现聚合}

由于聚合与组合类似都是“部分-整体”的关系,它们的实现几乎一致,其中的差异更多在于语义上。
在组合内,通常使用普通成员变量(或者是由组合类自身管理内存分配与释放的指针)来添加“部分”。
在聚合内,同样添加成员变量。然而这些成员变量通常是用于指向在类外部创建的对象的引用或者指针。
结果而言,一个聚合通常要么是使用指向的对象作为构造函数参数,要么是为空并在之后通过函数或者操作符添加子对象。

由于这些“部分”是存在于类的外部作用域,当类被销毁时,指针或者引用成员变量也会被销毁(但是不删除)。
结果而言,“部分”的本体仍然存在。

\begin{lstlisting}[language=C++]
#include <functional> // std::reference_wrapper
#include <iostream>
#include <string>
#include <vector>

class Teacher
{
  private:
  std::string m_name{};

  public:
  Teacher(const std::string& name)
      : m_name{name}
  {
  }

  const std::string& getName() const { return m_name; }
};

using Teachers = std::vector<std::reference_wrapper<Teacher>>;

class Department
{
  private:
  Teachers m_teachers{};

  public:
  Department(const Teachers teachers)
      : m_teachers{teachers}
  {
  }

  void add(Teacher& teacher) { m_teachers.push_back(teacher); }
};

int main(int argc, char const* argv[])
{
  Teacher jacob{"Jacob"};
  Teacher april{"April"};

  Teachers teachers{};

  {
    Department department{teachers};

    department.add(jacob);
    department.add(april);
  }

  std::cout << jacob.getName() << " still exists!\n";
  std::cout << april.getName() << " still exists!\n";

  return 0;
}
\end{lstlisting}

\subsubsection*{建模时选择正确的关系}

最佳实践: 实现最简单的关系类型用于满足项目需求,而不是现实世界中看起来对的那个。

\end{document}
