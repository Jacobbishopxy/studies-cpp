\documentclass[../../LearnCpp.tex]{subfiles}

\begin{document}

\asubsection{6}{Container classes}

\textbf{容器类} container class 是一种被设计用于存储并组织若干其他类型实例(别的类,或是基础类型)的类。
有不同类型的容器类,它们每个都有各种各样的优势,劣势以及限制。迄今为止看到最多的容器就是数组。
尽管 C++ 拥有内建的数组功能,程序员通常还是会使用数组容器类(`std::array` 或 `std::vector`),
因为它们提供了更多的便利。
不同于内建数组,数组容器类通常提供动态扩缩(当元素被添加或删除),
当传递给函数时记下它们的大小,并且进行边界检查。
这不仅使数组容器类比起普通数组更加的便利,同时也更加的安全。

容器类通常实现了最小标准的一系列功能。大多数设计优秀的容器都会包含以下功能:

\begin{itemize}
  \item 创建一个空容器(通过构造函数)
  \item 插入新对象进入容器
  \item 从容器中移除一个对象
  \item 报告容器中现有的对象数
  \item 清除容器
  \item 提供访问存储对象的能力
  \item 元素排序(可选)
\end{itemize}

有时候一些特定的容器类会省略上述的一些功能。
例如数组容器类通常省略了插入以及移除功能,因为它们很慢并且类的设计者并不鼓励这么使用。

容器类实现了 member-of 关系。例如,一个数组中的元素是 members-of(属于)这个数组的。

\subsubsection*{容器的类型}

容器类通常有两种不同的种类。
\textbf{值容器} value containers 作为组合 compositions 用于存储它们所持有的对象的副本(因此负责创建与销毁这些副本)。
\textbf{引用容器} reference containers 作为聚合 aggregations 用于存储其他对象的指针或引用(因此不负责创建或销毁这些对象)。

不同于现实生活的容器可以持有任何类型的对象,C++ 中的容器通常只能持有一种类型的数据。
尽管使用上有限制,容器还是非常的有用,同时它们也使得编程编的更加简单,安全,与快速。

\end{document}
