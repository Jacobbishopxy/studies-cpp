\documentclass[../../LearnCpp.tex]{subfiles}

\begin{document}

\asubsection{4}{Association}

\subsubsection*{关联}

被视为\textbf{关联},一个对象与另一个对象之间必须符合以下关系:

\begin{itemize}
    \item 关联对象(成员)除了关联关系以外,与对象(类)无其它关系
    \item 关联对象(成员)同一时间可以属于不止一个对象(类)
    \item 关联对象(成员)的存在\textit{没有}被对象(类)管理
    \item 关联对象(成员)有可能知道对象(类)的存在
\end{itemize}

不同于组合与聚合,”部分“是整个对象的一部分,在关联关系中,被关联的对象与对象没有其他关系。
类似于聚合,被关联对象可以同时属于若干对象,且不由这些对象关联。
然而不同于聚合,即总是单向关系的,关联关系中,可以使单向或者双向的(即两个对象感知对方)。

\subsubsection*{实现关联}

doctor\_patient.h:

\begin{lstlisting}[language=C++]
#include <functional>
#include <iostream>
#include <string>
#include <vector>

class Patient;

class Doctor
{
    private:
    std::string m_name{};
    std::vector<std::reference_wrapper<const Patient>> m_patient{};

    public:
    Doctor(const std::string& name)
        : m_name{name}
    {
    }

    void addPatient(Patient& patient);

    friend std::ostream& operator<<(std::ostream& out, const Doctor& doctor);

    const std::string& getName() const { return m_name; }
};

class Patient
{
    private:
    std::string m_name{};
    std::vector<std::reference_wrapper<const Doctor>> m_doctor{};

    void addDoctor(const Doctor& doctor) { m_doctor.push_back(doctor); }

    public:
    Patient(const std::string& name)
        : m_name{name}
    {
    }

    friend std::ostream& operator<<(std::ostream& out, const Patient& patient);

    const std::string& getName() const { return m_name; }

    friend void Doctor::addPatient(Patient& patient);
};

void Doctor::addPatient(Patient& patient)
{
    m_patient.push_back(patient);

    patient.addDoctor(*this);
}

std::ostream& operator<<(std::ostream& out, const Doctor& doctor)
{
    if (doctor.m_patient.empty())
    {
    out << doctor.m_name << " has no patients right now";
    return out;
    }

    out << doctor.m_name << " is seeing patients: ";
    for (const auto& patient : doctor.m_patient)
    out << patient.get().getName() << ' ';

    return out;
}

std::ostream& operator<<(std::ostream& out, const Patient& patient)
{
    if (patient.m_doctor.empty())
    {
    out << patient.getName() << " has no doctors right now";
    return out;
    }

    out << patient.m_name << " is seeing doctors: ";
    for (const auto& doctor : patient.m_doctor)
    out << doctor.get().getName() << ' ';

    return out;
}
\end{lstlisting}

\subsubsection*{自身关联}

有时候对象可能与同样类型的其他对象拥有关系。
这被称为\textbf{自身关联} reflexive association。
一个比较好的例子就是大学课程与其前置课程(同样也是大学课程)之间的关系。

\begin{lstlisting}[language=C++]
#include <string>
class Course
{
private:
    std::string m_name;
    const Course* m_prerequisite;

public:
    Course(const std::string& name, const Course* prerequisite = nullptr):
        m_name{ name }, m_prerequisite{ prerequisite }
    {
    }

};
\end{lstlisting}

这样就可以实现关联链(一个课程有前置课程,前置课程也有其前置课程等等)。

\subsubsection*{关联可以使非直接的}

之前所有的案例中,使用的要么是指针要么是引用来直接连接对象。
然而在关联关系中,这并不是严格需要的。任何种类的数据允许连个对象链接在一起就足够了。

car\_driver.h:

\begin{lstlisting}[language=C++]
#include <iostream>
#include <string>

class Car
{
  private:
  std::string m_name;
  int m_id;

  public:
  Car(const std::string& name, int id)
      : m_name{name}, m_id{id}
  {
  }

  const std::string& getName() const { return m_name; }
  int getId() const { return m_id; }
};

// CarLot 本质上是一个 static 的 Car 数组,以及一个用于获取它们的查询函数。
// 因为是 static 的,不需要为其分配一个 CarLot 类型的对象
class CarLot
{
  private:
  static Car s_carLot[4];

  public:
  CarLot(/* args */) = delete; // 确保不会创建一个 CarLot

  static Car* getCar(int id)
  {
    for (int count{0}; count < 4; ++count)
    {
      if (s_carLot[count].getId() == id)
      {
        return &(s_carLot[count]);
      }
    }

    return nullptr;
  }
};

Car CarLot::s_carLot[4]{
    {"Prius", 4}, {"Corolla", 17}, {"Accord", 84}, {"Matrix", 62}};

class Driver
{
  private:
  std::string m_name;
  int m_carId; // 可以通过 ID 而不是指针来关联 Car

  public:
  Driver(const std::string& name, int carId)
      : m_name{name}, m_carId{carId}
  {
  }

  const std::string& getName() const { return m_name; }
  int getCarId() const { return m_carId; }
};
\end{lstlisting}

\subsubsection*{组合 vs 聚合 vs 关联}

\begin{center}
    \begin{tiny}
        \begin{tabularx}{ 1\textwidth}{
                | >{\raggedright\arraybackslash}X
                | >{\raggedright\arraybackslash}X
                | >{\raggedright\arraybackslash}X
                | >{\raggedright\arraybackslash}X |
            }
            \hline
            Property                               & Composition    & Aggregation    & Association                     \\
            \hline
            Relationship type                      & Whole/part     & Whole/part     & Otherwise unrelated             \\
            Members can belong to multiple classes & No             & Yes            & Yes                             \\
            Members’ existence managed by class    & Yes            & No             & No                              \\
            Directionality                         & Unidirectional & Unidirectional & Unidirectional or bidirectional \\
            Relationship verb                      & Part-of        & Has-a          & Uses-a                          \\
            \hline
        \end{tabularx}
    \end{tiny}
\end{center}

\end{document}
