\documentclass[../../LearnCpp.tex]{subfiles}

\begin{document}

\asubsection{5}{Dependencies}

当一个对象的构建需要另一个对象的功能才能完成时,\textbf{依赖} dependency 则会出现。
这种关系比关联还要弱,不过任何对被依赖的对象的修改都有可能破坏依赖者的功能。
依赖永远是单向关系。

\begin{lstlisting}[language=C++]
#include <iostream>

class Point
{
private:
    double m_x{};
    double m_y{};
    double m_z{};

public:
    Point(double x=0.0, double y=0.0, double z=0.0): m_x{x}, m_y{y}, m_z{z}
    {
    }

    friend std::ostream& operator<<(std::ostream& out, const Point& point); // Point 依赖于 std::ostream
};

std::ostream& operator<< (std::ostream& out, const Point& point)
{
    // 因为 operator<< 是 Point 类的友元,可以直接访问 Point 的成员
    out << "Point(" << point.m_x << ", " << point.m_y << ", " << point.m_z << ')';

    return out;
}

int main()
{
    Point point1 { 2.0, 3.0, 4.0 };

    std::cout << point1; // 项目依赖 std::cout

    return 0;
}
\end{lstlisting}

\subsubsection*{依赖 vs 关联}

关于依赖与关联的差异,通常会有一些疑惑。

在 C++ 中,关联是在于两个类之间的类等级关系。
也就是说一个类与其关联的类作为成员,保持一个直接或间接的”连接“。

依赖通常来说不在类等级上表现 -- 也就是说被依赖的对象并不作为成员来进行连接。
而是,被依赖的对象通常在实例化时被需要,或者是作为一个参数传递进一个函数。

\end{document}
