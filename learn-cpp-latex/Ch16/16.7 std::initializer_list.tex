\documentclass[../../LearnCpp.tex]{subfiles}

\begin{document}

\asubsection{7}{std::initializer\_list}

考虑一个固定的整型数组:

\begin{lstlisting}[language=C++]
int array[5];
\end{lstlisting}

如果使用值来初始化这个数组,可以直接通过初始化列表语法:

\begin{lstlisting}[language=C++]
#include <iostream>

int main()
{
  int array[] { 5, 4, 3, 2, 1 }; // 初始化列表
  for (auto i : array)
    std::cout << i << ' ';

  return 0;
}
\end{lstlisting}

同样也可以对动态分配的数组生效:

\begin{lstlisting}[language=C++]
#include <iostream>

int main()
{
  auto* array{ new int[5]{ 5, 4, 3, 2, 1 } }; // 初始化列表
  for (int count{ 0 }; count < 5; ++count)
    std::cout << array[count] << ' ';
  delete[] array;

  return 0;
}
\end{lstlisting}

上一章中介绍了容器类的概念,并且展示了 IntArray 类持有整型数组的例子:

\begin{lstlisting}[language=C++]
#include <cassert> // assert()
#include <iostream>

class IntArray
{
private:
    int m_length{};
    int* m_data{};

public:
    IntArray() = default;

    IntArray(int length)
        : m_length{ length }
        , m_data{ new int[length]{} }
    {
    }

    ~IntArray()
    {
        delete[] m_data;
        // 这里不需要设置 m_data 为 null 或者 m_length 为 0,因为在这个函数完成之后,对象会立刻被销毁。
    }

    int& operator[](int index)
    {
        assert(index >= 0 && index < m_length);
        return m_data[index];
    }

    int getLength() const { return m_length; }
};

int main()
{
  // 如果尝试使用容器类进行初始化列表会发生什么呢?
  IntArray array { 5, 4, 3, 2, 1 }; // 这一行不能编译
  for (int count{ 0 }; count < 5; ++count)
    std::cout << array[count] << ' ';

  return 0;
}
\end{lstlisting}

代码不能被编译的原因是 \acode{IntArray} 不具有一个构造函数用于初始化列表。
结果还是只能通过单独的初始化每个元素:

\begin{lstlisting}[language=C++]
int main()
{
  IntArray array(5);
  array[0] = 5;
  array[1] = 4;
  array[2] = 3;
  array[3] = 2;
  array[4] = 1;

  for (int count{ 0 }; count < 5; ++count)
    std::cout << array[count] << ' ';

  return 0;
}
\end{lstlisting}

这很不好。

\subsubsection*{使用 std::initializer\_list 进行类的初始化}

当编译器看到初始化列表,则会自动转换它成为一个 \acode{std::initializer\_list} 类型的对象。
因此,如果创建一个接受 \acode{std::initializer\_list} 参数的构造函数,
就可以使用初始化列表来创建对象了。

\acode{std::initializer\_list} 位于 \acode{<initializer\_list>} 头文件。

关于 \acode{std::initializer\_list} 有几点需要知道的。
类似于 \acode{std::array} 或 \acode{std::vector},
需要通过尖括号告诉 \acode{std::initializer\_list} 数据的类型,
除非是直接对 \acode{std::initializer\_list} 进行初始化。
因此几乎见不到一个纯 \acode{std::initializer\_list},
而是类似于 \acode{std::initializer\_list<int>} 或者 \acode{std::initializer\_list<std::string>}。

其次,\acode{std::initializer\_list} 拥有一个 \acode{size()} 函数(取名不当),
其返回列表中元素的数量。
这对于需要知道传递的列表长度时非常有用。

现在看一下添加了入参为 \acode{std::initializer\_list} 构造函数,更新后的代码:

\begin{lstlisting}[language=C++]
#include <cassert> // assert()
#include <initializer_list> // std::initializer_list
#include <iostream>

class IntArray
{
private:
  int m_length {};
  int* m_data {};

public:
  IntArray() = default;

  IntArray(int length)
    : m_length{ length }
    , m_data{ new int[length]{} }
  {

  }

  IntArray(std::initializer_list<int> list) // 允许 IntArray 列表初始化
    : IntArray(static_cast<int>(list.size())) // 使用委派构造函数设置初始数组
  {
    // 现在从 list 开始初始化数组
    int count{ 0 };
    for (auto element : list)
    {
      m_data[count] = element;
      ++count;
    }
  }

  ~IntArray()
  {
    delete[] m_data;
  }

  IntArray(const IntArray&) = delete; // 为了避免浅拷贝
  IntArray& operator=(const IntArray& list) = delete; // 为了避免浅拷贝

  int& operator[](int index)
  {
    assert(index >= 0 && index < m_length);
    return m_data[index];
  }

  int getLength() const { return m_length; }
};

int main()
{
  IntArray array{ 5, 4, 3, 2, 1 }; // 初始化列表
  for (int count{ 0 }; count < array.getLength(); ++count)
    std::cout << array[count] << ' ';

  return 0;
}
\end{lstlisting}

现在来看看细节:

\begin{lstlisting}[language=C++]
IntArray(std::initializer_list<int> list)
  : IntArray(static_cast<int>(list.size()))
{
  int count{ 0 };
  for (int element : list)
  {
    m_data[count] = element;
    ++count;
  }
}
\end{lstlisting}

第一行:根据之前的描述使用尖括号表示列表中元素的类型。
本例中因为是 \acode{IntArray} 因此为 int。
注意这里没有传递 const 引用的 list。
类似于 \acode{std::string_view},\acode{std::initializer\_list} 是非常轻量的,同时其拷贝是相对廉价的。

第二行:通过委派构造函数(减少冗余代码)委派分配内存给 \acode{IntArray} 至另一个构造函数。
另一个构造函数需要知道数组的长度,因此这里传递了 \acode{list.size()},即列表中的元素数量。
注意 \acode{list.size()} 返回的是 \acode{size\_t}(即无符号的),因此还需要转型为有符号的 int。
这里使用的是直接初始化而不是花括号初始化,因为花括号初始化倾向于使用列表构造函数。
尽管构造函数可以被正确解析,使用直接初始化来初始化类更加的安全。

构造函数的函数体保留用于从列表中拷贝元素至 \acode{IntArray} 类中。
由于一些难以说明的原因,\acode{std::initializer\_list} 并没有提供通过下标(操作符[])来访问元素的功能。

然而缺乏下标功能可以被绕过。这里最简单的方法就是使用 for-each 循环。
通过 ranged-base 的 for 循环,可以遍历每个初始化列表中的元素,手动的拷贝这些元素进入内部的数组。

一个警告:初始化列表总是喜欢一个匹配的 \acode{initializer\_list} 构造函数胜过于其他的潜在匹配构造函数。
因此,如下的变量定义:

\begin{lstlisting}[language=C++]
IntArray array { 5 };
\end{lstlisting}

将匹配 \acode{IntArray(std::initializer\_list<int>)} 而不是 \acode{IntArray(int)}。
在列表构造函数被定义后,如果还希望匹配到 \acode{IntArray(int)},那么则需要拷贝初始化或者直接初始化。
这也与 \acode{std::vector} 以及其他容器类相似,它们都有列表构造函数以及同类单个入参的构造器:

\begin{lstlisting}[language=C++]
std::vector<int> array(5); // 调用 std::vector::vector(std::vector::size_type),5 value-initialized elements: 0 0 0 0 0
std::vector<int> array{ 5 }; // 调用 std::vector::vector(std::initializer_list<int>),1 element: 5
\end{lstlisting}

\subsubsection*{使用 std::initializer\_list 进行类的赋值}

同样也可以使用 \acode{std::initializer\_list} 复制操作符来获取 \acode{std::initializer\_list} 参数给一个类进行赋值。
这与上述例子类似。

注意如果要实现一个以 \acode{std::initializer\_list} 为入参的构造函数,就必须要确保满足以下至少一点:

\begin{enumerate}
  \item 提供重载列表赋值操作符
  \item 提供合适的深拷贝的拷贝复制操作符
  \item 删除拷贝复制操作符
\end{enumerate}

原因是:考虑以下代码中的类(即没有满足上述三点中的任何一点),与列表赋值声明一起:

\begin{lstlisting}[language=C++]
#include <cassert> // assert()
#include <initializer_list> // std::initializer_list
#include <iostream>

class IntArray
{
private:
  int m_length{};
  int* m_data{};

public:
  IntArray() = default;

  IntArray(int length)
    : m_length{ length }
    , m_data{ new int[length] {} }
  {

  }

  IntArray(std::initializer_list<int> list)
    : IntArray(static_cast<int>(list.size()))
  {
    int count{ 0 };
    for (auto element : list)
    {
      m_data[count] = element;
      ++count;
    }
  }

  ~IntArray()
  {
    delete[] m_data;
  }

  // IntArray(const IntArray&) = delete; // 为了避免浅拷贝
  // IntArray& operator=(const IntArray& list) = delete; // 为了避免浅拷贝

  int& operator[](int index)
  {
    assert(index >= 0 && index < m_length);
    return m_data[index];
  }

  int getLength() const { return m_length; }
};

int main()
{
  IntArray array{};
  array = { 1, 3, 5, 7, 9, 11 }; // 这里是列表赋值声明

  for (int count{ 0 }; count < array.getLength(); ++count)
    std::cout << array[count] << ' ';

  return 0;
}
\end{lstlisting}

首先,编译器会注意到接受 \acode{std::initializer\_list} 的赋值函数并不存在。
接着会查看其他可以使用的赋值函数,然后发现隐式提供的拷贝赋值操作符。
然而,这个函数仅可用于转换初始化列表成为 \acode{IntArray}。
因为 \acode{\{ 1, 3, 5, 7, 9, 11 \}} 是一个 \acode{std::initializer\_list},
编译器将会使用列表构造函数来转换一个初始化列表成为一个临时的 \acode{IntArray}。
接着讲调用隐式赋值操作符,即浅拷贝临时的 \acode{IntArray} 给数组对象。

这个时候,无论是临时的 \acode{IntArray} 的 \acode{m_data} 与 \acode{array->m\_data} 指向了相同的地址(因为浅拷贝的原因)。
已经可以看到后面会发生什么了。

在赋值声明的最后,临时的 \acode{IntArray} 被摧毁。
析构函数的调用删除了临时 \acode{IntArray} 中的 \acode{m\_data}。
也就留下 \acode{array->m\_data} 变为悬垂指针。
当尝试使用 \acode{array->m\_data} 时则会导致未定义行为。

最佳实践:如果提供了列表构造函数,再提供一个列表赋值是一个好主意。

\end{document}
