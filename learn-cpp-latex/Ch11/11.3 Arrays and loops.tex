\documentclass[../../LearnCpp.tex]{subfiles}

\begin{document}

\asubsection{3}{Arrays and loops}

当一个循环用于访问每个元素时,通常被称为\textbf{遍历}整个数组。

\begin{lstlisting}[language=C++]
constexpr int scores[]{ 84, 92, 76, 81, 56 };
constexpr int numStudents{ static_cast<int>(std::size(scores)) };
// const int numStudents{ sizeof(scores) / sizeof(scores[0]) }; // 当 C++17 不可用时请使用这种方式
int totalScore{ 0 };

// 使用循环计算总分数
for (int student{ 0 }; student < numStudents; ++student)
    totalScore += scores[student];

auto averageScore{ static_cast<double>(totalScore) / numStudents };
\end{lstlisting}

\begin{lstlisting}[language=C++]
#include <iostream>
#include <iterator> // std::size

int main()
{
    constexpr int scores[]{ 84, 92, 76, 81, 56 };
    constexpr int numStudents{ static_cast<int>(std::size(scores)) };

    int maxScore{ 0 }; // 保存最大的分数
    for (int student{ 0 }; student < numStudents; ++student)
    {
        if (scores[student] > maxScore)
        {
            maxScore = scores[student];
        }
    }

    std::cout << "The best score was " << maxScore << '\n';

    return 0;
}
\end{lstlisting}

\end{document}
