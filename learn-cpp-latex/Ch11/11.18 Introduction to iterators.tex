\documentclass[../../LearnCpp.tex]{subfiles}

\begin{document}

\asubsection{18}{Introduction to iterators}

遍历整个数组(或者其他结构)的数据在编程中非常常见。
现阶段覆盖了多种方式:通过循环以及索引(\acode{for-loops} 以及 \acode{while lopps}),
通过指针和指针算数,以及通过 \acode{range-based for-loops}:

\begin{lstlisting}[language=C++]
#include <array>
#include <cstddef>
#include <iostream>

int main()
{
    // C++17 中,变量 data 的类型被推导为 std::array<int, 7>
    // 如果报编译错误,详见下方的警告
    std::array data{ 0, 1, 2, 3, 4, 5, 6 };
    std::size_t length{ std::size(data) };

    // 带有显式索引的 while 循环
    std::size_t index{ 0 };
    while (index < length)
    {
        std::cout << data[index] << ' ';
        ++index;
    }
    std::cout << '\n';

    // 带有显式索引的 for 循环
    for (index = 0; index < length; ++index)
    {
        std::cout << data[index] << ' ';
    }
    std::cout << '\n';

    // 带有指针的 for 循环(注意:ptr 不能为 const,因为需要增加它)
    for (auto ptr{ &data[0] }; ptr != (&data[0] + length); ++ptr)
    {
        std::cout << *ptr << ' ';
    }
    std::cout << '\n';

    // ranged-based 的 for 循环
    for (int i : data)
    {
        std::cout << i << ' ';
    }
    std::cout << '\n';

    return 0;
}
\end{lstlisting}

警告:上述例子使用了一个 C++17 的特性名为 \acode{class template argument deduction} 用于从模版变量的初始化中推导模版参数。
例子中,当编译器看到 \acode{std::array data{0, 1, 2, 3, 4, 5, 6};} 时,
会推导成 \acode{std::array<int, 7> data {...}}。

\subsubsection*{迭代器}

\textbf{迭代器} iterator是一个设计用来穿越整个容器的对象(例如,数组中的值,或者字符串中的字符),提供了访问每个元素的能力。

一个容器可能会提供不同种类的迭代器。例如,一个数组可以提供一个前向迭代器以及一个反方向的后向迭代器。

一旦合适类型的迭代器被创建,程序员便可以使用迭代器所提供的接口来穿越并访问元素,而不再需要担心使用哪种方式或者数据是如何存储于容器中的。因为 C++ 迭代器通常使用相同的接口(操作符 ++ 来移动至下一个元素)并访问(操作符 \* 来访问当前元素),用户可以使用始终如一的方法来遍历不同类型的容器。

\subsubsection*{指针作为迭代器}

最简单的迭代器类型就是一个指针,(使用指针算法)作用于有序的内存:

\begin{lstlisting}[language=C++]
#include <array>
#include <iostream>

int main()
{
    std::array data{ 0, 1, 2, 3, 4, 5, 6 };

    auto begin{ &data[0] };
    // 注意指向超出了最后一个元素
    auto end{ begin + std::size(data) };

    // 带有指针的 for-loop
    for (auto ptr{ begin }; ptr != end; ++ptr) // ++ 用于移动至下一个元素
    {
        std::cout << *ptr << ' '; // 解引用获取当前元素的值
    }
    std::cout << '\n';

    return 0;
}
\end{lstlisting}

\subsubsection*{标准库迭代器}

遍历是常见的操作,因此所有的标准库容器都直接提供了遍历的支持。不需要用户计算起始以及终止点,可以直接通过函数 \acode{begin()} 与 \acode{end()} 来获取起始与终止点:

\begin{lstlisting}[language=C++]
#include <array>
#include <iostream>

int main()
{
    std::array array{ 1, 2, 3 };

    // 向数组请求起始与终止点(通过各自的成员函数)
    auto begin{ array.begin() };
    auto end{ array.end() };

    for (auto p{ begin }; p != end; ++p) // ++ 移动至下一个元素
    {
        std::cout << *p << ' '; // 解引用获取当前元素的值
    }
    std::cout << '\n';

    return 0;
}
\end{lstlisting}

\acode{iterator} 头文件同样也包含了泛型方法(\acode{std::begin} 与 \acode{std::end}):

\begin{lstlisting}[language=C++]
#include <array>
#include <iostream>
#include <iterator> // std::begin 与 std::end

int main()
{
    std::array array{ 1, 2, 3 };

    // 使用 std::begin 以及 std::end 来获取起始点以及终止点
    auto begin{ std::begin(array) };
    auto end{ std::end(array) };

    for (auto p{ begin }; p != end; ++p) // ++ 移动至下一个元素
    {
        std::cout << *p << ' '; // 解引用获取当前元素的值
    }
    std::cout << '\n';

    return 0;
}
\end{lstlisting}

现在暂时不用担心迭代器的类型,之后的章节中会详细讲解。
这里的关键点在于迭代器处理了遍历容器的所有细节。
用户仅需要做四件事情:起始点,终止点,操作符 ++ 移动迭代器至下一个元素(或者结束),
以及操作符 \* 用于获取当前值。

\subsubsection*{迭代器失效(悬垂迭代器)}

类似于指针已经引用,如果元素已经被遍历过了,迭代器也可以成为“悬垂”。
当这种情况发生时可以说改迭代器已经\textbf{失效} invalidated 了。
访问单个迭代器会导致未定义行为。

\begin{lstlisting}[language=C++]
#include <iostream>
#include <vector>

int main()
{
    std::vector v{ 1, 2, 3, 4, 5, 6, 7 };

    auto it{ v.begin() };

    ++it; // 移动至下一个元素
    std::cout << *it << '\n'; // ok: 打印 2

    v.erase(it); // 擦除现在正在遍历的元素

    // erase() 使迭代器擦除元素(以及其后面的元素)
    // 因此迭代器 "it" 不再有效

    ++it; // 未定义行为
    std::cout << *it << '\n'; // 未定义行为

    return 0;
}
\end{lstlisting}

\end{document}
