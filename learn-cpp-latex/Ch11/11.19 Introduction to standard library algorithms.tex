\documentclass[../../LearnCpp.tex]{subfiles}

\begin{document}

\asubsection{19}{Introduction to standard library algorithms}

由 C++ 算数标准库提供的功能主要分为三种类型:

\begin{itemize}
    \item \textbf{Inspectors} -- 用于观察(但不修改)容器中的数据。例如查询以及计数。
    \item \textbf{Mutators} -- 用于修改容器中的数据。例如排序以及乱序。
    \item \textbf{Facilitators} -- 用于由数据成员生成结果。例如乘法,或者以何种形式的方式排序。
\end{itemize}

\subsubsection*{使用 std::find 通过值寻找元素}

\begin{lstlisting}[language=C++]
#include <algorithm>
#include <array>
#include <iostream>

int main()
{
    std::array arr{ 13, 90, 99, 5, 40, 80 };

    std::cout << "Enter a value to search for and replace with: ";
    int search{};
    int replace{};
    std::cin >> search >> replace;

    // 省略输入检查

    // std::find 返回一个迭代器,指向找到的元素(或是容器的末尾)
    // 用一个变量存储它,使用类型推导做该迭代器的类型(因为不关心)
    auto found{ std::find(arr.begin(), arr.end(), search) };

    // 算法在找不到目标的情况下返回迭代器的末尾。
    // 使用成员函数 end() 访问
    if (found == arr.end())
    {
        std::cout << "Could not find " << search << '\n';
    }
    else
    {
        // 替换值
        *found = replace;
    }

    for (int i : arr)
    {
        std::cout << i << ' ';
    }

    std::cout << '\n';

    return 0;
}
\end{lstlisting}

\subsubsection*{使用 std::find\_if 寻找匹配符合某种条件的元素}

\begin{lstlisting}[language=C++]
#include <algorithm>
#include <array>
#include <iostream>
#include <string_view>

// 函数将返回 true 如果元素匹配
bool containsNut(std::string_view str)
{
    // 如果没有找到子字符串 std::string_view::find 返回 std::string_view::npos
    // 否则返回子字符串出现的索引。
    return (str.find("nut") != std::string_view::npos);
}

int main()
{
    std::array<std::string_view, 4> arr{ "apple", "banana", "walnut", "lemon" };

    // 检索数组查看是否有元素包含 “nut” 子字符串
    auto found{ std::find_if(arr.begin(), arr.end(), containsNut) };

    if (found == arr.end())
    {
        std::cout << "No nuts\n";
    }
    else
    {
        std::cout << "Found " << *found << '\n';
    }

    return 0;
}
\end{lstlisting}

\subsubsection*{使用 std::count 与 std::count\_if 统计出现次数}

\acode{std::count} 统计所有出现的元素; \acode{std::count\_if} 统计所有符合条件的元素。

\begin{lstlisting}[language=C++]
#include <algorithm>
#include <array>
#include <iostream>
#include <string_view>

bool containsNut(std::string_view str)
{
    return (str.find("nut") != std::string_view::npos);
}

int main()
{
    std::array<std::string_view, 5> arr{ "apple", "banana", "walnut", "lemon", "peanut" };

    auto nuts{ std::count_if(arr.begin(), arr.end(), containsNut) };

    std::cout << "Counted " << nuts << " nut(s)\n";

    return 0;
}
\end{lstlisting}

\subsubsection*{使用 std::sort 进行自定义排序}

\acode{std::sort} 可以接收一个函数作为第三个参数用于实现自定义排序。该函数接收两个用作比较的参数,如果第一个参数排序在第二个参数之前返回 true。默认情况下 \acode{std::sort} 进行的是升序排序。

\begin{lstlisting}[language=C++]
#include <algorithm>
#include <array>
#include <iostream>

bool greater(int a, int b)
{
    // 排序 @a 在 @b 之前,如果 @a 大于 @b
    return (a > b);
}

int main()
{
    std::array arr{ 13, 90, 99, 5, 40, 80 };

    // 传递 greater 函数给 std::sort
    std::sort(arr.begin(), arr.end(), greater);

    for (int i : arr)
    {
        std::cout << i << ' ';
    }

    std::cout << '\n';

    return 0;
}
\end{lstlisting}

\subsubsection*{使用 std::for\_each 对一个容器中所有元素操作}

\acode{std::for\_each} 接收一个列表作为输入,并应用自定义函数于每个元素。

\begin{lstlisting}[language=C++]
#include <algorithm>
#include <array>
#include <iostream>

void doubleNumber(int& i)
{
    i *= 2;
}

int main()
{
    std::array arr{ 1, 2, 3, 4 };

    std::for_each(arr.begin(), arr.end(), doubleNumber);

    for (int i : arr)
    {
        std::cout << i << ' ';
    }

    std::cout << '\n';

    return 0;
}
\end{lstlisting}

\subsubsection*{执行顺序}

注意标准库中的大部分算法并不保证执行的顺序。这类的算法,只关心用户传递的函数,并不假设特定顺序,因为不同的编译器顺序的调用可能不同。

下列算法保证顺序执行:\acode{std::for\_each},\acode{std::copy},\acode{std::copy\_backward},\acode{std::move} 与 \acode{std::move_backward}。

最佳实践:除非有特别的指出,不要假设标准库的算法能确保执行的顺序。

\end{document}
