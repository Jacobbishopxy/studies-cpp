\documentclass[../../LearnCpp.tex]{subfiles}

\begin{document}

\asubsection{9}{Pointer arithmetic and array indexing}

\subsubsection*{指针算数}

C++ 允许对指针进行整数加减法的操作。
如果 \acode{ptr} 指向一个整数,那么 \acode{ptr + 1} 则是 ptr 在内存中的下一个整数的地址;
\acode{ptr - 1} 则是上一个整数的地址。

注意 \acode{ptr + 1} 不会返回 \acode{ptr} 之后的\textbf{内存地址},
而是\textbf{下一个类型的对象}的内存地址。

当计算指针算数表达式的结果时,编译器总是会乘上对象的大小。这就叫做 \textbf{scaling}。

\begin{lstlisting}[language=C++]
#include <iostream>

int main()
{
    int value{ 7 };
    int* ptr{ &value };

    std::cout << ptr << '\n';
    std::cout << ptr+1 << '\n';
    std::cout << ptr+2 << '\n';
    std::cout << ptr+3 << '\n';

    return 0;
}
\end{lstlisting}

打印类似于:

\begin{lstlisting}
0012FF7C
0012FF80
0012FF84
0012FF88
\end{lstlisting}

而下面的代码:

\begin{lstlisting}[language=C++]
#include <iostream>

int main()
{
    short value{ 7 };
    short* ptr{ &value };

    std::cout << ptr << '\n';
    std::cout << ptr+1 << '\n';
    std::cout << ptr+2 << '\n';
    std::cout << ptr+3 << '\n';

    return 0;
}
\end{lstlisting}

打印类似于:

\begin{lstlisting}
0012FF7C
0012FF7E
0012FF80
0012FF82
\end{lstlisting}

\subsubsection*{指针算数,数组,以及 indexing 背后的魔法}

\begin{lstlisting}[language=C++]
#include <iostream>

int main()
{
     int array[]{ 9, 7, 5, 3, 1 };

     std::cout << &array[1] << '\n';  // 打印数组元素 1 的内存地址
     std::cout << array+1 << '\n';    // 打印数组指针 + 1 的内存地址

     std::cout << array[1] << '\n';   // 打印 7
     std::cout << *(array+1) << '\n'; // 打印 7(注意这里圆括号是必须的)

    return 0;
}
\end{lstlisting}

打印类似于:

\begin{lstlisting}
0017FB80
0017FB80
7
7
\end{lstlisting}

也就是说,当编译器看到下标操作符(\acode{[]}),则会转化数组成为指针加法!
归纳就是,当 n 是整数时,\acode{array[n]} 等同于 \acode{*(array + n)}。
下标操作符看起来更好看同时更加简单易用(不再需要记住圆括号)。

\subsubsection*{使用指针遍历数组}

\begin{lstlisting}[language=C++]
#include <iostream>
#include <iterator> // std::size

bool isVowel(char ch)
{
    switch (ch)
    {
    case 'A':
    case 'a':
    case 'E':
    case 'e':
    case 'I':
    case 'i':
    case 'O':
    case 'o':
    case 'U':
    case 'u':
        return true;
    default:
        return false;
    }
}

int main()
{
    char name[]{ "Mollie" };
    int arrayLength{ static_cast<int>(std::size(name)) };
    int numVowels{ 0 };

    // name + arrayLength 为数组中最后一个 char 的内存地址
    //
    for (char* ptr{ name }; ptr != (name + arrayLength); ++ptr)
    {
        if (isVowel(*ptr))
        {
            ++numVowels;
        }
    }

    std::cout << name << " has " << numVowels << " vowels.\n";

    return 0;
}
\end{lstlisting}

由于元素计算很常见,标准库提供了 \acode{std::count_if} 方法:

\begin{lstlisting}[language=C++]
#include <algorithm>
#include <iostream>
#include <iterator> // std::begin 与 std::end

bool isVowel(char ch)
{
    switch (ch)
    {
    case 'A':
    case 'a':
    case 'E':
    case 'e':
    case 'I':
    case 'i':
    case 'O':
    case 'o':
    case 'U':
    case 'u':
        return true;
    default:
        return false;
    }
}

int main()
{
    char name[]{ "Mollie" };

    // 遍历所有的元素并计数
    //
    auto numVowels{ std::count_if(std::begin(name), std::end(name), isVowel) };

    std::cout << name << " has " << numVowels << " vowels.\n";

    return 0;
}
\end{lstlisting}

\acode{std::begin} 返回一个从第一个元素开始的迭代器(指针);
\acode{std::end} 返回一个从最后一个元素开始的迭代器(指针),该迭代器仅作为标记,
访问它会导致未定义行为,因为它并没有指向一个真实的元素。

\acode{std::begin} 与 \acode{std::end} 仅仅作用于已知大小的数组。
如果数组衰退成指针,则可以手动计算其起始。

\begin{lstlisting}[language=C++]
// nameLength 是数组的元素数
std::count_if(name, name + nameLength, isVowel)

// 不要使用以下代码。访问非法索引会导致未定义行为
// std::count_if(name, &name[nameLength], isVowel)
\end{lstlisting}

\end{document}
