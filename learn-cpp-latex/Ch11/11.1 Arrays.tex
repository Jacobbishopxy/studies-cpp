\documentclass[../../LearnCpp.tex]{subfiles}

\begin{document}

\asubsection{1}{Arrays}

\textbf{数组} array是一种聚合类型允许用户通过单个标识符访问若干同样类型的变量。

\begin{lstlisting}[language=C++]
int testScore[30]{}; // 分配 30 个整数变量在固定的数组
\end{lstlisting}

任何 array 变量的声明中,使用方括号(`[]`)告诉编译器这是一个数组变量(而不是普通变量),
同时有多少个变量要分配(\textbf{数组长度} array length)。

上述例子声明了一个名为 testScore 并带有 30 长度的固定数组。
\textbf{固定数组}的长度在编译时已知。
当 testScore 实例化后,30 个整数将被分配。

数组中每个变量被称为\textbf{元素}。
元素没有独立的名称,而是通过\textbf{下标操作符} subscript operator([])以及\textbf{下标} subscript(或 index)来访问数组名称获取元素。
这个过程被称为 \textbf{subscripting} 或 \textbf{indexing} 数字。

\begin{lstlisting}[language=C++]
#include <iostream>

int main()
{
    int prime[5]{};
    prime[0] = 2;
    prime[1] = 3;
    prime[2] = 5;
    prime[3] = 7;
    prime[4] = 11;

    std::cout << "The lowest prime number is: " << prime[0] << '\n';
    std::cout << "The sum of the first 5 primes is: " << prime[0] + prime[1] + prime[2] + prime[3] + prime[4] << '\n';

    return 0;
}
\end{lstlisting}

最佳实践:显式实例化数组(即便元素类型是 self-initializing 的)。

规则:使用数组时请确保下标不要超出范围!

\end{document}
