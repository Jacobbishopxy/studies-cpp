\documentclass[../../LearnCpp.tex]{subfiles}

\begin{document}

\asubsection{13}{For-each loops}

在 11.3 中使用了 \textit{for 循环}来遍历数组中的所有元素:

\begin{lstlisting}[language=C++]
#include <iostream>
#include <iterator> // std::size

int main()
{
    constexpr int scores[]{ 84, 92, 76, 81, 56 };
    constexpr int numStudents{ std::size(scores) };

    int maxScore{ 0 };
    for (int student{ 0 }; student < numStudents; ++student)
    {
        if (scores[student] > maxScore)
        {
            maxScore = scores[student];
        }
    }

    std::cout << "The best score was " << maxScore << '\n';

    return 0;
}
\end{lstlisting}

有另一种简单且安全的循环被称为 \textbf{for-each} 循环(也被称为 \textbf{range-based for-loop}),
可以遍历数组中的所有元素(或者其它 list-type 结果)。

\subsubsection*{For-each 循环}

\textit{for-each} 声明的语法类似于:

\begin{lstlisting}
for (element_declaration : array)
   statement;
\end{lstlisting}

\begin{lstlisting}[language=C++]
#include <iostream>

int main()
{
    constexpr int fibonacci[]{ 0, 1, 1, 2, 3, 5, 8, 13, 21, 34, 55, 89 };
    for (int number : fibonacci) // 遍历 fibonacci 数组
    {
       std::cout << number << ' '; // 访问数组元素
    }

    std::cout << '\n';

    return 0;
}
\end{lstlisting}

\subsubsection*{For each 循环以及 auto 关键字}

由于元素声明需要与数组元素类型一致,这里使用 \acode{auto} 关键字是一个好主意,
让 C++ 帮助进行类型推导。

\begin{lstlisting}[language=C++]
#include <iostream>

int main()
{
    constexpr int fibonacci[]{ 0, 1, 1, 2, 3, 5, 8, 13, 21, 34, 55, 89 };
    for (auto number : fibonacci) // 类型为 auto,number 的类型由数组进行推导
    {
       std::cout << number << ' ';
    }

    std::cout << '\n';

    return 0;
}
\end{lstlisting}

\subsubsection*{For each 循环以及引用}

\begin{lstlisting}[language=C++]
std::string array[]{ "peter", "likes", "frozen", "yogurt" };
for (auto element : array) // 元素将被拷贝
{
    std::cout << element << ' ';
}
\end{lstlisting}

这就意味着遍历到的每个数组元素都会被拷贝,这就非常的昂贵,大多数时候仅仅只需要引用原来的元素。幸运的是可以使用引用:

\begin{lstlisting}[language=C++]
std::string array[]{ "peter", "likes", "frozen", "yogurt" };
for (auto& element: array) // & 符号可以使元素成为引用,阻止拷贝发生
{
    std::cout << element << ' ';
}
\end{lstlisting}

上述例子,元素会变成引用,另外任何对元素的修改都会影响被遍历的数组。因此有时候使用 \acode{const} 引用意为只需要可读功能:

\begin{lstlisting}[language=C++]
std::string array[]{ "peter", "likes", "frozen", "yogurt" };
for (const auto& element: array) // 元素现在是 const 引用
{
    std::cout << element << ' ';
}
\end{lstlisting}

最佳实践:在 for-each 循环元素声明中,如果元素不是基础类型,
因为性能原因,推荐使用引用或者 \acode{const} 引用。

\subsubsection*{For-each 不能作用于数组指针}

由于需要遍历整个数组,for-each 需要知道数组的大小,这就意味着数组长度已知。
因为数组会退化成指针而丢失大小,for-each 则不能工作!

\begin{lstlisting}[language=C++]
#include <iostream>

int sumArray(const int array[]) // array 为指针
{
    int sum{ 0 };

    for (auto number : array) // 编译错误,array 的大小不可知
    {
        sum += number;
    }

    return sum;
}

int main()
{
     constexpr int array[]{ 9, 7, 5, 3, 1 };

     std::cout << sumArray(array) << '\n'; // array 在这里退化成指针

     return 0;
}
\end{lstlisting}

动态数组也因为同样的原因不能工作。

\subsubsection*{如何获取当前元素的索引?}

\textit{For-each} 循环\textit{没有提供}直接获取数组当前元素索引的方法。
这是因为 \textit{for-each} 循环还可以用于很多不能直接索引的结构(例如链表)。

从 C++20 开始,range-based for-loops 可以使用\textbf{初始声明} init-statement,即与普通 for 循环一样。
可以使用初始声明来手动创建索引计数从而不去污染 for-loop 中的函数。

初始声明的语法如下:

\begin{lstlisting}
for (init-statement; element_declaration : array)
   statement;
\end{lstlisting}

\begin{lstlisting}[language=C++]
#include <iostream>

int main()
{
    std::string names[]{ "Alex", "Betty", "Caroline", "Dave", "Emily" }; // 学生名字
    constexpr int scores[]{ 84, 92, 76, 81, 56 };
    int maxScore{ 0 };

    for (int i{ 0 }; auto score : scores) // i 是当前元素的索引
    {
        if (score > maxScore)
        {
            std::cout << names[i] << " beat the previous best score of " << maxScore << " by " << (score - maxScore) << " points!\n";
            maxScore = score;
        }

        ++i;
    }

    std::cout << "The best score was " << maxScore << '\n';

    return 0;
}
\end{lstlisting}

打印:

\begin{lstlisting}
Alex beat the previous best score of 0 by 84 points!
Betty beat the previous best score of 84 by 8 points!
The best score was 92
\end{lstlisting}

需要注意的是如果有 \acode{continue} 在循环内,\acode{++i} 会被跳过而导致非预期结果,
因此需要确保 \acode{i} 在触发 \acode{continue} 之前自增。

在 C++20 之前,变量 \acode{i} 需要在循环外部声明,
这可能在函数内部有另一个 \acode{i} 的时候导致命名冲突。

\end{document}
