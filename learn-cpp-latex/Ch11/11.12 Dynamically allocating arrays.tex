\documentclass[../../LearnCpp.tex]{subfiles}

\begin{document}

\asubsection{12}{Dynamically allocating arrays}

除了动态分配单个值,也可以动态分配数组。
不同于固定数组需要在编译期知道数组长度,动态分配一个数组可以在运行时选择长度。

使用 new 和 delete(通常为 new[] 与 delete[])来进行数组的动态分配。

\begin{lstlisting}[language=C++]
#include <iostream>

int main()
{
    std::cout << "Enter a positive integer: ";
    int length{};
    std::cin >> length;

    int* array{ new int[length]{} }; // 使用 array new。注意这里的 length 不再需要是常量!

    std::cout << "I just allocated an array of integers of length " << length << '\n';

    array[0] = 5;

    delete[] array; // 使用 array delete 释放内存

    // 这里不需要设定 array 成为 nullptr/0,因为这里马上出作用域

    return 0;
}
\end{lstlisting}

动态分配数组的 length 必须为可以被转换成 \acode{std::size\_t} 的类型。
实际上使用 \acode{int} 长度也是可以的,因为 \acode{int} 会被转换成 \acode{std::size\_t}。

\subsubsection*{初始化动态分配数组}

\begin{lstlisting}[language=C++]
int* array{ new int[length]{} };
\end{lstlisting}

\subsubsection*{数组重新设定长度}

动态分配数组使得用户可以在分配是设定数组长度。
然而 C++ 没有提供内置的重设已被分配的数组长度的方法。
可以用别的方法来绕过这个限制,例如重新分配一个新的数组,拷贝所有元素,然后删除旧数组。
然而这很容易导致错误,特别是元素类型是 class(因为有特殊的规则用于管理它们的创建)。

因此不推荐这么做。如果需要这样的功能,C++ 在标准库中提供了 \acode{std::vector} 的可变大小的数组。

\end{document}
