\documentclass[../../LearnCpp.tex]{subfiles}

\begin{document}

\asubsection{16}{An introduction to std::array}

之前的章节谈到过固定与动态数组,尽管它们都内置于 C++,它们都有缺陷:
固定数组衰退成为指针,丢失数组长度信息;
动态数组内存释放的麻烦,以及重置数组长度所面临的风险。

为了解决这些问题,C++ 标准库包含了功能性的数组:
\acode{std::array} 以及 \acode{std::vector}。
本章讲解前者,下一章节讲解后者。

\subsubsection*{简介 std::array}

\acode{std::array} 提供固定数组的功能且传递给函数式不会退化,
其定义于 \acode{<array>} 头文件中的 \acode{std} 命名空间。

\begin{lstlisting}[language=C++]
#include <array>

std::array<int, 3> myArray; // 声明一个长度为 3 的整数数组
\end{lstlisting}

可以使用 initializer lists 或者 list initialization 来进行 \acode{std::array} 初始化:

\begin{lstlisting}[language=C++]
std::array<int, 5> myArray = { 9, 7, 5, 3, 1 }; // initializer list
std::array<int, 5> myArray2 { 9, 7, 5, 3, 1 };  // list initialization
\end{lstlisting}

有别于内建的固定数组,\acode{std::array} 不能省略数组长度:

\begin{lstlisting}[language=C++]
std::array<int, > myArray { 9, 7, 5, 3, 1 };  // 非法:必须提供数组长度
std::array<int> myArray { 9, 7, 5, 3, 1 };    // 非法:必须提供数组长度
\end{lstlisting}

然而从 C++17 开始,允许省略类型和长度。
它们可以被同时省略,但是不能仅省略其中之一,同时数组必须是显式的初始化:

\begin{lstlisting}[language=C++]
std::array myArray { 9, 7, 5, 3, 1 }; // 类型被推导成 std::array<int, 5>
std::array myArray { 9.7, 7.31 };     // 类型被推导成 std::array<double, 2>
\end{lstlisting}

从 C++20 开始,可以指定元素类型同时省略数组长度:

\begin{lstlisting}[language=C++]
auto myArray1 { std::to_array<int, 5>({ 9, 7, 5, 3, 1 }) }; // 指定类型与长度
auto myArray2 { std::to_array<int>({ 9, 7, 5, 3, 1 }) };    // 仅指定类型,推导长度
auto myArray3 { std::to_array({ 9, 7, 5, 3, 1 }) };         // 推导类型与长度
\end{lstlisting}

不幸的是,\acode{std::to_array} 比起直接创建 \acode{std::array} 而言更昂贵,
因为它实际上会从 C-style 数组中拷贝所有元素到 \acode{std::array}。
正因如此,应该避免使用 \acode{std::to_array},特别是创建很多次的环境下(例如在一个循环中)。

也可以对数组进行赋值:

\begin{lstlisting}[language=C++]
std::array<int, 5> myArray;
myArray = { 0, 1, 2, 3, 4 };    // okay
myArray = { 9, 8, 7 };          // okay,元素 3 和 4 设置为零!
myArray = { 0, 1, 2, 3, 4, 5 }; // 不允许,超出范围
\end{lstlisting}

使用下标操作符访问 \acode{std::array} 的值:

\begin{lstlisting}[language=C++]
std::cout << myArray[1] << '\n';
myArray[2] = 6;
\end{lstlisting}

与内建的固定数组一样,下标操作符不会有任何的范围检查,
如果提供了无效的索引,未定义行为会发生。

\acode{std::array} 提供了另一种元素访问的方式(\acode{at()} 函数)带有范围检查(运行时):

\begin{lstlisting}[language=C++]
std::array myArray { 9, 7, 5, 3, 1 };
myArray.at(1) = 6;  // 数组元素 1 有效, 设置第二个元素为 6
myArray.at(9) = 10; // 数组元素 9 无效,抛出运行时异常
\end{lstlisting}

\subsubsection*{长度与排序}

\acode{size()} 函数可以用于获取 \acode{std::array} 的长度:

\begin{lstlisting}[language=C++]
std::array myArray { 9.0, 7.2, 5.4, 3.6, 1.8 };
std::cout << "length: " << myArray.size() << '\n';
\end{lstlisting}

因为当传递给函数时 \acode{std::array} 不会衰退成指针,\acode{size()} 函数可用:

\begin{lstlisting}[language=C++]
#include <array>
#include <iostream>

void printLength(const std::array<double, 5>& myArray)
{
    std::cout << "length: " << myArray.size() << '\n';
}

int main()
{
    std::array myArray { 9.0, 7.2, 5.4, 3.6, 1.8 };

    printLength(myArray);

    return 0;
}
\end{lstlisting}

注意标准库使用“size”意为数组长度 -- 不要与 \acode{sizeof()} 固定数组搞混淆(即返回数组的内存大小)。
这里面的命名并不一致。

最佳实践:总是传递 \acode{std::array} 的引用,或者 \acode{const} 引用。

因为长度总是已知的,range-based for-loops 也可以用于 \acode{std::array}:

\begin{lstlisting}[language=C++]
std::array myArray{ 9, 7, 5, 3, 1 };

for (int element : myArray)
    std::cout << element << ' ';
\end{lstlisting}

可以使用 \acode{std::sort} 对 \acode{std::array} 进行排序,位于 \acode{<algorithm>} 头文件:

\begin{lstlisting}[language=C++]
#include <algorithm> // for std::sort
#include <array>
#include <iostream>

int main()
{
    std::array myArray { 7, 3, 1, 9, 5 };
    std::sort(myArray.begin(), myArray.end());      // 前向排序
    // std::sort(myArray.rbegin(), myArray.rend()); // 后向排序

    for (int element : myArray)
        std::cout << element << ' ';

    std::cout << '\n';

    return 0;
}
\end{lstlisting}

\subsubsection*{传递不同长度的 std::array 给函数}

\begin{lstlisting}[language=C++]
#include <array>
#include <cstddef>
#include <iostream>

// printArray 是一个模板函数
template <typename T, std::size_t size> // 模板化元素类型与长度
void printArray(const std::array<T, size>& myArray)
{
    for (auto element : myArray)
        std::cout << element << ' ';
    std::cout << '\n';
}

int main()
{
    std::array myArray5{ 9.0, 7.2, 5.4, 3.6, 1.8 };
    printArray(myArray5);

    std::array myArray7{ 9.0, 7.2, 5.4, 3.6, 1.8, 1.2, 0.7 };
    printArray(myArray7);

    return 0;
}
\end{lstlisting}

\subsubsection*{通过 size\_type 手动索引 std::array}

\begin{lstlisting}[language=C++]
#include <array>
#include <iostream>

int main()
{
    std::array myArray { 7, 3, 1, 9, 5 };

    // std::array<int, 5>::size_type 是 size() 的返回值!
    for (std::array<int, 5>::size_type i{ 0 }; i < myArray.size(); ++i)
        std::cout << myArray[i] << ' ';

    std::cout << '\n';

    return 0;
}
\end{lstlisting}

更好的可读性:

\begin{lstlisting}[language=C++]
#include <array>
#include <cstddef> // std::size_t
#include <iostream>

int main()
{
    std::array myArray { 7, 3, 1, 9, 5 };

    for (std::size_t i{ 0 }; i < myArray.size(); ++i)
        std::cout << myArray[i] << ' ';

    std::cout << '\n';

    return 0;
}
\end{lstlisting}

更好的方式是避免对 \acode{std::array} 进行手动索引,
而是用 range-based for-loops (或迭代器)。

\subsubsection*{结构体数组}

\begin{lstlisting}[language=C++]
#include <array>
#include <iostream>

struct House
{
    int number{};
    int stories{};
    int roomsPerStory{};
};

int main()
{
    std::array<House, 3> houses{};

    houses[0] = { 13, 4, 30 };
    houses[1] = { 14, 3, 10 };
    houses[2] = { 15, 3, 40 };

    for (const auto& house : houses)
    {
        std::cout << "House number " << house.number
                  << " has " << (house.stories * house.roomsPerStory)
                  << " rooms\n";
    }

    return 0;
}
\end{lstlisting}

初始化:

\begin{lstlisting}[language=C++]
// 与预期一致
std::array<House, 3> houses { // houses 的初始化器
    { // 额外一层花括号用于初始化位于 std::array 结构体内部的 C-style 的数组成员
        { 13, 4, 30 }, // initializer for array element 0
        { 14, 3, 10 }, // initializer for array element 1
        { 15, 3, 40 }, // initializer for array element 2
     }
};
\end{lstlisting}

\end{document}
