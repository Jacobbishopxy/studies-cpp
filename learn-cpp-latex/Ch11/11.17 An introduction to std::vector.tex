\documentclass[../../LearnCpp.tex]{subfiles}

\begin{document}

\asubsection{17}{An introduction to std::vector}

上一章节介绍了 \acode{std::array} 提供了 C++ 内建固定数组的功能以及更安全和使用的方式。

类似的,C++ 标准库提供了更安全和便利的动态数组的功能,名为 \acode{std::vector}。

不同于与固定数组类似的 \acode{std::array},\acode{std::vector} 拥有额外的能力。

\subsubsection*{简介 std::vector}

\begin{lstlisting}[language=C++]
#include <vector>

// 不需要声明长度
std::vector<int> array;
std::vector<int> array2 = { 9, 7, 5, 3, 1 };  // 使用 initializer list 初始化数组 (before C++11)
std::vector<int> array3 { 9, 7, 5, 3, 1 };    // 使用 uniform initialization 初始化数组

// 如 std::array,C++17 开始可以省略类型
std::vector array4 { 9, 7, 5, 3, 1 }; // 推导成 std::vector<int>
\end{lstlisting}

与 \acode{std::array} 类似,可以通过下标访问也可以通过 \acode{at()} 函数访问:

\begin{lstlisting}[language=C++]
array[6] = 2;     // 无边界检查
array.at(7) = 3;  // 有边界检查
\end{lstlisting}

\subsubsection*{自清理防止内存泄漏}

当向量变量离开作用域时,会自动释放其内存。

\begin{lstlisting}[language=C++]
void doSomething(bool earlyExit)
{
    int* array{ new int[5] { 9, 7, 5, 3, 1 } }; // 通过 new 分配内存

    if (earlyExit)
        return; // 退出函数,但是没有调用内存释放

    // 做一些事情

    delete[] array; // 永远没有被调用
}
\end{lstlisting}

然而如果 \acode{array} 是 \acode{std::vector} 的话内存泄漏便不会发生,
这是因为一旦 \acode{array} 离开作用域,内存就会立刻被释放(无论函数是否提前退出),
这就使得 \acode{std::vector} 更加的安全。

\subsubsection*{Vectors 记忆其长度}

不同于内建的动态数组,不知道其自身的长度,std::vector 一直追踪其自身长度。
可以通过 \acode{size()} 函数查询向量的长度:

\begin{lstlisting}[language=C++]
#include <iostream>
#include <vector>

void printLength(const std::vector<int>& array)
{
    std::cout << "The length is: " << array.size() << '\n';
}

int main()
{
    std::vector array { 9, 7, 5, 3, 1 };
    printLength(array);

    std::vector<int> empty {};
    printLength(empty);

    return 0;
}
\end{lstlisting}

如同 \acode{std::array},\acode{size()} 返回一个嵌套类型 \acode{size_type},为一个非负整数。

\subsubsection*{修改向量的长度}

变化内建的动态数组是非常复杂的,而 \acode{std::vector} 则非常简单,
只需要调用 \acode{resize()} 函数。

\begin{lstlisting}[language=C++]
#include <iostream>
#include <vector>

int main()
{
    std::vector array { 0, 1, 2 };
    array.resize(5); // 设置长度为 5

    std::cout << "The length is: " << array.size() << '\n';

    for (int i : array)
        std::cout << i << ' ';

    std::cout << '\n';

    return 0;
}
\end{lstlisting}

打印:

\begin{lstlisting}
The length is: 5
0 1 2 0 0
\end{lstlisting}

向量也可以变小:

\begin{lstlisting}[language=C++]
#include <vector>
#include <iostream>

int main()
{
    std::vector array { 0, 1, 2, 3, 4 };
    array.resize(3); // 设置长度为 3

    std::cout << "The length is: " << array.size() << '\n';

    for (int i : array)
        std::cout << i << ' ';

    std::cout << '\n';

    return 0;
}
\end{lstlisting}

打印:

\begin{lstlisting}
The length is: 3
0 1 2
\end{lstlisting}

\subsubsection*{初始化特定大小的向量}

\begin{lstlisting}[language=C++]
#include <iostream>
#include <vector>

int main()
{
    // 使用直接初始化,可以创建一个有 5 个元素的向量,每个元素都为 0。
    // 如果使用的是大括号初始化,向量拥有 1 个为 5 的元素。
    std::vector<int> array(5);

    std::cout << "The length is: " << array.size() << '\n';

    for (int i : array)
        std::cout << i << ' ';

    std::cout << '\n';

    return 0;
}
\end{lstlisting}

\end{document}
