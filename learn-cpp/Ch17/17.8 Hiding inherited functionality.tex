\documentclass[../../LearnCpp.tex]{subfiles}

\begin{document}

\asubsection{8}{Hiding inherited functionality}

\subsubsection*{修改继承成员的访问等级}

C++ 允许程序员在派生类中修改继承成员的访问等级。
这是通过在不同访问(域)中,对需要修改属性的成员使用 \textit{using} 声明所达成的。

\begin{lstlisting}[language=C++]
#include <iostream>

class Base
{
private:
    int m_value {};

public:
    Base(int value)
        : m_value { value }
    {
    }

protected:
    void printValue() const { std::cout << m_value; }
};

class Derived: public Base
{
public:
    Derived(int value)
        : Base { value }
    {
    }

    // Base::printValue 本身是 protected 继承的,一次不能被公共访问
    // 但是这里通过 using 声明修改成为了公共访问
    using Base::printValue; // 注意:这里没有圆括号
};
\end{lstlisting}

\subsubsection*{隐藏功能}

在 C++ 中,除了修改源代码,移除或者限制基类的功能是不可能的。
然而在派生类中可以隐藏基类的已有功能,这样可以不再被派生类访问。
简单的修改相关的访问限定符即可达成。

\begin{lstlisting}[language=C++]
#include <iostream>
class Base
{
public:
  int m_value {};
};

class Derived : public Base
{
private:
  using Base::m_value;

public:
  Derived(int value)
  // 不可以初始化 m_value,因为它是 Base 成员(Base 必须初始化它)
  {
    // 但是可以对它进行赋值
    m_value = value;
  }
};

int main()
{
  Derived derived { 7 };

  // 以下不能工作是因为 m_value 已经被重新定义为私有
  std::cout << derived.m_value;

  return 0;
}
\end{lstlisting}

同样也可以在派生类中,标记成员函数为 deleted,来确保不会通过派生类来调用它:

\begin{lstlisting}[language=C++]
#include <iostream>
class Base
{
private:
  int m_value {};

public:
  Base(int value)
    : m_value { value }
  {
  }

  int getValue() const { return m_value; }
  };

class Derived : public Base
  {
public:
  Derived(int value)
    : Base { value }
  {
  }


  int getValue() = delete; // mark this function as inaccessible
  };

int main()
  {
  Derived derived { 7 };

  // The following won't work because getValue() has been deleted!
  std::cout << derived.getValue();

  return 0;
}
\end{lstlisting}

\end{document}
