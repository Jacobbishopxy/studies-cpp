\documentclass[../../LearnCpp.tex]{subfiles}

\begin{document}

\asubsection{2}{STL containers overview}

现如今 STL 库中使用的最普遍的功能就是 STL 容器类。
STL 包含了很多不同的容器类可供用于不同的场景。
通常来说,容器类有三种分类:序列容器,关联容器,以及容器适配器。

\subsubsection*{序列容器}

序列容器是一种用于在容器中维护元素顺序的容器类。
序列容器定义的特征是可以根据位置插入元素。最常见的例子就是数组。

从 C++11 开始,STL 包含 6 种序列容器:
\acode{std::vector},\acode{std::deque},\acode{std::array},\acode{std::list},\acode{std::forward_list},以及 \acode{std::basic_string}。

- \textbf{vector} 类允许通过操作符[] 进行随机访问其中的元素,插入与移除 vector 末端的元素通常很快。

\begin{lstlisting}[language=C++]
  #include <vector>
  #include <iostream>

  int main()
  {

      std::vector<int> vect;
      for (int count=0; count < 6; ++count)
          vect.push_back(10 - count); // insert at end of array

      for (int index=0; index < vect.size(); ++index)
          std::cout << vect[index] << ' ';

      std::cout << '\n';
  }
  \end{lstlisting}

- \textbf{deque}(发音“deck”)是一个双端队列类,实现为动态数组可以从两端扩展

\begin{lstlisting}[language=C++]
  #include <iostream>
  #include <deque>

  int main()
  {
      std::deque<int> deq;
      for (int count=0; count < 3; ++count)
      {
          deq.push_back(count); // insert at end of array
          deq.push_front(10 - count); // insert at front of array
      }

      for (int index=0; index < deq.size(); ++index)
          std::cout << deq[index] << ' ';

      std::cout << '\n';
  }
  \end{lstlisting}

- \textbf{list} 是一种特殊类型的序列容器,被称为双向链表,即容器中的每个元素包含了指向上一个元素以及下一个元素的指针。链表仅提供访问两端的功能 -- 没有提供随机访问。如果想要找到中间的值,则需要从某一段开始遍历直至达到希望找到的元素。链表的优势是在知道位置时,插入元素非常的快。之后的章节将会详细讲解。

- \textbf{string} 可以被视为字符类型(或 wchar)的向量。

\subsubsection*{关联容器}

关联容器会自动对输入进其的元素进行排序。默认情况下,关联容器通过操作符< 进行元素比较。

- \textbf{set}是一种用于存储唯一元素的容器。根据容器中元素的值进行的排列。

- \textbf{multiset}是允许重复元素的 set。

- \textbf{map}(也被称为关联数组)是一个 set,其中每个元素都是一对,被称为键/值。键用于排序和索引,必须唯一;值就是数据。

- \textbf{multimap}(也被称为字典)是一个允许重复键的 map。

\subsubsection*{容器适配器}

容器适配器是特殊的预定义的容器,用于适配指定的用法。这其中有趣的是用户可以选择哪种序列容器被使用。

- \textbf{stack} 是一种容器,其元素操作为 LIFO(后进先出),元素在容器的尾部被插入(pushed)与移除(popped)。Stack 使用 deque 作为其默认序列容器(看起来很奇怪,因为 vector 看上去更符合直觉),不过也可以使用 vector 或 list。

- \textbf{queue} 是一种容器,其元素操作为 FIFO(先进先出),元素在容器尾部插入(pushed)并在头部移除(popped)。Queue 使用 deque 作为其默认实现,不过也可以使用 list。

- \textbf{priority queue} 是一种元素维持在排序(通过操作符<)状态下的 queue。当元素被插入,该元素会在 queue 中被排序。移除头部元素返回的则是最高优先级的项。

\end{document}
