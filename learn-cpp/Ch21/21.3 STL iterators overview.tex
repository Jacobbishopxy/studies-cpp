\documentclass[../../LearnCpp.tex]{subfiles}

\begin{document}

\asubsection{3}{STL iterators overview}

\textbf{迭代器 iterator}是一种对象可以穿过(遍历)一个容器类而不需要用户知道容器是如何实现的。
很多类(特别是 list 和关联类),遍历器是主要的元素访问方式。

迭代器最好的可视化是作为一个指针指向给定容器中的元素,并带有一系列重载的操作符提供了良好的功能:

\begin{itemize}
    \item \textbf{操作符*} -- 解引用迭代器返回迭代器当前指向的元素
    \item \textbf{操作符++} -- 移动迭代器到容器中的下一个元素。大多数迭代器同样也提供操作符--用于移动到上一个元素。
    \item \textbf{操作符== 与操作符!=} -- 基础比较符用于判断两个迭代器是否指向同一个元素。为了比较两个值,首先需要解引用迭代器,接着使用比较操作符。
    \item \textbf{操作符=} -- 指派迭代器至新的位置(通常是容器元素的开始或结尾)。为了指派,首先需要解引用迭代器,接着使用指派操作符。
\end{itemize}

每个容器包含了四种基础成员函数用于使用操作符=:

\begin{itemize}
    \item \textbf{begin()} 返回一个迭代器表示容器的起始元素。
    \item \textbf{end()} 返回一个迭代器表示容器的末尾元素之后的位置。
    \item \textbf{cbegin()} 与 \acode{begin()} 相同,只读。
    \item \textbf{cend()} 与 \acode{end()} 相同,只读。
\end{itemize}

\acode{end()} 不指向 list 的最后一个元素看起来有点奇怪,但是这主要是让循环更简单:
遍历所有可继续的元素直到触及 \acode{end()},接着就知道完成了。

最后,所有容器提供(至少)两种类型的迭代器:

\begin{itemize}
    \item \textbf{container::iterator} 提供读/写迭代器
    \item \textbf{container::const\_iterator} 提供只读迭代器
\end{itemize}

遍历一个 vector:

\begin{lstlisting}[language=C++]
#include <iostream>
#include <vector>

int main()
{
    std::vector<int> vect;
    for (int count=0; count < 6; ++count)
        vect.push_back(count);

    std::vector<int>::const_iterator it; // 声明一个只读迭代器
    it = vect.cbegin(); // 赋值 it 来开始 vector
    while (it != vect.cend()) // while it 直到触及结尾
        {
        std::cout << *it << ' '; // 打印所指向的元素
        ++it; // 迭代器指向下一个元素
        }

    std::cout << '\n';
}
\end{lstlisting}

遍历一个 list:

\begin{lstlisting}[language=C++]
#include <iostream>
#include <list>

int main()
{

    std::list<int> li;
    for (int count=0; count < 6; ++count)
        li.push_back(count);

    std::list<int>::const_iterator it; // 声明一个迭代器
    it = li.cbegin(); // 赋值 it 来开始 list
    while (it != li.cend()) // while it 直到触及结尾
    {
        std::cout << *it << ' '; // 打印所指向的元素
        ++it; // 迭代器指向下一个元素
    }

    std::cout << '\n';
}
\end{lstlisting}

遍历一个 set:

\begin{lstlisting}[language=C++]
#include <iostream>
#include <set>

int main()
{
    std::set<int> myset;
    myset.insert(7);
    myset.insert(2);
    myset.insert(-6);
    myset.insert(8);
    myset.insert(1);
    myset.insert(-4);

    std::set<int>::const_iterator it; // 声明一个迭代器
    it = myset.cbegin(); // 赋值 it 来开始 set
    while (it != myset.cend()) // while it 直到触及结尾
    {
        std::cout << *it << ' '; // 打印所指向的元素
        ++it; // 迭代器指向下一个元素
    }

    std::cout << '\n';
}
\end{lstlisting}

遍历 map 会有一点棘手,maps 和 multimaps 获取的是一对元素(定义为 \acode{std::pair}),
可以使用 \acode{make_pair()} 帮助函数来创建对,
\acode{std::pair} 允许通过 \acode{first} 与 \acode{second} 成员函数来访问元素。

\begin{lstlisting}[language=C++]
#include <iostream>
#include <map>
#include <string>

int main()
{
    std::map<int, std::string> mymap;
    mymap.insert(std::make_pair(4, "apple"));
    mymap.insert(std::make_pair(2, "orange"));
    mymap.insert(std::make_pair(1, "banana"));
    mymap.insert(std::make_pair(3, "grapes"));
    mymap.insert(std::make_pair(6, "mango"));
    mymap.insert(std::make_pair(5, "peach"));

    auto it{ mymap.cbegin() }; // 声明一个 const 迭代器并赋值 it 来开始 vector
    while (it != mymap.cend()) // while it 直到触及结尾
    {
        std::cout << it->first << '=' << it->second << ' '; // 打印所指向的元素
        ++it; // 迭代器指向下一个元素
    }

    std::cout << '\n';
}
\end{lstlisting}

\end{document}
