\documentclass[../../LearnCpp.tex]{subfiles}

\begin{document}

\asubsection{7}{External linkage and variable forward declarations}

上一章中讲到 \acode{internal linkage} 限制了标识符仅可以使用在单个文件中。
这一章将探讨 \acode{external linkage}。

带有 \acode{external linkage} 的标识符同时可视并可用于其定义的文件,以及其他代码文件(通过前向声明)。
这种情况下,带有 \acode{external linkage} 的标识符是真正意味上的“全局”,可在项目的任何位置使用。

\subsubsection*{函数默认带有 external linkage}

2.8 中学到的用户可以在其他的文件中,从其定义的地方调用函数。这是因为函数默认是 external linkage 的。

为了在其他文件中调用,用户必须为需要调用的函数放置一个前向声明 \acode{forward declaration}。
前向声明可以告诉编译器函数的存在,linker 则关联函数调用至其真正定义的地方。

\subsubsection*{全局变量带有 external linkage}

带有 external linkage 的全局变量有时被称为 \textbf{external variables}。
用户可以使用 \acode{extern} 关键字使一个全局变量 external(以便其它文件访问)。

\subsubsection*{变量通过 extern 关键字前向声明}

为了在其它文件中可以使用到 external 的全局变量,用户在其他文件中必须放置该变量的前向声明 \acode{forward declaration}。
而对于变量而言,为其创建一个前向声明也是通过 \acode{extern} 关键字(并且不带初始值)。

\subsubsection*{文件作用域 vs. 全局作用域}

“文件作用域”与“全局作用域”这两个概念有可能搞混淆,这是因为它们都是被非正式的使用。

考虑以下程序:

global.cpp:

\begin{lstlisting}[language=C++]
int g_x { 2 }; // 默认为 external linkage
// g_x 在此离开作用域
\end{lstlisting}

main.cpp:

\begin{lstlisting}[language=C++]
extern int g_x; // 前向声明 g_x -- g_x 可以被用在这里(超出 global.cpp 的文件作用域)

int main()
{
    std::cout << g_x << '\n'; // 应该打印 2

    return 0;
}
// g_x 的前向声明在这里结束
\end{lstlisting}

变量 \acode{g_x} 在 \acode{global.cpp} 中拥有文件作用域 -- 它可以使用在定义起始的位置,直到文件结束,
但是它不可以直接在 \acode{global.cpp} 外可视。

在 \acode{main.cpp} 中,\acode{g_x} 的前置定义同样也有文件作用域 -- 它可以使用在定义起始的位置,直到文件结束。

然而,非正式的“文件作用域”更经常被用于带有 internal linkage 的全局变量,
而“全局作用域”被用于带有 external linkage 的全局变量(因为它们可以被用于程序的任何地方,
只需要有正确的前向声明)。

\end{document}
