\documentclass[../../LearnCpp.tex]{subfiles}

\begin{document}

\asubsection{15}{Unnamed and inline namespaces}

C++ 支持两种变体的命名空间值得了解。

\subsubsection*{无名(匿名)命名空间}

一个\textbf{无名命名空间}(也被称为\textbf{匿名命名空间})是一种没有定义名称的命名空间,例如:

\begin{lstlisting}[language=C++]
#include <iostream>

namespace // 无名命名空间
{
    void doSomething() // 只可以在本文件中访问
    {
        std::cout << "v1\n";
    }
}

int main()
{
    doSomething(); // 可以在不使用命名空间前缀的情况下调用 doSomething()

    return 0;
}
\end{lstlisting}

在 \acode{unnamed namespace} 中的所有声明的内容被视为其父命名空间的一部分。
因此即使 \acode{doSomething} 函数定义在 \acode{unnamed namespace} 中,
函数本身是可以在父命名空间中方芬(在本案例中即 \acode{global namespace}),
这也是为何可以在 \acode{main} 里调用 \acode{doSomething} 而不需要任何限定符。

看上去这里的 \acode{unnamed namespace} 没什么用,但是其中的所有标识符都可以被视为带有 \acode{internal linkage},
这就意味着对于文件外而言它们是不可见的。

对于函数而言,无名命名空间的作用等同于 \acode{static functions},即:

\begin{lstlisting}[language=C++]
#include <iostream>

static void doSomething() // 仅在此文件中可以被访问
{
    std::cout << "v1\n";
}

int main()
{
    doSomething(); // 可以不使用命名空间前缀调用 doSomething()

    return 0;
}
\end{lstlisting}

\acode{unnamed namespace} 通常而言用于大量内容需要确保定义在指定文件中,
这样简化了需要对内容中的每一条加上 \acode{static} 声明。
\acode{unnamed namespace} 同样维护 \acode{user-defined types}(之后章节将覆盖)在本地文件中,
这是没有其他别的代替方案可以做到的。

\subsubsection*{内联命名空间}

内联命名空间通常用于版本内容。类似于 \acode{unnamed namespace},
所有定义在 \acode{inline namespace} 中的内容视为父命名空间的一部分。
然而 \acode{inline namespace} 不会提供 \acode{internal linkage}。

只需要 \acode{inline} 关键字就可以定义内联命名空间:

\begin{lstlisting}[language=C++]
#include <iostream>

inline namespace v1 // 定义一个名为 v1 的内联命名空间
{
    void doSomething()
    {
        std::cout << "v1\n";
    }
}

namespace v2 // 定义一个名为 v2 的命名空间
{
    void doSomething()
    {
        std::cout << "v2\n";
    }
}

int main()
{
    v1::doSomething(); // 调用 v1 的 doSomething()
    v2::doSomething(); // 调用 v2 的 doSomething()

    doSomething(); // 调用内联命名空间的 doSomething(),即 v1

    return 0;
}
\end{lstlisting}

下次修改版本时,可以直接去掉 v1 的 \acode{inline} 并在 v2 命名空间前加上,
这样使得 \acode{doSomething()} 直接调用的是 v2 版本。

\end{document}
