\documentclass[../../LearnCpp.tex]{subfiles}

\begin{document}

\asubsection{1}{Template classes}

\subsubsection*{模版与容器类}

16.6 章节中学习到了如何使用组合 composition 来实现可以包含若干其他类实例的容器类。
这里是一个简化版的代码:

\begin{lstlisting}[language=C++]
#ifndef INTARRAY_H
#define INTARRAY_H

#include <cassert>

class IntArray
{
private:
    int m_length{};
    int* m_data{};

public:

    IntArray(int length)
    {
        assert(length > 0);
        m_data = new int[length]{};
        m_length = length;
    }

    // 不希望 IntArray 的拷贝允许被创建
    IntArray(const IntArray&) = delete;
    IntArray& operator=(const IntArray&) = delete;

    ~IntArray()
    {
        delete[] m_data;
    }

    void erase()
    {
        delete[] m_data;
        // 这里需要确保设置 m_data 为空指针,否则将会仍然指向已释放的内存!
        m_data = nullptr;
        m_length = 0;
    }

    int& operator[](int index)
    {
        assert(index >= 0 && index < m_length);
        return m_data[index];
    }

    int getLength() const { return m_length; }
};

#endif
\end{lstlisting}

那么如果是想要创建一个 double 的数组呢?
与 \acode{IntArray} 相比,仅有实质性不同的在于数据类型不同。
创建一个模板类类似于创建模板函数。

\begin{lstlisting}[language=C++]
#ifndef ARRAY_H
#define ARRAY_H

#include <cassert>

template <typename T> // 新增
class Array
{
private:
    int m_length{};
    T* m_data{}; // 修改类型为 T

public:

    Array(int length)
    {
        assert(length > 0);
        m_data = new T[length]{}; // 分配类型为 T 的对象的数组
        m_length = length;
    }

    Array(const Array&) = delete;
    Array& operator=(const Array&) = delete;

    ~Array()
    {
        delete[] m_data;
    }

    void erase()
    {
        delete[] m_data;
        m_data = nullptr;
        m_length = 0;
    }

    T& operator[](int index) // 现在返回的是 T&
    {
        assert(index >= 0 && index < m_length);
        return m_data[index];
    }

    // 模板化的 getLength() 函数在下方定义
    int getLength() const;
};

// 成员函数定义在类的外部,需要它们自己的模版声明
template <typename T>
int Array<T>::getLength() const // 注意类名为 Array<T>,而不是 Array
{
  return m_length;
}

#endif
\end{lstlisting}

可以看到上述代码几乎与 \acode{IntArray} 相同,
除了添加了模版声明以外,以及使用 T 更换了数据类型。

注意除此之外还定义了 \acode{getLength()} 函数在类声明之外。
这并不是必须的,但是新手程序员通常会卡在这里的语法上,因此这个案例具有指导意义。
每个模板成员函数定义在类声明之外就需要其自身的模版声明。
同时,注意模版数组的名称现在是 \acode{Array<T>},而不是 \acode{Array}。

\begin{lstlisting}[language=C++]
#include <iostream>
#include "Array.h"

int main()
{
  Array<int> intArray { 12 };
  Array<double> doubleArray { 12 };

  for (int count{ 0 }; count < intArray.getLength(); ++count)
  {
    intArray[count] = count;
    doubleArray[count] = count + 0.5;
  }

  for (int count{ intArray.getLength() - 1 }; count >= 0; --count)
    std::cout << intArray[count] << '\t' << doubleArray[count] << '\n';

  return 0;
}
\end{lstlisting}

打印:

\begin{lstlisting}
11     11.5
10     10.5
9       9.5
8       8.5
7       7.5
6       6.5
5       5.5
4       4.5
3       3.5
2       2.5
1       1.5
0       0.5
\end{lstlisting}

模板类的实例化与模板函数一样 --
编译器拓印一份所需的拷贝,将模板参数替换为用户所需的真实数据类型,接着编译该副本。
如果没有使用模板类,编译器甚至不会编译它。

模板类用来实现容器类是理想的,因为它高度期望容器可以对不同类型的数据有效,同样的模板也避免了大量重复代码。
尽管语法很挫,同时报错信息可能会很隐晦,然而模板类是 C++ 中真正最好的最有用的特性。

\subsubsection*{标准库中的模板类}

现在已经覆盖了模板类,那么理解 \acode{std::vector<int>} 的真正意义就不难了 --
它实际上就是一个模板类,二 \acode{int} 则是模板的类型参数!
标准库中存在着大量的预定义的模板类可供使用。之后的章节会覆盖到它们。

\subsubsection*{拆分模板类}

一个模板既不是类也不是函数 -- 它是拓印用于创建类或者函数的。
因此,它的工作方式与普通的类或者函数不一样。大多数情况下,这并不是什么问题。
然而有一个地方是开发者通常容易出错的。

非模板类时,通常的过程是放置类定义与头文件中,其成员函数定义放在同名的源文件中。
这样的方式,类的源代码会以分隔项目文件的方式被编译。然而使用模板时,这样并不能工作。
考虑下面的代码:

Array.h:

\begin{lstlisting}[language=C++]
#ifndef ARRAY_H
#define ARRAY_H

#include <cassert>

template <typename T>
class Array
{
private:
    int m_length{};
    T* m_data{};

public:

    Array(int length)
    {
        assert(length > 0);
        m_data = new T[length]{};
        m_length = length;
    }

    Array(const Array&) = delete;
    Array& operator=(const Array&) = delete;

    ~Array()
    {
        delete[] m_data;
    }

    void erase()
    {
        delete[] m_data;

        m_data = nullptr;
        m_length = 0;
    }

    T& operator[](int index)
    {
        assert(index >= 0 && index < m_length);
        return m_data[index];
    }

    int getLength() const;
};

# endif
\end{lstlisting}

Array.cpp:

\begin{lstlisting}[language=C++]
#include "Array.h"

template <typename T>
int Array<T>::getLength() const // 注意类名称为 Array<T>,而不是 Array
{
  return m_length;
}
\end{lstlisting}

main.cpp:

\begin{lstlisting}[language=C++]
#include <iostream>
#include "Array.h"

int main()
{
  Array<int> intArray(12);
  Array<double> doubleArray(12);

  for (int count{ 0 }; count < intArray.getLength(); ++count)
  {
    intArray[count] = count;
    doubleArray[count] = count + 0.5;
  }

  for (int count{ intArray.getLength() - 1 }; count >= 0; --count)
    std::cout << intArray[count] << '\t' << doubleArray[count] << '\n';

  return 0;
}
\end{lstlisting}

上述程序可以被编译但是会有一个 linker 错误:

\begin{lstlisting}
unresolved external symbol "public: int __thiscall Array<int>::getLength(void)" (?GetLength@?\$Array@H@@QAEHXZ)
\end{lstlisting}

为了让编译器使用模板,编译器必须同时看到模板定义(不仅只是声明)以及用于实例化模板的模板类型。
同时不要忘记 C++ 是单独的编译文件。
当 \acode{Array.h} 头文件被 main 引入,模板类的定义被拷贝进 \acode{main.cpp}。
当编译器如果看到两个模板实例,\acode{Array<int>} 与 \acode{Array<double>} 时,
编译器会实例化它们,并将它们编译成 \acode{main.cpp} 中的一部分。
然而当编译器分开编译 \acode{Array.cpp} 时,它并不知道需要的是 \acode{Array<int>} 与 \acode{Array<double>},
因此模板函数永远不会被实例化。
因此获得了 linker 错误,这是因为编译器没有发现 \acode{Array<int>::getLength()} 与 \acode{Array<double>::getLength()} 的定义。

有一些方法可以绕过这个问题。

最简单的方法就是把所有模板类的代码放到头文件中(这个案例中就是将 \acode{Array.cpp} 的内容放进 \acode{Array.h} 中位于类的下方)。
这样当 \#include 头文件时,所有模板代码都会在同一个地方。这么做的好处就是简单。
这么做的坏处是如果模板类在很多地方被使用的时候,将会得到大量模板类的本地副本,
这增加了编译时间以及 link 次数(linker 应该会移除重复的定义,因此并不会使可执行文件膨胀起来)。
这是推荐的解决方案,直到编译器或 link 次数开始成为一个问题时。

如果将 \acode{Array.cpp} 代码放进 \acode{Array.h} 头文件使得头文件非常大且凌乱,
另一种做法是移动 \acode{Array.cpp} 至新的名为 \acode{Array.inl} 文件中(.inl 意为内联 inline),
接着将 \acode{Array.inl} 包含在 \acode{Array.h} 头文件的尾部(头文件保护符之内)。
这将返回与将所有代码放入头文件中,产生同样的结果,并给代码带来了更好的管理。

小贴士:如果使用的是 .inl 方法并触发关于重复定义的编译错误时,编译器大概率是将 .inl 文件视为了代码文件而成为了项目的一部分。
这个结果是 .inl 被编译了两次:第一次是编译了 .inl,还有一次是当 .cpp 文件引用 .inl 文件时被编译。如果这种情况发生了,
那么则需要从 build 中排除 .inl 文件。这通常是可以由编辑器中对该文件点击右键并选择其属性来完成的。

另一种方法是使用三文件处理法。模板类定义放在头文件,模板类成员函数放在代码文件,接着添加第三个文件,包含所有所需的实例化的类:

templates.cpp:

\begin{lstlisting}[language=C++]
// 确保所有 Array 模板定义都可以被看见
#include "Array.h"
#include "Array.cpp" // 这里破坏了最佳实践,但是仅在此一处

// 在这里 #include 其他所有需要的 .h 与 .cpp 模板定义

template class Array<int>;    // 显式实例化模板 Array<int>
template class Array<double>; // 显式实例化模板 Array<double>

// 这里实例化其他模板
\end{lstlisting}

这个方法有可能更高效(取决于编译器和 linker 是如何处理模板与重复定义的),
但是需要为每个项目都维护一个 \acode{templates.cpp} 文件。

\end{document}
