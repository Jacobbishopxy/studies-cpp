\documentclass[../../LearnCpp.tex]{subfiles}

\begin{document}

\asubsection{2}{Template non-type parameters}

前一节学到了如何使用模板类型变量来创建类型独立的函数与类。
模板类型参数是一个占位类型用于替换传入的类型参数。

然而模板类型参数并不是唯一可用的模板参数。
模板类与函数可以使用其他类型的模板参数,被称为非类型参数。

\subsubsection*{非类型参数}

一个模板非类型参数是一种参数类型是预定义的并且由 constexpr 值传递给入参的模板参数。

一个非类型参数可以是以下任意一个类型:

\begin{itemize}
    \item 整型
    \item 枚举类型
    \item 一个类对象的指针或引用
    \item 一个函数的指针或引用
    \item 一个类成员函数的指针或引用
    \item sts::nullptr\_t
    \item 浮点类型(C++20 起)
\end{itemize}

下面的例子中,创建了一个非动态(static)数组类,同时用到了类型参数与非类型参数。
类型参数控制静态数组的数据类型,而整型的非类型参数控制静态数组的大小。

\begin{lstlisting}[language=C++]
#include <iostream>

template <typename T, int size> // size 是一个整型的非类型参数
class StaticArray
{
private:
    // 非类型参数控制数组大小
    T m_array[size] {};

public:
    T* getArray();

    T& operator[](int index)
    {
        return m_array[index];
    }
};

// 展示带有非类型参数的类成员函数是如何定义在类定义外部的
template <typename T, int size>
T* StaticArray<T, size>::getArray()
{
    return m_array;
}

int main()
{
    // 声明整型数组,大小为 12
    StaticArray<int, 12> intArray;

    // 顺序填数,倒序打印
    for (int count { 0 }; count < 12; ++count)
        intArray[count] = count;

    for (int count { 11 }; count >= 0; --count)
        std::cout << intArray[count] << ' ';
    std::cout << '\n';

    // 声明浮点型数组,大小为 4
    StaticArray<double, 4> doubleArray;

    for (int count { 0 }; count < 4; ++count)
        doubleArray[count] = 4.4 + 0.1 * count;

    for (int count { 0 }; count < 4; ++count)
        std::cout << doubleArray[count] << ' ';

    return 0;
}
\end{lstlisting}

上述代码中一个值得注意的点就是没有动态分配 \acode{m_array} 成员变量!
这是因为任何给定的 \acode{StaticArray} 类,\acode{size} 必须是 constexpr。
例如,如果实例化一个 \acode{StaticArray<int, 12>},编译器替换 \acode{size} 为 12。
因此 \acode{m_array} 的类型是 \acode{int[12]},即可被静态分配。

\end{document}
