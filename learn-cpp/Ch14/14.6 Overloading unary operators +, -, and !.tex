\documentclass[../../LearnCpp.tex]{subfiles}

\begin{document}

\asubsection{6}{Overloading unary operators +, -, and !}

\subsubsection*{重载一元操作符}

不同于迄今为止已经见到过得操作符,正号(+)、负号(-)以及逻辑非(!)操作符都是一元操作符,
这就意味着它们仅可以对一个运算成员进行计算。
因为它们运算的对象就是应用的对象,通常一元操作符重载的实现都是以成员函数的方式。
这三种运算的实现行为一致。

现在看一下上一节所用的例子来进行操作符- 重载:

\begin{lstlisting}[language=C++]
#include <iostream>

class Cents
{
private:
    int m_cents {};

public:
    Cents(int cents): m_cents{cents} {}

    // 以成员函数的方式重载 -Cents
    Cents operator-() const;

    int getCents() const { return m_cents; }
};

// 注意:该函数为成员函数!
Cents Cents::operator-() const
{
    return -m_cents; // 由于返回类型是一个 Cents,这里使用了 Cents(int) 构造函数,进行了从 int 到 Cents 的一个隐式转换
}

int main()
{
    const Cents nickle{ 5 };
    std::cout << "A nickle of debt is worth " << (-nickle).getCents() << " cents\n";

    return 0;
}
\end{lstlisting}

这应该是很直接的。重载的负号操作符(-)是一个以成员函数方式实现的一元操作符,因此其不需要参数(它运算 *this 对象)。
该函数返回一个与原始对象一样的对象,其值带有负号。
因为操作符- 不修改 Cents 对象,我们可以(应该)使其成为一个 const 函数(有此可在 const Cents 对象上进行调用)。

注意这里的负号操作符- 与减法操作符- 应该是没有歧义的 -- 因为它们的参数不同。

以下是另一个案例。操作符! 是逻辑非操作符 -- 如果一个表达式运算类似于“true”,操作符! 将返回 false,反之亦然。
我们通常会在布尔变量上看到这样的操作:

\begin{lstlisting}[language=C++]
if (!isHappy)
  std::cout << "I am not happy!\n";
else
  std::cout << "I am so happy!\n";
\end{lstlisting}

对于整数,0 解析为 false,其它为 true,因此操作符! 应用在除零以外的整数上都为 true,零为 false。

延展这个概念,在“false“,”zero“ 或者其他默认初始状态下,使用操作符! 解析为 true。

以下案例展示了重载了操作符- 以及操作符! 的自定义 Point 类:

\begin{lstlisting}[language=C++]
#include <iostream>

class Point
{
private:
    double m_x {};
    double m_y {};
    double m_z {};

public:
    Point(double x=0.0, double y=0.0, double z=0.0):
        m_x{x}, m_y{y}, m_z{z}
    {
    }

    // 转换 Point 为与之相等的负值
    Point operator- () const;

    // 如果 point 是原点返回 true
    bool operator! () const;

    double getX() const { return m_x; }
    double getY() const { return m_y; }
    double getZ() const { return m_z; }
};

Point Point::operator- () const
{
    return { -m_x, -m_y, -m_z };
}

bool Point::operator! () const
{
    return (m_x == 0.0 && m_y == 0.0 && m_z == 0.0);
}

int main()
{
    Point point{}; // use default constructor to set to (0.0, 0.0, 0.0)

    if (!point)
        std::cout << "point is set at the origin.\n";
    else
        std::cout << "point is not set at the origin.\n";

    return 0;
}
\end{lstlisting}

打印:

\begin{lstlisting}
point is set at the origin.
\end{lstlisting}

\end{document}
