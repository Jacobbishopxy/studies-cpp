\documentclass[../../LearnCpp.tex]{subfiles}

\begin{document}

\asubsection{5}{Overloading operators using member functions}

在 14.2 中学习了使用友元函数重载算法操作符,同样也学习了使用普通函数重载操作符。
很多操作符可以有另一种方式被重载:成员函数。

使用成员函数重载操作符类似于使用友元函数的方式,但是有几点要求:

\begin{itemize}
    \item 重载操作符必须是左侧运算对象的成员函数
    \item 左侧运算对象变成隐式的 *this 对象
    \item 其它的运算对象变为函数入参
\end{itemize}

转换一个友元重载操作符成为成员重载操作符也很简单:

\begin{enumerate}
    \item 重载操作符定义为成员函数而不是友元
    \item 左侧参数被移除,因为该参数现在变为了隐式 *this 对象
    \item 在函数体内,所有左侧参数的引用可以被移除
\end{enumerate}

\begin{lstlisting}[language=C++]
#include <iostream>

class Cents
{
private:
    int m_cents {};

public:
    Cents(int cents)
        : m_cents { cents } { }

    // Overload Cents + int
    Cents operator+ (int value);

    int getCents() const { return m_cents; }
};

// 注意:该函数是一个成员函数!
// 友元版本中的 cents 参数现在是隐式 *this 参数
Cents Cents::operator+ (int value)
{
    return Cents { m_cents + value };
}

int main()
{
    Cents cents1 { 6 };
    Cents cents2 { cents1 + 2 };
    std::cout << "I have " << cents2.getCents() << " cents.\n";

    return 0;
}
\end{lstlisting}

\subsubsection*{不是所有东西都能以友元函数的方式重载}

赋值(=),下标([]),函数调用(()),以及成员选择(->)操作符必须以成员函数重载。

\subsubsection*{不是所有东西都能以成员函数的方式重载}

14.4 重载 I/O 操作符中学习了重载操作符<<,它是不能作为成员函数的。
因为重载的运算成员必须是左侧的成员。

\subsubsection*{使用普通、友元、或成员函数进行重载的时机}

大多数情况下,C++ 允许用户自己决定到底是使用普通、友元、或成员函数来进行重载。
然而总有一种情况是优于另外两个的。

当处理二元操作符且不需要修改左侧运算成员(例如操作符+),普通函数或友元函数更为推荐,
因为它可以用于所有的参数类型(即使左侧运算成员不是一个类对象,或者是一个不可变的类)。
普通或友元函数可以从“对称性”获益,因为索引运算成员是显式参数(而不是左侧运算成员是 *this)。

当处理二元操作符且需要修改左侧运算成员时(例如操作符+=),成员函数通常更为推荐。
这些情况下,左侧运算成员通常是一个类类型,且拥被修改的对象通过 *this 指向是自然而然的。
因为右侧运算成员变成了显式参数,被修改与被计算的是显而易见的。

一元运算通常也是作为成员函数进行重载,因为它们不需要参数。

\end{document}
