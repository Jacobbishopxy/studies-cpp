
\documentclass[../../LearnCpp.tex]{subfiles}

\begin{document}

\asubsection{8}{Overloading the increment and decrement operators}

重载增加(\acode{++})与减少(\acode{--})操作符非常的直观,仅有一个小例外。
它们实际上两个版本的增加与减少操作符:前置增加以及减少(例如 \acode{++x; --y;})以及后置的增加与减少(例如 \acode{x++; y--;})。

由于增加与减少操作符都是一元操作符,它们要修改操作成员,因此最好是作为成员函数进行重载。这里处理前置的版本,因为它们更直观。

\subsubsection*{重载前置增加与减少}

\begin{lstlisting}[language=C++]
#include <iostream>

class Digit
{
private:
    int m_digit;
public:
    Digit(int digit=0)
        : m_digit{digit}
    {
    }

    Digit& operator++();
    Digit& operator--();

    friend std::ostream& operator<< (std::ostream& out, const Digit& d);
};

Digit& Digit::operator++()
{
    // If our number is already at 9, wrap around to 0
    if (m_digit == 9)
        m_digit = 0;
    // otherwise just increment to next number
    else
        ++m_digit;

    return *this;
}

Digit& Digit::operator--()
{
    // If our number is already at 0, wrap around to 9
    if (m_digit == 0)
        m_digit = 9;
    // otherwise just decrement to next number
    else
        --m_digit;

    return *this;
}

std::ostream& operator<< (std::ostream& out, const Digit& d)
{
	out << d.m_digit;
	return out;
}

int main()
{
    Digit digit(8);

    std::cout << digit;
    std::cout << ++digit;
    std::cout << ++digit;
    std::cout << --digit;
    std::cout << --digit;

    return 0;
}
\end{lstlisting}

打印:

\begin{lstlisting}
89098
\end{lstlisting}

注意返回了 *this。重载的增加与减少操作符返回当前的隐式对象,这样若干操作符可以被”串联“起来。

\subsubsection*{重载后置增加与减少}

通常而言,重名的函数可以通过不同的参数数量以及/或不同的参数类型来进行重载。
然而考虑前置与后置的增加与减少操作符,它们都有相同的名字,并且都是接受同样类型的一个参数。
那么如何区分它们的重载呢?

C++ 语言规范提供了一个特殊的方案:编译器检查重载操作符是否有一个 int 参数。如果有,操作符是后置重载;
如果重载的操作符没有参数,操作符则是前置重载。

\begin{lstlisting}[language=C++]
class Digit
{
private:
    int m_digit;
public:
    Digit(int digit=0)
        : m_digit{digit}
    {
    }

    Digit& operator++(); // 前置不带参数
    Digit& operator--(); // 前置不带参数

    Digit operator++(int); // 后置带有 int 参数
    Digit operator--(int); // 后置带有 int 参数

    friend std::ostream& operator<< (std::ostream& out, const Digit& d);
};

// 没有参数意味着前置 operator++
Digit& Digit::operator++()
{
    // 如果处于 9,处理成 0
    if (m_digit == 9)
        m_digit = 0;
    // 否则增加一个数
    else
        ++m_digit;

    return *this;
}

// 没有参数意味着前置 operator--
Digit& Digit::operator--()
{
    // 如果处于 0,处理成 9
    if (m_digit == 0)
        m_digit = 9;
    // 否则减少一个数
    else
        --m_digit;

    return *this;
}

// int 参数意味着后置 operator++
Digit Digit::operator++(int)
{
    // 通过当前的 digit 创建一个临时的变量
    Digit temp{*this};

    // 使用前置操作符来增加 digit
    ++(*this); // 应用操作符

    // 返回临时结果
    return temp; // 返回保存的状态
}

// int 参数意味着后置 operator--
Digit Digit::operator--(int)
{
    // 通过当前的 digit 创建一个临时的变量
    Digit temp{*this};

    // 使用前置操作符来减少 digit
    --(*this); // 应用操作符

    // 返回临时的结果
    return temp; // 返回保存的状态
}

std::ostream& operator<< (std::ostream& out, const Digit& d)
{
	out << d.m_digit;
	return out;
}

int main()
{
    Digit digit(5);

    std::cout << digit;
    std::cout << ++digit; // 调用 Digit::operator++();
    std::cout << digit++; // 调用 Digit::operator++(int);
    std::cout << digit;
    std::cout << --digit; // 调用 Digit::operator--();
    std::cout << digit--; // 调用 Digit::operator--(int);
    std::cout << digit;

    return 0;
}
\end{lstlisting}

打印:

\begin{lstlisting}
5667665
\end{lstlisting}

这里有几个有趣的点。首先,注意通过整数虚拟参数区分了前置与后置操作符。其次由于虚拟参数并没有在函数实现中使用,因此不需要为其命名。
这就是告诉编译器对待该变量为占位符,意味着不会因为声明了变量却不使用而产生警告。

其次,注意前置与后置操作符的功能一致 -- 它们都是增加或减少对象。它们俩的区别在于返回的值。前置操作符返回被增加或减少的对象。
因此重载它们非常的直接,仅需简单的增加或减少成员变量,接着返回 *this。

后置操作符需要返回的是增加或减少发生前的状态。这就带来了一点迷惑 -- 如果要返回修改前的状态,那么增加或减少对象就不可能。
也就是说如果在修改对象前返回对象,修改的行为将永远不会被调用。

这个问题的通常解决方案是使用临时变量来存储修改前的状态。那么对象可以被修改,最后临时变量被返回给调用者。
这种方式下,调用者获取的是对象在修改之前的拷贝,而对象本身已经被修改了。
注意这就意味着返回值必须是非引用的,因为不能返回本地变量的引用,它们会在函数结束时被销毁。
同样注意这就意味着后置操作符通常效率会比前置操作符更低一些,因为整个过程中增加了临时变量的实例化,并且值返回而不是引用返回。

最后注意后置操作符都调用了前置操作符的函数。这样减少了重复代码,同样使得未来的修改变得更简单。

\end{document}
