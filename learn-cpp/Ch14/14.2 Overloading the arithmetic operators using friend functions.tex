\documentclass[../../LearnCpp.tex]{subfiles}

\begin{document}

\asubsection{2}{Overloading the arithmetic operators using friend functions}

C++ 中用到最多的操作符之一就有运算操作符 -- 即加法操作符(+),减法操作符(-),乘法操作符(*)与除法操作符(/)。
注意所有的运算操作符都是二元操作符 -- 即需要两个运算对象 -- 位于操作符的两侧。
这四个操作符都以相同的方式的重载。

事实上有三种不同的方法用于重载操作符:以成员函数的方式,以友元函数的方式,以及普通函数的方式。
本节将详细讲述友元函数的方式(因为它更符合直觉)。
下一节讲述普通函数的方式,接着再讲解成员函数的方式。

\subsubsection*{使用友元函数重载操作符}

考虑以下代码:

\begin{lstlisting}[language=C++]
class Cents
{
private:
  int m_cents {};

public:
  Cents(int cents) : m_cents{ cents } { }
  int getCents() const { return m_cents; }
};
\end{lstlisting}

下面的例子展示了如何重载操作符加号(+)来使两个“Cents”对象相加:

\begin{lstlisting}[language=C++]
#include <iostream>

class Cents
{
private:
  int m_cents {};

public:
  Cents(int cents) : m_cents{ cents } { }

  // 使用友元函数进行 Cents + Cents 的相加
  friend Cents operator+(const Cents& c1, const Cents& c2);

  int getCents() const { return m_cents; }
};

// 注意:该函数不是一个成员函数!
Cents operator+(const Cents& c1, const Cents& c2)
{
  // 使用 Cents 的构造函数与操作符+(int, int)
  // 可以直接访问 m_cents 因为其为友元
  return Cents{c1.m_cents + c2.m_cents};
}

int main()
{
  Cents cents1{ 6 };
  Cents cents2{ 8 };
  Cents centsSum{ cents1 + cents2 };
  std::cout << "I have " << centsSum.getCents() << " cents.\n";

  return 0;
}
\end{lstlisting}

打印:

\begin{lstlisting}
I have 14 cents.
\end{lstlisting}

重载减法操作符(-)同样也很简单:

\begin{lstlisting}[language=C++]
#include <iostream>

class Cents
{
private:
  int m_cents {};

public:
  Cents(int cents) : m_cents{ cents } { }

  // 使用友元函数进行 Cents + Cents 的相加
  friend Cents operator+(const Cents& c1, const Cents& c2);

  // 使用友元函数进行 Cents - Cents 的相减
  friend Cents operator-(const Cents& c1, const Cents& c2);

  int getCents() const { return m_cents; }
};

// 注意:该函数不是一个成员函数!
Cents operator+(const Cents& c1, const Cents& c2)
{
  // 使用 Cents 的构造函数与操作符+(int, int)
  // 可以直接访问 m_cents 因为其为友元
  return Cents{c1.m_cents + c2.m_cents};
}

// 注意:该函数不是一个成员函数!
Cents operator-(const Cents& c1, const Cents& c2)
{
  // 使用 Cents 的构造函数与操作符-(int, int)
  // 可以直接访问 m_cents 因为其为友元
  return Cents(c1.m_cents - c2.m_cents);
}

int main()
{
  Cents cents1{ 6 };
  Cents cents2{ 2 };
  Cents centsSum{ cents1 - cents2 };
  std::cout << "I have " << centsSum.getCents() << " cents.\n";

  return 0;
}
\end{lstlisting}

重载乘法操作符(*)与除法操作符(/)同样也很简单。

\subsubsection*{友元函数可以定义在类的内部}

尽管友元函数不是类的成员函数,它们仍然可以被定义在类的内部:

\begin{lstlisting}[language=C++]
#include <iostream>

class Cents
{
private:
  int m_cents {};

public:
  Cents(int cents) : m_cents{ cents } { }

  friend Cents operator+(const Cents& c1, const Cents& c2)
  {
    return Cents{c1.m_cents + c2.m_cents};
  }

  int getCents() const { return m_cents; }
};

int main()
{
  Cents cents1{ 6 };
  Cents cents2{ 8 };
  Cents centsSum{ cents1 + cents2 };
  std::cout << "I have " << centsSum.getCents() << " cents.\n";

  return 0;
}
\end{lstlisting}

通常来说并不推荐这么做,因为重要的函数定义放在分开的 .cpp 文件中更好。
而之后的教程中这么使用是为了代码的连续性。

\subsubsection*{为不同类型的运算对象重载操作符}

通常情况下,会有不同类型的运算对象的需要。例如 Cents(4) 与整数 6 相加返回 Cents(10)。

当 C++ 解析 \acode{x + y} 表达式时,\acode{x} 变为第一个参数,\acode{y} 变为第二个参数。
当 \acode{x} 与 \acode{y} 的类型相同,那么 \acode{x + y} 还是 \acode{y + x} 并没有区别,相同类型的操作符+ 会被调用。
然而当运算对象的类型不同时,\acode{x + y} 与 \acode{y + x} 并不相同。

例如,\acode{Cents(4) + 6} 将调用操作符+(Cents, int),
而 \acode{6 + Cents(4)} 将调用操作符+(int, Cents)。
因此为不同类型的二元进行重载操作符,实际上是需要编写两个函数 -- 以对应两种情况。

\begin{lstlisting}[language=C++]
#include <iostream>

class Cents
{
private:
  int m_cents {};

public:
  Cents(int cents) : m_cents{ cents } { }

  // 使用友元函数进行 Cents + int 的相加
  friend Cents operator+(const Cents& c1, int value);

  // 使用友元函数进行 int + Cents 的相加
  friend Cents operator+(int value, const Cents& c1);


  int getCents() const { return m_cents; }
};

// 注意:该函数不是一个成员函数!
Cents operator+(const Cents& c1, int value)
{
  // 使用 Cents 的构造函数与操作符+(Cents&, int)
  // 可以直接访问 m_cents 因为其为友元
  return { c1.m_cents + value };
}

// 注意:该函数不是一个成员函数!
Cents operator+(int value, const Cents& c1)
{
  // 使用 Cents 的构造函数与操作符+(int ,Cents&)
  // 可以直接访问 m_cents 因为其为友元
  return { c1.m_cents + value };
}

int main()
{
  Cents c1{ Cents{ 4 } + 6 };
  Cents c2{ 6 + Cents{ 4 } };

  std::cout << "I have " << c1.getCents() << " cents.\n";
  std::cout << "I have " << c2.getCents() << " cents.\n";

  return 0;
}
\end{lstlisting}

注意两个重载函数有着相同的实现 -- 这是因为它们做的事情是一样的,不同的只是参数的顺序。

\subsubsection*{另一个例子}

接下来看另一个例子:

\begin{lstlisting}[language=C++]
#include <iostream>

class MinMax
{
private:
  int m_min {}; // 已经见到的 min 值
  int m_max {}; // 已经见到的 max 值

public:
  MinMax(int min, int max)
    : m_min { min }, m_max { max }
  { }

  int getMin() const { return m_min; }
  int getMax() const { return m_max; }

  friend MinMax operator+(const MinMax& m1, const MinMax& m2);
  friend MinMax operator+(const MinMax& m, int value);
  friend MinMax operator+(int value, const MinMax& m);
};

MinMax operator+(const MinMax& m1, const MinMax& m2)
{
  // 获取 m1 与 m2 中的最小值
  int min{ m1.m_min < m2.m_min ? m1.m_min : m2.m_min };

  // 获取 m1 与 m2 中的最大值
  int max{ m1.m_max > m2.m_max ? m1.m_max : m2.m_max };

  return { min, max };
}

MinMax operator+(const MinMax& m, int value)
{
  // 获取 m 与 value 中的最小值
  int min{ m.m_min < value ? m.m_min : value };

  // 获取 m 与 value 中的最大值
  int max{ m.m_max > value ? m.m_max : value };

  return { min, max };
}

MinMax operator+(int value, const MinMax& m)
{
  // 调用 operator+(MinMax, int)
  return { m + value };
}

int main()
{
  MinMax m1{ 10, 15 };
  MinMax m2{ 8, 11 };
  MinMax m3{ 3, 12 };

  MinMax mFinal{ m1 + m2 + 5 + 8 + m3 + 16 };

  std::cout << "Result: (" << mFinal.getMin() << ", " <<
    mFinal.getMax() << ")\n";

  return 0;
}
\end{lstlisting}

\subsubsection*{使用其他操作符来实现操作符}

上述例子中,注意定义的操作符+(int, MinMax) 中调用了操作符+(MinMax, int),即得出相同结果。
这使得操作符+(int, MinMax) 的实现变成了一行代码,大大的减少了重复的代码,也同样使得函数更容易理解。

通常可以通过调用其他重载操作符来定义重载操作符。当这样做可以简化代码时,这样的做法则是提倡的。

\end{document}
