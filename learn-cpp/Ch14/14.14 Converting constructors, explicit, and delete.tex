\documentclass[../../LearnCpp.tex]{subfiles}

\begin{document}

\asubsection{14}{Converting constructors, explicit, and delete}

默认情况下 C++ 视任何构造函数为隐式转换操作符。考虑下面代码:

\begin{lstlisting}[language=C++]
#include <cassert>
#include <iostream>

class Fraction
{
private:
	int m_numerator;
	int m_denominator;

public:
	Fraction(int numerator = 0, int denominator = 1)
		 : m_numerator(numerator), m_denominator(denominator)
	{
		assert(denominator != 0);
	}

	Fraction(const Fraction& copy)
		: m_numerator(copy.m_numerator), m_denominator(copy.m_denominator)
	{
		std::cout << "Copy constructor called\n";
	}

	friend std::ostream& operator<<(std::ostream& out, const Fraction& f1);
	int getNumerator() { return m_numerator; }
	void setNumerator(int numerator) { m_numerator = numerator; }
};

void printFraction(const Fraction& f)
{
	std::cout << f;
}

std::ostream& operator<<(std::ostream& out, const Fraction& f1)
{
	out << f1.m_numerator << '/' << f1.m_denominator;
	return out;
}

int main()
{
	printFraction(6);

	return 0;
}
\end{lstlisting}

尽管函数 \acode{printFraction()} 预期一个 Fraction,我们提供的却是字面值整数 6。由于 Fraction 拥有一个构造函数将获取一个整数,
编译器将隐式转化字面值 6 成为一个 Fraction 对象。
这么做是借助于初始化 \acode{printFraction()} 参数 f 时使用了 \acode{Fraction(int, int)} 构造函数。

因此上述代码打印:

\begin{lstlisting}
6/1
\end{lstlisting}

这个隐式转换作用于所有的初始化(直接与拷贝)。

\subsubsection*{explicit 关键字}

在 Fraction 案例中隐式转换是有意义的,但在其他情况下可能并不可取,或是导致非预期的行为:

\begin{lstlisting}[language=C++]
#include <string>
#include <iostream>

class MyString
{
private:
	std::string m_string;
public:
	MyString(int x) // 分配 x 大小的字符串
	{
		m_string.resize(x);
	}

	MyString(const char* string) // 分配字符串用于存储字符串的值
	{
		m_string = string;
	}

	friend std::ostream& operator<<(std::ostream& out, const MyString& s);

};

std::ostream& operator<<(std::ostream& out, const MyString& s)
{
	out << s.m_string;
	return out;
}

void printString(const MyString& s)
{
	std::cout << s;
}

int main()
{
	MyString mine = 'x'; // 可编译并使用 MyString(int)
	std::cout << mine << '\n';

	printString('x'); // 可编译并使用 MyString(int)
	return 0;
}
\end{lstlisting}

上述例子中,用户尝试通过一个字符来初始化字符串。
由于字符是整数家庭的一部分,编译器使用了转换的构造函数 \acode{MyString(int)} 来隐式转换 char 成为 MyString。
该程序打印 MyString 非预期的结果。同样的,调用 \acode{printString('x')} 导致隐式转换返回同样的问题。

解决该问题的一种方法是使构造函数(以及转换函数)通过 explicit 关键字使它们变成显式的,做法是放置其在函数名前。
构造函数与转换函数是显式的情况下将不会使用\textit{隐式}构造函数或拷贝初始化:

\begin{lstlisting}[language=C++]
#include <string>
#include <iostream>

class MyString
{
private:
  std::string m_string;
public:
  // explicit 关键字使该构造函数的隐式转换变为不合法
  explicit MyString(int x) // 分配 x 大小的字符串
  {
    m_string.resize(x);
  }

  MyString(const char* string) // 分配字符串用于存储字符串的值
  {
    m_string = string;
  }

  friend std::ostream& operator<<(std::ostream& out, const MyString& s);

};

std::ostream& operator<<(std::ostream& out, const MyString& s)
{
  out << s.m_string;
  return out;
}

void printString(const MyString& s)
{
  std::cout << s;
}

int main()
{
  MyString mine = 'x'; // 编译错误,因为 MyString(int) 现在是 explicit 的,且没有任何与之匹配的
  std::cout << mine;

  printString('x'); // 编译错误,因为 MyString(int) 不能用于隐式转换

  return 0;
}
\end{lstlisting}

上述的程序不能通过编译,因为 \acode{MyString(int)} 变成了显式的了,且合适的转化构造函数不能被找到用于隐式转换 'x' 成为一个 MyString。

然而注意让一个构造函数 explicit 仅阻止\textit{隐式}转换。显式转换(通过 casting)仍然是被允许的:

\begin{lstlisting}[language=C++]
std::cout << static_cast<MyString>(5); // 允许:显式转换 5 至 MyString(int)
\end{lstlisting}

直接或标准初始化仍然会转换参数来匹配(标准初始化不会做狭窄转换,但是会做其他类型的转换):

\begin{lstlisting}[language=C++]
MyString str{'x'}; // 允许:初始化参数可能仍然被隐式转换来进行匹配
\end{lstlisting}

最佳实践:explicit 构造函数与用户定义转换成员函数可以阻止隐式转换错误。

\subsubsection*{delete 关键字}

MyString 的例子中,希望完全禁止 'x' 被转换成 MyString(无论是隐式或显式,因为结果并不符合直觉)。
一种办法是添加 \acode{MyString(char)} 构造函数,并使其私有:

\begin{lstlisting}[language=C++]
#include <string>
#include <iostream>

class MyString
{
private:
  std::string m_string;

  MyString(char) // MyString(char) 类型的对象不能在类的外部进行构造
  {
  }

public:
  // explicit 关键字使得该构造函数的隐式转换不合法
  explicit MyString(int x) // allocate string of size x
  {
    m_string.resize(x);
  }

  MyString(const char* string) // 分配字符串用于存储值
  {
    m_string = string;
  }

  friend std::ostream& operator<<(std::ostream& out, const MyString& s);

};

std::ostream& operator<<(std::ostream& out, const MyString& s)
{
  out << s.m_string;
  return out;
}

int main()
{
  MyString mine('x'); // 编译错误,因为 MyString(char) 是私有的
  std::cout << mine;
  return 0;
}
\end{lstlisting}

然而构造函数仍然可以在类的内部使用(私有访问仅对于调用该函数的非成员生效)。

一个更好的方案来解决这个问题就是使用 “delete” 关键字:

\begin{lstlisting}[language=C++]
#include <string>
#include <iostream>

class MyString
{
private:
  std::string m_string;

public:
  MyString(char) = delete; // 任何使用该构造函数的都视为错误

  explicit MyString(int x)
  {
    m_string.resize(x);
  }

  MyString(const char* string)
  {
    m_string = string;
  }

  friend std::ostream& operator<<(std::ostream& out, const MyString& s);

};

std::ostream& operator<<(std::ostream& out, const MyString& s)
{
  out << s.m_string;
  return out;
}

int main()
{
  MyString mine('x'); // 编译错误,因为 MyString(char) 是 deleted 的
  std::cout << mine;
  return 0;
}
\end{lstlisting}

\end{document}
