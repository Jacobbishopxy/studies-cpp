\documentclass[../../LearnCpp.tex]{subfiles}

\begin{document}

\asubsection{7}{Overloading the comparison operators}

在 5.6 的关系符以及浮点数比较中讨论了六种比较操作符。重载这些比较操作符相对来说比较简单,因为它们遵循相同的模式。

因为比较操作符同样都是二元操作符,并且不会修改左侧运算成员,因此我们将以友元函数的方式进行重载。

这里是一个重载了操作符== 与操作符!= 的 Car 类案例:

\begin{lstlisting}[language=C++]
#include <iostream>
#include <string>
#include <string_view>

class Car
{
private:
    std::string m_make;
    std::string m_model;

public:
    Car(std::string_view make, std::string_view model)
        : m_make{ make }, m_model{ model }
    {
    }

    friend bool operator== (const Car& c1, const Car& c2);
    friend bool operator!= (const Car& c1, const Car& c2);
};

bool operator== (const Car& c1, const Car& c2)
{
    return (c1.m_make == c2.m_make &&
            c1.m_model == c2.m_model);
}

bool operator!= (const Car& c1, const Car& c2)
{
    return (c1.m_make != c2.m_make ||
            c1.m_model != c2.m_model);
}

int main()
{
    Car corolla{ "Toyota", "Corolla" };
    Car camry{ "Toyota", "Camry" };

    if (corolla == camry)
        std::cout << "a Corolla and Camry are the same.\n";

    if (corolla != camry)
        std::cout << "a Corolla and Camry are not the same.\n";

    return 0;
}
\end{lstlisting}

上述代码非常的直观。

那么操作符< 与操作符> 呢?它们的直观意义并不明确,因此不去定义这些操作符更好。

最佳实践:仅去定义符合直觉的重载操作符。

然而有一个特例。如果希望对 Cars 列表进行排序呢?这种情况下,我们希望能重载比较操作符用于排序。
例如对 Cars 重载操作符< 意为根据字母顺序进行排序。

一些标准库中的容器类需要重载操作符< 这样它们可以维护元素的排序。

这里是另一个重载了 6 个所有比较操作符的例子:

\begin{lstlisting}[language=C++]
#include <iostream>

class Cents
{
private:
    int m_cents;

public:
    Cents(int cents)
	: m_cents{ cents }
	{}

    friend bool operator== (const Cents& c1, const Cents& c2);
    friend bool operator!= (const Cents& c1, const Cents& c2);

    friend bool operator< (const Cents& c1, const Cents& c2);
    friend bool operator> (const Cents& c1, const Cents& c2);

    friend bool operator<= (const Cents& c1, const Cents& c2);
    friend bool operator>= (const Cents& c1, const Cents& c2);
};

bool operator== (const Cents& c1, const Cents& c2)
{
    return c1.m_cents == c2.m_cents;
}

bool operator!= (const Cents& c1, const Cents& c2)
{
    return c1.m_cents != c2.m_cents;
}

bool operator< (const Cents& c1, const Cents& c2)
{
    return c1.m_cents < c2.m_cents;
}

bool operator> (const Cents& c1, const Cents& c2)
{
    return c1.m_cents > c2.m_cents;
}

bool operator<= (const Cents& c1, const Cents& c2)
{
    return c1.m_cents <= c2.m_cents;
}

bool operator>= (const Cents& c1, const Cents& c2)
{
    return c1.m_cents >= c2.m_cents;
}

int main()
{
    Cents dime{ 10 };
    Cents nickel{ 5 };

    if (nickel > dime)
        std::cout << "a nickel is greater than a dime.\n";
    if (nickel >= dime)
        std::cout << "a nickel is greater than or equal to a dime.\n";
    if (nickel < dime)
        std::cout << "a dime is greater than a nickel.\n";
    if (nickel <= dime)
        std::cout << "a dime is greater than or equal to a nickel.\n";
    if (nickel == dime)
        std::cout << "a dime is equal to a nickel.\n";
    if (nickel != dime)
        std::cout << "a dime is not equal to a nickel.\n";

    return 0;
}
\end{lstlisting}

\subsubsection*{最小化比较的冗余}

上述例子可以注意到每个重载比较操作符有多么的相似。重载比较操作符倾向有高度的冗余,并且随着实现的复杂性增加,冗余随之增加。

幸运的是,很多比较操作符可以使用其他比较操作符来实现:

\begin{itemize}
  \item 操作符!= 可以通过 !(操作符==) 来实现
  \item 操作符> 可以通过与之相反的 < 来实现
  \item 操作符>= 可以通过 !(操作符<) 来实现
  \item 操作符<= 可以通过 !(操作符>) 来实现
\end{itemize}

这就意味着我们仅需要实现操作符== 以及操作符<,那么其它的四个比较操作符可以根据这两个操作符来被定义!以下是更新后的 Cents 案例:

\begin{lstlisting}[language=C++]
#include <iostream>

class Cents
{
private:
    int m_cents;

public:
    Cents(int cents)
        : m_cents{ cents }
    {}

    friend bool operator== (const Cents& c1, const Cents& c2);
    friend bool operator!= (const Cents& c1, const Cents& c2);

    friend bool operator< (const Cents& c1, const Cents& c2);
    friend bool operator> (const Cents& c1, const Cents& c2);

    friend bool operator<= (const Cents& c1, const Cents& c2);
    friend bool operator>= (const Cents& c1, const Cents& c2);

};

bool operator== (const Cents& c1, const Cents& c2)
{
    return c1.m_cents == c2.m_cents;
}

bool operator!= (const Cents& c1, const Cents& c2)
{
    return !(operator==(c1, c2));
}

bool operator< (const Cents& c1, const Cents& c2)
{
    return c1.m_cents < c2.m_cents;
}

bool operator> (const Cents& c1, const Cents& c2)
{
    return operator<(c2, c1);
}

bool operator<= (const Cents& c1, const Cents& c2)
{
    return !(operator>(c1, c2));
}

bool operator>= (const Cents& c1, const Cents& c2)
{
    return !(operator<(c1, c2));
}

int main()
{
    Cents dime{ 10 };
    Cents nickel{ 5 };

    if (nickel > dime)
        std::cout << "a nickel is greater than a dime.\n";
    if (nickel >= dime)
        std::cout << "a nickel is greater than or equal to a dime.\n";
    if (nickel < dime)
        std::cout << "a dime is greater than a nickel.\n";
    if (nickel <= dime)
        std::cout << "a dime is greater than or equal to a nickel.\n";
    if (nickel == dime)
        std::cout << "a dime is equal to a nickel.\n";
    if (nickel != dime)
        std::cout << "a dime is not equal to a nickel.\n";

    return 0;
}
\end{lstlisting}

如此便在需要修改的时候,仅需要更新操作符== 以及操作符< 而不是所有的六种比较操作符!

\end{document}
