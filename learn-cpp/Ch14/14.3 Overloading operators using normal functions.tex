\documentclass[../../LearnCpp.tex]{subfiles}

\begin{document}

\asubsection{3}{Overloading operators using normal functions}

上一节中讲到了使用友元函数来重载操作符。
这很方便因为可以给与直接访问需要操作的类内部成员的权利。
然而当不需要访问时,可以使用普通函数来编写重载函数。

\begin{lstlisting}[language=C++]
#include <iostream>

class Cents
{
private:
  int m_cents{};

public:
  Cents(int cents)
    : m_cents{ cents }
  {}

  int getCents() const { return m_cents; }
};

// 注意:该函数不是成员函数也不是友元函数!
Cents operator+(const Cents& c1, const Cents& c2)
{
  // 使用 Cents 构造函数与操作符+(int, int)
  // 这里并不需要直接访问私有成员
  return Cents{ c1.getCents() + c2.getCents() };
}

int main()
{
  Cents cents1{ 6 };
  Cents cents2{ 8 };
  Cents centsSum{ cents1 + cents2 };
  std::cout << "I have " << centsSum.getCents() << " cents.\n";

  return 0;
}
\end{lstlisting}

由于普通函数和友元函数效果完全一样(仅在访问私有成员时拥有不同等级),
通常来说不会区分它们。一个区别在于友元函数声明在类内部也可以作为原型;
而普通函数的版本,用户需要提供自己的函数原型。

Cents.h:

\begin{lstlisting}[language=C++]
#ifndef CENTS_H
#define CENTS_H

class Cents
{
private:
  int m_cents{};

public:
  Cents(int cents)
    : m_cents{ cents }
  {}

  int getCents() const { return m_cents; }
};

// 需要显式提供操作符+ 的原型,这样使得其他文件夹知道该重载的存在
Cents operator+(const Cents& c1, const Cents& c2);

# endif
\end{lstlisting}

Cents.cpp:

\begin{lstlisting}[language=C++]
#include "Cents.h"

// 注意:该函数不是成员函数也不是友元函数!
Cents operator+(const Cents& c1, const Cents& c2)
{
  // 使用 Cents 构造函数与操作符+(int, int)
  // 这里并不需要直接访问私有成员
  return { c1.getCents() + c2.getCents() };
}
\end{lstlisting}

main.cpp:

\begin{lstlisting}[language=C++]
#include "Cents.h"
#include <iostream>

int main()
{
  Cents cents1{ 6 };
  Cents cents2{ 8 };
  Cents centsSum{ cents1 + cents2 }; // 如果在 Cents.h 中没有原型,这里会编译错误
  std::cout << "I have " << centsSum.getCents() << " cents.\n";

  return 0;
}
\end{lstlisting}

通常而言,如果存在成员函数可用的情况,普通函数比友元函数更为推荐(越少的函数解除类的内部越好)。
然而,不要仅仅为了使用普通函数来重载操作符而去添加额外的访问函数,直接使用友元函数!

最佳实践:在不需要添加额外的访问函数的情况下,推荐使用普通函数而不是友元函数来重载操作符。

\end{document}
