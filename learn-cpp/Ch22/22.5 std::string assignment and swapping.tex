\documentclass[../../LearnCpp.tex]{subfiles}

\begin{document}

\asubsection{5}{std::string assignment and swapping}

\subsubsection*{字符串赋值}

给字符串赋值的最简单方法是使用重载操作符= 函数。同样也有一个成员函数 \acode{assign()} 复制了一些功能。

\begin{quotation}
  \textbf{string\& string::operator=(const string\& str)}

  \textbf{string\& string::assign(const string\& str)}

  \textbf{string\& string::operator=(const char* str)}

  \textbf{string\& string::assign(const char* str)}

  \textbf{string\& string::operator=(char c)}

  \begin{itemize}
    \item 这些函数指派不同类型的值给字符串
    \item 这些函数返回 *this 以便它们可以被串联
    \item 注意没有单个字符的 assign() 函数
  \end{itemize}

  \begin{lstlisting}[language=C++]
  std::string sString;

  // Assign a string value
  sString = std::string("One");
  std::cout << sString << '\n';

  const std::string sTwo("Two");
  sString.assign(sTwo);
  std::cout << sString << '\n';

  // Assign a C-style string
  sString = "Three";
  std::cout << sString << '\n';

  sString.assign("Four");
  std::cout << sString << '\n';

  // Assign a char
  sString = '5';
  std::cout << sString << '\n';

  // Chain assignment
  std::string sOther;
  sString = sOther = "Six";
  std::cout << sString << ' ' << sOther << '\n';
  \end{lstlisting}

  输出:

  \begin{lstlisting}
  One
  Two
  Three
  Four
  5
  Six Six
  \end{lstlisting}
\end{quotation}

\acode{assign()} 成员函数还有一些其他的样式:

\begin{quotation}
  \textbf{string\& string::assign(const string\& str, size\_type index, size\_type len)}

  \begin{itemize}
    \item 赋值 str 的子字符串,从 index 开始,长度为 len
    \item 如果索引超出边界,抛出 out\_of\_range 异常
    \item 返回 *this 以便可以被串联
  \end{itemize}

  \begin{lstlisting}[language=C++]
  const std::string sSource("abcdefg");
  std::string sDest;

  sDest.assign(sSource, 2, 4); // assign a substring of source from index 2 of length 4
  std::cout << sDest << '\n';
  \end{lstlisting}

  输出:

  \begin{lstlisting}
  cdef
  \end{lstlisting}
\end{quotation}

\begin{quotation}
  \textbf{string\& string::assign(const char* chars, size\_type len)}

  \begin{itemize}
    \item 从 C-style 字符数组中赋值 len 长的字符
    \item 如果结果超出字符最大数值,抛出 length\_error 异常
    \item 返回 *this 以便可以被串联
  \end{itemize}

  \begin{lstlisting}[language=C++]
  std::string sDest;

  sDest.assign("abcdefg", 4);
  std::cout << sDest << '\n';
  \end{lstlisting}

  输出:

  \begin{lstlisting}
  abcd
  \end{lstlisting}

  该函数有潜在风险,不建议使用
\end{quotation}

\begin{quotation}
  \textbf{string\& string::assign(size\_type len, char c)}

  \begin{itemize}
    \item 赋值 len 长度的字符 c
    \item 如果结果超出字符最大数值,抛出 length\_error 异常
    \item 返回 *this 以便可以被串联
  \end{itemize}

  \begin{lstlisting}[language=C++]
  std::string sDest;

  sDest.assign(4, 'g');
  std::cout << sDest << '\n';
  \end{lstlisting}

  输出:

  \begin{lstlisting}
  gggg
  \end{lstlisting}
\end{quotation}

\subsubsection*{Swapping}

如果拥有两个字符串并想让它们值互换,有两个名为 \acode{swap()} 的函数可供使用:

\begin{quotation}
  \textbf{void string::swap(string\& str)}

  \textbf{void swap(string\& str1, string\& str2)}

  \begin{itemize}
    \item 两个函数可以互换字符串的值。成员函数 swaps 的是 *this 与 str,全局函数 swaps 的是 str1 与 str2
    \item 这俩函数效率高,互换字符的场景下不要使用赋值,而是这俩函数
  \end{itemize}

  \begin{lstlisting}[language=C++]
  std::string sStr1("red");
  std::string sStr2("blue");

  std::cout << sStr1 << ' ' << sStr2 << '\n';
  swap(sStr1, sStr2);
  std::cout << sStr1 << ' ' << sStr2 << '\n';
  sStr1.swap(sStr2);
  std::cout << sStr1 << ' ' << sStr2 << '\n';
  \end{lstlisting}

  输出:

  \begin{lstlisting}
  red blue
  blue red
  red blue
  \end{lstlisting}
\end{quotation}

\end{document}
