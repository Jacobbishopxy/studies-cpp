\documentclass[../../LearnCpp.tex]{subfiles}

\begin{document}

\asubsection{14}{Void pointers}

\textbf{void 指针},又称通用指针,是一种特殊类型的指针可以用作于指向任何数据类型的对象!

\begin{lstlisting}[language=C++]
void* ptr; // ptr 为 void pointer
\end{lstlisting}

void 指针可以指向任何数据类型的对象:

\begin{lstlisting}[language=C++]
int nValue;
float fValue;

struct Something
{
    int n;
    float f;
};

Something sValue;

void* ptr;
ptr = &nValue; // 有效
ptr = &fValue; // 有效
ptr = &sValue; // 有效
\end{lstlisting}

然而因为 void 指针不知道它所指向的对象类型,解引用 void 指针是不合法的。
相反,void 指针必须先强制转换成其它类型的指针才可以进行解引用操作。

\begin{lstlisting}[language=C++]
int value{ 5 };
void* voidPtr{ &value };

// std::cout << *voidPtr << '\n'; // 不合法:void 指针解引用

int* intPtr{ static_cast<int*>(voidPtr) }; // 强制转换为 int 指针

std::cout << *intPtr << '\n'; // 解引用
\end{lstlisting}

接下来显而易见的问题就是:如果一个 void 指针不知道其指向的是什么,如何进行强制转换呢?
根本上而言,用户需要一直进行追踪。

\begin{lstlisting}[language=C++]
#include <iostream>
#include <cassert>

enum class Type
{
    tInt, // 注意:这里不能使用 int,因为是关键字
    tFloat,
    tCString
};

void printValue(void* ptr, Type type)
{
    switch (type)
    {
    case Type::tInt:
        std::cout << *static_cast<int*>(ptr) << '\n'; // 强制转换 int 指针并解引用
        break;
    case Type::tFloat:
        std::cout << *static_cast<float*>(ptr) << '\n'; // 强制转换 float 指针并解引用
        break;
    case Type::tCString:
        std::cout << static_cast<char*>(ptr) << '\n'; // 强制转换 char 指针(无解引用)
        // std::cout 将 char* 视为 C-style 字符串
        // 如果在这里解引用了,则只会打印单个字符
        break;
    default:
        assert(false && "type not found");
        break;
    }
}

int main()
{
    int nValue{ 5 };
    float fValue{ 7.5f };
    char szValue[]{ "Mollie" };

    printValue(&nValue, Type::tInt);
    printValue(&fValue, Type::tFloat);
    printValue(szValue, Type::tCString);

    return 0;
}
\end{lstlisting}

\subsubsection*{混合 void 指针}

void 指针可以设置空值:

\begin{lstlisting}[language=C++]
void* ptr{ nullptr };
\end{lstlisting}

最佳实践:通常而言,除了必须使用时,应该避免使用 void 指针。

\end{document}
