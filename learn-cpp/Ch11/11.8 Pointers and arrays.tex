\documentclass[../../LearnCpp.tex]{subfiles}

\begin{document}

\asubsection{8}{Pointers and arrays}

当一个固定数组用于一个表达式中时,固定数值则会 \textbf{decay}(被隐式转换)成为一个指向数组第一个元素的指针。

\begin{lstlisting}[language=C++]
#include <iostream>

int main()
{
    int array[5]{ 9, 7, 5, 3, 1 };

    // 打印数组第一个元素的地址
    std::cout << "Element 0 has address: " << &array[0] << '\n';

    // 打印数组 decay 后的指针的值
    std::cout << "The array decays to a pointer holding address: " << array << '\n';


    return 0;
}
\end{lstlisting}

打印结果为:

\begin{lstlisting}
Element 0 has address: 0042FD5C
The array decays to a pointer holding address: 0042FD5C
\end{lstlisting}

一个数组和一个数组的指针是完全相同的,这是 C++ 中一个常见的谬误,它们并不相同。
上述的案例中,数组的类型是“int[5]”,并且它的“值”是数组元素本身。
而数组的指针类型是“int\*”,其值为数组第一个元素的地址。

数组中所有的元素仍然可以通过指针进行访问,
但是由数组类型派生出来的信息(例如数组的长度)不可被指针所访问。

不过实际上,大多数情况下,用户可以将固定数组视为指针。

\begin{lstlisting}[language=C++]
int array[5]{ 9, 7, 5, 3, 1 };

// 数组解引用返回第一个元素
std::cout << *array; // 打印 9!

char name[]{ "Jason" }; // C-style 字符串(同样也是一个数组)
std::cout << *name << '\n'; // 打印 'J'
\end{lstlisting}

注意我们\textit{实际上}并没有对数组自身解引用。
数组(类型 \acode{int[5]})被隐式转换成了一个指针(类型 \acode{int*}),
解引用了指针获取了在其内存地址的值(即数组第一个元素的值)。

同样也可以让指针指向数组:

\begin{lstlisting}[language=C++]
#include <iostream>

int main()
{
    int array[5]{ 9, 7, 5, 3, 1 };
    std::cout << *array << '\n'; // 打印 9

    int* ptr{ array };
    std::cout << *ptr << '\n'; // 打印 9

    return 0;
}
\end{lstlisting}

这是因为数组退化成为了一个 \acode{int*} 类型的指针,新的指针(同样也是 \acode{int*})拥有同样的类型。

这里有一些例子来解释,固定数组与指针的不同点。

最主要的区别是再是用 sizeof() 操作符。
对固定数组使用时,其返回整个数组的大小(数组长度 \* 元素大小);
而对指针使用时,返回的是指针的大小。

\begin{lstlisting}[language=C++]
#include <iostream>

int main()
{
    int array[5]{ 9, 7, 5, 3, 1 };

    std::cout << sizeof(array) << '\n'; // 打印 sizeof(int) * array length

    int* ptr{ array };
    std::cout << sizeof(ptr) << '\n'; // 打印指针大小

    return 0;
}
\end{lstlisting}

结果:

\begin{lstlisting}
20
4
\end{lstlisting}

第二个区别在于使用 address-of 操作符(\acode{&})。
获取指针的地址返回的是指针变量的内存地址;获取数组地址返回的是整个数组的指针,
该指针同样指向数组的第一个元素,但是类型信息是不同的。

\begin{lstlisting}[language=C++]
#include <iostream>

int main()
{
    int array[5]{ 9, 7, 5, 3, 1 };
    std::cout << array << '\n';   // 类型 int[5],打印 009DF9D4
    std::cout << &array << '\n';  // 类型 int(*)[5],打印 009DF9D4

    std::cout << '\n';

    int* ptr{ array };
    std::cout << ptr << '\n';     // 类型 int*,打印 009DF9D4
    std::cout << &ptr << '\n';    // 类型 int**,打印 009DF9C8

    return 0;
}
// h/t to reader PacMan for this example
\end{lstlisting}

11.2 中提到过拷贝大数组是非常昂贵的,因此在传递数组至函数时 C++ 不会对数组进行拷贝,
而是衰退成指针,该指针被传递:

\begin{lstlisting}[language=C++]
#include <iostream>

void printSize(int* array)
{
    // 数组在这里被视为指针
    std::cout << sizeof(array) << '\n'; // 打印的是指针的大小,而不是数组的大小!
}

int main()
{
    int array[]{ 1, 1, 2, 3, 5, 8, 13, 21 };
    std::cout << sizeof(array) << '\n'; // 打印 sizeof(int) * array length

    printSize(array); // 数组参数在这里衰退成指针

    return 0;
}
\end{lstlisting}

注意,即使参数声明是固定数组,数组被隐式转换成指针照样发生:

\begin{lstlisting}[language=C++]
#include <iostream>

// C++ 将隐式转换 array[] 至 *array
void printSize(int array[])
{
    // array 在这里被视为指针,而不是固定数组
    std::cout << sizeof(array) << '\n'; // 打印的是指针的大小,而不是数组的大小!
}

int main()
{
    int array[]{ 1, 1, 2, 3, 5, 8, 13, 21 };
    std::cout << sizeof(array) << '\n'; // will print sizeof(int) * array length

    printSize(array); // 数组参数在这里衰退成指针

    return 0;
}
\end{lstlisting}

上述例子中,C++ 隐式转换参数数组语法(\acode{[]})成为指针语法(\acode{*}),
这就意味着下面两个函数声明是完全相同的:

\begin{lstlisting}[language=C++]
void printSize(int array[]);
void printSize(int* array);
\end{lstlisting}

最佳实践:推荐使用指针语法(\acode{*})。

最后值得注意的是当数组是结构体或者类里面的一部分的时候,
传递整个结构体或类至函数时,它们是不会衰退成指针的。

\end{document}
