\documentclass[../../LearnCpp.tex]{subfiles}

\begin{document}

\asubsection{11}{Dynamic memory allocation with new and delete}

C++ 支持三种基本类型的内存分配:

\begin{itemize}
    \item \textbf{静态内存分配} static memory allocation发生在静态与全局变量。
          这些类型的变量内存仅在程序运行时分配一次,并持续程序整个过程。
    \item \textbf{自动内存分配} automatic memory allocation发生在函数参数与本地变量。
          这些类型的变量内存在进入相关的代码块时进行分配,在代码块结束时释放。
    \item \textbf{动态内存分配} dynamic memory allocation本文详解。
\end{itemize}

静态分配和自动分配有两点相同:

\begin{itemize}
    \item 变量/数组的大小在编译期就必须知道
    \item 内存分配和内存释放是自动的(当变量实例化/销毁时)
\end{itemize}

\textbf{动态内存分配}是运行程序时向操作系统请求内存的一种方式。
这些内存不是从有限的栈内存而来,而是由操作系统所管理的更大的\textbf{堆}内存提供。

\subsubsection*{动态分配单个变量}

为了动态分配*单个*变量,需要使用 \textbf{new} 操作符:

\begin{lstlisting}[language=C++]
new int; // 动态分配一个整数(并丢弃其值)
\end{lstlisting}

上述是向操作系统请求一个整数的内存。
new 操作符创建使用了该内存的对象,并返回一个包含了其被分配的内存\textit{地址}的指针。

通常来说会将返回值分配给用户自己的指针变量,这样便于之后访问:

\begin{lstlisting}[language=C++]
int* ptr{ new int };
\end{lstlisting}

可以通过指针访问内存间接进行操作:

\begin{lstlisting}[language=C++]
*ptr = 7; // 赋值 7 至内存地址
\end{lstlisting}

\subsubsection*{初始化一个动态分配的变量}

当动态分配一个变量时,可以选择直接初始化或者标准初始化:

\begin{lstlisting}[language=C++]
int* ptr1{ new int (5) };   // 使用直接初始化
int* ptr2{ new int { 6 } }; // 使用标准初始化
\end{lstlisting}

\subsubsection*{删除单个变量}

当使用完动态分配的变量时,需要显式的告诉 C++ 释放内存。
对于单个变量而已,使用 \textbf{delete} 操作符:

\begin{lstlisting}[language=C++]
// 假设 ptr 之前已经被 new 操作符分配过了
delete ptr;     // 将 ptr 指向的内存还给操作系统
ptr = nullptr;  // 设置 ptr 成为空指针
\end{lstlisting}

\subsubsection*{悬垂指针}

C++ 不会对内存释放或是删除指针的值做任何的保证。
大多数情况下,操作系统返回的内存总是包含与之前其返回的值一致,
同时指针会指向被内存释放的地址。

指向被内存释放地址的指针被称为\textbf{悬垂指针} dangling pointer。
间接通过或删除悬垂指针将会发生未定义行为。

\begin{lstlisting}[language=C++]
#include <iostream>

int main()
{
    int* ptr{ new int }; // 动态分配整数
    *ptr = 7; // 值放入内存地址

    delete ptr; // 返回操作系统的内存,ptr 现在是悬垂指针

    std::cout << *ptr; // 间接通过悬垂指针将会发生未定义行为
    delete ptr; // 再次进行内存释放将会发生未定义行为

    return 0;
}
\end{lstlisting}

释放内存可能会创建若干个悬垂指针:

\begin{lstlisting}[language=C++]
#include <iostream>

int main()
{
    int* ptr{ new int{} }; // 动态分配整数
    int* otherPtr{ ptr };  // otherPtr 现在指向同一个内存地址

    delete ptr; // 返回操作系统的内存,ptr 和 otherPtr 现在都是悬垂指针
    ptr = nullptr; // ptr 设为 nullptr

    // 然而 otherPtr 仍然还是一个悬垂指针!

    return 0;
}
\end{lstlisting}

最佳实践:设置被删除的指针为空指针,除非该指针会立刻离开作用域。

\subsubsection*{操作符 new 可能会失败}

当向操作系统请求内存时,在很罕见的情况下,操作系统有可能没有任何内存可以被分配。

默认情况下,如果失败 \textit{bad\_alloc} 异常会被抛出。
如果这个异常不能被正确的处理,那么程序则会被终结。

多数情况下,抛出异常是非预期的,因此有另一种方式的 new 可以在内存不能被分配时返回空指针。
通过在 new 关键字与分配的类型之间增加常量 std::nothrow:

\begin{lstlisting}[language=C++]
int* value { new (std::nothrow) int }; // 当整数分配失败时,值会被设为空指针
\end{lstlisting}

注意如果尝试间接通过该指针,未定义行为将出现(大概率程序崩溃)。
因此最佳实践是检查所有的内存请求来确保它们成功了再去使用内存分配:

\begin{lstlisting}[language=C++]
int* value { new (std::nothrow) int{} }; // 请求一个整数的内存
if (!value) // 处理返回 null 时的情况
{
    // 错误处理
    std::cerr << "Could not allocate memory\n";
}
\end{lstlisting}

\subsubsection*{空指针与动态内存分配}

在处理动态内存分配时,空指针(设为 nullptr 的指针)实际上很有用。

\begin{lstlisting}[language=C++]
// 如果 ptr 没有被分配,则进行分配
if (!ptr)
    ptr = new int;
\end{lstlisting}

删除空指针没有影响,因此不需要下面的代码:

\begin{lstlisting}[language=C++]
if (ptr)
    delete ptr;
\end{lstlisting}

而是可以直接:

\begin{lstlisting}[language=C++]
delete ptr;
\end{lstlisting}

\subsubsection*{内存泄漏}

动态分配的内存会一直存在直到显式的释放内存或者程序结束(操作系统进行清理)。
然而,用于保存动态分配地址的指针遵循着普通的作用域规则。
这样的不匹配会导致有趣的问题:

\begin{lstlisting}[language=C++]
void doSomething()
{
    int* ptr{ new int{} };
}
\end{lstlisting}

这个函数进行了整数的内存分配,但是永远没有被删除。
因为指针变量仅仅是普通的变量,当函数结束时,ptr 离开作用域。
正因为 ptr 是唯一保存动态分配地址的变量,当 ptr 被销毁时便不再有指向动态分配内存的引用。
这就意味着程序“失去了”动态分配的内存。结果该动态分配的整数不能再被删除。

这被称为\textbf{内存泄漏}。
内存泄漏发生在将内存还给操作系统之前,程序丢失了动态分配内存的地址,操作系统不能使用该内存,
因为该内存被视为仍然在程序中使用。

尽管离开作用域会导致内存泄漏,还有其他一些方法也会导致内存泄漏:

\begin{lstlisting}[language=C++]
int value = 5;
int* ptr{ new int{} }; // 分配内存
ptr = &value; // 丢失旧地址,导致内存泄漏
\end{lstlisting}

上述可以在重新赋值前删除指针进行修复:

\begin{lstlisting}[language=C++]
int value{ 5 };
int* ptr{ new int{} }; // 分配内存
delete ptr;   // 将内存还给操作系统
ptr = &value; // 重新赋值
\end{lstlisting}

同样的通过双分配也会导致内存泄漏:

\begin{lstlisting}[language=C++]
int* ptr{ new int{} };
ptr = new int{}; // 丢失旧地址,导致内存泄漏
\end{lstlisting}

\end{document}
