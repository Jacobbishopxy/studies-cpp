\documentclass[../../LearnCpp.tex]{subfiles}

\begin{document}

\asubsection{5}{Constructors}

当类(或结构体)的所有成员都是公有时,可以使用列表初始化来进行\textbf{聚合初始化} aggregate initialization 来初始化类(或结构体)。

\begin{lstlisting}[language=C++]
class Foo
{
public:
    int m_x {};
    int m_y {};
};

int main()
{
    Foo foo { 6, 7 }; // 列表初始化

    return 0;
}
\end{lstlisting}

然而一旦让任意一个成员变量为私有时,便不再允许使用聚合初始化了。
这也符合常理:如果不能直接访问变量(因为其为私有),则不应该允许直接初始化它。

\subsubsection*{构造函数}

\textbf{构造函数}是一种特殊的类成员函数,在创建一个该类的对象时会被自动调用。
构造函数通常通过用户提供的值进行初始化类的成员变量,
或者是构造类时任何必要的步骤(例如打开文件或者数据库)。

在一个构造函数被执行后,对象应该处于一个被定义好的,以及可使用的状态。

有别于普通的成员函数,构造函数的命名需要遵守规定的规则:

\begin{itemize}
    \item 与类同名(大小写一致)
    \item 没有返回值(void 也不行)
\end{itemize}

\subsubsection*{默认构造函数以及默认初始化}

一个不带有任何参数的构造函数(或者所有参数带有默认值)被称为\textbf{默认构造器}。
如果用户没有提供初始化值,则调用默认构造函数。

\begin{lstlisting}[language=C++]
#include <iostream>

class Fraction
{
private:
    int m_numerator {};
    int m_denominator {};

public:
    Fraction() // 默认构造函数
    {
        m_numerator = 0;
        m_denominator = 1;
    }

    int getNumerator() { return m_numerator; }
    int getDenominator() { return m_denominator; }
    double getValue() { return static_cast<double>(m_numerator) / m_denominator; }
};

int main()
{
    Fraction frac{}; // 调用 Fraction() 默认构造函数
    std::cout << frac.getNumerator() << '/' << frac.getDenominator() << '\n';

    return 0;
}
\end{lstlisting}

\subsubsection*{值初始化}

上述代码,使用了值初始化 value-initialization 初始化类对象。

\begin{lstlisting}[language=C++]
Fraction frac {}; // 使用空的花括号进行值初始化
\end{lstlisting}

同样也可以使用默认初始化 default-initialization 来初始化类对象。

\begin{lstlisting}[language=C++]
Fraction frac; // 默认初始化 default-initialization,调用默认构造函数
\end{lstlisting}

大多数情况下,值初始化与默认初始化一个类对象会返回同样的结果:默认构造函数被调用。

大多数程序员偏向使用默认初始化而不是值初始化一个类对象。
这是因为当使用值初始化时,在一些确定的情况下,编译器调用默认构造函数之前,
会零初始化 zero-initialize 类成员,这就带来一些性能损失(C++ 程序员不会关心他们不使用的特性)。

\begin{lstlisting}[language=C++]
#include <iostream>

class Fraction
{
private:
    // 移除初始化器
    int m_numerator;
    int m_denominator;

public:
    // 移除默认构造函数

    int getNumerator() { return m_numerator; }
    int getDenominator() { return m_denominator; }
    double getValue() { return static_cast<double>(m_numerator) / m_denominator; }
};

int main()
{
    Fraction frac;
    // frac 没有被初始化,访问其成员会导致未定义的行为
    std::cout << frac.getNumerator() << '/' << frac.getDenominator() << '\n';

    return 0;
}
\end{lstlisting}

最佳实践:推荐值初始化 value-initialization 而不是默认初始化 default-initialization。

\subsubsection*{使用带有参数的构造函数进行直接初始化和列表初始化}

由于默认构造函数方便了用户可以确保以合理的默认值进行初始化,
而多数时候又希望类的实例化可以带有指定的入参。
幸运的是,构造函数同样也可以带有参数。

\begin{lstlisting}[language=C++]
#include <cassert>

class Fraction
{
private:
    int m_numerator {};
    int m_denominator {};

public:
    Fraction() // 默认构造函数
    {
         m_numerator = 0;
         m_denominator = 1;
    }

    // 带有两个参数的构造函数,一个带有默认值
    Fraction(int numerator, int denominator=1)
    {
        assert(denominator != 0);
        m_numerator = numerator;
        m_denominator = denominator;
    }

    int getNumerator() { return m_numerator; }
    int getDenominator() { return m_denominator; }
    double getValue() { return static_cast<double>(m_numerator) / m_denominator; }
};
\end{lstlisting}

注意现在有两个构造函数:一个是默认构造函数,一个是带有两个入参的构造函数。
它们可以和平的共存是因为函数重载。
实际上可以定义很多个构造函数,只要它们都是唯一的签名(入参的数量与类型)。

那么如何使用带有参数的构造函数呢?非常简单,使用列表初始化或者直接初始化:

\begin{lstlisting}[language=C++]
Fraction fiveThirds{ 5, 3 };  // 列表初始化 list initialization,调用 Fraction(int, int)
Fraction threeQuarters(3, 4); // 直接初始化 Direct initialization,也调用 Fraction(int, int)
\end{lstlisting}

同样的,这里推荐列表初始化。
在后面章节会说明原因(模板与 \acode{std::initializer_list})何时使用直接初始化。
有一些其它特别的构造函数可能会导致花括号初始化不一样,这种情况下需要使用直接初始化。
之后的章节会进行讲解。

注意构造器的第二个参数带有默认值,因为下面代码也是合法的:

\begin{lstlisting}[language=C++]
Fraction six{ 6 }; // 调用 Fraction(int, int) 构造函数,第二个参数使用默认值 1
\end{lstlisting}

最佳实践:推荐使用花括号进行类对象的初始化。

\subsubsection*{使用等号与类进行拷贝初始化}

与基础变量相似,类也可以被拷贝初始化。

\begin{lstlisting}[language=C++]
Fraction six = Fraction{ 6 }; // 拷贝初始化 Fraction,将调用 Fraction(6, 1)
Fraction seven = 7; // 拷贝初始化 Fraction。编译器将尝试找到转换 7 成为一个 Fraction 的方式,将唤起 Fraction(7, 1) 构造函数
\end{lstlisting}

然而建议不要使用这样的方式进行初始化类,因为可能会变得更低效。
尽管直接初始化,列表初始化,以及拷贝初始化都可以作用于基础类型,
类的拷贝初始化却不一致(尽管最终结果通常一致)。
后续章节将会探讨更多的细节。

\subsubsection*{减少构造函数}

上述带有两个构造函数的代码可以简化成:

\begin{lstlisting}[language=C++]
#include <cassert>

class Fraction
{
private:
    int m_numerator {};
    int m_denominator {};

public:
    // 默认构造函数
    Fraction(int numerator=0, int denominator=1)
    {
        assert(denominator != 0);

        m_numerator = numerator;
        m_denominator = denominator;
    }

    int getNumerator() { return m_numerator; }
    int getDenominator() { return m_denominator; }
    double getValue() { return static_cast<double>(m_numerator) / m_denominator; }
};
\end{lstlisting}

\subsubsection*{隐式生成默认构造函数}

如果类没有构造函数,C++ 则会为类自动生成一个公有的默认构造器。
有时这被称为\textbf{隐式构造函数} implicit constructor(或者隐式生成构造函数)。

\begin{lstlisting}[language=C++]
class Date
{
private:
    int m_year{ 1900 };
    int m_month{ 1 };
    int m_day{ 1 };

    // 没有用户提供的构造函数,编译器生成默认构造器
};

int main()
{
    Date date{};

    return 0;
}
\end{lstlisting}

如果类拥有其它的构造函数,那么隐式生成构造函数则不会被提供:

\begin{lstlisting}[language=C++]
class Date
{
private:
    int m_year{ 1900 };
    int m_month{ 1 };
    int m_day{ 1 };

public:
    Date(int year, int month, int day) // 普通的非默认构造函数
    {
        m_year = year;
        m_month = month;
        m_day = day;
    }

    // 没有隐式构造函数提供,因为用户已经定义了
};

int main()
{
    Date date{}; // 错误:不可以实例化对象,因为默认构造函数不存在,同时编译器也不会生成它
    Date today{ 2020, 1, 19 }; // today 被初始化为 Jan 19th, 2020

    return 0;
}
\end{lstlisting}

在拥有其它构造函数的情况下希望拥有默认构造函数,
可以选择为所有入参添加默认值,或是显式的定义一个默认构造函数。

还有第三种选择:使用 \acode{default} 关键字告诉编译器创建一个默认构造函数:

\begin{lstlisting}[language=C++]
class Date
{
private:
    int m_year{ 1900 };
    int m_month{ 1 };
    int m_day{ 1 };

public:
    // 告诉编译器创建一个默认构造函数,即使提供了其他用户定义的构造函数
    Date() = default;

    Date(int year, int month, int day) // 普通的非默认构造函数
    {
        m_year = year;
        m_month = month;
        m_day = day;
    }
};

int main()
{
    Date date{}; // date 被初始化为 Jan 1st, 1900
    Date today{ 2020, 10, 14 }; // today 被初始化为 Oct 14th, 2020

    return 0;
}
\end{lstlisting}

最佳实践:如果 \acode{class} 拥有构造函数,
同时还需要默认构造函数时(例如因为所有的成员使用非 static 成员初始化),
使用 \acode{= default}。

\subsubsection*{包含类成员的类}

\begin{lstlisting}[language=C++]
#include <iostream>

class A
{
public:
    A() { std::cout << "A\n"; }
};

class B
{
private:
    A m_a; // B 作为成员变量,包含了 A

public:
    B() { std::cout << "B\n"; }
};

int main()
{
    B b;
    return 0;
}
\end{lstlisting}

\subsubsection*{构造函数笔记}

很多新程序员会迷惑构造函数是否会创建对象。
它们不会 -- 编译器设置好对象的内存分配先于构造函数的调用。

构造函数实际上服务于两个目的。

\begin{enumerate}
    \item 构造函数决定谁被允许创建一个类(class)类型的对象。
          也就是说一个类的对象仅可以在匹配构造函数时被创建。
    \item 构造函数可以用于初始化对象。一个构造函数是否初始化是由程序员所决定的。
          在语法上让一个构造函数不进行初始化是合法的(构造函数仍然允许对象被创建)。
\end{enumerate}

然而,与初始化所有本地变量的最佳实践类似,在创建对象时初始化所有成员变量也是最佳实践。
这可以通过一个构造函数或者非 static 成员初始化完成。

最佳实践:总是初始化对象中的所有成员变量。

最后,构造函数仅意在创建对象时用于初始化。
用户不应该调用一个构造函数来重新初始化一个已存在的对象。
虽然它有可能通过编译,但是结果可能并不是用户预期的那样(相反的是,编译器将创建一个临时对象然后再销毁它)。

\end{document}
