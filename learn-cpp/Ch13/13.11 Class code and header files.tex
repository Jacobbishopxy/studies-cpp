\documentclass[../../LearnCpp.tex]{subfiles}

\begin{document}

\asubsection{11}{Class code and header files}

\subsubsection*{在类定义外定义成员函数}

在类变得更长以及更复杂的时候,所有成员函数定义在类的内部会导致类变得难以管理以及使用。
使用一个已经写好的类只需要明白其共有接口(共有成员函数),而不是类在底下是怎么工作的。
成员函数实现的细节正是这些。

幸运的是 C++ 提供一种分离类的“声明”部分与“实现”部分的的方法。
这便是定义类成员函数在类所定义的外部。
为了这么做,只需要简单的使用普通函数那样去定义成员函数,
并使得类名称作为该函数的前缀,并由操作符(\acode{::})隔开。

\begin{lstlisting}[language=C++]
class Date
{
private:
    int m_year;
    int m_month;
    int m_day;

public:
    Date(int year, int month, int day);

    void SetDate(int year, int month, int day);

    int getYear() { return m_year; }
    int getMonth() { return m_month; }
    int getDay()  { return m_day; }
};

// Date 构造函数
Date::Date(int year, int month, int day)
{
    SetDate(year, month, day);
}

// Date 成员函数
void Date::SetDate(int year, int month, int day)
{
    m_month = month;
    m_day = day;
    m_year = year;
}
\end{lstlisting}

这以及很直接了。因为访问函数通常只需要一行,它们通常留在内定义里,即使它们可以被移到外面去。

这里是另一个例子包含了通过成员初始化列表定义的外部定义的构造函数:

\begin{lstlisting}[language=C++]
class Calc
{
private:
    int m_value = 0;

public:
    Calc(int value=0);

    Calc& add(int value);
    Calc& sub(int value);
    Calc& mult(int value);

    int getValue() { return m_value; }
};

Calc::Calc(int value): m_value{value}
{
}

Calc& Calc::add(int value)
{
    m_value += value;
    return *this;
}

Calc& Calc::sub(int value)
{
    m_value -= value;
    return *this;
}

Calc& Calc::mult(int value)
{
    m_value *= value;
    return *this;
}
\end{lstlisting}

\subsubsection*{放置类定义于头文件中}

按照传统,类定义放在与类同名的头文件中,而定义在类外部的成员函数放在与类同名的 .cpp 文件中。

Date.h:

\begin{lstlisting}[language=C++]
#ifndef DATE_H
#define DATE_H

class Date
{
private:
    int m_year;
    int m_month;
    int m_day;

public:
    Date(int year, int month, int day);

    void SetDate(int year, int month, int day);

    int getYear() { return m_year; }
    int getMonth() { return m_month; }
    int getDay()  { return m_day; }
};

#endif
\end{lstlisting}

Date.cpp:

\begin{lstlisting}[language=C++]
#include "Date.h"

// Date 构造函数
Date::Date(int year, int month, int day)
{
    SetDate(year, month, day);
}

// Date 成员函数
void Date::SetDate(int year, int month, int day)
{
    m_month = month;
    m_day = day;
    m_year = year;
}
\end{lstlisting}

现在任何其他头文件或者代码文件希望使用 Date 类仅需简单的 \acode{\#include "Date.h"}。
注意 Date.cpp 同样也需要被编译到任何使用了 Date.h 的项目中,
这样 linker 才能知道 Date 是如何实现的。

\textbf{定义类在一个头文件中会违反 one-definition 规则吗?}

这并不会。如果头文件拥有正确的头文件保护符,它不可能在同一文件中被引入多次。

类型(包括类)是免于 one-definition 规则的。
因此 \#including 类进若干代码文件中不会有问题(如果有问题,类就没有那么有用了)。

\textbf{定义成员函数在头文件中会违反 one-definition 规则吗?}

这得看情况。成员函数定义在类定义处被视为隐式内联。
内联函数免于 one-definition 规则。
这意味着定义小型的成员函数(例如访问函数)在类定义处是没有问题的。

成员函数定义在类的外部被视为普通函数,也服从了 one-definition 的规则。
因此,这些函数应该被定义于代码文件,而不是头文件。
有一个例外是模板函数,它也被视为隐式内联。

\textbf{那么什么是应该定义在头文件 vs cpp 文件,以及什么是在类定义内 vs 外部?}

将所有成员函数定义在头文件的类中,虽然可以编译,还是有很多弊端。
首先如上提到的,类定义被塞满了很凌乱。
其次,如果修改了任何头文件中的代码,则需要重新编译所有引用该头文件的文件。
这会带来涟漪效果,即一个微小的改动会导致整个程序需要被重新编译(很慢)。
如果只是修改了 .cpp 文件的代码,只需要重新编译该文件即可!

因此,推荐以下做法:

\begin{itemize}
    \item 对于仅使用在一个文件中并没有复用的类,定义在需要使用的单个 .cpp 文件中。
    \item 对于需要在若干文件使用的类或者是普通的复用,定义在与类名称相同的 .h 文件中。
    \item 小型的成员函数(小型的构造函数或析构函数,访问函数,等等)可以在类中定义。
    \item 非小型成员函数应该定义在与类同名的 .cpp 文件中。
\end{itemize}

成员函数的默认参数应该被定义在类定义处(在头文件中),它们可以被任何 \#includes 该头文件的看到。

在编写库时,分隔类定义与类实现是非常常见的。

除开开源软件(.h 与 .cpp 文件同时被提供),大多数三方库仅提供头文件,以及预编译的库文件。
这么做有以下几个原因:1)link 预编译库要比每次需要重新编译快很多;
2)单个预编译的库可以被很多应用共享,而编译的代码被编译成每一个使用它的可执行文件(文件大小膨胀);
3)知识版权(不希望代码被偷窃)。

将文件分割成声明(头文件)以及实现(代码文件)不仅仅是好的形式,在创建用户自定义库的时候也变得更简单了。

\end{document}
