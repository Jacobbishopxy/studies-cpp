\documentclass[../../LearnCpp.tex]{subfiles}

\begin{document}

\asubsection{4}{Variable assignment and initialization}

\subsubsection*{初始化}

\begin{itemize}
  \item Default initialization:当没有初始值时,称为\textbf{默认初始化}。大多数情况下,默认初始化会给变量留下一个待定值。
  \item Copy initialization:当提供了一个初始值在等号后,称为\textbf{拷贝初始化}。继承于 C 语言。在现代 C++ 中不再常用。
  \item Direct initialization:当初始值在圆括号中提供,称为\textbf{直接初始化}。同拷贝初始化一样,现代 C++ 中不再常用。
  \item Brace initialization:现代 C++ 初始化变量的方式是使用花括号,称为\textbf{大括号初始化},
        也可被称为\textbf{统一初始化} uniform initialization 或者\textbf{列表初始化} list initialization。
\end{itemize}


\begin{lstlisting}[language=C++]
int width { 5 };    // 推荐,直接大括号初始化
int height = { 6 }; // 拷贝大括号初始化
int depth {};       // 值初始化
\end{lstlisting}


\subsubsection*{值初始化与零初始化}

当一个变量使用空的大括号初始化时,\textbf{值初始化}便发生了。
大多数情况下,\textbf{值初始化}将初始化变量为零(或者空,如果给定的类型更适合时)。
这种情况下被称为\textbf{零初始化}。

何时使用 { 0 } vs {}?

使用显式初始化值,如果用户真正的使用该值。

\begin{lstlisting}[language=C++]
int x { 0 };    // 显式初始化值为 0
std::cout << x; // 使用 0 值
\end{lstlisting}

使用值初始化如果值是临时的且会被覆盖的。

\begin{lstlisting}[language=C++]
int x {};       // 值初始化
std::cin >> x;  // 用户立即替换该值
\end{lstlisting}

\subsubsection*{初始化多个变量}

\begin{lstlisting}[language=C++]
int a = 5, b = 6;       // 拷贝初始化
int c( 7 ), d( 8 );     // 直接初始化
int e { 9 }, f { 10 };  // 推荐,大括号初始化
\end{lstlisting}

\end{document}
