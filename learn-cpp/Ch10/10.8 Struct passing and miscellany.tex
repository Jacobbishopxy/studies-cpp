\documentclass[../../LearnCpp.tex]{subfiles}

\begin{document}

\asubsection{8}{Struct passing and miscellany}

\subsubsection*{structs 传递(通过引用)}

使用 structs 而不是单个变量的最大好处就是可以传递整个 struct 给需要其成员的函数。
structs 通常通过(const)引用传递来避免拷贝。

\begin{lstlisting}[language=C++]
#include <iostream>

struct Employee
{
    int id {};
    int age {};
    double wage {};
};

void printEmployee(const Employee& employee) // 注意这里传递引用
{
    std::cout << "ID:   " << employee.id << '\n';
    std::cout << "Age:  " << employee.age << '\n';
    std::cout << "Wage: " << employee.wage << '\n';
}

int main()
{
    Employee joe { 14, 32, 24.15 };
    Employee frank { 15, 28, 18.27 };

    // 打印 Joe 的信息
    printEmployee(joe);

    std::cout << '\n';

    // 打印 Frank 的信息
    printEmployee(frank);

    return 0;
}
\end{lstlisting}

\subsubsection*{返回 structs}

设想如果需要一个函数返回三维的笛卡尔坐标:

\begin{lstlisting}[language=C++]
#include <iostream>

struct Point3d
{
    double x { 0.0 };
    double y { 0.0 };
    double z { 0.0 };
};

Point3d getZeroPoint()
{
    // 创建一个变量并返回
    Point3d temp { 0.0, 0.0, 0.0 };
    return temp;
}

int main()
{
    Point3d zero{ getZeroPoint() };

    if (zero.x == 0.0 && zero.y == 0.0 && zero.z == 0.0)
        std::cout << "The point is zero\n";
    else
        std::cout << "The point is not zero\n";

    return 0;
}
\end{lstlisting}

\subsubsection*{返回未命名 structs}

上述的 `getZeroPoint()` 创建了一个命名对象(`temp`)并返回,这里可以改进一下返回临时(未命名)对象:

\begin{lstlisting}[language=C++]
Point3d getZeroPoint()
{
    return Point3d { 0.0, 0.0, 0.0 }; // 返回未命名的 Point3d
}
\end{lstlisting}

这种情况下,临时的 Point3d 被创建,拷贝后返回给调用者,然后在表达式结束时销毁。那么在函数有显式的返回值(`Point3d`)而不是类型推导(`auto` 作为返回值)时,甚至可以省略返回时的声明:

\begin{lstlisting}[language=C++]
Point3d getZeroPoint()
{
    // 已经指定了函数的返回值类型
    // 因此这里便不在需要再次声明
    return { 0.0, 0.0, 0.0 }; // 返回未命名的 Point3d
}
\end{lstlisting}

注意因为该案例中返回的都是 0 值,可以使用空花括号来返回一个值初始化的 Point3d:

\begin{lstlisting}[language=C++]
Point3d getZeroPoint()
{
    return {};
}
\end{lstlisting}

\subsubsection*{带有程序定义成员的 structs}

C++ 中,structs(以及 classes)可以拥有程序定义类型的成员。这里有两种方法。

首先定义一个程序定义类型(全局作用域中),接着使用其作为另一个程序定义类型的成员:

\begin{lstlisting}[language=C++]
#include <iostream>

struct Employee
{
    int id {};
    int age {};
    double wage {};
};

struct Company
{
    int numberOfEmployees {};
    Employee CEO {}; // Employee 为 Company struct 内部的 struct
};

int main()
{
    Company myCompany{ 7, { 1, 32, 55000.0 } }; // 嵌套的初始化列表用于初始化 Employee
    std::cout << myCompany.CEO.wage; // 打印 CEO 薪水
}
\end{lstlisting}

其次,类型同样也可以嵌套在另一个类型内部:

\begin{lstlisting}[language=C++]
#include <iostream>

struct Company
{
    struct Employee // 通过 Company::Employee 进行访问
    {
        int id{};
        int age{};
        double wage{};
    };

    int numberOfEmployees{};
    Employee CEO{}; // Employee 为 Company struct 内部的 struct
};

int main()
{
    Company myCompany{ 7, { 1, 32, 55000.0 } }; // 嵌套的初始化列表用于初始化 Employee
    std::cout << myCompany.CEO.wage; // 打印 CEO 薪水
}
\end{lstlisting}

这种做法更常见于 classes,详见 13.17。

\subsubsection*{Struct 大小与数据结构偏移}

通常而言,struct 的大小是所有成员大小之和,不过也不总是如此!

\begin{lstlisting}[language=C++]
#include <iostream>

struct Foo
{
    short a {};
    int b {};
    double c {};
};

int main()
{
    std::cout << "The size of Foo is " << sizeof(Foo) << '\n';

    return 0;
}
\end{lstlisting}

在很多操作系统上,short 为 2 个字节,int 为 4 个字节,已经 double 为 8 哥字节,因此预期的 `sizeof(Foo)` 应该为 2 + 4 + 8 = 14 字节。然而在作者机器上打印的是:The size of Foo is 16。

一个 struct 的大小只能说其最小是所有变量大小之和,但是往往会更大!因为性能的原因,编译器偶尔会在 struct 之间增加间隔(被称为**padding**)。

上述的 `Foo` 中,编译器无形的添加 2 个字节在 `a` 后,使得结构体变成了 16 字节而非 14。

这实际上会对 struct 大小产生很大的影响:

\begin{lstlisting}[language=C++]
#include <iostream>

struct Foo1
{
    short a{};
    short qq{}; // 注意:qq 在此定义
    int b{};
    double c{};
};

struct Foo2
{
    short a{};
    int b{};
    double c{};
    short qq{}; // 注意:qq 在此定义
};

int main()
{
    std::cout << "The size of Foo1 is " << sizeof(Foo1) << '\n';
    std::cout << "The size of Foo2 is " << sizeof(Foo2) << '\n';

    return 0;
}
\end{lstlisting}

注意 `Foo1` 与 `Foo2` 拥有相同成员,唯一的区别是 `qq` 定义的顺序不同。程序打印:

\begin{lstlisting}
The size of Foo1 is 16
The size of Foo2 is 24
\end{lstlisting}

\end{document}
