\documentclass[../../LearnCpp.tex]{subfiles}

\begin{document}

\asubsection{7}{Default member initialization}

当定义 struct(或 class)类型时,可以给每个成员提供默认初始值作为类型定义的一部分。
这个过程被称为\textbf{非 static 成员初始化} non-static member initialization,
同时初始值被称为\textbf{默认成员初始化器} default member initializer。

\begin{lstlisting}[language=C++]
struct Something
{
    int x;       // 无初始化值(坏)
    int y {};    // 默认值初始化
    int z { 2 }; // 显式默认值
};

int main()
{
    Something s1; // s1.x 没有被初始化, s1.y 为 0,以及 s1.z 为 2

    return 0;
}
\end{lstlisting}

\subsubsection*{显式初始化值优先于默认值}

\begin{lstlisting}[language=C++]
struct Something
{
    int x;
    int y {};
    int z { 2 };
};

int main()
{
    Something s2 { 5, 6, 7 }; // 为 s2.x,s2.y,以及 s2.z 显式初始化(默认值没有被使用)

    return 0;
}
\end{lstlisting}

\subsubsection*{当默认值存在时在初始化列表中缺失的初始化器}

\begin{lstlisting}[language=C++]
struct Something
{
    int x;
    int y {};
    int z { 2 };
};

int main()
{
    Something s3 {}; // 值初始化 s3.x,s3.y 与 s3.z 使用默认值

    return 0;
}
\end{lstlisting}

\subsubsection*{总是为成员提供默认值}

\begin{lstlisting}[language=C++]
struct Fraction
{
  int numerator { }; // 应该在这里使用 { 0 } 但是因为是例子,因此这里用值初始化
  int denominator { 1 };
};

int main()
{
  Fraction f1;          // f1.numerator 值初始化 0, f1.denominator 默认为 1
  Fraction f2 {};       // f2.numerator 值初始化 0, f2.denominator 默认为 1
  Fraction f3 { 6 };    // f3.numerator 初始化为 6, f3.denominator 默认为 1
  Fraction f4 { 5, 8 }; // f4.numerator 初始化为 5, f4.denominator 初始化为 8

  return 0;
}
\end{lstlisting}

最佳实践:为所有成员提供默认值。这样可以确保在即使变量的定义没有包含初始化列表的情况下,
所有成员也可以被初始化,。

\subsubsection*{aggregates 的默认初始化 vs 值初始化}

\begin{lstlisting}[language=C++]
Fraction f1;          // f1.numerator 值初始化为 0, f1.denominator 默认为 1
Fraction f2 {};       // f2.numerator 值初始化为 0, f2.denominator 默认为 1
\end{lstlisting}

最佳实践:如果没有显式的初始化值提供给 aggregate,那么推荐使用值初始化(通过空的花括号初始化器)
而不是默认初始化(不带花括号)。

\end{document}
