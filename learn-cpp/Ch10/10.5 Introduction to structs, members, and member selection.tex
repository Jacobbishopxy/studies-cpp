\documentclass[../../LearnCpp.tex]{subfiles}

\begin{document}

\asubsection{5}{Introduction to structs, members, and member selection}

\textbf{结构体} struct是程序定义类型,允许用户打包若干变量在同一个类型内。

\subsubsection*{定义结构体}

\begin{lstlisting}[language=C++]
struct Employee
{
    int id {};
    int age {};
    double wage {};
};
\end{lstlisting}

\acode{struct} 关键字告诉编译器正在定义一个名为 \acode{Employee} 的结构体。

在花括号内,定义该结构体内所包含的变量。这些变量是结构体的一部分,
被称为\textbf{数据成员} data members(或\textbf{成员变量} member variables)。

最后以分号结束结构体类型的定义。

\subsubsection*{定义结构体对象}

\begin{lstlisting}[language=C++]
Employee joe; // Employee 为类型,joe 为变量名
\end{lstlisting}

\subsubsection*{访问成员}

\begin{lstlisting}[language=C++]
struct Employee
{
    int id {};
    int age {};
    double wage {};
};

int main()
{
    Employee joe;

    return 0;
}
\end{lstlisting}

上述例子中,\acode{joe} 意为整个结构体对象(包含了成员变量)。
访问成员变量需要使用\textbf{成员选择操作符} member selection operator(\acode{.})
介于结构体变量名与成员名之间。

\begin{lstlisting}[language=C++]
#include <iostream>

struct Employee
{
    int id {};
    int age {};
    double wage {};
};

int main()
{
    Employee joe;

    joe.age = 32;

    std::cout << joe.age << '\n';

    return 0;
}
\end{lstlisting}

\begin{lstlisting}[language=C++]
#include <iostream>

struct Employee
{
    int id {};
    int age {};
    double wage {};
};

int main()
{
    Employee joe;
    joe.id = 14;
    joe.age = 32;
    joe.wage = 60000.0;

    Employee frank;
    frank.id = 15;
    frank.age = 28;
    frank.wage = 45000.0;

    int totalAge { joe.age + frank.age };

    if (joe.wage > frank.wage)
        std::cout << "Joe makes more than Frank\n";
    else if (joe.wage < frank.wage)
        std::cout << "Joe makes less than Frank\n";
    else
        std::cout << "Joe and Frank make the same amount\n";

    // Frank 升职了
    frank.wage += 5000.0;

    // 今天是 Joe 的生日
    ++joe.age; // 使用 pre-increment 来增加 Joe 的年龄

    return 0;
}
\end{lstlisting}

\end{document}
