\documentclass[../../LearnCpp.tex]{subfiles}

\begin{document}

\asubsection{4}{Virtual destructors, virtual assignment, and overriding virtualization}

\subsubsection*{虚析构函数}

尽管 C++ 为所有类都提供了默认的析构函数,用户有时也需要自定义的析构函数(特别是如果类需要释放内存)。
那么如果在处理继承的时候,析构函数应该\textbf{总是}虚函数。

\begin{lstlisting}[language=C++]
#include <iostream>
class Base
{
public:
    ~Base() // 注意:非虚函数
    {
        std::cout << "Calling ~Base()\n";
    }
};

class Derived: public Base
{
private:
    int* m_array;

public:
    Derived(int length)
      : m_array{ new int[length] }
    {
    }

    ~Derived() // 注意:非虚函数(编译器有可能会警告)
    {
        std::cout << "Calling ~Derived()\n";
        delete[] m_array;
    }
};

int main()
{
    Derived* derived { new Derived(5) };
    Base* base { derived };

    delete base;

    return 0;
}
\end{lstlisting}

注意:如果编译上述代码,编译器可能会警告析构函数为非虚函数(这个例子中是故意的)。

由于 \acode{base} 是 \acode{Base} 的指针,当 \acode{base} 被删除时,
程序会检测 \acode{Base} 的析构函数是否为虚函数。
如果不是,则会假设仅需调用 \acode{Base} 析构函数。可以看到上述例子会打印:

\begin{lstlisting}
Calling ~Base()
\end{lstlisting}

然而这里事非常希望 delete 函数调用 Derived 的析构函数(也会依次调用 Base 的析构函数),
否则 m\_array 不会被删除。
那么这里可以使 Base 的析构函数变为虚函数:

\begin{lstlisting}[language=C++]
#include <iostream>
class Base
{
public:
    virtual ~Base() // 注意:virtual 关键字
    {
        std::cout << "Calling ~Base()\n";
    }
};

class Derived: public Base
{
private:
    int* m_array;

public:
    Derived(int length)
      : m_array{ new int[length] }
    {
    }

    virtual ~Derived() // 注意:virtual 关键字
    {
        std::cout << "Calling ~Derived()\n";
        delete[] m_array;
    }
};

int main()
{
    Derived* derived { new Derived(5) };
    Base* base { derived };

    delete base;

    return 0;
}
\end{lstlisting}

现在打印:

\begin{lstlisting}
Calling ~Derived()
Calling ~Base()
\end{lstlisting}

规则:当处理继承时,应该使任何析构函数都显式的成为虚函数。

正如普通的虚函数成员那样,如果基类函数是虚函数,所有派生的重写都被视为虚函数,无论它们是否有指定。
派生类中创建一个空的析构函数并标记为虚函数并不是必须的。

注意如果希望基类拥有一个虚函数的析构函数并且是为空,可以这样进行定义:

\begin{lstlisting}[language=C++]
virtual ~Base() = default; // 生成一个 默认的虚化析构函数
\end{lstlisting}

\subsubsection*{虚化赋值}

赋值操作符也可以被虚化。然而不同于析构函数使用虚化总是一个好主意,虚化赋值操作符会带来很多 bug。
结果而言,还是推荐赋值为非虚化函数,也是为了简化的目的。

\subsubsection*{忽略虚化}

罕见的情况下希望忽略函数的虚化。例如:

\begin{lstlisting}[language=C++]
class Base
{
public:
    virtual ~Base() = default;
    virtual const char* getName() const { return "Base"; }
};

class Derived: public Base
{
public:
    virtual const char* getName() const { return "Derived"; }
};
\end{lstlisting}

有些情况下希望指向 \acode{Derived} 对象的 \acode{Base} 类型指针调用的是 \acode{Base::getName()} 而不是 \acode{Derived::getName()}。
仅需使用作用域解析操作符就可以做到:

\begin{lstlisting}[language=C++]
#include <iostream>
int main()
{
    Derived derived;
    const Base& base { derived };
    // 调用 Base::getName() 而不是虚化的 Derived::getName()
    std::cout << base.Base::getName() << '\n';

    return 0;
}
\end{lstlisting}

可能并不需要经常这么做,不过这是一个可以了解到的可行方案。

\subsubsection*{应该让所有的析构函数虚化吗?}

这是一个新手程序员经常会问到的问题。
正如最开始的例子中展示,如果基类析构函数并没有被标记为 virtual,
那么如果程序员在之后删除嘞基类类型指向派生类对象的指针时,程序则会有泄漏内存的风险。
其中一种避免问题发生的方式就是标记所有析构函数为 virtual。但是这应该吗?

简单来说是应该的,因为之后这样使用任何类当做基类 --
但是这会带来性能惩罚(虚化指针被添加到类的所有实例中)。
因此用户需要权衡成本。

最终建议:

\begin{itemize}
    \item 如果希望自定义类被继承,请确保析构函数为虚函数。
    \item 如果不希望自定义类被继承,请标记 final。这样就可以阻止被其他类继承,而不遵守类本身的限制。
\end{itemize}

\end{document}
