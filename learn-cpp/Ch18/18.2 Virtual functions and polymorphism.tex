\documentclass[../../LearnCpp.tex]{subfiles}

\begin{document}

\asubsection{2}{Virtual functions and polymorphism}

\textbf{虚函数}是一种特殊类型的函数,当被调用时解析基类与派生类之间最-派生版本的函数。
这个能力也被称为\textbf{多态} polymorphism。
如果派生函数拥有相同签名(名称,入参类型,以及是否为 const)以及相同的返回类型,则被视为匹配。
这样的函数被称为\textbf{重写} overrides。

要是函数成为虚函数,仅需要简单的在函数声明前加上 virtual 关键字即可。

\begin{lstlisting}[language=C++]
#include <iostream>
#include <string_view>

class Base
{
public:
    virtual std::string_view getName() const { return "Base"; } // 注意额外的 virtual 关键字
};

class Derived: public Base
{
public:
    virtual std::string_view getName() const { return "Derived"; }
};

int main()
{
    Derived derived;
    Base& rBase{ derived };
    std::cout << "rBase is a " << rBase.getName() << '\n';

    return 0;
}
\end{lstlisting}

打印:

\begin{lstlisting}
rBase is a Derived
\end{lstlisting}

因为 \acode{rBase} 是 \acode{Derived} 对象中 \acode{Base} 部分的引用,
当 \acode{rBase.getName()} 被计算时,通常会是 \acode{Base::getName()}。
然而由于 \acode{Base::getName()} 是虚函数,
这就告诉程序去查看在 \acode{Base} 与 \acode{Derived} 之前是否有任何更-派生版本的函数。
这个情况下即 \acode{Derived::getName()}!

下面是一个稍微复杂一点的例子:

\begin{lstlisting}[language=C++]
#include <iostream>
#include <string_view>

class A
{
public:
    virtual std::string_view getName() const { return "A"; }
};

class B: public A
{
public:
    virtual std::string_view getName() const { return "B"; }
};

class C: public B
{
public:
    virtual std::string_view getName() const { return "C"; }
};

class D: public C
{
public:
    virtual std::string_view getName() const { return "D"; }
};

int main()
{
    C c;
    A& rBase{ c };
    std::cout << "rBase is a " << rBase.getName() << '\n';

    return 0;
}
\end{lstlisting}

打印:

\begin{lstlisting}
rBase is a C
\end{lstlisting}

\begin{lstlisting}[language=C++]
#include <iostream>
#include <string>
#include <string_view>

class Animal
{
protected:
    std::string m_name;

    // 使该构造函数为 protected,因为不希望任何人可以直接创建 Animal 对象,
    // 但是仍然希望派生类可以使用它。
    Animal(const std::string& name)
        : m_name{ name }
    {
    }

public:
    const std::string& getName() const { return m_name; }
    virtual std::string_view speak() const { return "???"; }
};

class Cat: public Animal
{
public:
    Cat(const std::string& name)
        : Animal{ name }
    {
    }

    virtual std::string_view speak() const { return "Meow"; }
};

class Dog: public Animal
{
public:
    Dog(const std::string& name)
        : Animal{ name }
    {
    }

    virtual std::string_view speak() const { return "Woof"; }
};

void report(const Animal& animal)
{
    std::cout << animal.getName() << " says " << animal.speak() << '\n';
}

int main()
{
    Cat cat{ "Fred" };
    Dog dog{ "Garbo" };

    report(cat);
    report(dog);

    return 0;
}
\end{lstlisting}

打印:

\begin{lstlisting}
Fred says Meow
Garbo says Woof
\end{lstlisting}

同样的:

\begin{lstlisting}[language=C++]
Cat fred{ "Fred" };
Cat misty{ "Misty" };
Cat zeke{ "Zeke" };

Dog garbo{ "Garbo" };
Dog pooky{ "Pooky" };
Dog truffle{ "Truffle" };

// 设置一个 animals 指针的数组,同时放入 Cat 与 Dog 对象的指针
Animal* animals[]{ &fred, &garbo, &misty, &pooky, &truffle, &zeke };

for (const auto* animal : animals)
    std::cout << animal->getName() << " says " << animal->speak() << '\n';
\end{lstlisting}

打印:

\begin{lstlisting}[language=C++]
Fred says Meow
Garbo says Woof
Misty says Meow
Pooky says Woof
Truffle says Woof
Zeke says Meow
\end{lstlisting}

\subsubsection*{虚函数的返回类型}

在普通的情况下,虚函数的返回类型以及其的重写必须匹配。考虑以下例子:

\begin{lstlisting}[language=C++]
class Base
{
public:
    virtual int getValue() const { return 5; }
};

class Derived: public Base
{
public:
    virtual double getValue() const { return 6.78; }
};
\end{lstlisting}

这种情况下,\acode{Derived::getValue()} 不会被视为 \acode{Base::getValue()} 的重写而导致编译错误。

\subsubsection*{构造函数与析构函数不要调用虚函数}

当派生类被创建时,基类部分会首先被创建。
如果调用了基类构造函数中的虚函数,而派生部分并未被创建,这就会无法调用派生版本的函数。
在 C++ 中,则还是会调用基类版本的函数。

析构函数同理,如果在基类析构函数中调用虚函数,还是只会使用基类版本的函数,因为派生部分已经被销毁了。

最佳实践:永远不要在构造函数与析构函数中调用虚函数。

\subsubsection*{虚函数的缺陷}

由于大部分时间都希望函数是虚函数,那么为什么不让所有函数都称为虚函数呢?
答案是因为低性能 -- 虚函数的解析要长于正常函数。
除此以外,编译器还需要为每个拥有一个或多个虚函数的类对象分配额外的指针。

\end{document}
