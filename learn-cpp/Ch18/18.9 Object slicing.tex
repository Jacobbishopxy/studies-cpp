\documentclass[../../LearnCpp.tex]{subfiles}

\begin{document}

\asubsection{9}{Object slicing}

现在来看之前的案例:

\begin{lstlisting}[language=C++]
#include <iostream>

class Base
{
protected:
    int m_value{};

public:
    Base(int value)
        : m_value{ value }
    {
    }

    virtual const char* getName() const { return "Base"; }
    int getValue() const { return m_value; }
};

class Derived: public Base
{
public:
    Derived(int value)
        : Base{ value }
    {
    }

    const char* getName() const override { return "Derived"; }
};

int main()
{
    Derived derived{ 5 };
    std::cout << "derived is a " << derived.getName() << " and has value " << derived.getValue() << '\n';

    Base& ref{ derived };
    std::cout << "ref is a " << ref.getName() << " and has value " << ref.getValue() << '\n';

    Base* ptr{ &derived };
    std::cout << "ptr is a " << ptr->getName() << " and has value " << ptr->getValue() << '\n';

    return 0;
}
\end{lstlisting}

上述例子中,\acode{derived} 的 \acode{ref} 引用与 \acode{ptr} 指针,
拥有一个 \acode{Base} 部分,以及一个 \acode{Derived} 部分。
因为 \acode{ref} 与 \acode{ptr} 的类型都是 \acode{Base},
它们只能看到 \acode{derived} 的 \acode{Base} 部分 --
而 \acode{Derived} 部分仍然存在,就是不能简单的通过 \acode{ref} 与 \acode{ptr} 来看见。
然而通过虚函数的使用,我们可以访问最-派生的函数版本。
因此,上述程序打印:

\begin{lstlisting}
derived is a Derived and has value 5
ref is a Derived and has value 5
ptr is a Derived and has value 5
\end{lstlisting}

那么如果不设置 \acode{Base} 引用或者指针给 \acode{Derived} 对象,
而是简单的*赋值*一个 \acode{Derived} 对象给 \acode{Base} 对象呢?

\begin{lstlisting}[language=C++]
int main()
{
    Derived derived{ 5 };
    Base base{ derived }; // 这里会发生什么?
    std::cout << "base is a " << base.getName() << " and has value " << base.getValue() << '\n';

    return 0;
}
\end{lstlisting}

注意 \acode{derived} 拥有 \acode{Base} 部分以及 \acode{Derived} 部分。
当赋值一个 \acode{Derived} 对象给 \acode{Base} 对象时,只有 \acode{Base} 部分会被拷贝,
而 \acode{Derived} 部分并没有被拷贝。
上述例子中,\acode{base} 获取了 \acode{derived} 的 \acode{Base} 部分的拷贝。
\acode{Derived} 部分实际上是被“切掉了 sliced off”。
因此,将 \acode{Derived} 类对象赋值给 \acode{Base} 类对象被称为\textbf{对象切片} object slicing(或是简称切片)。

因为变量 \acode{base} 没有 \acode{Derived} 部分,
因此 \acode{base.getName()} 被解析为 \acode{Base::getName()}。

上述代码打印:

\begin{lstlisting}
base is a Base and has value 5
\end{lstlisting}

然而不正确的使用切片可以导致若干不同方向的预想不到的结果。下面来看一下这些情况。

\subsubsection*{切片与函数}

可能会认为上述的例子很蠢。毕竟为什么会有人赋值 \acode{derived} 给 \acode{base} 呢?
然而切片的发生更可能出现在函数上。

考虑以下函数:

\begin{lstlisting}[language=C++]
void printName(const Base base) // 注意:base 是值传递而不是引用传递
{
    std::cout << "I am a " << base.getName() << '\n';
}
\end{lstlisting}

这个函数非常的简单,const 的 \acode{Base} 对象作为入参被值传递。如果像以下这样调用函数:

\begin{lstlisting}[language=C++]
int main()
{
    Derived d{ 5 };
    printName(d); // 啊,并没有察觉这是值传递

    return 0;
}
\end{lstlisting}

编写程序的时候很有可能会没有注意 \acode{base} 是一个值入参而不是引用。
因此,当调用 \acode{printName(d)} 时,
虽然预期的是 \acode{base.getName()} 调用虚函数最终打印出“I am a Derived”,
但是实际上并没有。

当然避免该问题的发生也很简单只需要使入参变为引用即可(这也是另一个传递类的引用而不是值的原因)。

\begin{lstlisting}[language=C++]
void printName(const Base& base) // 注意:base 现在是引用传递
{
    std::cout << "I am a " << base.getName() << '\n';
}

int main()
{
    Derived d{ 5 };
    printName(d);

    return 0;
}
\end{lstlisting}

\subsubsection*{向量切片}

另一个新手程序员可能经常会犯的错误是尝试在 \acode{std::vector} 上实现多态。

\begin{lstlisting}[language=C++]
#include <vector>

int main()
{
  std::vector<Base> v{};
  v.push_back(Base{ 5 }); // 添加 Base 对象至向量
  v.push_back(Derived{ 6 }); // 添加 Derived 对象至向量

  // 打印所有向量中的元素
  for (const auto& element : v)
    std::cout << "I am a " << element.getName() << " with value " << element.getValue() << '\n';

  return 0;
}
\end{lstlisting}

打印:

\begin{lstlisting}
I am a Base with value 5
I am a Base with value 6
\end{lstlisting}

与之前例子相似,因为 std::vector 声明的类型是 \acode{Base},当 \acode{Derived(6)} 添加至向量时,其被切片了。

修复这个问题可能有点困难。很多新手程序员会尝试创建一个引用对象的向量,类似于:

\begin{lstlisting}[language=C++]
std::vector<Base&> v{};
\end{lstlisting}

然而这并不能编译。向量中的元素必须是可赋值的,而引用并不能被赋值(仅可初始化)。

一种处理的方法是使用指针:

\begin{lstlisting}[language=C++]
#include <iostream>
#include <vector>

int main()
{
  std::vector<Base*> v{};

  Base b{ 5 }; // b 与 d 不能是匿名对象
  Derived d{ 6 };

  v.push_back(&b); // 添加 Base 对象至向量
  v.push_back(&d); // 添加 Derived 对象至向量

  // 打印所有向量中的元素
  for (const auto* element : v)
    std::cout << "I am a " << element->getName() << " with value " << element->getValue() << '\n';

  return 0;
}
\end{lstlisting}

打印:

\begin{lstlisting}
I am a Base with value 5
I am a Derived with value 6
\end{lstlisting}

这就能工作了!其中有一些注意项。
首先,\acode{nullptr} 现在是一个合法的选项,这有可能并不是期望的。
其次,需要处理指针语义,这会变得笨拙。
不过这样的好处是,允许了动态内存分配的可能性,这对于有些对象可能会离开作用域来说很有用。

另一个选项是使用 \acode{std::reference_wrapper},即一个模仿一个可重新赋值的引用的类:

\begin{lstlisting}[language=C++]
#include <functional> // std::reference_wrapper
#include <iostream>
#include <vector>

class Base
{
protected:
  int m_value{};

public:
  Base(int value)
    : m_value{ value }
  {
  }

  virtual const char* getName() const { return "Base"; }
  int getValue() const { return m_value; }
};

class Derived : public Base
{
public:
  Derived(int value)
    : Base{ value }
  {
  }

  const char* getName() const override { return "Derived"; }
};

int main()
{
  std::vector<std::reference_wrapper<Base>> v{}; // 可重新赋值的 Base 引用的向量

  Base b{ 5 }; // b 与 d 不能是匿名对象
  Derived d{ 6 };

  v.push_back(b); // 添加 Base 对象至向量
  v.push_back(d); // 添加 Derived 对象至向量

  // 打印向量中的所有元素
  // 使用 .get() 从 std::reference_wrapper 中获取元素
  for (const auto& element : v) // 元素类型为 const std::reference_wrapper<Base>&
    std::cout << "I am a " << element.get().getName() << " with value " << element.get().getValue() << '\n';

return 0;
}
\end{lstlisting}

\subsubsection*{Frankenobject}

上述例子中讲述的都是由于派生类被切片了所带来的错误结果。
现在来看一下另一个仍然在派生类中存在的危险案例!

\begin{lstlisting}[language=C++]
int main()
{
    Derived d1{ 5 };
    Derived d2{ 6 };
    Base& b{ d2 };

    b = d1; // 这一行是有问题的

    return 0;
}
\end{lstlisting}

前三行非常的直接:创建两个 \acode{Derived} 对象,对第二个对象设置 \acode{Base} 引用。

第四行出了问题。因为 \acode{b} 指向 \acode{d2},又赋值了 \acode{d1} 给 \acode{b},有人可能会认为 \acode{d1} 将会给 \acode{d2} 一个拷贝 -- 如果 \acode{b} 是 \acode{Derived} 的情况并不是这样。如果 \acode{b} 是 \acode{Base},C++ 中的操作符 \acode{=} 默认不会提供类的虚化。因此,\acode{d1} 只用 \acode{Base} 部分被拷贝进了 \acode{d2}。

结果就是 \acode{d2} 有了 \acode{d1} 的 \acode{Base} 部分以及 \acode{d2} 的 \acode{Derived} 部分。在这个特定的例子中,这并没有问题(因为 \acode{Derived} 类自身没有数据),但是大多数情况下,用户创建了一个 Frankenobject -- 由若干对象的不同部分组成的对象。更坏的是,这没有简单的办法来阻止这样的事情发生。

\subsubsection*{总结}

尽管 C++ 支持通过对象切片的方式将派生对象赋值给基类对象,
通常而言,这并不能带来什么好处而都是麻烦,因此应该尽量避免切片。
请确保所有函数入参是引用(或者指针),同时在处理派生类时尽量避免任何值传递类型。

\end{document}
