\documentclass[../../LearnCpp.tex]{subfiles}

\begin{document}

\asubsection{14}{Compile-time constants, constant expressions, and constexpr}

\subsubsection*{常数表达}

\textbf{常数表达} constant expression 是一种表达式,可以被编译器在编译期计算。
常数表达中,所有的值需要在编译期可知(同时所有的操作符以及函数都必须支持编译器计算)。

当编译器遇到一个常数表达时,则在编译期计算表达式,接着用其计算结果替换掉表达式。

\subsubsection*{编译期常数}

\textbf{编译期常数}是在编译期计算出的常数。
例如'1','2.3' 以及 "Hello, world!" 都可以是编译期常数。

那么常数变量呢?常数变量可以是编译期常数,也可以不是。

\subsubsection*{编译期常数变量}

一个常数变量的初始化是常数表达式,那么它是编译期变量。

\begin{lstlisting}[language=C++]
#include <iostream>

int main()
{
  const int x { 3 };  // x 是一个编译期常数
  const int y { 4 };  // y 是一个编译期常数

  const int z { x + y }; // x + y 是一个编译期表达式

  std::cout << z << '\n';

  return 0;
}
\end{lstlisting}

因为 x 与 y 它们的初始值是常数表达式,那么它们则是编译期常数,这意味着 \acode{x + y} 同样是常数表达式。
因此当编译器工作时,会计算 \acode{x + y} 的值,并将常数表达式替换为计算的结果 \acode{7}。

\subsubsection*{运行时常数变量}

任何通过非常数表达式的 const 变量皆为运行时常数。
\textbf{运行时常数} runtime constants 是在运行时才能计算得出的常数。

\begin{lstlisting}[language=C++]
#include <iostream>

int getNumber()
{
    std::cout << "Enter a number: ";
    int y{};
    std::cin >> y;

    return y;
}

int main()
{
    const int x{ 3 };           // x 是一个编译期常数

    const int y{ getNumber() }; // y 是一个运行时常数

    const int z{ x + y };       // x + y 是一个运行时表达式
    std::cout << z << '\n';     // 这同样也是一个运行时表达式

    return 0;
}
\end{lstlisting}

尽管 y 是一个 const,其初始值(即 `getNumber()` 的返回值)直到运行时才知道。
因此,y 是运行时常数,而不是编译期常数。因此 \acode{x + y} 也是一个运行时表达式。

\subsubsection*{constexpr 关键字}

当用户声明一个常数变量,编译期将会隐式的追踪其是否为一个运行时或者是编译期常数。
多数情况下,这不会作为优化的目标,但是有一些奇怪的案例中,C++ 需要编译期常数而非运行时常数(稍后将会提到)。

因为编译期常数通常允许更好的优化(同时很少的坏处),在可能的情况下,用户通常会选择使用编译期常数。

使用 const 时,变量会成为那种常数是根据初始化是否为编译期表达式来决定的。
因为它们的定义看起来都一样,用户很容易误认一个运行时常数为编译期常数。
上面的例子中,很难知道 y 是否为编译期常数 -- 除非用户去检查 \acode{getNumber()} 的返回类型。

幸运的是,用户可以从编译期中获得帮助,来确保编译时常数符合预期。
为了达到这个目的,用户需要使用 \acode{constexpr} 关键字而不是 \acode{const} 变量声明。
\textbf{constexpr}(常数表达式 constant expression 的简称)变量只允许是编译期常数。
如果其初始值不是一个常数表达式,编译器则会报错。

\begin{lstlisting}[language=C++]
#include <iostream>

int five()
{
    return 5;
}

int main()
{
    constexpr double gravity { 9.8 }; // ok: 9.8 是一个常数表达式
    constexpr int sum { 4 + 5 };      // ok: 4 + 5 是一个常数表达式
    constexpr int something { sum };  // ok: sum 是一个常数表达式

    std::cout << "Enter your age: ";
    int age{};
    std::cin >> age;

    constexpr int myAge { age };      // 编译错误:age 不是一个常数表达式
    constexpr int f { five() };       // 编译错误:five() 的返回值不是一个常数表达式

    return 0;
}
\end{lstlisting}

最佳实践:

\begin{itemize}
  \item 任何变量在初始化后,不会变化的并且其初始化器在编译期可知,则为 \acode{constexpr}。
  \item 任何变量在初始化后,不会变化的并且其初始化器在编译期不可知,则为 \acode{const}。
\end{itemize}

\subsubsection*{常数合并常数的子表达}

观察以下案例:

\begin{lstlisting}[language=C++]
#include <iostream>

int main()
{
  constexpr int x { 3 + 4 }; // 3 + 4 是一个常数表达式
  std::cout << x << '\n';    // 这是运行时表达式

  return 0;
}
\end{lstlisting}

\acode{3 + 4} 是常数表达式,因此编译器会在编译期计算并将其结果 \acode{7} 替换掉表达式。
编译器之后将会优化 \acode{x},替换 \acode{std::cout << x << '\n'} 为 \acode{std::cout << 7 << '\n'}。
其返回表达式则在运行时执行。

然而,因为 \acode{x} 只会用到一次,用户很可能在编写代码的时候这样写:

\begin{lstlisting}[language=C++]
#include <iostream>

int main()
{
  std::cout << 3 + 4 << '\n'; // 这是运行时表达式

  return 0;
}
\end{lstlisting}

由于 \acode{std::cout << 3 + 4 << '\n'} 不是一个常数表达式。
那么有理由怀疑常数的子表达式 \acode{3 + 4} 是否还是可以在编译期优化。
这里的回答通常是“是”。
编译器一直能够优化常数表达式,也包含了那些可以在编译期就决定好了的变量(编译期常数以及 constexpr 变量)。

让变量成为 constexpr 可以确保这些变量在编译时就能知道,
因此有资格获得合并常数用于它们的表达式(即使在非常数表达式中)。

\end{document}
