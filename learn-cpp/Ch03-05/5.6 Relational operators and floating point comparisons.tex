
\documentclass[../../LearnCpp.tex]{subfiles}

\begin{document}

\asubsection{6}{Relational operators and floating point comparisons}

\textbf{关系操作符} relational operators 是用于比较两个值的操作符。一共有六种:

\begin{center}
  \begin{tiny}
    \begin{tabularx}{ 1\textwidth}{
        | >{\raggedright\arraybackslash}X
        | >{\raggedright\arraybackslash}X
        | >{\raggedright\arraybackslash}X
        | >{\raggedright\arraybackslash}X |
      }
      \hline
      操作符                    & 符号 & 形式     & 操作                                                       \\
      \hline
      Greater than           & >  & x > y  & true if x is greater than y, false otherwise             \\
      Less than              & <  & x < y  & true if x is less than y, false otherwise                \\
      Greater than or equals & >= & x >= y & true if x is greater than or equal to y, false otherwise \\
      Less than or equals    & <= & x <= y & true if x is less than or equal to y, false otherwise    \\
      Equality               & == & x == y & true if x equals y, false otherwise                      \\
      Inequality             & != & x != y & true if x does not equal y, false otherwise              \\
      \hline
    \end{tabularx}
  \end{tiny}
\end{center}

\subsubsection*{比较计算后的浮点值会出问题}

考虑以下程序:

\begin{lstlisting}[language=C++]
#include <iostream>

int main()
{
    double d1{ 100.0 - 99.99 }; // should equal 0.01 mathematically
    double d2{ 10.0 - 9.99 }; // should equal 0.01 mathematically

    if (d1 == d2)
        std::cout << "d1 == d2" << '\n';
    else if (d1 > d2)
        std::cout << "d1 > d2" << '\n';
    else if (d1 < d2)
        std::cout << "d1 < d2" << '\n';

    return 0;
}
\end{lstlisting}

变量 d1 与 d2 应该都为 0.01,然而程序打印了非预期的结果:

\begin{lstlisting}[language=C++]
d1 > d2
\end{lstlisting}

如果在调试器中检查 d1 与 d2 的值,那么可能会得到 d1 = 0.0100000000000005116 以及 d2 = 0.0099999999999997868。
两个数字都接近 0.01,但是 d1 要大一些,而 d2 要小一些。

如果需要高精度,那么使用任何关系操作符来比较浮点数是很危险的。这是因为浮点数并不精确,且对其进行小的取整可能会导致非预期的结果。

\subsubsection*{浮点数等式}

使用等式操作符(!= 以及 ==)的问题更大。考虑操作符== 是在运算成员完全相等的情况下返回 true。
由于最小的取整错误都会导致两个浮点数并不相等,操作符== 具有返回 false 的风险,即便预期是 true 的情况。

因为这个原因,浮点运算成员使用这类操作符应该通常都被避免。

\end{document}
