\documentclass[../../LearnCpp.tex]{subfiles}

\begin{document}

\asubsection{4}{Lvalue references to const}

上一章节中讨论了左值引用只可以绑定到可变的左值上。这就意味着下面代码不合法:

\begin{lstlisting}[language=C++]
int main()
{
    const int x { 5 }; // x 是一个不可变的 (const) 左值
    int& ref { x }; // 错误:ref 不可以绑定不可变左值

    return 0;
}
\end{lstlisting}

\subsubsection*{const 的左值引用}

定义左值引用时使用 \acode{const} 关键字,即告诉左值引用其引用的对象为 const。
这种引用被称为 \textbf{const} 值的左值引用(有时也称为 \textbf{reference to const} 或者 \textbf{const reference})。

\begin{lstlisting}[language=C++]
int main()
{
    const int x { 5 };    // x 为不可变左值
    const int& ref { x }; // 正确:ref 是 const 值的左值引用

    return 0;
}
\end{lstlisting}

因为 const 左值引用视为 const 的引用,它们可以被访问但是不可以被修改:

\begin{lstlisting}[language=C++]
#include <iostream>

int main()
{
    const int x { 5 };    // x 为不可变左值
    const int& ref { x }; // 正确:ref 是 const 值的左值引用

    std::cout << ref << '\n'; // 正确:可以访问 const 对象
    ref = 6;                  // 错误:不可以修改 const 对象

    return 0;
}
\end{lstlisting}

\subsubsection*{通过可变左值初始化 const 左值引用}

const 左值引用可以绑定可变左值。这种情况下,当通过引用访问时,被引用的对象视为 const(尽管其根本的对象为非 const):

\begin{lstlisting}[language=C++]
#include <iostream>

int main()
{
    int x { 5 };              // x 为可变左值
    const int& ref { x };     // 正确:可以绑定可变左值给 const 引用

    std::cout << ref << '\n'; // 正确:可以通过 const 引用访问对象
    ref = 7;                  // 错误:不可以通过 const 引用修改对象

    x = 6;                    // 正确:x 为可变左值,仍然可以通过原有的标识符进行修改

    return 0;
}
\end{lstlisting}

最佳实践:更加推荐 \acode{lvalue references to const} 而不是 \acode{lvalue references to non-const} 除非需要通过引用来修改对象。

\subsubsection*{通过右值初始化 const 左值引用}

const 左值引用也可以绑定右值:

\begin{lstlisting}[language=C++]
#include <iostream>

int main()
{
    const int& ref { 5 }; // 可行: 5 为右值

    std::cout << ref << '\n'; // 打印 5

    return 0;
}
\end{lstlisting}

上述代码中,一个临时的对象被创建后用右值进行了初始化,const 引用绑定临时对象。

\textbf{临时对象}(有时也被称为\textbf{匿名对象})是一个为临时使用(接着销毁)而在单个表达式中创建的对象。临时对象完全没有作用域(合理,因为域是标识符的属性,而临时对象没有标识符)。这就意味着临时对象仅可以在其创建的地方使用,因为没有其他别的办法指向它。

\subsubsection*{const 引用绑定临时对象延长了临时对象的生命周期}

临时对象通常在其被创建的表达式结束时被销毁。

然而,考虑一下上述例子中如果临时对象创建时其右值 \acode{5} 在初始化 \acode{ref} 的表达式结束时被销毁。
\acode{ref} 引用将会变为左悬垂(引用的对象被销毁),这时尝试访问 \acode{ref} 则会得到未定义行为。

为了避免这样悬垂引用的情况,C++ 有一个特别的规则:当 const 左值引用绑定了临时对象,
临时对象的声明周期被延长到与引用的生命周期相匹配。

\begin{lstlisting}[language=C++]
#include <iostream>

int main()
{
    const int& ref { 5 }; // 临时对象的生命周期被延长与 ref 一致

    std::cout << ref << '\n'; // 因此可以在此处安全使用

    return 0;
} // ref 和临时对象在此同时被销毁
\end{lstlisting}

重点:左值引用仅可以绑定可变左值。const 左值引用可以绑定可变左值,不可变左值,以及右值。
这使得它们成为更加灵活的引用类型。

那么为什么 C++ 运行 const 引用绑定右值呢?请看下一章节。

\end{document}
