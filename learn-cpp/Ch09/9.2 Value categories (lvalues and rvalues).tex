\documentclass[../../LearnCpp.tex]{subfiles}

\begin{document}

\asubsection{2}{Value categories (lvalues and rvalues)}

\subsubsection*{表达式的属性}

为了帮助判断表达式如何被计算以及可以在哪里被使用,C++ 中所有的表达式都有两个属性:类型和值类别。

\subsubsection*{表达式的类型}

\begin{lstlisting}[language=C++]
#include <iostream>

int main()
{
    auto v1 { 12 / 4 }; // int / int => int
    auto v2 { 12.0 / 4 }; // double / int => double

    return 0;
}
\end{lstlisting}

注意表达式的类型必须可以在编译期被决定(否则类型检查与类型推导不能工作) --
然而,表达式的值既可以在编译期(如果表达式为 constexpr)或是运行时(非 constexpr)。

\subsubsection*{表达式的值类别}

表达式(或子表达式)的\textbf{值类别}表明一个表达式是否解析为一个值,一个函数,或者其他类型的对象。

C++11 之前,只有两种值类别: \acode{lvalue} 和 \acode{rvalue}。

C++11 中,添加了三种新的值类别(\acode{glvalue},\acode{prvalue} 和 \acode{xvalue})用于支持名为 \acode{move sematics} 的新特性。

\subsubsection*{左值与右值表达式}

\textbf{lvalue}(发音 ell-value,为 left value 或 locator value 的缩写,
有时写为 l-value),是计算为可识别的对象或者函数(或 bit-field)的表达式。

C++ 标准中的 “identity” 并没有被很好的定义。实体 entity(例如对象或函数)拥有 identity
可以从其他类似的实体中区分出来(通常的做法是比较实体的地址)。

带有 identities 的实体可以通过标识符,引用,或者指针来进行访问,同时相较于单个表达式或声明而言,
它们通常都拥有较长的生命周期。

\begin{lstlisting}[language=C++]
#include <iostream>

int main()
{
    int x { 5 };
    int y { x }; // x 为左值表达式

    return 0;
}
\end{lstlisting}

由于常数被引用进语言中,左值则有两种子类型:\textbf{可变 lvalue}以及\textbf{不可变 lvalue}。

\begin{lstlisting}[language=C++]
#include <iostream>

int main()
{
    int x{};
    const double d{};

    int y { x }; // x 为可变左值表达式
    const double e { d }; // d 为不可变左值表达式

    return 0;
}
\end{lstlisting}

\textbf{rvalue}(发音 arr-value,为 right value 的缩写,有时写为 r-value)是非 l-value 的表达式。
常见的 rvalues 包括字面值(除了 C-style 的字符串字面值,它是 lvalues)以及函数和操作符的返回值。
rvalues 不能被区分(意味着它们必须立刻被使用掉),同时仅存在于其表达式所在的作用域。

\begin{lstlisting}[language=C++]
#include <iostream>

int return5()
{
    return 5;
}

int main()
{
    int x{ 5 }; // 5 为右值表达式
    const double d{ 1.2 }; // 1.2 为右值表达式

    int y { x }; // x 为可变左值表达式
    const double e { d }; // d 为不可变左值表达式
    int z { return5() }; // return5() 为右值表达式(因为其返回值是由值返回的)

    int w { x + 1 }; // x + 1 为右值表达式
    int q { static_cast<int>(d) }; // 转换 d 为 int 是右值表达式

    return 0;
}
\end{lstlisting}

关联内容:所有的左值和右值表达式可以在\href{https://en.cppreference.com/w/cpp/language/value\_category}{这里}找到。

\subsubsection*{左值转换至右值}

\begin{lstlisting}[language=C++]
int main()
{
    int x{ 1 };
    int y{ 2 };

    x = y; // y 为可变左值而不是右值,这是合法的

    return 0;
}
\end{lstlisting}

这里合法是因为左值被隐式转换成右值,因此左值可以被使用在右值所需要在的地方。

考虑以下代码:

\begin{lstlisting}[language=C++]
int main()
{
    int x { 2 };

    x = x + 1;

    return 0;
}
\end{lstlisting}

这里变量 \acode{x} 被用作于两个不同的上下文中。
在分配符左侧,\acode{x} 是左值表达式用于计算变量 x。
在分配符右侧,\acode{x + 1} 为右值表达式用于计算出值 \acode{3}。

现在覆盖了左值的概念,接下来学习第一个组合类型:\acode{lvalue reference}。

重要概念:判断左值与右值的经验法则:左值表达式是计算变量或其他可辨识对象,它们的存在超出表达式范围;
右值表达式计算字面值或函数与操作符的返回值,它们在表达式结束后舍弃。

\end{document}
