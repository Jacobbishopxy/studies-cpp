\documentclass[../../LearnCpp.tex]{subfiles}

\begin{document}

\asubsection{6}{Introduction to pointers}

\subsubsection*{address-of 操作符(\&)}

尽管变量使用的内存地址默认不会暴露给用户,还是可以访问其信息。
\textbf{address-of} 操作符(\&)返回其内存地址:

\begin{lstlisting}[language=C++]
#include <iostream>

int main()
{
    int x{ 5 };
    std::cout << x << '\n';  // 打印 x 变量的值
    std::cout << &x << '\n'; // 打印 x 变量的内存地址

    return 0;
}
\end{lstlisting}

对象使用若干字节地址的情况下,address-of 将会返回对象使用的第一个字节的地址。

\subsubsection*{解引用操作符(*)}

单独获取变量的地址并不是很有用。

最有用的做法是通过地址访问其所存储的值。
\textbf{解引用操作符} dereference operator(*)(同时偶尔会被称为 \textbf{indirection operator})
返回给定内存地址的值作为左值:

\begin{lstlisting}[language=C++]
#include <iostream>

int main()
{
    int x{ 5 };
    std::cout << x << '\n';  // 打印变量 x 的值
    std::cout << &x << '\n'; // 打印变量 x 的内存地址

    std::cout << *(&x) << '\n'; // 打印变量 x 在内存地址的值(括号是不需要的,仅方便阅读)

    return 0;
}
\end{lstlisting}

\subsubsection*{指针}

\textbf{指针}是一个对象用于存储\textit{内存地址}(通常来说是其他的变量)作为其值。
这允许用户存储其它的对象在之后使用。

题外话:现代 C++ 中,指针有时会被称为“裸指针 raw pointers”或者“dumb pointers”,
这是为了区别于“智能指针 smart pointers”(后续章节详解)。

与引用类型的声明相似,指针类型的声明使用星号(*):

\begin{lstlisting}[language=C++]
int;  // 普通 int
int&; // int 值的左值引用

int*; // int 值的指针(保存一个整数值的地址)
\end{lstlisting}

创建指针类型的变量如下:

\begin{lstlisting}[language=C++]
int main()
{
    int x { 5 };    // 普通变量
    int& ref { x }; // 整数(绑定 x)的引用

    int* ptr;       // 整数的指针

    return 0;
}
\end{lstlisting}

注意星号是声明指针的语法一部分,而不是使用解引用操作符。

最佳实践:当声明一个指针类型,把星号放在类型后。

\subsubsection*{指针实例化}

与普通变量一样,指针默认*不会*被初始化的。
一个没有被初始化的指针有时会被称为\textbf{野指针}。
野指针包含一个垃圾地址,对野指针进行解引用将会造成未定义行为。
正因如此,应该总是让指针初始化一个已知值。

最佳实践:总是初始化指针。

\begin{lstlisting}[language=C++]
int main()
{
    int x{ 5 };

    int* ptr;        // 一个未初始化指针(存储了垃圾地址)
    int* ptr2{};     // 一个空指针(下一章节讨论)
    int* ptr3{ &x }; // 一个初始化了变量 x 地址的指针

    return 0;
}
\end{lstlisting}

因为指针保存地址,当初始化或为指针分配值时,值必须为一个地址。
通常来说,指针用来存储另一个变量的地址(可以使用 address-of 操作符(\&)来获取)。

一旦指针存储了其他对象的地址,则可以使用解引用操作符(*)来访问该地址的值了。例如:

\begin{lstlisting}[language=C++]
#include <iostream>

int main()
{
    int x{ 5 };
    std::cout << x << '\n'; // 打印变量 x 的值

    int* ptr{ &x }; // ptr 保存 x 的地址
    std::cout << *ptr << '\n'; // 使用解引用操作符打印 ptr 保存的地址的值

    return 0;
}
\end{lstlisting}

\subsubsection*{指针与赋值}

指针赋值有两种不同的方式:

\begin{enumerate}
  \item 修改指针指向(通过分配一个新地址)
  \item 修改其指向的值(解引用指针后赋值一个新值)
\end{enumerate}

\begin{lstlisting}[language=C++]
#include <iostream>

int main()
{
    int x{ 5 };
    int* ptr{ &x }; // ptr 实例化指向 x

    std::cout << *ptr << '\n'; // 打印指向的地址的值(x 的地址)

    int y{ 6 };
    ptr = &y; // // 修改 ptr 指向 y

    std::cout << *ptr << '\n'; // 打印指向的地址的值(y 的地址)

    return 0;
}
\end{lstlisting}

\begin{lstlisting}[language=C++]
#include <iostream>

int main()
{
    int x{ 5 };
    int* ptr{ &x }; // ptr 实例化指向 x

    std::cout << x << '\n';    // 打印 x 的值
    std::cout << *ptr << '\n'; // 打印指向的地址的值(x 的地址)

    *ptr = 6; // ptr 存储地址的对象被赋值 6(注意 ptr 在这里被解引用了)

    std::cout << x << '\n';
    std::cout << *ptr << '\n'; // 打印指向的地址的值(x 的地址)

    return 0;
}
\end{lstlisting}

\subsubsection*{指针行为很想左值引用}

\begin{lstlisting}[language=C++]
#include <iostream>

int main()
{
    int x{ 5 };
    int& ref { x };  // x 的引用
    int* ptr { &x }; // x 的指针

    std::cout << x;
    std::cout << ref;           // 使用引用打印 x 的值(5)
    std::cout << *ptr << '\n';  // 使用指针打印 x 的值(5)

    ref = 6; // 使用引用修改 x 的值
    std::cout << x;
    std::cout << ref;           // 使用引用打印 x 的值(6)
    std::cout << *ptr << '\n';  // 使用指针打印 x 的值(6)

    *ptr = 7; // 使用指针修改 x 的值
    std::cout << x;
    std::cout << ref;           // 使用引用打印 x 的值(7)
    std::cout << *ptr << '\n';  // 使用指针打印 x 的值(7)

    return 0;
}
\end{lstlisting}

指针与引用还有一些其它的不同值得注意:

\begin{itemize}
  \item 引用必须被初始化,指针则不需要(但是也应该初始化)
  \item 引用不是对象,指针则是
  \item 引用不可被重复位(修改自身引用其它别的),指针可以被修改其指向
  \item 引用必须总是绑定一个对象,指针可以指向空(下一章节讲解案例)
  \item 引用是“安全的”(悬垂引用之外),指针固有危险(下一章节详解)
\end{itemize}

\subsubsection*{address-of 操作符返回一个指针}

值得注意的是 address-of 操作符(\&)不会作为字面值返回其引用的对象。
相反的是,其返回一个指针包含了其被引用者的地址,而该指针的类型有参数派生出来(例如 \acode{int} 地址会返回 \acode{int} 指针的地址)。

\begin{lstlisting}[language=C++]
#include <iostream>
#include <typeinfo>

int main()
{
  int x{ 4 };
  std::cout << typeid(&x).name() << '\n'; // 打印 &x 的类型(int *)

  return 0;
}
\end{lstlisting}

而 gcc 这里打印的是“pi”(pointer to int)。
因为 typeid().name() 的返回结果是编译器所决定的,
用户的编译器有可能打印不同的东西,但是都是相同的意思。

\subsubsection*{指针的大小}

指针的大小是根据编译后的可执行架构所决定的 -- 32-bit 的可执行使用 32-bit 的内存地址,
指针的大小为 32 bits(4 个字节);64-bit 的可执行,指针的大小为 64 bits(8 个字节)。

\subsubsection*{悬垂指针}

与悬垂引用类似,\textbf{悬垂指针} dangling pointer 是指针存储地址的对象不再有效(例如对象被销毁)。
解引用一个悬垂指针会导致未定义的行为。

以下是创建悬垂指针的例子:

\begin{lstlisting}[language=C++]
#include <iostream>

int main()
{
    int x{ 5 };
    int* ptr{ &x };

    std::cout << *ptr << '\n'; // 有效

    {
        int y{ 6 };
        ptr = &y;

        std::cout << *ptr << '\n'; // 有效
    } // y 离开作用域,ptr 现在悬垂了

    std::cout << *ptr << '\n'; // 解引用一个悬垂指针导致未定义行为

    return 0;
}
\end{lstlisting}

\end{document}
