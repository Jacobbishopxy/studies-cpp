\documentclass[../../LearnCpp.tex]{subfiles}

\begin{document}

\asubsection{1}{Input and output (IO) streams}

输入与输出的功能并没有定义在 C++ 语言的核心库中,而是通过 C++ 标准库来提供(并因此存在于 std 命名空间中)。
之前的章节中已经引入了很多次 iostream 库头文件并使用了 cin 与 cout 对象来进行简单的 I/O。
这一节中我们将更为深入的了解其中的细节。

\subsubsection*{iostream 库}

当用户引入了 iostream 头文件时可以访问整个提供了 I/O 功能所有的类(包括一个已经命名为 iostream 的类)。

这些类的等级体系是由若干继承构成的(也是之前讲述过尽可能需要避免的),而 iostream 库是被精心设计并得到了极致的测试,
就是为了避免任何典型的多重继承问题,因此用户可以放心的使用它们。

\subsubsection*{Streams}

抽象的角度,\textbf{流} stream 是一系列的字节可用作于序列式的访问。随着时间流逝,一个流可能会潜在的生成或消费无限数量的数据。

通常而言会处理两种不同类型的刘。\textbf{输入流} input streams 用于存储数据生产者的数据输入,
例如键盘,文件,或者网络。例如用户可能在键盘上按下一个键而程序当前并未预期有任何的输入。相比于忽视用户的输入,数据则是被放入输入流中,
直到程序准备好使用该数据。

相反,\textbf{输出流} output streams 用作于存储特定数据消费者的输出,例如一个监控器,文件,或者是打印机。当将数据写入输出设备时,
设备有可能并未对数据接收就绪 -- 例如在程序写入数据只输出流时,打印机可能仍然在启动。数据将会留存在输出流中,直到打印机开始消费它。

某些设备,例如文件或者网络,能够同时作为输入与输出源。

对于程序员而言,刘的好处是只需要学习如何与流交互,便可以与不同类型设备进行数据读写。
而流与真实设备是如何挂钩的则交给运行环境或者操作系统来操心。

\subsubsection*{C++ 中的输入/输出}

尽管 ios 类通常是从 ios\_base 派生而来,而 ios 是用户直接使用的最基础的类。ios 类定义了一大堆东西是通用于输入与输出流的。
之后的章节将会详细进行了解。

\textbf{istream} 类是处理输入流的主要类。通过输入流,\textbf{提取操作符(>>)} extraction operator>> 用于从流中移除值。
这有道理:当用户在键盘上按下按键,键代码被放置进输入流中。接着程序从流中提取该值,使其可被使用。

\textbf{ostream} 类是处理输出流的主要类。通过输出流,\textbf{插入操作符(<<)} insertion operator<< 用于放置值值流中。
这同样有道理:用户插入值至流中,接着数据消费者(例如显示器)使用它们。

\textbf{iostream} 类可以同时处理输入与输出,允许双向 I/O。

\subsubsection*{C++ 中的标准 streams}

\textbf{标准流} 是一个预连接的流提供给计算机程序其环境。C++ 提供了四种标准流对象。

\begin{enumerate}
  \item cin -- 一个 istream 对象与标准输入关联(通常而言是键盘)
  \item cout -- 一个 ostream 对象与标准输出关联(通常而言是显示器)
  \item cerr -- 一个 ostream 对象与标准错误关联(通常而言是显示器),提供了非缓存的输出
  \item clog -- 一个 ostream 对象与标准错误关联(通常而言是显示器),提供了缓存输出
\end{enumerate}

非缓存输出通常是立刻处理的,而缓存输出通常是存储起来并以块的方式输出。由于 clog 没有被经常使用,在标准流的列表中经常被省略。

\end{document}
