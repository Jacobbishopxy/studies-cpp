\documentclass[../../LearnCpp.tex]{subfiles}

\begin{document}

\asubsection{5}{Stream states and input validation}

\subsubsection*{流状态}

ios\_base 类包含若干状态 flags 用于标记不同情况下可能的状态:

\begin{center}
  \begin{tiny}
    \begin{tabularx}{ 1\textwidth}{
        | >{\raggedright\arraybackslash}X
        | >{\raggedright\arraybackslash}X |
      }
      \hline
      Flag    & Meaning                                                                                              \\
      \hline
      goodbit & Everything is okay                                                                                   \\
      badbit  & Some kind of fatal error occurred (e.g. the program tried to read past the end of a file)            \\
      eofbit  & The stream has reached the end of a file                                                             \\
      failbit & A non-fatal error occurred (e.g. the user entered letters when the program was expecting an integer) \\
      \hline
    \end{tabularx}
  \end{tiny}
\end{center}

尽管这些 flags 存在于 ios\_base 中,由于 ios 是从 ios\_base 派生而来的,且 ios 相比于 ios\_base 只需少量打字,
因此我们通常通过 ios 使用它们(例如 \acode{std::ios::failbit})。

ios 同样提供了一系列成员函数用于方便的使用这些状态:

\begin{center}
  \begin{tiny}
    \begin{tabularx}{ 1\textwidth}{
        | >{\raggedright\arraybackslash}X
        | >{\raggedright\arraybackslash}X |
      }
      \hline
      Member          & function	Meaning                                                       \\
      \hline
      good()          & Returns true if the goodbit is set (the stream is ok)                  \\
      bad()           & Returns true if the badbit is set (a fatal error occurred)             \\
      eof()           & Returns true if the eofbit is set (the stream is at the end of a file) \\
      fail()          & Returns true if the failbit is set (a non-fatal error occurred)        \\
      clear()         & Clears all flags and restores the stream to the goodbit state          \\
      clear(state)    & Clears all flags and sets the state flag passed in                     \\
      rdstate()       & Returns the currently set flags                                        \\
      setstate(state) & Sets the state flag passed in                                          \\
      \hline
    \end{tabularx}
  \end{tiny}
\end{center}

处理 bit 最常用的是 failbit,当用户的输入无效时它会被设置。例如,考虑以下程序:

\begin{lstlisting}[language=C++]
std::cout << "Enter your age: ";
int age {};
std::cin >> age;
\end{lstlisting}

注意该程序预期用户输入一个整数,然而如果用户输入非数值数据,例如 “Alex”,cin 将无法提取任何东西到 age,此时 failbit 将被设置。

如果错误出现且一个流被设置了除了 goodbit 以外的任何标记,之后对该流进行的所有流操作将被忽略。这个状态可以通过调用 \acode{clear()} 函数来进行清除。

\subsubsection*{输入校验}

\textbf{输入校验} input validation 是检查用户输入是否符合某些条件的一个过程。输入校验通常可以拆分为两种类型:字符串与数值。

通过字符串校验,我们接受所有用户输入为字符串,然后根据其是否被正确的格式化来接受或拒绝该字符串。例如,如果要求用户输入一个电话号码,我们希望确保输入的数据有十位。
在大多数语言中(特别是脚本语言例如 Perl 以及 PHP),这是通过正则表达来完成的。C++ 标准库同样也拥有一个正则表达库。因为正则表达式相比于手动字符串校验而言要慢很多,
它们通常仅在不考虑性能(编译时间与运行时间)或者手动校验太笨重时使用。

通过数值校验,我们通常将用户输入的数值确保在特定的范围内(例如在 0 到 20 之间)。然而不同于字符串校验,用户输入的完全不是数字是有可能的 -- 我们还需要处理这种情况。

C++ 提供了一系列函数供用户决定特定字符是数值还是字符。下面这些函数存在于 \acode{cctype} 头文件中:

\begin{center}
  \begin{tiny}
    \begin{tabularx}{ 1\textwidth}{
        | >{\raggedright\arraybackslash}X
        | >{\raggedright\arraybackslash}X |
      }
      \hline
      Function           & Meaning                                                                         \\
      \hline
      std::isalnum(int)  & Returns non-zero if the parameter is a letter or a digit                        \\
      std::isalpha(int)  & Returns non-zero if the parameter is a letter                                   \\
      std::iscntrl(int)  & Returns non-zero if the parameter is a control character                        \\
      std::isdigit(int)  & Returns non-zero if the parameter is a digit                                    \\
      std::isgraph(int)  & Returns non-zero if the parameter is printable character that is not whitespace \\
      std::isprint(int)  & Returns non-zero if the parameter is printable character (including whitespace) \\
      std::ispunct(int)  & Returns non-zero if the parameter is neither alphanumeric nor whitespace        \\
      std::isspace(int)  & Returns non-zero if the parameter is whitespace                                 \\
      std::isxdigit(int) & Returns non-zero if the parameter is a hexadecimal digit (0-9, a-f, A-F)        \\
      \hline
    \end{tabularx}
  \end{tiny}
\end{center}

\subsubsection*{字符串校验}

现在让我们做一个简单的要求用户输入字符串的校验案例。校验的标准是用户仅输入字母字符或空格。如果任何其他类型的输入出现,输入则被拒绝。

当遇到变量长度输入时,最好校验字符串的方式(在不使用一个正则表达式的库的情况下)是遍历每个字符并确保它符合校验标准。
以下案例便是这么做的,或者更确切的说是 \acode{std::all_of} 做的。

\begin{lstlisting}[language=C++]
#include <algorithm> // std::all_of
#include <cctype> // std::isalpha, std::isspace
#include <iostream>
#include <ranges>
#include <string>
#include <string_view>

bool isValidName(std::string_view name)
{
  return std::ranges::all_of(name, [](char ch) {
    return (std::isalpha(ch) || std::isspace(ch));
  });

  // Before C++20, without ranges
  // return std::all_of(name.begin(), name.end(), [](char ch) {
  //    return (std::isalpha(ch) || std::isspace(ch));
  // });
}

int main()
{
  std::string name{};

  do
  {
    std::cout << "Enter your name: ";
    std::getline(std::cin, name); // get the entire line, including spaces
  } while (!isValidName(name));

  std::cout << "Hello " << name << "!\n";
}
\end{lstlisting}

注意这段代码并不完美:用户可能会输入“asf w jweo s di we ao” 或者其它垃圾信息,或者更坏的是一堆空格。
我们可以强化校验标准来避免上述问题。

现在让我们看一下另一个例子,用户输入电话号码。不同于用户名,对于每个字符的长度与校验标准都是一致的,
电话号码是固定长度的,但是校验标准根据字符所处的位置不同。因此我们需要不同的方法来处理电话号码的输入。
编写的函数将会检查用户的输入是否与预定义的模板匹配。该模板的工作原理如下:

\begin{itemize}
  \item \# 将匹配任意用户输入的数值
  \item @ 将匹配任何用户输入的字母
  \item \_ 将匹配任何空值
  \item ? 将匹配任意值
\end{itemize}

除此之外,用户的输入必须完全与模板匹配。

那么如果询问函数匹配模板 \acode{(###)###-####},意味着预期用户输入一个 \acode{(} 字符,三个数字,一个 \acode{)} 字符,
一个空格,三个数字,一杠,以及四个数字。如果出现任何不匹配,输入将被拒绝。

\begin{lstlisting}[language=C++]
#include <algorithm> // std::equal
#include <cctype> // std::isdigit, std::isspace, std::isalpha
#include <iostream>
#include <map>
#include <ranges>
#include <string>
#include <string_view>

bool inputMatches(std::string_view input, std::string_view pattern)
{
    if (input.length() != pattern.length())
    {
        return false;
    }

    // This table defines all special symbols that can match a range of user input
    // Each symbol is mapped to a function that determines whether the input is valid for that symbol
    static const std::map<char, int (*)(int)> validators{
      { '#', &std::isdigit },
      { '_', &std::isspace },
      { '@', &std::isalpha },
      { '?', [](int) { return 1; } }
    };

    // Before C++20, use
    // return std::equal(input.begin(), input.end(), pattern.begin(), [](char ch, char mask) -> bool {
    // ...

    return std::ranges::equal(input, pattern, [](char ch, char mask) -> bool {
        if (auto found{ validators.find(mask) }; found != validators.end())
        {
            // The pattern's current element was found in the validators. Call the
            // corresponding function.
            return (*found->second)(ch);
        }
        else
        {
            // The pattern's current element was not found in the validators. The
            // characters have to be an exact match.
            return (ch == mask);
        }
        });
}

int main()
{
    std::string phoneNumber{};

    do
    {
        std::cout << "Enter a phone number (###) ###-####: ";
        std::getline(std::cin, phoneNumber);
    } while (!inputMatches(phoneNumber, "(###) ###-####"));

    std::cout << "You entered: " << phoneNumber << '\n';
}
\end{lstlisting}

使用该函数可以强制用户匹配特殊的格式。然而,该函数仍然若干约束:如果 \#,@,\_,以及 ? 是合法的用户字符输入,该函数不能正常工作,
因为这些符号被赋予了特殊的含义。另外,有别于正则表达式,缺少一个模板符号用作于表示“一个数值字符可以被输入”。因此,这样一个模板不能被
用于保证用户输入两个单词间添加一个空格,因为它不能处理单词变量的长度。正因上述问题,非模板处理通常更为合适。

\subsubsection*{数值校验}

\begin{lstlisting}[language=C++]
#include <iostream>
#include <limits>

int main()
{
    int age{};

    while (true)
    {
        std::cout << "Enter your age: ";
        std::cin >> age;

        if (std::cin.fail()) // no extraction took place
        {
            std::cin.clear(); // reset the state bits back to goodbit so we can use ignore()
            std::cin.ignore(std::numeric_limits<std::streamsize>::max(), '\n'); // clear out the bad input from the stream
            continue; // try again
        }

        if (age <= 0) // make sure age is positive
            continue;

        break;
    }

    std::cout << "You entered: " << age << '\n';
}
\end{lstlisting}

如果不想这样的输入有效,我们需要做一些额外的工作。幸运的是,之前的例子已经给出了一半的解决方案。我们可以使用 \acode{gcount()} 函数来
决定多少字符被忽略。如果输出是有效的,\acode{gcount()} 应该返回 1(新行字符已被弃用了)。如果返回的数大于 1,那么用户的输入并不正确,
我们需要问他们要新的输入。以下是例子:


\begin{lstlisting}[language=C++]
#include <iostream>
#include <limits>

int main()
{
    int age{};

    while (true)
    {
        std::cout << "Enter your age: ";
        std::cin >> age;

        if (std::cin.fail()) // no extraction took place
        {
            std::cin.clear(); // reset the state bits back to goodbit so we can use ignore()
            std::cin.ignore(std::numeric_limits<std::streamsize>::max(), '\n'); // clear out the bad input from the stream
            continue; // try again
        }

        std::cin.ignore(std::numeric_limits<std::streamsize>::max(), '\n'); // clear out any additional input from the stream
        if (std::cin.gcount() > 1) // if we cleared out more than one additional character
        {
            continue; // we'll consider this input to be invalid
        }

        if (age <= 0) // make sure age is positive
        {
            continue;
        }

        break;
    }

    std::cout << "You entered: " << age << '\n';
}
\end{lstlisting}

\subsubsection*{数值校验作为一个字符串}

上述相当大的例子仅作用于简单的工作。另一种处理数值输入的方式是将它们读取成一个字符串。以下例子就是使用这样的方法论:

\begin{lstlisting}[language=C++]
#include <charconv> // std::from_chars
#include <iostream>
#include <optional>
#include <string>
#include <string_view>

std::optional<int> extractAge(std::string_view age)
{
  int result{};
  auto end{ age.data() + age.length() };

  // Try to parse an int from age
  if (std::from_chars(age.data(), end, result).ptr != end)
  {
    return {};
  }

  if (result <= 0) // make sure age is positive
  {
    return {};
  }

  return result;
}

int main()
{
  int age{};

  while (true)
  {
    std::cout << "Enter your age: ";
    std::string strAge{};
    std::getline(std::cin >> std::ws, strAge);

    if (auto extracted{ extractAge(strAge) })
    {
      age = *extracted;
      break;
    }
  }

  std::cout << "You entered: " << age << '\n';
}
\end{lstlisting}

这种处理方式的工作量大小取决于验证参数和限制。

如此可见,C++ 中的输入验证需要做大量的工作。幸运的是,很多这样的任务(例如作为一个字符串的数值校验)可以简单的转换成函数,即可以在不同的情况下进行复用。

\end{document}
