\documentclass[../../LearnCpp.tex]{subfiles}

\begin{document}

\asubsection{2}{Input with istream}

iostream 库相当的复杂 -- 因此我们不能在此教程中做到完全覆盖。然而我们会展示最常见的功能。本节将深入不同角度的输入类(istream)。

\subsubsection*{提取操作符}

在之前的很多章节已经见到过了,我们可以使用提取操作符(>>)来入去输入流中的信息。C++ 已经为所有内建数据类型预定义了提取操作符,
其中我们也学习了如何为用户自定义的类进行重载提取操作符。

当读取字符串时,提取操作符中的一个最常见的问题就是如何避免输入超出缓存。例如:

\begin{lstlisting}[language=C++]
char buf[10];
std::cin >> buf;
\end{lstlisting}

如果用户输入了 18 个字符呢?缓存会溢出,坏事发生。通常来说,对于用户输入多少字符做任何的预期都是一个坏的主意。

其中一个解决该问题的方案是使用调制器。\textbf{调制器} manipulator 是一个当应用了提取符(>>)或插入(\acode{setw} 位于 iomanip 头文件中)的流时,
用作修改流的对象,其可用于限制字符读取进流的数量。使用 \acode{setw()} 只需要简单的提供一个最大字符数作为读取的参数,并插入至输入声明中。

\begin{lstlisting}[language=C++]
#include <iomanip>
char buf[10];
std::cin >> std::setw(10) >> buf;
\end{lstlisting}

该程序现在只会从流中读取前 9 个字符(为结束符留位置)。任何剩余的字符将被留在流中,直到下一次提取。

\subsubsection*{提取与空白字符}

作为提醒,提取操作符跳过空白字符(空格,tabs,以及换行)。

\begin{lstlisting}[language=C++]
int main()
{
    char ch;
    while (std::cin >> ch)
        std::cout << ch;

    return 0;
}
\end{lstlisting}

当用户进行以下输入:

\begin{lstlisting}
Hello my name is Alex
\end{lstlisting}

提取操作符跳过空格与换行,输出:

\begin{lstlisting}
HellomynameisAlex
\end{lstlisting}

大多数时候并不想抛弃空格,为此 istream 类提供了一些函数来帮助我们。其中一个最常用的是 \textbf{get()} 函数:

\begin{lstlisting}[language=C++]
int main()
{
    char ch;
    while (std::cin.get(ch))
        std::cout << ch;

    return 0;
}
\end{lstlisting}

现在再输入:

\begin{lstlisting}
Hello my name is Alex
\end{lstlisting}

那么输出则是:

\begin{lstlisting}
Hello my name is Alex
\end{lstlisting}

\acode{std::get()} 同样还有一个字符串版本,用作带有上限的字符读取:


\begin{lstlisting}[language=C++]
int main()
{
    char strBuf[11];
    std::cin.get(strBuf, 11);
    std::cout << strBuf << '\n';

    return 0;
}
\end{lstlisting}

现在输入:

\begin{lstlisting}
Hello my name is Alex
\end{lstlisting}

其输出则是:

\begin{lstlisting}
Hello my n
\end{lstlisting}

注意只读取了前 10 个字符(我们需要留一个字符给终止符),而剩下的字符仍然留在输入流中。

\acode{get()} 函数中有一点需要注意的是其不会读取换行字符!这会导致一些非预期的结果:

\begin{lstlisting}[language=C++]
int main()
{
    char strBuf[11];
    // Read up to 10 characters
    std::cin.get(strBuf, 11);
    std::cout << strBuf << '\n';

    // Read up to 10 more characters
    std::cin.get(strBuf, 11);
    std::cout << strBuf << '\n';
    return 0;
}
\end{lstlisting}

如果用户输入:

\begin{lstlisting}
Hello!
\end{lstlisting}

程序输出:

\begin{lstlisting}
Hello!
\end{lstlisting}

接着结束!为什么它不要求另外十个字符呢?这是因为第一个 \acode{get()} 读取到换行接着停止了。
第二个 \acode{get()} 看到 cin 流中仍然有输入并尝试读取。但是第一个字符是换行,因此立刻停止。

因此有另一个称为 \textbf{getline()} 的函数,其功能与 \acode{get()} 一致,但是会读取换行。

\begin{lstlisting}[language=C++]
int main()
{
    char strBuf[11];
    // Read up to 10 characters
    std::cin.getline(strBuf, 11);
    std::cout << strBuf << '\n';

    // Read up to 10 more characters
    std::cin.getline(strBuf, 11);
    std::cout << strBuf << '\n';
    return 0;
}
\end{lstlisting}

这个代码正如预期,即使用户输入了带有换行的字符串。

如果需要知道上一次调用 \acode{getline()} 提取了多少字符,使用 \textbf{gcount()}:

\begin{lstlisting}[language=C++]
int main()
{
    char strBuf[100];
    std::cin.getline(strBuf, 100);
    std::cout << strBuf << '\n';
    std::cout << std::cin.gcount() << " characters were read" << '\n';

    return 0;
}
\end{lstlisting}

\subsubsection*{std::string 的一个特殊版本的 getline()}

还有一个特殊版本的 \acode{getline()} 位于 istream 类之外,用于读取 std::string 变量中的值。
这个特殊的版本并不是 ostream 或是 istream 中的成员,它是位于 string 头文件中。

\begin{lstlisting}[language=C++]
#include <string>
#include <iostream>

int main()
{
    std::string strBuf;
    std::getline(std::cin, strBuf);
    std::cout << strBuf << '\n';

    return 0;
}
\end{lstlisting}

\subsubsection*{一些额外的有用的 istream 函数}

还有一些有用的输入函数用户可能需要会用到:

\begin{itemize}
    \item \textbf{ignore()} 丢弃流中的第一个字符
    \item \textbf{ignore(int nCount)} 丢弃流中前 nCount 的字符
    \item \textbf{peek()} 允许从流中读取字符而不移除它
    \item \textbf{unget()} 返回最后一个字符并不消耗它,这样在下次调用时可以再次读取
    \item \textbf{putback(char ch)} 允许将一个字符放入流中,下次调用时读取
\end{itemize}

\end{document}
