\documentclass[../../LearnCpp.tex]{subfiles}

\begin{document}

\asubsection{4}{Stream classes for strings}

迄今为止所接触到的所有 I/O 例子都是 cout 写入或者 cin 读取。然而还有另外一套关于字符串流的类,允许用户对字符串进行插入(<<)以及提取(>>)操作符。
类似于 istream 与 ostream,字符串流提供了一个缓存用于存储数据。然而不同于 cin 与 cout,这些流并没有连接到一个 I/O 管道(例如键盘,显示器灯)。
字符串流的一个最主要使用时缓存用于展示的输出,或者处理一行一行的输入。

字符串有六种流类:istringstream(从 istream 派生),ostringstream(从 ostream 派生),以及 stringstream(从 iostream 派生)用于读取与写入
普通字符宽度的字符串,wistringstream,wostringstream,以及 wstringstream 用于读取与写入宽字符的字符串。需要 \#include \acode{sstream} 头文件,
才能使用 stringstreams。

有两种存储 stringstream 数据的方式:

\begin{enumerate}
      \item 使用插入(<<)操作符:
            \begin{lstlisting}[language=C++]
        std::stringstream os;
        os << "en garde!\n"; // insert "en garde!" into the stringstream
        \end{lstlisting}
      \item 使用 str(string) 函数来设置缓存的值:
            \begin{lstlisting}[language=C++]
        std::stringstream os;
        os.str("en garde!"); // set the stringstream buffer to "en garde!"
        \end{lstlisting}
\end{enumerate}

有两种类似的获取 stringstream 数据的方式:

\begin{enumerate}
      \item 使用 \acode{str()} 函数来获取缓存的数据:
            \begin{lstlisting}[language=C++]
        std::stringstream os;
        os << "12345 67.89\n";
        std::cout << os.str();
        \end{lstlisting}
            打印:
            \begin{lstlisting}
        12345 67.89
        \end{lstlisting}
      \item 使用提取(>>)操作符:
            \begin{lstlisting}[language=C++]
        std::stringstream os;
        os << "12345 67.89"; // insert a string of numbers into the stream

        std::string strValue;
        os >> strValue;

        std::string strValue2;
        os >> strValue2;

        // print the numbers separated by a dash
        std::cout << strValue << " - " << strValue2 << '\n';
        \end{lstlisting}
            打印:
            \begin{lstlisting}
        12345 - 67.89
        \end{lstlisting}
\end{enumerate}

\subsubsection*{字符串与数字间的转换}

由于插入与提取操作符明白如何与所有基础数据类型工作,我们可以使用它们来转换字符成为数字,反之亦然。

首先看一下数值转字符串:

\begin{lstlisting}[language=C++]
std::stringstream os;

int nValue{ 12345 };
double dValue{ 67.89 };
os << nValue << ' ' << dValue;

std::string strValue1, strValue2;
os >> strValue1 >> strValue2;

std::cout << strValue1 << ' ' << strValue2 << '\n';
\end{lstlisting}

打印:

\begin{lstlisting}
12345 67.89
\end{lstlisting}

再是数值字符串转数值:

\begin{lstlisting}[language=C++]
std::stringstream os;
os << "12345 67.89"; // insert a string of numbers into the stream
int nValue;
double dValue;

os >> nValue >> dValue;

std::cout << nValue << ' ' << dValue << '\n';
\end{lstlisting}

打印:

\begin{lstlisting}
12345 67.89
\end{lstlisting}

\subsubsection*{清空一个 stringstream 用作重用}

有若干种办法可以清空一个 stringstream 缓存:

\begin{enumerate}
      \item 使用 \acode{str()} 与空的 C-style 字符串,设置其为空字符串:
            \begin{lstlisting}[language=C++]
      std::stringstream os;
      os << "Hello ";

      os.str(""); // erase the buffer

      os << "World!";
      std::cout << os.str();
      \end{lstlisting}
      \item 使用 \acode{str()} 与空的 \acode{std::string} 对象,设置其为空字符串:
            \begin{lstlisting}[language=C++]
      std::stringstream os;
      os << "Hello ";

      os.str(std::string{}); // erase the buffer

      os << "World!";
      std::cout << os.str();
      \end{lstlisting}
\end{enumerate}

它们都会打印以下结果:

\begin{lstlisting}
World!
\end{lstlisting}

当清理一个 stringstream 时,调用 \acode{clear()} 函数通常是一个好主意:

\begin{lstlisting}[language=C++]
std::stringstream os;
os << "Hello ";

os.str(""); // erase the buffer
os.clear(); // reset error flags

os << "World!";
std::cout << os.str();
\end{lstlisting}

\end{document}
