\documentclass[../../LearnCpp.tex]{subfiles}

\begin{document}

\asubsection{7}{Forward declarations and definitions}

\subsubsection*{使用前向声明}

\textbf{前向声明} forward declaration 允许用户在真正定义标识符之前告知编译器该标识符的存在。

通过\textbf{函数声明} function declaration的语句(也可称为\textbf{函数原型})来编写函数的前向声明:

\begin{lstlisting}[language=C++]
int add(int x, int y);  // 函数声明包含了返回值,名称,入参类型,以及分号。没有函数体!
\end{lstlisting}

那么可以这样使用:

\begin{lstlisting}[language=C++]
#include <iostream>

int add(int x, int y);

int main()
{
    std::cout << "The sum of 3 and 4 is: " << add(3, 4) << '\n';
    return 0;
}

int add(int x, int y)
{
    return x + y;
}
\end{lstlisting}


\subsubsection*{其他类型的前向声明}

前向声明经常用于函数上。
然而,前向声明也可以用于其它的 C++ 标识符,例如变量或者用户定义类型。
它们有与函数声明不同的语法,后续章节将会讲到。

\subsubsection*{声明 vs. 定义}

\textbf{定义}为真正实现了(函数或者类型)或者实例化了(变量)标识符。
为了满足 \textbf{linker},定义是必须的。
如果标识符没有提供定义,linker 则会报错。

\textbf{单定义规则} one definition rule(简写 ODR)是 C++ 中熟知的规则。拥有三个部分:

\begin{itemize}
    \item 给定的一个\textit{文件}中,函数、变量、类型或者模板只允许有一种定义。
    \item 给定的一个\textit{程序}中,变量或者普通函数只有一种定义。区别在于程序可以拥有若干文件(下个章节)。
    \item 类型,模板,内联函数以及内联变量允许有独立的定义于不同的文件中(后续章节)。
\end{itemize}

\textbf{声明}为语句用于告知\textit{编译器}标识符的存在以及其类型信息。

\end{document}
