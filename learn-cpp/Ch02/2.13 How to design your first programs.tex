\documentclass[../../LearnCpp.tex]{subfiles}

\begin{document}

\asubsection{13}{How to design your first programs}

\subsubsection*{设计步骤 1:定义目标}

用一至两句换表达用户所需的结果,例如;

\begin{itemize}
    \item 允许用户管理一组用户名称以及关联的电话号码。
    \item 生成随机的地下城且具有好看的洞穴。
    \item 生成一组拥有高分红的股票推荐。
    \item 为从塔顶落下的球需要多长时间着陆而建模。
\end{itemize}

\subsubsection*{设计步骤 2: 定义需求}

需求是对于约束以及能力双方而言的一个好听的词,前者是结果所需的约束条件(例如预算,时间轴,空间,内存等等),后者是程序展示的可以满足用户需求的能力。
注意需求类似于专注于”什么(what)“,而不是”如何(how)“。

例如:

\begin{itemize}
    \item 电话号码需要被存储,所以它们之后可以被重播。
    \item 随机的地下城应该总是包含一种通关方式。
    \item 股票推荐应该利用历史的价格数据。
    \item 用户可以进入塔楼的更高层。
    \item 需要 7 天之内的可测试版本。
    \item 程序应该在用户提交需求后的 10 秒之内产生结果。
    \item 程序应该仅有 0.1\% 的崩溃率。
\end{itemize}

一个简单的问题可能会引出很多需求,同时解决方案并不会”结束“知道满足所有的需求。

\subsubsection*{设计步骤 3:定义工具,目标,以及备用计划}

如果是一个有经验的程序员,则会有其他的一些步骤在此,例如:

\begin{itemize}
    \item 定义目标架构以及/或者操作系统用于运行程序。
    \item 决定哪些工具能被使用。
    \item 决定是个人编写还是集体编写程序。
    \item 定义测试/反馈/发布的策略。
    \item 定义如何备份代码。
\end{itemize}

\subsubsection*{设计步骤 4:拆解困难问题成细小的简单问题}

真实世界中,用户所执行的任务通常来说很复杂。
直接尝试解决这些任务具有很大挑战,这种情况下需要使用\textbf{自上而下}的方法来解决问题。
也就是说,不是直接解决单个复杂的任务,而是拆解任务至若干子任务,它们每个都可以简单的独立完成。
如果子任务解决起来还是很困难,则继续拆解,最终得到可控的最小任务。

\subsubsection*{设计步骤 5:构想出事件的流程}

项目有了结构之后便是要决定如何将所有的任务关联起来,即决定执行事件的序列。

\subsubsection*{实现步骤 1:规划 main 函数}

例如:

\begin{lstlisting}[language=C++]
int main()
{
//    doBedroomThings();
//    doBathroomThings();
//    doBreakfastThings();
//    doTransportationThings();

    return 0;
}
\end{lstlisting}

或者:

\begin{lstlisting}[language=C++]
int main()
{
    // Get first number from user
    // getUserInput();

    // Get mathematical operation from user
    // getMathematicalOperation();

    // Get second number from user
    // getUserInput();

    // Calculate result
    // calculateResult();

    // Print result
    // printResult();

    return 0;
}
\end{lstlisting}

\subsubsection*{实现步骤 2:实现每个函数}

每个函数需要:

\begin{enumerate}
    \item 定义函数原型(输入和输出)
    \item 编写函数
    \item 测试函数
\end{enumerate}

\subsubsection*{实现步骤 3:最终测试}

测试整个项目。

\subsubsection*{其他编写程序时的建议}

\begin{itemize}
    \item \textbf{程序总是可以简单的启动}。
    \item \textbf{逐步增加功能}。
    \item \textbf{同一时间关注一块领域}。
    \item \textbf{编写代码的同时做好测试}。
    \item \textbf{不要预想着前期代码就完美}。
\end{itemize}

\end{document}
