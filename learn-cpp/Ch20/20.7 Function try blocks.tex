\documentclass[../../LearnCpp.tex]{subfiles}

\begin{document}

\asubsection{7}{Function try blocks}

try 以及 catch blocks 可以应付大多数情况,但是有一种特殊的情况不适用。考虑下面例子:

\begin{lstlisting}[language=C++]
#include <iostream>

class A
{
private:
  int m_x;
public:
  A(int x) : m_x{x}
  {
    if (x <= 0)
      throw 1; // 这里抛出异常
  }
};

class B : public A
{
public:
  B(int x) : A{x}
  {
    // 如果 A 的创建失败了会发生什么,以及该如何应对?
  }
};

int main()
{
  try
  {
    B b{0};
  }
  catch (int)
  {
    std::cout << "Oops\n";
  }
}
\end{lstlisting}

\subsubsection*{函数 try blocks}

函数 try blocks 是设计用来对整个函数体建立异常而不是一个代码块,其语法稍微有点难以描述,且看下面例子:

\begin{lstlisting}[language=C++]
#include <iostream>

class A
{
private:
  int m_x;
public:
  A(int x) : m_x{x}
  {
    if (x <= 0)
      throw 1; // 这里抛出异常
  }
};

class B : public A
{
public:
  B(int x) try : A{x} // 注意这里额外的 try 关键字
  {
  }
  catch (...) // 注意这与函数体本身处于同级
  {
    // 成员初始化列表的异常或是构造函数体在这里被捕获

    std::cerr << "Exception caught\n";

    throw; // 重新抛出现有异常
  }
};

int main()
{
  try
  {
    B b{0};
  }
  catch (int)
  {
    std::cout << "Oops\n";
  }
}
\end{lstlisting}

打印:

\begin{lstlisting}
Exception caught
Oops
\end{lstlisting}

首先,注意额外的“try”关键字位于成员初始化列表之前。
这代表着其所有的后续(直到函数结束)应该被视为在 try block 之内。

其次,注意关联的 catch block 是位于整个函数相同缩进的级别。
任何在 try 关键字与函数结尾中抛出的异常都应该在这里被捕获。

当上面的程序运行时,变量 \acode{b} 开始构造,调用 \acode{B} 的构造函数(使用了函数 try)。
\acode{B} 的构造函数调用 \acode{A} 的构造函数,接着抛出一个异常。
因为 \acode{A} 的构造函数并未处理这个异常,异常传递给栈上的 \acode{B} 构造函数,
其被 \acode{B} 构造函数同等级所捕获。
catch block 打印“Exception caught”,接着重新抛出当前异常给栈,
即被 \acode{main()} 中的 catch block 捕获,并打印“Oops”。

\subsubsection*{函数 catch blocks 的限制}

对于常规的 catch block(函数内部)而言,有三种选项:
可以抛出一个新的异常,重新抛出当前异常,或者是解决该异常(通过一个返回声明,或者使控制触及 catch block 底部)。

对于一个构造函数而言,函数等级的 catch block 必须抛出一个新异常或者抛出已有异常 --
也就是不允许解决异常!返回声明同样也不被运行,同时触及 catch block 底部将隐式的重新抛出异常。

对于一个结构函数而言,函数等级的 catch block 可以抛出异常,重新抛出异常,
或者是通过一个返回声明解决当前异常。触及 catch block 的底部将会隐式的重新抛出异常。

对于其他函数而言,函数等级的 catch block 可以抛出异常,
重新抛出异常,或者是通过一个返回声明解决当前异常。
触及 catch block 的底部将会隐式的解析非值(\acode{void})返回的函数,
以及对值返回的函数产生未定义行为!

下面的表格总结了限制与函数等级 catch blocks 的行为:

\begin{center}
  \begin{tiny}
    \begin{tabularx}{ 1\textwidth}{
        | >{\raggedright\arraybackslash}X
        | >{\raggedright\arraybackslash}X
        | >{\raggedright\arraybackslash}X |
      }
      \hline
      Function type                & Can resolve exceptions via return statement & Behavior at end of catch block \\
      Constructor                  & No, must throw or rethrow                   & Implicit rethrow               \\
      Destructor                   & Yes                                         & Implicit rethrow               \\
      Non-value returning function & Yes                                         & Resolve exception              \\
      Value-returning function     & Yes                                         & Undefined behavior             \\
    \end{tabularx}
  \end{tiny}
\end{center}

由于在 catch block 结尾的不同行为依赖于函数类型(包括值返回函数的未定义行为),
因此推荐永远不要让控制触及 catch block 的结尾,并总是显示抛出异常,重新抛出异常或者是返回。

最佳实践:应当避免函数等级的 catch block 触及结尾,而是显示的抛出异常,重新抛出异常或者是返回。

\subsubsection*{函数 try blocks 可以捕获基类以及当前类异常}

上面的例子中,如果 \acode{A} 或 \acode{B} 构造函数抛出异常,
则会被 \acode{B} 的构造函数的 try block 所捕获。

下面代码从 \acode{B} 类中抛出异常,而不是从 \acode{A}。

\begin{lstlisting}[language=C++]
#include <iostream>

class A
{
private:
  int m_x;
public:
  A(int x) : m_x{x}
  {
  }
};

class B : public A
{
public:
  B(int x) try : A{x} // 注意这里额外的 try 关键字
  {
    if (x <= 0) // 从 A 中移动到 B
      throw 1;  // 以及移动改行
  }
  catch (...)
  {
    std::cerr << "Exception caught\n";

    // 如果异常没有在这里被显式的抛出,当前异常将会被隐式重新抛出
  }
};

int main()
{
  try
  {
    B b{0};
  }
  catch (int)
  {
    std::cout << "Oops\n";
  }
}
\end{lstlisting}

可以获得相同的输出:

\begin{lstlisting}
Exception caught
Oops
\end{lstlisting}

\subsubsection*{不要使用函数尝试清理资源}

当一个对象的构造函数失败嘞,该类的析构函数并不会被调用。
因此用户可能倾向使用一个函数 try block 在失败之前清理之前分配过的资源。
然而指向失败对象的成员被视为未定义行为,因为对象在 catch block 之前就已经“死了”。
这就意味着不可以在一个类结束后还使用函数 try 来进行清理。
如果希望在类结束后清理,遵循标准规则来清理类,且抛出异常。

函数 try 主要用于日志失败之前通过传递异常给栈,或者改变抛出异常的类型。

\end{document}
