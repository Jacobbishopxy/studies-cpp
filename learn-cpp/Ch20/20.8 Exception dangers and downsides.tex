\documentclass[../../LearnCpp.tex]{subfiles}

\begin{document}

\asubsection{8}{Exception dangers and downsides}

异常有很多的好处,但同时也存在一些潜在的缺陷!

\subsubsection*{清理资源}

新手程序员在使用异常时遇到最大的问题就是清理资源。考虑下面代码:

\begin{lstlisting}[language=C++]
#include <iostream>

try
{
    openFile(filename);
    writeFile(filename, data);
    closeFile(filename);
}
catch (const FileException& exception)
{
    std::cerr << "Failed to write to file: " << exception.what() << '\n';
}
\end{lstlisting}

如果 \acode{WriteFile()} 失败并抛出一个 \acode{FileException} 会发生什么?
这个时间点已经打开了文件,现在的控制流调到了 \acode{FileException} 句柄,即打印错误并退出。
注意文件永远没有被关闭!这个例子应该重写成:

\begin{lstlisting}[language=C++]
#include <iostream>

try
{
    openFile(filename);
    writeFile(filename, data);
}
catch (const FileException& exception)
{
    std::cerr << "Failed to write to file: " << exception.what() << '\n';
}

// 确保文件被关闭
closeFile(filename);
\end{lstlisting}

该类型的错误通常以另一种形式出现,即当处理动态分配内存时:

\begin{lstlisting}[language=C++]
#include <iostream>

try
{
    auto* john { new Person{ "John", 18, PERSON_MALE } };
    processPerson(john);
    delete john;
}
catch (const PersonException& exception)
{
    std::cerr << "Failed to process person: " << exception.what() << '\n';
}
\end{lstlisting}

如果 \acode{processPerson()} 抛出异常,控制流调转到捕获句柄。
结果 john 永远没有被释放!
这个例子比上一个例子更为棘手 --
因为 john 是属于 try block 本地的,它会在 try block 退出时离开作用域。
这就意味着异常处理完全不能访问 john(它已经被销毁了),因此没有办法释放其内存。

然而有两种相关的简单方法来修复这个问题。首先,在 try block 外声明 john,
因此其不会在 try block 退出时离开作用域:

\begin{lstlisting}[language=C++]
#include <iostream>

Person* john{ nullptr };

try
{
    john = new Person("John", 18, PERSON_MALE);
    processPerson(john);
}
catch (const PersonException& exception)
{
    std::cerr << "Failed to process person: " << exception.what() << '\n';
}

delete john;
\end{lstlisting}

第二种方法是使用一个本地的类变量,其明白在离开作用域时改如何清理自身(通常称为智能指针)。
标准库为了这个目的提供了一个名为 \acode{std::unique_ptr} 的类。
\textbf{std::unique\_ptr} 是一个模板类其存储一个指针,在离开作用域时释放该指针。

\begin{lstlisting}[language=C++]
#include <iostream>
#include <memory> // std::unique_ptr

try
{
    auto* john { new Person("John", 18, PERSON_MALE) };
    std::unique_ptr<Person> upJohn { john }; // upJohn 现在拥有 john

    ProcessPerson(john);

    // 当 upJohn 离开作用域,它会删除 john
}
catch (const PersonException& exception)
{
    std::cerr << "Failed to process person: " << exception.what() << '\n';
}
\end{lstlisting}

下一章节将会详细讲解智能指针。

\subsubsection*{异常与析构函数}

不同于构造函数,抛出异常在表示对象创建没有成功时非常的有用,但是异常应该\textit{永远不}在析构函数中抛出。

问题出现在当一个异常从析构函数中抛出时的栈展开过程中。
编译器会进入一种状态不知道是继续栈展开过程还是处理新的异常。
最终的结果就是程序立刻被终结。

规则:析构函数不应该抛出异常。

\subsubsection*{性能考虑}

异常确实具有小的性能成本。
它们增加了可执行的大小,同时由于需要执行额外的检查,这也可能会造成运行变慢。
然而异常所带来的主要性能惩罚适当其真实被抛出的时候。
这种情况下,栈必须被展开同时查找合适的异常处理,也就是一个相对昂贵的操作。

作为记录,一些现代计算机架构支持被称为零成本异常的异常模型。
如果支持的话,就没有额外的运行时成本在非错误情况(也是最关心性能的地方)。
然而,它们在发现异常时带来了更大的惩罚。

\subsubsection*{何时使用异常}

异常处理的最佳使用是当以下条件皆满足时:

\begin{itemize}
    \item 低频率的出错
    \item 错误很严峻,且不处理就没法继续
    \item 错误不能在其出现处被处理
    \item 没有其他的方案来返回错误码给调用者
\end{itemize}

\end{document}
