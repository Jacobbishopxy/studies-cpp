\documentclass[../../LearnCpp.tex]{subfiles}

\begin{document}

\asubsection{1}{The need for exceptions}

在之前的处理错误的章节中地道过使用 \acode{assert()},\acode{std::cerr},以及 \acode{exit()} 来处理异常。
然而,有一个更重要的方式将在本章覆盖:异常。

\subsubsection*{当返回码为失败时}

当编写可复用的代码时,错误处理事必须得。一个最常见的方式来处理潜在的错误就是通过返回码。例如:

\begin{lstlisting}[language=C++]
#include <string_view>

int findFirstChar(std::string_view string, char ch)
{
    // 遍历字符串中每个字符
    for (std::size_t index{ 0 }; index < string.length(); ++index)
        // 如果匹配 ch,返回其索引
        if (string[index] == ch)
            return index;

    // 如果不匹配,返回 -1
    return -1;
}
\end{lstlisting}

这个函数返回字符串中第一个匹配 \acode{ch} 字符的索引。如果没有找到则返回 -1 作为错误指示。

这种方法的主要优点是,它非常的简单。然而在重要场合下使用返回码的若干缺陷会快速暴露出来:

首先,返回值的含义是模糊的 -- 如果一个函数返回 -1,那么它到底是代表错误,还是一个真实有效的值?

其次,函数仅可以返回一个值,那么如果希望既返回一个函数结果又返回一个错误码呢?

再者,一系列的代码中有很多地方可以出错,错误码需要经常被检查。考虑下面的代码切片,涉及到传递一个文本文件:

\begin{lstlisting}[language=C++]
std::ifstream setupIni { "setup.ini" }; // 打开 setup.ini 用于读
// 如果文件不能被打开(例如文件缺失)返回一些某错误枚举
if (!setupIni)
    return ERROR_OPENING_FILE;

// 开始从文件中读取内容
if (!readIntegerFromFile(setupIni, m_firstParameter)) // 尝试从文件中读取整型值
    return ERROR_READING_VALUE; // 返回代表不能读取值的枚举值

if (!readDoubleFromFile(setupIni, m_secondParameter)) // 尝试从文件中读取双精度值
    return ERROR_READING_VALUE;

if (!readFloatFromFile(setupIni, m_thirdParameter))   // 尝试从文件中读取浮点值
    return ERROR_READING_VALUE;
\end{lstlisting}

暂时还未覆盖文件访问,无须担心上述代码如何工作的 --
仅需要注意每个调用都需要错误检查并返回给调用者。
现在试想一下有二十种类型的参数 --
那么按照上面的做法要检查二十次并返回 ERROR\_READING\_VALUE 二十次!

第四,错误码并不与构造函数契合。
如果创建一个对象时构造函数内发生错误应该怎么办?
构造函数没有返回类型传递回状态指示,通过引用参数传递回的话又麻烦又必须要显示检查。
另外,即使做了这些,对象仍然会被创建,那么就需要处理它或者销毁它。

最后,当错误码返回给调用者时,调用者并不能总是处理错误。
如果调用者不想处理错误,那么要么是忽略它(这种情况下就永远失去了),要么是再向上返回错误码。
这相当的胡乱同时也带来了之前提到过的那些问题。

\subsubsection*{异常}

异常处理提供了一种机制用来从通常代码的控制流中解耦错误或者其它特殊情况。
它使处理错误在合适以及如何处理上,拥有更多的自由,消除了大部分(如果不是全部的)的错误码带来的混乱。

\end{document}
