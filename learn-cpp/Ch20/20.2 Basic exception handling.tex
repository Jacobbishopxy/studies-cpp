\documentclass[../../LearnCpp.tex]{subfiles}

\begin{document}

\asubsection{2}{Basic exception handling}

C++ 中的异常处理是由三个关键字关联工作的:\textit{throw},\textit{try} 以及 \textit{catch}。

\subsubsection*{抛出异常}

在 C++ 中,一个 \textit{throw} 声明用于发出异常或者错误情况出现了的信号。
一个异常的出现同样被称为 \textit{raising} 一个异常。例如:

\begin{lstlisting}[language=C++]
throw -1; // throw a literal integer value
throw ENUM_INVALID_INDEX; // throw an enum value
throw "Can not take square root of negative number"; // throw a literal C-style (const char*) string
throw dX; // throw a double variable that was previously defined
throw MyException("Fatal Error"); // Throw an object of class MyException
\end{lstlisting}

每个声明作为一个信号代表某种类型的问题出现了且需要被解决。

\subsubsection*{查看异常}

抛出异常仅是异常处理过程的一部分。
C++ 中,使用 \textit{try} 关键字来定义一块声明(成为 \textit{try block}),
检查其中任何代码块中可能抛出异常的地方。

\begin{lstlisting}[language=C++]
try
{
    // Statements that may throw exceptions you want to handle go here
    throw -1; // here's a trivial throw statement
}
\end{lstlisting}

注意 try block 并没有定义如何去处理异常。

\subsubsection*{处理异常}

真实处理异常的地方是在 catch block(s)。
\textit{catch} 关键字用于定义单个数据类型的异常处理。

\begin{lstlisting}[language=C++]
catch (int x)
{
    // Handle an exception of type int here
    std::cerr << "We caught an int exception with value" << x << '\n';
}
\end{lstlisting}

如果参数不会在 catch block 中用到,则可以被省略:

\begin{lstlisting}[language=C++]
catch (double) // note: no variable name since we don't use it in the catch block below
{
    // Handle exception of type double here
    std::cerr << "We caught an exception of type double" << '\n';
}
\end{lstlisting}

\subsubsection*{throw,try 以及 catch}

\begin{lstlisting}[language=C++]
#include <iostream>
#include <string>

int main()
{
    try
    {
        // Statements that may throw exceptions you want to handle go here
        throw -1; // here's a trivial example
    }
    catch (int x)
    {
        // Any exceptions of type int thrown within the above try block get sent here
        std::cerr << "We caught an int exception with value: " << x << '\n';
    }
    catch (double) // no variable name since we don't use the exception itself in the catch block below
    {
        // Any exceptions of type double thrown within the above try block get sent here
        std::cerr << "We caught an exception of type double" << '\n';
    }
    catch (const std::string&) // catch classes by const reference
    {
        // Any exceptions of type std::string thrown within the above try block get sent here
        std::cerr << "We caught an exception of type std::string" << '\n';
    }

    std::cout << "Continuing on our merry way\n";

    return 0;
}
\end{lstlisting}

\subsubsection*{catch block 通常用来做什么}

如果一个异常被导航到了一个 catch block,它被视为“已处理”即使该 catch block 是空的。
然而,通常希望的是 catch block 能做一些有用的事情。有三种通常的做法:

首先,可以打印错误(输出到终端或是日志文件)。

其次,可以返回一个值或者错误码给调用者。

第三,可以抛出另一个异常。
因为 catch block 是在 try block 外部,新的抛出异常并没有被 try block 处理 --
它会被另一个包围其的 try block 处理。

\end{document}
