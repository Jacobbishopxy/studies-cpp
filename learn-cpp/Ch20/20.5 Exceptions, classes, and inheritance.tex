\documentclass[../../LearnCpp.tex]{subfiles}

\begin{document}

\asubsection{5}{Exceptions, classes, and inheritance}

\subsubsection*{异常与成员函数}

在成员函数中,异常同样非常有用,甚至在重载操作符中更有用。
考虑下列的重载 \acode{[]} 操作符作为简易整型数组类:

\begin{lstlisting}[language=C++]
int& IntArray::operator[](const int index)
{
    return m_data[index];
}
\end{lstlisting}

尽管这个函数在有效范围能工作的很好,但是它急缺一个好的错误检查。
可以通过断言声明来确保其有效性:

\begin{lstlisting}[language=C++]
int& IntArray::operator[](const int index)
{
    assert (index >= 0 && index < getLength());
    return m_data[index];
}
\end{lstlisting}

现在如果用户传入了无效索引,项目会导致断言错误。
不幸的是因为重载操作符有其特定的需求的数量和类型参数用于返回,
通过向调用者返回错误码或者布尔值的做法非常的不灵活。
然而,由于异常并不会修改函数签名,它们在此的作用意义重大。

\begin{lstlisting}[language=C++]
int& IntArray::operator[](const int index)
{
    if (index < 0 || index >= getLength())
        throw index;

    return m_data[index];
}
\end{lstlisting}

\subsubsection*{当构造函数失败时}

构造函数是另一个使用异常非常有用的地方。
如果一个构造函数因为某些原因(例如用户无效输入)必须失败时,
简单的抛出一个异常就可以指示创建对象失败了。
这种情况下对象的构造流产了,所有的类成员(已被创建的,以及之前在构造函数体内初始化的)会被照常被析构。

然而类的析构函数永远不会被调用(因为对象从未完成构造函数)。
因为析构函数从未执行,因此不可依赖析构函数来清理已被分配的内存。

那么问题就来了,应该怎么做当构造函数中分配了内存之后且构造函数结束之前异常出现了。
如何确保已被分配的资源被正确的清理?一个方法是使用 try block 包裹任何可能出现失败的代码,
使用关联的 catch block 来捕获异常并做出必要的清理,接着重新抛出异常。
然而这添加了一堆杂乱的东西,且非常容易出错,特别是当类分配了若干资源时。

幸运的是,还有一个更好的办法。
利用了类成员即使在构建函数失败时也会被析构的优势,
如果资源分配处于类成员之中(而不是处于构造函数之中),那么这些成员可以在被析构时清理自身。

\begin{lstlisting}[language=C++]
#include <iostream>

class Member
{
public:
  Member()
  {
    std::cerr << "Member allocated some resources\n";
  }

  ~Member()
  {
    std::cerr << "Member cleaned up\n";
  }
};

class A
{
private:
  int m_x {};
  Member m_member;

public:
  A(int x) : m_x{x}
  {
    if (x <= 0)
      throw 1;
  }

  ~A()
  {
    std::cerr << "~A\n"; // 不应该被调用到
  }
};


int main()
{
  try
  {
    A a{0};
  }
  catch (int)
  {
    std::cerr << "Oops\n";
  }

  return 0;
}
\end{lstlisting}

打印:

\begin{lstlisting}
Member allocated some resources
Member cleaned up
Oops
\end{lstlisting}

上述代码中,当类 A 抛出一个异常,所有 A 的成员都被析构,
\acode{m_member} 的析构函数被调用,提供了清理其分配的任何资源的机会。

这也是 RAII 被高度提倡的部分原因 -- 即使在异常环境中,实现 RAII 了的类有能力清理自身。

然而像是创建一个 Member 这样的自定义类来管理资源分配并不高效。
幸运的是,C++ 标准库带来的符合 RAII 类来管理通常的资源类型,
例如文件(\acode{std::fstream} 23.6 中覆盖)以及动态内存(\acode{std::unique_ptr}
以及其他类型的智能指针 M.1 中覆盖)。

例如,与其这么做:

\begin{lstlisting}[language=C++]
class Foo
private:
    int* ptr; // Foo 将处理 分配/释放
\end{lstlisting}

不如这么做:

\begin{lstlisting}[language=C++]
class Foo
private:
    std::unique_ptr<int> ptr; // std::unique_ptr 将处理 分配/释放
\end{lstlisting}

前一种情况中,如果 Foo 构造函数在 ptr 分配动态内存后失败了,Foo 则需要负责清理,这会带来挑战。
后一种情况中,Foo 构造函数在 ptr 分配动态内存后失败了,ptr 的析构函数会执行并将内存返回给系统。
Foo 不需要做任何的显示清理,因为资源处理被委派给了符合 RAII 的成员!

\subsubsection*{异常类}

使用基础数据类型做异常类型的一个最大问题就是他们是本身的意义是模糊的。
一个更大的问题是当多个声明或函数调用存在一个 try block 时会引起歧义。

一个解决该问题的方法是使用异常类。一个\textbf{异常类}就是一个被设计成抛出异常的普通的类。

\begin{lstlisting}[language=C++]
#include <iostream>
#include <string>
#include <string_view>

class ArrayException
{
private:
  std::string m_error;

public:
  ArrayException(std::string_view error)
    : m_error{ error }
  {
  }

  const std::string& getError() const { return m_error; }
};

class IntArray
{
private:
  int m_data[3]{};

public:
  IntArray() {}

  int getLength() const { return 3; }

  int& operator[](const int index)
  {
    if (index < 0 || index >= getLength())
      throw ArrayException{ "Invalid index" };

    return m_data[index];
  }
};

int main()
{
  IntArray array;

  try
  {
    int value{ array[5] };
  }
  catch (const ArrayException& exception)
  {
    std::cerr << "An array exception occurred (" << exception.getError() << ")\n";
  }
}
\end{lstlisting}

使用这样的一个类就可以返回带有问题描述信息的异常,因此报错时可以提供异常信息。
因为 \acode{ArrayException} 拥有其自身类型,可以通过其类型指定捕获异常。

\subsubsection*{异常与继承}

由于抛出类作为异常是可行的,类又是可以由其他类派生而来,
因此需要考虑清楚当使用继承的类作为异常时回发生什么。
事实上,异常处理将不仅仅匹配指定类型,它们也会匹配从指定类型中派生而来的类!

\begin{lstlisting}[language=C++]
#include <iostream>

class Base
{
public:
    Base() {}
};

class Derived: public Base
{
public:
    Derived() {}
};

int main()
{
    try
    {
        throw Derived();
    }
    catch (const Base& base)
    {
        std::cerr << "caught Base";
    }
    catch (const Derived& derived)
    {
        std::cerr << "caught Derived";
    }

    return 0;
}
\end{lstlisting}

上述的例子抛出的是 \acode{Derived} 类型的异常,然而输出的却是:

\begin{lstlisting}
caught Base
\end{lstlisting}

为什么会这样呢?

首先,派生类会被当成基类被捕获。因为 \acode{Derived} 由 \acode{Base} 派生而来,
\acode{Derived} 是一个 \acode{Base}。
其次当 C++ 尝试寻找一个被抛出异常的处理句柄,它是顺序进行的。
因此,C++ 做的第一件事就是检查异常处理是否匹配 \acode{Base},如果匹配,
则执行 \acode{Base} 的 catch block!而 \acode{Derived} 的 catch block 永远不会被执行。

为了使其符合预期,需要调转一下 catch blocks 的顺序:

\begin{lstlisting}[language=C++]
#include <iostream>

class Base
{
public:
    Base() {}
};

class Derived: public Base
{
public:
    Derived() {}
};

int main()
{
    try
    {
        throw Derived();
    }
    catch (const Derived& derived)
    {
        std::cerr << "caught Derived";
    }
    catch (const Base& base)
    {
        std::cerr << "caught Base";
    }

    return 0;
}
\end{lstlisting}

规则:处理派生类的异常句柄需要被排列在基类之前。

\subsubsection*{std::exception}

在标准库中又很多类与操作符在失败时会抛出异常。
例如,操作符 \acode{new} 可以在不能分配足够内存时抛出 \acode{std::bad_alloc}。
一个失败的 \acode{dynamic_cast} 将抛出 \acode{std::bad_cast}。
到 C++20 时已经有 28 中不同的异常类可以被抛出,更多的异常类会添加在随后的语言标准。

好消息是所有的这些异常类都是从单个名为 \textbf{std::exception}
(定义在 \acode{<exception>} 头文件中)派生而来的。\acode{std::exception}
是一个小型接口类,设计用于服务所有被 C++ 标准库抛出异常的基类。

大部分时间,当一个由标准库抛出的异常,
用户不需要关心其是否为一个不好的内存分配,一个不好的转型或其他。
仅需要关注是什么灾难性的错误导致程序崩溃。
感谢 \acode{std::exception},用户可以设置其错误处理,并一次性捕获其所有的派生异常!

\begin{lstlisting}[language=C++]
#include <cstddef> // for std::size_t
#include <exception> // for std::exception
#include <iostream>
#include <limits>
#include <string> // for this example

int main()
{
    try
    {
        // Your code using standard library goes here
        // We'll trigger one of these exceptions intentionally for the sake of the example
        std::string s;
        s.resize(std::numeric_limits<std::size_t>::max()); // will trigger a std::length_error or allocation exception
    }
    // This handler will catch std::exception and all the derived exceptions too
    catch (const std::exception& exception)
    {
        std::cerr << "Standard exception: " << exception.what() << '\n';
    }

    return 0;
}
\end{lstlisting}

有时候会需要某个特定类型的异常,这种情况可以添加该指定类型。

\begin{lstlisting}[language=C++]
try
{
     // code using standard library goes here
}
// This handler will catch std::length_error (and any exceptions derived from it) here
catch (const std::length_error& exception)
{
    std::cerr << "You ran out of memory!" << '\n';
}
// This handler will catch std::exception (and any exception derived from it) that fall
// through here
catch (const std::exception& exception)
{
    std::cerr << "Standard exception: " << exception.what() << '\n';
}
\end{lstlisting}

\subsubsection*{直接使用标准异常}

没有任何直接抛出 \acode{std::exception} 的情况,用户也不可以。
然而如果符合需求,则可以随便抛出标准库中其他类型的异常类。
可以在 \href{https://en.cppreference.com/w/cpp/error/exception}{cppreference} 中找到所有的标准异常。

\acode{std::runtime_error}(被包含在 \acode{stdexcept} 头文件中)是一个好的选择,
因为它拥有一个泛型名称,同时其构造函数可以接收自定义信息:

\begin{lstlisting}[language=C++]
#include <exception> // for std::exception
#include <iostream>
#include <stdexcept> // for std::runtime_error

int main()
{
  try
  {
    throw std::runtime_error("Bad things happened");
  }
  // This handler will catch std::exception and all the derived exceptions too
  catch (const std::exception& exception)
  {
    std::cerr << "Standard exception: " << exception.what() << '\n';
  }

  return 0;
}
\end{lstlisting}

\subsubsection*{从 std::exception 或 std::runtime\_error 中派生自定义的类}

用户当然也可以从 \acode{std::exception} 中派生自定义的类,
并重写虚函数 \acode{what()} const 成员函数。

\begin{lstlisting}[language=C++]
#include <exception> // std::exception
#include <iostream>
#include <string>
#include <string_view>

class ArrayException : public std::exception
{
private:
  std::string m_error{}; // 处理自定义字符串

public:
  ArrayException(std::string_view error)
    : m_error{error}
  {
  }

  // std::exception::what() 返回一个 const char*
  const char* what() const noexcept override { return m_error.c_str(); }
};

class IntArray
{
private:
  int m_data[3] {};

public:
  IntArray() {}

  int getLength() const { return 3; }

  int& operator[](const int index)
  {
    if (index < 0 || index >= getLength())
      throw ArrayException("Invalid index");

    return m_data[index];
  }
};

int main()
{
  IntArray array;

  try
  {
    int value{ array[5] };
  }
  catch (const ArrayException& exception) // 派生的 catch blocks 先行
  {
    std::cerr << "An array exception occurred (" << exception.what() << ")\n";
  }
  catch (const std::exception& exception)
  {
    std::cerr << "Some other std::exception occurred (" << exception.what() << ")\n";
  }
}
\end{lstlisting}

注意虚函数 \acode{what()} 有一个限定符 \acode{noexcept}
(意为函数保证其自身不会抛出任何异常)。因此重写应该也带上该限定符。

因为 \acode{std::runtime_error} 已经拥有了字符串处理能力,它同样也是派生异常的基类。

\begin{lstlisting}[language=C++]
#include <exception> // std::exception
#include <iostream>
#include <stdexcept> // std::runtime_error
#include <string>

class ArrayException : public std::runtime_error
{
public:
  // std::runtime_error 接收 const char* 空终止字符串
  // std::string_view 可能不是空终止字符串,因此在这里并不是一个好的选择
  // ArrayException 将获取一个 const std::string&,因为其保证是一个空终止字符串,
  // 同时可以被转换成一个 const char*
  ArrayException(const std::string& error)
    : std::runtime_error{ error.c_str() } // std::runtime_error 将处理字符串
  {
  }

  // 不再需要重写 what() 因为可以使用 std::runtime_error::what()
};

class IntArray
{
private:
  int m_data[3]{};

public:
  IntArray() {}

  int getLength() const { return 3; }

  int& operator[](const int index)
  {
    if (index < 0 || index >= getLength())
      throw ArrayException("Invalid index");

    return m_data[index];
  }
};

int main()
{
  IntArray array;

  try
  {
    int value{ array[5] };
  }
  catch (const ArrayException& exception) // 派生的 catch blocks 先行
  {
    std::cerr << "An array exception occurred (" << exception.what() << ")\n";
  }
  catch (const std::exception& exception)
  {
    std::cerr << "Some other std::exception occurred (" << exception.what() << ")\n";
  }
}
\end{lstlisting}

用户是否要创建自定义的异常类完全根据自身选择,无论是使用标准异常类,
亦或是从 \acode{std::exception} 或 \acode{std::runtime_error} 派生自定义异常类。
所有有效的解决方案取决于目的。

\end{document}
