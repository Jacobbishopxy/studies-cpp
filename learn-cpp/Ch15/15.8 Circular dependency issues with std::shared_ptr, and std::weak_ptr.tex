\documentclass[../../LearnCpp.tex]{subfiles}

\begin{document}

\asubsection{8}{Circular dependency issues with std::shared\_ptr and std::weak\_ptr}

上一节讲述了 \acode{std::shared\_ptr} 允许用户拥有若干智能指针协同拥有相同的资源。
然而在某些情况下,这会变得有问题的。
考虑以下案例,当共享指针指向两个独立对象,其中又各自指向对方:

\begin{lstlisting}[language=C++]
#include <iostream>
#include <memory> // std::shared_ptr
#include <string>

class Person
{
  std::string m_name;
  std::shared_ptr<Person> m_partner; // 初始创建为空

public:
  Person(const std::string &name): m_name(name)
  {
    std::cout << m_name << " created\n";
  }
  ~Person()
  {
    std::cout << m_name << " destroyed\n";
  }

  friend bool partnerUp(std::shared_ptr<Person> &p1, std::shared_ptr<Person> &p2)
  {
    if (!p1 || !p2)
      return false;

    p1->m_partner = p2;
    p2->m_partner = p1;

    std::cout << p1->m_name << " is now partnered with " << p2->m_name << '\n';

    return true;
  }
};

int main()
{
  auto lucy { std::make_shared<Person>("Lucy") }; // 创建一个名为 "Lucy" 的 Person
  auto ricky { std::make_shared<Person>("Ricky") }; // 创建一个名为 "Ricky" 的 Person

  partnerUp(lucy, ricky); // 使 "Lucy" 指向 "Ricky",反之亦然

  return 0;
}
\end{lstlisting}

上述例子中动态分配两个 Person,“Lucy” 与 “Ricky” 使用 \acode{make\_shared()}
(确保 \acode{lucy} 和 \acode{rick} 在 \acode{main()} 结束时销毁)。
接着使他们 partner,即在 “Lucy” 内设置 \acode{std::shared\_ptr} 指向 “Ricky”,
在 “Ricky” 内指向 “Lucy”。
共享指针意味着被共享,因此 lucy 共享指针和 rick 的 m\_partner 共享指针共同指向 “Lucy”(反之亦然)。

然而这个程序并不能像预期那样执行:

\begin{lstlisting}
Lucy created
Ricky created
Lucy is now partnered with Ricky
\end{lstlisting}

在 \acode{partnerUp()} 被调用,有两个共享指针指向 “Ricky”(ricky 以及 Lucy 的 m\_partner)
以及两个共享指针指向 “Lucy”(lucy 以及 Ricky 的 m\_partner)。

在 \acode{main()} 结束时,ricky 共享指针首先离开作用域。
当这个发生时,ricky 检查是否有任何共享指针协调所有 ”Ricky“。有(Lucy 的 m\_partner),
所以 ”Ricky“ 没有被释放(如果释放了,Lucy 的 m\_partner 会变成悬垂指针)。
这个时候,有一个功效指针指向 ”Ricky“(Lucy 的 m\_partner)以及两个共享指针指向 ”Lucy“(lucy 以及 Ricky 的 m\_partner)。

接着 lucy 的共享指针离开作用域,同样的事发生了。
lucy 的共享指针检查是否有其他任何共享指针协同所有 “Lucy”。
有(Ricky 的 m\_partner),所以 “Lucy” 没有被释放。
这个时候,仍然有一个共享指针指向 “Lucy”(Ricky 的 m\_partner)以及一个共享指针指向 “Ricky”(Lucy 的 m\_partner)。

接着程序结束 -- ”Lucy“ 或 ”Ricky“ 没有任何一个被释放!
本质上,”Lucy“ 维系了 ”Ricky“ 不被销毁,”Ricky“ 维系了 ”Lucy“ 不被销毁。

事实证明这可以发生的任何事件共享指针形成一个循环引用。

\subsubsection*{循环引用}

\textbf{循环引用} circular reference(同样被称为 \textbf{cyclical reference} 或 \textbf{cycle})是一系列的引用,
其中每个对象引用下一个,最后一个对象引用回第一个,最终导致了引用环。
引用不需要是真实的 C++ 引用 -- 它们可以是指针,唯一 IDs,或者任何其他用于身份识别的对象。

在共享指针的语境中,引用为指针。

这完全就是上述例子中的:”Lucy“ 指向 ”Ricky“,同时 ”Ricky“ 指向 ”Lucy“。
对于三指针而言,A 指向 B,B 指向 C,C 又指向 A 也会得到相同的结论。
共享指针形成的环所造成的实际效果是每个对象都指向下一个对象 -- 最后一个对象指向第一个对象。
因此,这种情况下没有对象会被释放,因为它们都会认为其它对象仍然需要自身!

\subsubsection*{一个还原案例}

实际上这种循环引用问题甚至会发生在单个 \acode{std::shared\_ptr} 上 --
\acode{std::shared\_ptr} 引用的对象包含自身仍然是一个循环(仅为还原)。
尽管它完全不像会在现实中发生,这里是一个额外的例子作为理解:

\begin{lstlisting}[language=C++]
#include <iostream>
#include <memory> // std::shared_ptr

class Resource
{
public:
  std::shared_ptr<Resource> m_ptr {}; // 初始化为空

  Resource() { std::cout << "Resource acquired\n"; }
  ~Resource() { std::cout << "Resource destroyed\n"; }
};

int main()
{
  auto ptr1 { std::make_shared<Resource>() };

  ptr1->m_ptr = ptr1; // m_ptr 现在共享了 Resource,包含了自身

  return 0;
}
\end{lstlisting}

上述例子中,当 \acode{ptr1} 离开作用域,\acode{Resource} 不会被释放,
因为 \acode{Resource} 的 \acode{m\_ptr} 共享 \acode{Resource}。
这个时候 \acode{Resource} 能被释放的唯一办法就是设置 \acode{m\_ptr} 指向其它(这样就没有任何贡献 \acode{Resource} 的了)。
但是不能访问 \acode{m\_ptr} 因为 \acode{ptr} 离开了最重要,所以不再可能做到了。
\acode{Resouce} 变为了内存泄漏。

所以程序打印:

\begin{lstlisting}
Resource acquired
\end{lstlisting}

这样就没了。

\subsubsection*{那么 std::weak\_ptr 是什么呢?}

\acode{std::weak\_ptr} 被设计用来解决上述描述的”循环所有权“问题。
一个 \acode{std::weak\_ptr} 是一个观测者 --
它可以观测并像 \acode{std::shared\_ptr}(或者其他 \acode{std::weak\_ptr})那样访问同一个对象,
但是不会被认为成一个所有者。
牢记当一个 \acode{std::shared} 指针离开作用域,它仅会考虑是否有其他的 \acode{std::shared\_ptr} 协同所有对象,
而 \acode{std::weak\_ptr} 则不会考虑!

现在通过 \acode{std::weak\_ptr} 来解决之前的问题:

\begin{lstlisting}[language=C++]
#include <iostream>
#include <memory> // std::shared_ptr 与 std::weak_ptr
#include <string>

class Person
{
  std::string m_name;
  std::weak_ptr<Person> m_partner; // 注意:这里现在是一个 std::weak_ptr

public:
  Person(const std::string &name): m_name(name)
  {
    std::cout << m_name << " created\n";
  }
  ~Person()
  {
    std::cout << m_name << " destroyed\n";
  }

  friend bool partnerUp(std::shared_ptr<Person> &p1, std::shared_ptr<Person> &p2)
  {
    if (!p1 || !p2)
      return false;

    p1->m_partner = p2;
    p2->m_partner = p1;

    std::cout << p1->m_name << " is now partnered with " << p2->m_name << '\n';

    return true;
  }
};

int main()
{
  auto lucy { std::make_shared<Person>("Lucy") };
  auto ricky { std::make_shared<Person>("Ricky") };

  partnerUp(lucy, ricky);

  return 0;
}
\end{lstlisting}

现在代码行为正常了:

\begin{lstlisting}
Lucy created
Ricky created
Lucy is now partnered with Ricky
Ricky destroyed
Lucy destroyed
\end{lstlisting}

\subsubsection*{使用 std::weak\_ptr}

\acode{std::weak\_ptr} 的缺陷是它不能直接使用(它们并没有操作符->)。
为了使用一个 \acode{std::weak\_ptr},
必须先转换其成为一个 \acode{std::shared\_ptr}。
可以使用 \acode{lock()} 成员函数进行转换。

\begin{lstlisting}[language=C++]
#include <iostream>
#include <memory> // std::shared_ptr 与 std::weak_ptr
#include <string>

class Person
{
  std::string m_name;
  std::weak_ptr<Person> m_partner; // 注意:这里现在是一个 std::weak_ptr

public:

  Person(const std::string &name) : m_name(name)
  {
    std::cout << m_name << " created\n";
  }
  ~Person()
  {
    std::cout << m_name << " destroyed\n";
  }

  friend bool partnerUp(std::shared_ptr<Person> &p1, std::shared_ptr<Person> &p2)
  {
    if (!p1 || !p2)
      return false;

    p1->m_partner = p2;
    p2->m_partner = p1;

    std::cout << p1->m_name << " is now partnered with " << p2->m_name << '\n';

    return true;
  }

  const std::shared_ptr<Person> getPartner() const { return m_partner.lock(); } // 使用 lock() 转换 weak_ptr 成为 shared_ptr
  const std::string& getName() const { return m_name; }
};

int main()
{
  auto lucy { std::make_shared<Person>("Lucy") };
  auto ricky { std::make_shared<Person>("Ricky") };

  partnerUp(lucy, ricky);

  auto partner = ricky->getPartner(); // 获取 shared_ptr 为 Ricky 的 partner
  std::cout << ricky->getName() << "'s partner is: " << partner->getName() << '\n';

  return 0;
}
\end{lstlisting}

打印:

\begin{lstlisting}
Lucy created
Ricky created
Lucy is now partnered with Ricky
Ricky's partner is: Lucy
Ricky destroyed
Lucy destroyed
\end{lstlisting}

不需要担心 \acode{std::shared\_ptr} 变量 ”partner“ 循环依赖,
因为它仅为函数内部的一个本地变量。
它最终会在函数结束时离开作用域,同时引用计数减少了 1。

\subsubsection*{std::weak\_ptr 的悬垂指针}

因为 \acode{std::weak\_ptr} 并不会维持其所持的资源存活,
那么对于一个 \acode{std::weak\_ptr} 而言,
留下它指向一个已被 \acode{std::shared\_ptr} 释放的对象也是有可能的。
这样的 \acode{std::weak\_ptr} 变成了悬垂,使用它会导致未定义行为。

\begin{lstlisting}[language=C++]
#include <iostream>
#include <memory>

class Resource
{
public:
  Resource() { std::cerr << "Resource acquired\n"; }
  ~Resource() { std::cerr << "Resource destroyed\n"; }
};

auto getWeakPtr()
{
  auto ptr{ std::make_shared<Resource>() }; // Resource 获取

  return std::weak_ptr{ ptr };
} // ptr 离开作用域, Resource 被销毁

int main()
{
  std::cerr << "Getting weak_ptr...\n";

  auto ptr{ getWeakPtr() }; // 悬垂

  std::cerr << "Done.\n";
}
\end{lstlisting}

上述例子中,在 \acode{getWeakPtr()} 中使用 \acode{std::make\_shared()}
来创建一个名为 \acode{ptr} 并所有 \acode{Resource} 对象的 \acode{std::shared\_ptr} 变量。
函数返回一个 \acode{std::weak\_ptr} 给调用者,其不增加引用计数。
由于 \acode{ptr} 是本地变量,它在函数结束时离开作用域,即减少引用计数至 0,
同时释放 \acode{Resource} 对象。
返回的 \acode{std::weak\_ptr} 变成了悬垂指针,其指向的 \acode{Resource} 被释放过了。

\end{document}
