\documentclass[../../LearnCpp.tex]{subfiles}

\begin{document}

\asubsection{5}{std::move\_if\_noexcept}

20.9 异常规范与 noexcept 中覆盖了 \acode{noexcept} 异常说明符与操作符,即本篇的基础。

同样也覆盖了 \acode{strong exception guarantee},
其用于保证一个函数若是被异常打断,不会有内存泄漏同时程序状态也不会改变。
特别是所有的构造函数都应该是强异常保证的,因此程序的状态不会在对象的构造函数失败后有所改变。

\subsubsection*{移动构造函数异常问题}

在拷贝对象的情况下,拷贝如果由于某些原因(例如机器内存不足)失败了,
被拷贝的对象不会有任何的危害,因为原始对象在创建拷贝时并不需要被改变。
这样程序可以弃置失败的拷贝并继续执行。
\acode{strong exception guarantee} 是被支持的。

现在考虑移动对象的情况。
一个移动的操作转移了给定对象的所有权至目标对象。
如果移动操作在转移所有权之后被异常打断,那么原始对象会处于一个被修改状态。
这对于原始对象是临时对象而言并不是问题,因为在其被移动之后总是会被抛弃 --
但是对于非临时对象而言,这就破坏了原始对象。
为了遵从 \acode{strong exception guarantee},我们需要移动资源回去给原始对象,
但是如果第一次移动就失败了,就不能保证移动回去还能成功。

那么如何能给予移动构造函数 \acode{strong exception guarantee} 呢?
在移动构造函数中避免抛出异常很容易,但是其自身有可能唤起其他的构造函数是 \acode{potentially throwing} 的。
以 \acode{std::pair} 作为移动构造函数为例,需要移动每个字对象进入新的 pair 对象。

\begin{lstlisting}[language=C++]
// Example move constructor definition for std::pair
// Take in an 'old' pair, and then move construct the new pair's 'first' and 'second' subobjects from the 'old' ones
template <typename T1, typename T2>
pair<T1,T2>::pair(pair&& old)
  : first(std::move(old.first)),
    second(std::move(old.second))
{}
\end{lstlisting}

那么现在使用两个类,\acode{MoveClass} 与 \acode{CopyClass},
将它们 \acode{pair} 在一起来证明移动构造函数的 \acode{strong exception guarantee} 问题:

\begin{lstlisting}[language=C++]
#include <iostream>
#include <utility> // std::pair, std::make_pair, std::move, std::move_if_noexcept
#include <stdexcept> // std::runtime_error

class MoveClass
{
private:
  int* m_resource{};

public:
  MoveClass() = default;

  MoveClass(int resource)
    : m_resource{ new int{ resource } }
  {}

  // 拷贝构造函数
  MoveClass(const MoveClass& that)
  {
    // 深拷贝
    if (that.m_resource != nullptr)
    {
      m_resource = new int{ *that.m_resource };
    }
  }

  // 移动构造函数
  MoveClass(MoveClass&& that) noexcept
    : m_resource{ that.m_resource }
  {
    that.m_resource = nullptr;
  }

  ~MoveClass()
  {
    std::cout << "destroying " << *this << '\n';

    delete m_resource;
  }

  friend std::ostream& operator<<(std::ostream& out, const MoveClass& moveClass)
  {
    out << "MoveClass(";

    if (moveClass.m_resource == nullptr)
    {
      out << "empty";
    }
    else
    {
      out << *moveClass.m_resource;
    }

    out << ')';

    return out;
  }
};


class CopyClass
{
public:
  bool m_throw{};

  CopyClass() = default;

  // 拷贝构造函数在 m_throw 为 'true' 时抛出异常
  CopyClass(const CopyClass& that)
    : m_throw{ that.m_throw }
  {
    if (m_throw)
    {
      throw std::runtime_error{ "abort!" };
    }
  }
};

int main()
{
  // 创建一个没有任何问题的 std::pair
  std::pair my_pair{ MoveClass{ 13 }, CopyClass{} };

  std::cout << "my_pair.first: " << my_pair.first << '\n';

  // 但是当移动 pair 至另一个 pair 时,问题出现了
  try
  {
    my_pair.second.m_throw = true; // 触发拷贝构造函数的异常

    // 下一行将抛出异常
    std::pair moved_pair{ std::move(my_pair) }; // 之后将注释掉这一行
    // std::pair moved_pair{ std::move_if_noexcept(my_pair) }; // 之后将反注释这一行

    std::cout << "moved pair exists\n"; // 永不会打印
  }
  catch (const std::exception& ex)
  {
      std::cerr << "Error found: " << ex.what() << '\n';
  }

  std::cout << "my_pair.first: " << my_pair.first << '\n';

  return 0;
}
\end{lstlisting}

打印:

\begin{lstlisting}
destroying MoveClass(empty)
my_pair.first: MoveClass(13)
destroying MoveClass(13)
Error found: abort!
my_pair.first: MoveClass(empty)
destroying MoveClass(empty)
\end{lstlisting}

现在来探讨一下具体发生了些什么。

第一个打印出的行展示了用于初始化 \acode{my\_pair} 的临时 \acode{MoveClass} 对象,
其在 \acode{my\_pair} 实例化声明被执行之后立刻被销毁。
它为 \acode{empty} 是因为在 \acode{my\_pair} 中的 \acode{MoveClass} 子对象是由移动构造的,
在下一行的打印中证明了 \acode{my\_pair.first} 包含了 \acode{MoveClass} 对象的值 \acode{13}。

有趣的事情发生在第三行。
通过拷贝构造函数创建 \acode{moved\_pair} 的 \acode{CopyClass} 子对象(它没有移动构造函数),
但是拷贝构造函数抛出了异常,
因为修改了 Boolean 标记。\acode{moved\_pair} 的构造由于异常的原因失败了,
同时其早前构造的成员函数被删除。这个情况下 \acode{MoveClass} 成员被销毁,
打印 \acode{destroying MoveClass(13)},
接着就看到了 \acode{Error found: abort!} 信息被打印出来。

当尝试再次打印 \acode{my\_pair.first} 时,显示出 \acode{MoveClass} 成员为空。
因为 \acode{moved\_pair} 的初始化使用过 \acode{std::move} 完成的,
而 \acode{MoveClass} 成员(拥有移动构造函数)被移动构造了且 \acode{my\_pair.first} 变为空值。

最后 \acode{my\_pair} 在 \acode{main()} 结束前销毁。

\subsubsection*{std::move\_if\_noexcept 来拯救}

注意上述的问题本来是可以避免的,如果 \acode{std::pair} 尝试的是拷贝而不是移动,
这种情况下 \acode{moved\_pair} 将会构造失败,但是 \acode{my\_pair} 并不会被修改。

但是拷贝有性能成本,因此不想为所有对象都付出这种代价 --
如果安全的话,理想的情况是使用移动,其次才是拷贝。

幸运的是 C++ 有两种机制,可以组合在一起使用。

首先 \acode{noexcept} 函数是 no-throw/no-fail 的,它们是隐式符合 \acode{strong exception guarantee} 标准。
因此 \acode{noexcept} 移动构造函数是被保证可以成功的。

其次是使用标准库的函数 \acode{std::move\_if\_noexcept} 在移动构造函数中
(或者对象时仅可移动且没有拷贝构造函数时)不会抛出异常,
那么 \acode{std::move\_if\_noexcept} 将与 \acode{std::move} 完全相同(且返回被转换成右值的对象)。
否则 \acode{std::move\_if\_noexcept} 将返回对象的一个普通的左值引用。

关键点:如果对象拥有一个 noexcept 移动构造函数,那么 \acode{std::move\_if\_noexcept} 将返回一个可移动的右值,
否则将返回可拷贝的左值。
可以使用 \acode{noexcept} 说明符作为连接 \acode{std::move\_if\_noexcept},
在当强异常保证存在时,使用移动语义(否则使用拷贝语义)。

现在更新之前例子中的代码:

\begin{lstlisting}[language=C++]
//std::pair moved_pair{std::move(my_pair)}; // 现在注释掉这一行
std::pair moved_pair{std::move_if_noexcept(my_pair)}; // 并反注释这一行
\end{lstlisting}

现在运行程序再打印:

\begin{lstlisting}
destroying MoveClass(empty)
my_pair.first: MoveClass(13)
destroying MoveClass(13)
Error found: abort!
my_pair.first: MoveClass(13)
destroying MoveClass(13)
\end{lstlisting}

可以看到在异常抛出之后,子对象 \acode{my\_pair.first} 仍然指向值 \acode{13}。

\acode{std::pair} 的移动构造函数不是 \acode{noexcept},
因此 \acode{std::move\_if\_noexcept} 返回 \acode{my\_pair} 为一个左值引用。
这导致 \acode{moved\_pair} 通过拷贝构造函数(而不是移动构造函数)进行创建。
拷贝构造函数可以安全的抛出异常,因为它不会修改原始对象。

标准库进出使用 \acode{std::move\_if\_noexcept} 来优化 \acode{noexcept} 函数。
例如 \acode{std::vector::resize} 将使用移动语义如果元素的类型是 \acode{noexcept} 移动构造函数,否则使用拷贝语义。
这就意味着 \acode{std::vector} 通常对于拥有 \acode{noexcept} 移动构造函数的对象来说,其操作会更快。

警告:如果一个类型同时是潜在抛出异常的移动语义,以及删除了拷贝语义(拷贝构造函数和拷贝赋值操作符不可用),
那么 \acode{std::move\_if\_noexcept} 则会放弃强保证同时唤起移动语义。
这种条件性的放弃强保证是普遍存在于标准库的容器类中的,因为它们时常使用 \acode{std::move\_if\_noexcept}。

\end{document}
