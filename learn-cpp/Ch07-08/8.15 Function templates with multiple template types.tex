\documentclass[../../LearnCpp.tex]{subfiles}

\begin{document}

\asubsection{15}{Function templates with multiple template types}

\subsubsection*{使用 static\_cast 转换参数来匹配类型}

\begin{lstlisting}[language=C++]
#include <iostream>

template <typename T>
T max(T x, T y)
{
    return (x > y) ? x : y;
}

int main()
{
    std::cout << max(static_cast<double>(2), 3.5) << '\n'; // 转换 int 至 double 用以调用 max(double, double)

    return 0;
}
\end{lstlisting}

\subsubsection*{提供实际类型}

\begin{lstlisting}[language=C++]
#include <iostream>

template <typename T>
T max(T x, T y)
{
    return (x > y) ? x : y;
}

int main()
{
    std::cout << max<double>(2, 3.5) << '\n'; // 提供了实际类型 double,编译器不再使用模版类型推导

    return 0;
}
\end{lstlisting}

\subsubsection*{带有若干模版类型参数的函数模版}

\begin{lstlisting}[language=C++]
#include <iostream>

template <typename T, typename U> // 使用两个模版类型参数 T 和 U
T max(T x, U y) // x 为 T 类型, y 为 U 类型
{
    return (x > y) ? x : y; // 这里出现了缩窄变化的问题
}

int main()
{
    std::cout << max(2, 3.5) << '\n';

    return 0;
}
\end{lstlisting}

\begin{lstlisting}[language=C++]
#include <iostream>

template <typename T, typename U>
auto max(T x, U y)
{
    return (x > y) ? x : y;
}

int main()
{
    std::cout << max(2, 3.5) << '\n';

    return 0;
}
\end{lstlisting}

\subsubsection*{简化函数模版 C++20}

C++20 引入关键字 \acode{auto} 新用法:当 \acode{auto} 关键字用于普通函数的参数类型,
编译器会自动转换函数成为函数模版,其 auto 参数变为独立模版类型参数。
这个方法创建的函数模版被称为**\textbf{简化函数模版}**。

例如,在 C++20 中简写:

\begin{lstlisting}[language=C++]
auto max(auto x, auto y)
{
    return (x > y) ? x : y;
}
\end{lstlisting}

代替之间的:

\begin{lstlisting}[language=C++]
template <typename T, typename U>
auto max(T x, U y)
{
    return (x > y) ? x : y;
}
\end{lstlisting}

如果想要每个模版类型参数作为独立类型,推荐这种去除模版参数声明的形式,使得代码更简洁已读。

最佳实践:如果每个 auto 参数都需要独立的模版类型时,请使用简写函数模版(在语言标准设为 C++20 或更新的情况下)。

\end{document}
