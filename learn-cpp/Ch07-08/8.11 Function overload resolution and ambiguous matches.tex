\documentclass[../../LearnCpp.tex]{subfiles}

\begin{document}

\asubsection{11}{Function overload resolution and ambiguous matches}

\begin{lstlisting}[language=C++]
#include <iostream>

void print(int x)
{
     std::cout << x << '\n';
}

void print(double d)
{
     std::cout << d << '\n';
}

int main()
{
     print('a'); // char 不匹配 int 或 double
     print(5L); // long 不匹配 int 或 double

     return 0;
}
\end{lstlisting}

而 \acode{char} 或 \acode{long} 可以被隐式类型转换成 \acode{int} 或 \acode{double}。
那么它们分别用到哪种转换呢?

\subsubsection*{解析重载函数调用}

如果一个函数调用的是重载函数,编译器会经过以下一系列规则来决定使用最为匹配的函数。

每个步骤中,编译器会对入参应用不同类型的类型转换。
每次转换被应用时,编译器会检查是否有相匹配的重载函数。
在所有类型的转换被应用后以及所有的匹配被检查后,这个过程就结束了。
这样会得到以下三种结果中的一个:

\begin{itemize}
  \item 没有匹配的函数被找到。编译器移动至下一个步骤。
  \item 单个匹配的函数被找到。该函数被认为是最好的匹配。匹配过程现在完成嘞,剩下的步骤不再执行。
  \item 多个匹配函数被找到。编译器会报匹配有歧义的错误。之后会讨论这种结果。
\end{itemize}

\subsubsection*{参数匹配序列}

第一步,编译器尝试找到最匹配的。其中分为两个阶段。
首先,编译器查找是否有重载函数,检查调用者的入参类型是否严格匹配重载函数的入参类型。

编译器将一些 trivial 的转换应用于函数调用的参数。
\textbf{trivial conversions} 是一系列固定的转换规则用于修改类型(而不修改值本身)达到找到匹配的目的。

\begin{lstlisting}[language=C++]
void print(const int)
{
}

void print(double)
{
}

int main()
{
    int x { 0 };
    print(x); // x trivially 转换成 const int

    return 0;
}
\end{lstlisting}

进阶:转换一个非引用类型至一个引用类型(或者相反)同样也是 trivial conversion。

通过 trivial conversion 的匹配被视为精确匹配。

第二步,如果没有确切匹配被找到,编译器则会尝试应用数值提升 numeric promotion 来匹配入参。

\begin{lstlisting}[language=C++]
void print(int)
{
}

void print(double)
{
}

int main()
{
    print('a'); // 提升用于匹配 print(int)
    print(true); // 提升用于匹配 print(int)
    print(4.5f); // 提升用于匹配 print(double)

    return 0;
}
\end{lstlisting}

第三步,如果通过数值提升 numeric promotion 没有匹配上,编译器则会尝试对入参应用数值转换 numeric conversions。

\begin{lstlisting}[language=C++]
#include <string> // std::string

void print(double)
{
}

void print(std::string)
{
}

int main()
{
    print('a'); // 'a' 转换后匹配 print(double)

    return 0;
}
\end{lstlisting}

上述例子中,因为没有 \acode{print(char)}(精准匹配),也没有 \acode{print(int)}(promotion 匹配),
\acode{'a'} 则被数值转换为 double 进而匹配 \acode{print(double)}。

重点:应用数值提升 numeric promotions 的匹配优先于任何应用数值转换 numeric conversions 的匹配。

第四步,如果通过数值转化 numeric conversion 也没有匹配上,编译器会尝试寻找用户自定义的转换。
尽管暂时还没有讲到用户自定义转换,某些类型(例如,类)可以定义隐式转换成其他类型。

\begin{lstlisting}[language=C++]
// 暂时还没有讲到类
class X // 这里定义了名为 X 的类型
{
public:
    operator int() { return 0; } // 这里是 X 到 int 的用户自定义转换
};

void print(int)
{
}

void print(double)
{
}

int main()
{
    X x; // 这里创建类型为 X (名为 x) 的对象
    print(x); // x 通过用户自定义转换,被转换成 int 类型

    return 0;
}
\end{lstlisting}

相关内容:14.11 重载类型转换 overloading typecasts。

第五步,如果用户自定义转换没有匹配上,编译器则会匹配使用省略号的函数。

相关内容:12.6 省略号(以及为什么要避免它)。

第六步,如果上述都没有匹配上,编译器则会报无法找到匹配函数的错误。

\subsubsection*{模棱两可的匹配}

在没有重载函数的情况下,每个函数调用会找到该函数,或者匹配失败同时报编译器错误。

在有重载函数的情况下,则会有第三种可能发生:\acode{ambiguous match} 可能被找到。
\textbf{模棱两可匹配} ambiguous match出现在编译期找到两个或以上的函数可以同时匹配。
这时编译器会停止匹配并且报模棱两可函数调用的错误。

例如:

\begin{lstlisting}[language=C++]
void print(int x)
{
}

void print(double d)
{
}

int main()
{
    print(5L); // 5L 是 long 类型

    return 0;
}
\end{lstlisting}

另一个模棱两可匹配:

\begin{lstlisting}[language=C++]
void print(unsigned int x)
{
}

void print(float y)
{
}

int main()
{
    print(0); // int 可以被数值转换成 unsigned int 或者 float
    print(3.14159); // double 可以被数值转换为 unsigned int 或者 float

    return 0;
}
\end{lstlisting}

\subsubsection*{解决模棱两可匹配}

由于模棱两可匹配是编译期错误,因此在编译程序前需要消除歧义。这里有几种方法:

\begin{enumerate}
  \item 总是定义简单的重载函数,其入参类型是调用函数的类型。
  \item 或者是显式转换歧义入参成为函数定义的类型。
  \item 如果参数是字面值,使用后缀来确保字面值是正确的类型。
\end{enumerate}

\end{document}
